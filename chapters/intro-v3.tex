% Maybe something like

``What are we made of?''. This question has occupied the minds of many great thinkers and scientists throughout history. It is also written on the T-shirts of the researchers who participated in the 2016 Summer Student Programme at CERN, celebrating the very trait that brought them together: Curiosity. Indeed, the urge to continuously learn about the world seems deeply rooted in humanity.
%The field of \emph{particle physics} sits at the core of all this, asking the most fundamental questions: ``What is nature made of and what are its fundamental laws?''.
% The field of \emph{particle physics} is at the core of this, seeking to answer one of the most profound questions: ``What are the fundamental laws of nature?''.
The field of \emph{particle physics} is at the core of this and plays an integral role in the search for an answer to one of the most profound questions: ``What are the fundamental laws of nature?''.
%The field of \emph{particle physics} is at the center of all this, asking questions at the most fundamental level: ``What is nature made of and what are its fundamental laws?''.

In the search for these laws, the Standard Model (SM) of particle physics provides the best theory to date describing the behavior of all known fundamental particles. 
% The urge to continuously learn about the world seems deeply rooted in humanity.
% %As children we explore our neighborhood, as adults we learn about different professions and cultures, and as we grow older we learn about the lives of the younger generation. 
% The field of \emph{particle physics} has taken on the challenge of learning about all fundamental laws of nature. 
% In this quest, the Standard Model of particle physics (SM) provides the best theory to date describing the behavior of all known fundamental particles. 
The SM includes three of the four known fundamental forces and has been extremely successful in predicting and describing correctly various experimental measurements. 
The culmination of this success was the discovery of a new particle by the ATLAS and CMS collaborations in 2012~\cite{HIGG-2012-27,CMS-HIG-12-028}.
To date, ten years after this discovery, all experimental measurements show consistency of the observed particle with the properties of the long sought-after Higgs boson predicted by the SM almost 50 years earlier.
%Despite this success, however, the SM cannot explain several observable phenomena, such as the asymmetry of matter and antimatter in the universe, and much remains to be understood about the nature of the Higgs boson.

Despite this success, however, the SM cannot explain several observable phenomena, such as the apparent asymmetry between matter and antimatter in the universe, and theoretical puzzles persist, such as the unification of the SM and Einstein's theory of gravity.
Many of the open questions can be directly linked to the Higgs boson, whose exact nature remains to be understood: Is the Higgs boson a fundamental particle? Why is the mass of the Higgs boson so small? Are there more particles similar to the Higgs boson?

Many theories beyond the SM that attempt to address these shortcomings predict two major modifications: First, they posit the existence of new fundamental particles, which motivates the direct search for experimental signatures of these particles; and second, they predict changes in the Higgs boson phenomenology predicted by the SM, which calls for a precise study of the properties of the Higgs boson. 
The latter becomes particularly interesting since no direct discovery of new fundamental particles other than the Higgs boson has been made in recent years. 
Specifically, the precise measurement of the Higgs boson's production and decay processes is crucial to constrain theories beyond the SM. 
%This can also reveal effects that new phenomena not directly accessible in current experiments may have in the experimentally measurable realm, thereby indirectly paving the way to new discoveries. 
This can also reveal effects that, for example, new particles too heavy to be produced in current experiments may have on SM processes at lower energies, thereby indirectly paving the way to new discoveries. 
% This may also reveal effects that new phenomena -- which are not directly accessible in experiments -- may have in the experimentally measurable domain, thereby indirectly paving the way to new discoveries. 
% It may also reveal effects that new phenomena -- which are not directly accessible in experiments -- may have in the experimentally accessible domain. %accessible energies.
% Specifically, the precise measurement of the Higgs boson's production and decay processes is crucial to constrain theories beyond the SM and may indirectly pave the way to new discoveries. 
%It also allows for measuring potential effects of new phenomena that are not directly within experimental reach at lower energies. 
% Specifically, the precise measurement of the Higgs boson's production and decay processes is crucial to constrain theories beyond the SM and may also reveal possible deviations from the SM predictions, thereby indirectly paving the way to new, undiscovered phenomena. 
%It also allows for measuring potential effects of new phenomena that are not directly within experimental reach at lower energies. 
% Many theories beyond the SM that attempt to address these shortcomings have two main implications: First, they posit the existence of new fundamental particles, and second, they predict changes in the Higgs boson phenomenology predicted by the SM. 
% The latter is the target of this thesis and one of the most promising research directions, in particular, since no direct discovery of new particles has been made in recent years. 
% Many theories beyond the SM that attempt to address these shortcomings predict changes in the Higgs boson phenomenology predicted by the SM. 
% This makes the study of the Higgs boson one of the most promising research directions, in particular, since no direct discovery of new particles has been made in recent years. 


% Especially since no direct discovery of new particles has been made in recent years, the precise study of the Higgs boson becomes crucial, as it may reveal possible deviations from the SM that could indirectly pave the way to new, undiscovered phenomena. 
%The main goal of experimental particle physics is to probe these theories to pave the way to a more complete understanding of nature.
%Despite this success, however, there are several questions the SM cannot answer, and much remains to be understood about the nature of the Higgs boson.
% Especially since no direct discovery of new particles has been made in recent years, the study of the Higgs boson is among the most promising research directions.
% In particular, the precise measurement of the interactions of the Higgs boson with all other fundamental particles becomes crucial. 
% This helps to constrain theories beyond the SM, and may also reveal possible deviations from the SM predictions that could pave the way to new, undiscovered phenomena. 
% This allows theories beyond the SM to be constrained and allows potential deviations from the SM predictions to be measured, possibly paving the way to new, undiscovered phenomena. 

%The rate at which Higgs bosons are produced is predicted by the SM, making it an important task to test these predictions. This can be achieved by counting the number of Higgs bosons produced in high energy particle collider experiments. 
The properties of the Higgs boson can be studied in high energy physics (HEP) particle collider experiments. 
The only collider that is currently able to produce enough Higgs bosons for study is the Large Hadron Collider (LHC), colliding protons at center-of-mass energies of up to $\sqrt{s} = 13\,$TeV.\footnote{Even higher energies are planned for future operating cycles of the LHC.}
In the proton-proton ($pp$) collision events, various physical processes occur with different probabilities dictated by the principles of quantum mechanics.
%In the proton-proton ($pp$) collision events, various physical processes occur with different probabilities determined by quantum mechanical phenomena.
%The rate at which various physical processes occur in the proton-proton ($pp$) collisions events is determined by quantum mechanical phenomena.
%The physical processes occurring in the proton-proton ($pp$) collisions are of probabilistic nature, and their outcomes are measured and counted with large detector facilities built around the interaction points. 
Their outcomes are measured and counted with large detector facilities built around the interaction points. 
%The outcomes of the $pp$ collision events are measured and counted with large detector facilities built around the interaction points. 
%The outcome of the proton-proton ($pp$) collision events are measured with large detector facilities built around the interaction points. 
The ATLAS experiment is one such detector, designed as a multipurpose detector to measure a variety of detector signals produced, for example, by $pp$ collision events in which a Higgs boson was created.
% cross section definition 
% a measure of the probability that a specific process will take place
% after they are created in the $pp$ collisions.
%Given the probabilistic nature of these physical processes, the counting of the number of Higgs bosons is an essential task of particle physics experiments. 
% The production of Higgs bosons is a probabilistic process, making it important to measure the production rates precisely. 
%but particularly well-suited to measuring the kind of detector signals that Higgs bosons generate after being produced in the $pp$ collisions.

The Higgs boson decays almost instantaneously after being produced. Hence, only its decay products can be measured in the detector. 
Since the Higgs boson couples to all massive particles, it can be produced in various production processes and has multiple possible decay modes, which leads to a rich phenomenology of Higgs boson physics. 
% The rate at which Higgs bosons are produced is an essential property to determine experimentally and quantified by the measurement of \emph{cross sections}. 
The rate at which Higgs bosons are produced is an essential property to determine experimentally and is quantified by the measurement of \emph{cross sections}. 

This thesis presents cross-section measurements of Higgs boson production identified by its decay to a pair of $W$ bosons.
The measurements target the gluon fusion (ggF) and vector-boson fusion (VBF) production modes of the Higgs boson, measuring their cross sections inclusively as well as differentially in various kinematic regions.
The analysis of \HWW\ decays enables some of the most sensitive measurements of Higgs boson processes at the LHC, as it is the second most likely decay mode of the Higgs boson, and provides an excellent control over other types of $pp$ collision events with similar detector signatures considered as backgrounds. 
%\TDinote{The latter is ensured by selecting events with two leptons in the final state to target \HWWdet decays with leptonically decaying $W$ bosons.}{Maybe too specific for intro?}
In particular, the VBF, \HWW process enables the most sensitive measurements of the coupling strength of the Higgs boson to vector bosons at the LHC. 
% . In addition, leptons originating from the $W$ boson decays provide an excellent control over other types of $pp$ collision events considered as backgrounds.
% For this reason, the analysis of \HWWdet decays enables some of the most sensitive measurements of Higgs boson production at the LHC.
The analysis is conducted with $pp$ collision data recorded during the second period of data taking at the LHC between 2015 and 2018, known as \RunTwo. The dataset corresponds to an integrated luminosity of 139\,\ifb\ and was recorded at a center-of-mass energy of $\sqrt{s} = 13\,$TeV.
%The ggF and VBF production cross sections are measured inclusively as well as differentially in various kinematic regions, using the framework of Simplified Template Cross Sections. 
The measurements are performed in a way that allows them to be combined with results obtained for other Higgs boson processes to ultimately provide the most precise and comprehensive measurements of the properties of the Higgs boson.
%for example, measurements of $H \to ZZ$ and $H \to \gamma\gamma$ decays. 
%Combined measurements like these 
% Furthermore, the Higgs boson measurements are used as baseline for theory-agnostic interpretations of the data for example in the framework of Effective Field Theories.

The success of the \HWW analysis is determined by two main factors: 
First, the ability to precisely estimate the expected number of events from SM sources given a set of event selection criteria applied to LHC data collected by the ATLAS experiment; and second, the ability to distinguish collision events with interesting Higgs boson processes from other types of events.
In this thesis, studies are presented that work towards both of these goals.
A main contribution to the first task is the precise measurement of the energy resolution of jets, which allows selecting Higgs boson events more accurately and reduces measurement uncertainties.
The second challenge is addressed in the analysis of the VBF signal.
%, which is difficult to distinguish from non-VBF signal events. 
A supervised machine learning algorithm is used to develop a deep neural network that recognizes patterns of the VBF signal to distinguish it from background events.
% One may argue, by teaching machines how to learn, humans are able to learn more about the fundamental laws of nature. 
%These patterns are too complex and multidimensional for simple analysis strategies to be efficient, so machines are taught to learn them for us.
%separate Higgs events from backgrounds events. 
This contributes significantly to achieving more precise measurements of the VBF production mode of the Higgs boson.
%and thus to making progress in learning about the fundamental laws of nature.

% 1. Epic Intro
% - The drive to learn seems deeply rooted into humanity. 
% - Learning sits at the core of humanity. 
%  Learning the fundamental laws of nature is the challenge elementary particle physics has set itself up for. 
% The quest for learning how nature behaves is driven by curiosity only. 

% 2. Particle physics / SM
% The current best theory to describe the behavior of the fundamental building blocks of matter, called elementary particles, is known as the Standard Model. 
% It has been hugely successful in describing various experimental measurements correctly. 
% The Higgs boson discovery in 2012 can be regarded as the culmination of success the success story. 
% It updated the status of particle physics as a whole and largely influences future research directions. 
% There open problems and much remains to be understood!

% 3. Higgs takes special place
% The Higgs boson sits at the core of the theory, being connected to 10 out of the 19 free parameters that need to be determined by experimental data.  
% This makes it very important to understand all aspects of Higgs boson physics precisely. In particular, the probability of the Higgs boson to be produced, measured in terms of cross sections, or the interactions with other particles.
% A comprehensive study of the Higgs boson and all its interactions with other particles is therefore essential. 

% 4. Experiments (LHC,  ATLAS)
% Large particle colliders allow testing SM predictions and looking for new phenomena. 
% ATLAS detector

% 5. Analysis 
% This thesis presents a measurement of Higgs production cross sections in its decay to a pair of $W$ bosons. 
% The HWW decay is the second most likely one and provides an excellent opportunity to study the kinematics of specifically the gluon fusion as well as the vector boson fusion production mode of the Higgs boson.

% The success of most physics analyses is determined by two main factors: 
% (i) 
% - the ability to cleverly select collision events so the amount of signal events that are to be measured is large compared to the other ``background'' events.  
% - The ability to distinguish interesting Higgs boson signal events from other types of events.
% (ii) the ability to precisely estimate the expected contributions given the selection criteria (query), with as little uncertainty as possible.

% 6. JER and Higgs cross sections
% For the latter, this thesis presents a measurement of the noise term of the jet energy resolution, allowing to select and disentangle events at a finer resolution. 
% For the former, this thesis presents the development of a deep neural network that discriminates Higgs boson events produced via vector boson fusion. A supervised learning algorithm is used for a neural net to learn to recognize patterns from labelled example data and separate Higgs events from backgrounds events. These patterns are too complex and multidimensional for a human to find, so machines are taught to learn them for us. 
% This allows measuring the amount of VBF, HWW events produced at unprecedented precision. 

% 7. deviations, STXS, further interpretations. 
% The precision measurements of HWW decays can be used for further combinations and interpretations. 

