
The urge to continuously learn about the world seems deeply rooted in humanity.
%As children we explore our neighborhood, as adults we learn about different professions and cultures, and as we grow older we learn about the lives of the younger generation. 
\emph{Particle physics} has taken on the challenge of learning about the fundamental laws of nature. 
In this quest, the Standard Model of particle physics (SM) provides the best theory to date describing the behavior of all known fundamental particles. 
The SM includes three of the four fundamental forces of nature and has been extremely successful in predicting and describing correctly an enormous number of experimental measurements. 
The culmination of this success was the discovery of a new particle by the ATLAS and CMS collaborations in 2012~\cite{HIGG-2012-27,CMS-HIG-12-028}.
To date, ten years after this discovery, all experimental measurements show consistency of the observed particle with the properties of the long sought-after Higgs boson predicted by the SM almost 50 years earlier.
Despite this success, however, there are several questions the SM cannot answer, and much remains to be understood about the nature of the Higgs boson.
Several observable phenomona cannot be explained, such as the asymmetry between matter and antimatter in the universe, and theoretical puzzles prevail, such as the unification of the SM and Einstein's theory of gravity, which is still pending.
Other open questions are directly related to the Higgs boson and need further study: Is the Higgs boson a fundamental particle? Are there more particles similar to the Higgs boson? Why is the mass of the Higgs boson so small? 
There are several theories beyond the SM that attempt to address these shortcomings. 
The main goal of experimental particle physics is to probe these theories and thus pave the way to a more complete understanding of nature.
Many of the theories posit the existence of new fundamental particles and also predict implications on the Higgs boson phenomenology.
% no direct discovery of new fundamental particles has been made in recent years. 
In particular since no direct discovery of new particles has been made in recent years, the study of the Higgs boson is therefore among the most important research directions. 
%as the said theories do not only predict new particles, but also predict implications on the Higgs boson phenomenology.
Precisely measuring the properties of the Higgs boson allows theories to be studied with high precision and allows potential deviations from the Standard Model predictions to be measured, possibly pointing to new, undiscovered phenomena. 

The properties of the Higgs boson can be studied in high energy particle collider experiments. 
The Large Hadron Collider (LHC) is the most powerful particle accelerator to date, colliding protons at center-of-mass energies of up to $\sqrt{s} = 13\,$TeV. It is the only collider that is currently able to produce enough Higgs bosons for study.
%\footnote{Even higher energies are planned for the next operating cycle of the LHC, which will begin at the turn of 2022 and 2023.}
The outcome of the proton-proton ($pp$) collision events are measured with large detector facilities built around the interaction points. 
The ATLAS experiment is one such detector, designed as a multipurpose detector but particularly well-suited to measuring the kind of detector signals that Higgs bosons generate after being produced in the $pp$ collisions.

The Higgs boson decays almost instantaneously after being produced. Hence, only its decay products can be measured in the detector. Since the Higgs boson couples to all massive particles, it can be produced in various production processes and has multiple possible decay modes, which leads to a rich phenomenology of Higgs boson physics. 

This thesis presents cross-section measurements of Higgs boson production in its decay to a pair of $W$ bosons.
The \HWW\ decay is the second most likely decay mode of the Higgs boson. In addition, leptons originating from the $W$ boson decays provide an excellent control over other types of $pp$ collision events considered as backgrounds.
For this reason, the analysis of \HWWdet decays enables some of the most sensitive measurements of Higgs boson production at the LHC.
The \HWWdet analysis presented is conducted with $pp$ collision data recorded during the second period of data taking at the LHC between 2015 and 2018, known as \RunTwo. The dataset corresponds to center-of-mass energies of $\sqrt{s} = 13\,$TeV and an integrated luminosity of 139\ifb.
The measurement targets the gluon fusion (ggF) and vector boson fusion (VBF) production of the Higgs boson, measuring their cross sections inclusively as well as differentially in various kinematic regions.
%The ggF and VBF production cross sections are measured inclusively as well as differentially in various kinematic regions, using the framework of Simplified Template Cross Sections. 
The measurements are performed in a way that allows them to be combined with measurements of other Higgs boson processes to ultimately provide the most precise measurements.
%for example, measurements of $H \to ZZ$ and $H \to \gamma\gamma$ decays. 
%Combined measurements like these 
% Furthermore, the Higgs boson measurements are used as baseline for theory-agnostic interpretations of the data for example in the framework of Effective Field Theories.

The success of the \HWWdet analysis is determined by two main factors: 
(i) the ability to precisely estimate the expected number of events given a set of event selection criteria, and (ii) the ability to distinguish collision events with interesting Higgs boson events from other types of events.
In this thesis, studies are presented that work towards both of those goals.
A contribution to the first task is made by measuring the energy resolution of jets as precisely as possible, which allows selecting Higgs boson events more accurately and reduces measurement uncertainties.
The second challenge is addressed in \HWWdet decays, with the Higgs boson produced via VBF.
The VBF signal events are difficult to distinguish from non-VBF events. 
A supervised machine learning algorithm is therefore used to develop a deep neural network that learns to recognize patterns of the VBF signal and the VBF signal to be discriminated from the backgrounds.
One may argue, by teaching machines how to learn, humans are able to learn more about the fundamental laws of nature. 
%These patterns are too complex and multidimensional for simple analysis strategies to be efficient, so machines are taught to learn them for us.
%separate Higgs events from backgrounds events. 

These studies contribute to making progress on understanding the world around us. 

% 1. Epic Intro
% - The drive to learn seems deeply rooted into humanity. 
% - Learning sits at the core of humanity. 
%  Learning the fundamental laws of nature is the challenge elementary particle physics has set itself up for. 
% The quest for learning how nature behaves is driven by curiosity only. 

% 2. Particle physics / SM
% The current best theory to describe the behavior of the fundamental building blocks of matter, called elementary particles, is known as the Standard Model. 
% It has been hugely successful in describing various experimental measurements correctly. 
% The Higgs boson discovery in 2012 can be regarded as the culmination of success the success story. 
% It updated the status of particle physics as a whole and largely influences future research directions. 
% There open problems and much remains to be understood!

% 3. Higgs takes special place
% The Higgs boson sits at the core of the theory, being connected to 10 out of the 19 free parameters that need to be determined by experimental data.  
% This makes it very important to understand all aspects of Higgs boson physics precisely. In particular, the probability of the Higgs boson to be produced, measured in terms of cross sections, or the interactions with other particles.
% A comprehensive study of the Higgs boson and all its interactions with other particles is therefore essential. 

% 4. Experiments (LHC,  ATLAS)
% Large particle colliders allow testing SM predictions and looking for new phenomena. 
% ATLAS detector

% 5. Analysis 
% This thesis presents a measurement of Higgs production cross sections in its decay to a pair of $W$ bosons. 
% The HWW decay is the second most likely one and provides an excellent opportunity to study the kinematics of specifically the gluon fusion as well as the vector boson fusion production mode of the Higgs boson.

% The success of most physics analyses is determined by two main factors: 
% (i) 
% - the ability to cleverly select collision events so the amount of signal events that are to be measured is large compared to the other ``background'' events.  
% - The ability to distinguish interesting Higgs boson signal events from other types of events.
% (ii) the ability to precisely estimate the expected contributions given the selection criteria (query), with as little uncertainty as possible.

% 6. JER and Higgs cross sections
% For the latter, this thesis presents a measurement of the noise term of the jet energy resolution, allowing to select and disentangle events at a finer resolution. 
% For the former, this thesis presents the development of a deep neural network that discriminates Higgs boson events produced via vector boson fusion. A supervised learning algorithm is used for a neural net to learn to recognize patterns from labelled example data and separate Higgs events from backgrounds events. These patterns are too complex and multidimensional for a human to find, so machines are taught to learn them for us. 
% This allows measuring the amount of VBF, HWW events produced at unprecedented precision. 

% 7. deviations, STXS, further interpretations. 
% The precision measurements of HWW decays can be used for further combinations and interpretations. 

