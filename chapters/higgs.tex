\chapter{Higgs Boson Physics}
\label{chap:higgs}
% The SM has been extremely successful in making predictions that were later confirmed by experimental measurements. Examples are the observation of the $W$ and $Z$ bosons in 1983 by the UA1 and UA2 collaborations, or the top quark discovery in 1995 by the CDF and D0 experiments. \todo{Maybe references}
% The observation of the $W$ and $Z$ bosons by the UA1 and UA2 collaborations in 1983 are just two examples of this.
% The discovery of the long sought-after Higgs boson in 2012 by the ATLAS and CMS collaborations~\cite{Aad:2012tfa,Chatrchyan:2012xdj} marks the highlight of this success story. 
% The success story culminated in the discovery of a new particle by the ATLAS and CMS collaborations in 2012~\cite{HIGG-2012-27,CMS-HIG-12-028}. To date, ten years after this discovery, the experimental evidence is overwhelming that the observed particle is the long sought-after Higgs boson predicted by the SM.

% The particle discovered in 2012~\cite{HIGG-2012-27,CMS-HIG-12-028} is now commonly referred to as the Higgs boson, since all measurements of its properties made so far are consistent with the properties of Higgs boson properties predicted by the SM.

% After the discovery of a new particle in 2012~\cite{HIGG-2012-27,CMS-HIG-12-028}, subsequent measurements showed consistency of the observed particle with the Higgs boson predicted by the SM. 
% However, the uncertainties on many of the measurements are still sizable and much remains to be understood. 
The Higgs boson plays a special role in the SM in many ways.
It is the only fundamental particle in the SM with a spin of zero and is connected to a large fraction of the free parameters (see \cref{subsec:final-lagrangian}). 
% It is responsible for the only interaction that distinguishes between the generations of fermions, and
It is the only boson that is not a consequence of the principle of local gauge invariance, but arises from the introduction of EWSB via a scalar doublet field. 

The unique nature of the Higgs boson makes it a prime candidate for searching for physics beyond the SM and thus relevant to many of the open fundamental questions~\cite{2019BHeinemann}. 
Some dark matter models, for example, predict interactions of the Higgs boson with yet unknown particles that are candidates for dark matter~\cite{Baumgart_2009,Kaplan_2009,Dienes_2012}. 
Other models such as two-Higgs-doublet models~\cite{Branco_2012}\footnote{The main motivation for considering two-Higgs-doublet models is supersymmetry as well as the fact that it introduces new potential sources for CP violation.} hypothesize the existence of additional particles similar to the Higgs boson as well as deviations from the SM predictions for the Higgs boson.
%This would provide new sources of CP violation and
%This is assumed in many supersymmetric models and hypothesizes new sources of CP violation. 

% The Higgs boson is mostly recognized as a result of the Higgs mechanism that provides masses to the $W$ and $Z$ bosons. 
The existence of the Higgs boson also solves a theoretical problem related to $WW$ scattering.
The cross section of the scattering of longitudinally polarized $W$ bosons ($W_LW_L \to W_LW_L$) would diverge at high energies in the absence of the Higgs boson. These divergencies cancel only if the coupling of the Higgs boson to the $W$ bosons is exactly as predicted by the SM.

These, among others, are important reasons for making precision measurements of the Higgs sector to test the predictions of the SM with ever-increasing precision and possibly find deviations from them. 
%This provides stringent constraints on models beyond the SM and may reveal deviations from the SM that could prove to be a gateway to the discovery of new physics. 

%This year (2022) marks the ten years anniversary of the Higgs boson discovery which is celebrated with a paper \cite{NaturePaper} that summarizes all state-of-the art Higgs measurements.
%The paper demonstrates that the observation of the Higgs in 2012 was only the beginning of an era, the era of Higgs precision measurements.
This section provides an overview of the phenomenology of Higgs boson physics. \Cref{subsec:higgschannels} summarizes the different production and decay modes of the Higgs boson at the LHC, and discusses their experimental accessibility. \Cref{subsec:xsec-measurements} outlines different types of Higgs boson measurements performed at the LHC, before \cref{subsec:higgs-exp-status} concludes with a brief overview of the experimental status of Higgs boson physics at the time of writing.
%Many more details on the status of Higgs boson physics can be found in \cref{PDG2020}.

% From PDG:
% All these BSM scenarios can have important effects on the phenomenology of the Higgs boson.


% So far, there is no indication of deviations from the SM. 
% - Properties of the Higgs boson in agreement with the spin-parity JP = 0+ predicted by the SM. 
% - Cross sections all in agreement with the SM predictions. Firmly establishing VBF production in this thesis.
% - There is no indication that the found particle is not the SM. 
%The contrary, the experimental evidence is overwhelming that the measured particle is indeed the particle predicted by the SM.

% Wikipedia
%Since the Higgs field is scalar, the Higgs boson has no spin. The Higgs boson is also its own antiparticle, is CP-even, and has zero electric and color charge.[163]
% The Standard Model spin-parity JP = 0+
% \subsection{The role of the Higgs boson in the SM}
% \todo{Maybe add such a section -> Look for resources first!}

% \subsection{Motivation for Higgs physics}
% The Higgs boson plays a special role in the SM. 
% It is the only fundamental particle with a spin of zero and connected to a large fraction of the free parameters of the SM (see \cref{subsec:final-lagrangian}). 
% It is also the only boson that is not a consequence of the principle of local gauge invariance, but is added in an ad-hoc solution in the mechanism of EWSB. 
% Furthermore, the Higgs boson is a prime candidate for searching for physics beyond the SM and thus relevant to many of the open fundamental questions \cite{2019BHeinemann}.\footnote{This includes, for example, the search for dark matter (e.g. \ccite{Baumgart_2009,Kaplan_2009,Dienes_2012}) by testing models that predict interactions of the Higgs boson with yet unknown particles that are candidates for dark matter; or investigations of the process of EWSB itself by testing two-Higgs-doublet models~\cite{Branco_2012} that predict the existence of additional particles similar to the Higgs boson.}
% %\cite{2019BHeinemann} 

% The above-mentioned provides important reasons for performing precision measurements of the Higgs sector to test the predictions of the SM at an ever greater statistical precision and thus providing stringent constraints on models beyond the SM.
% Only with a precise knowledge of Higgs processes and an understanding of what exact role the Higgs boson plays in nature, fundamental physics can come closer to answering some of the open fundamental questions.  

% Precisely measuring the properties and cross sections of the Higgs boson is needed in order to understand the exact role it plays in nature and thus getting closer to answering some of the open fundamental questions. 

% From gianotti
% o gain even deeper insights into the Higgs boson and its role in fundamental physics

%OR: Precisely measuring the properties and cross sections of the Higgs boson is needed in order to understand the exact role it plays in nature and thus getting closer to answering some of the open fundamental questions. 

% From nature
% The measurements of those production and decay rates probe the strength of the interactions between the Higgs boson and the particles involved. This allows a test of an essential prediction of the SM: that the interaction strengths scale with particle masses.

% provide stringent constraints on many models of new phenomena beyond the Standard Model.


\section{Higgs Boson Production and Decay Modes}
\label{subsec:higgschannels}
The Higgs boson directly couples to all massive particles and therefore has various production and decay modes.
The relative contributions of the various modes (or \emph{channels}) are dependent on the mass of the Higgs boson, which is determined to be $m_H = 125.38 \pm 0.14\,\GeV$ in the most precise measurement to date performed by the CMS collaboration~\cite{CMS-HIG-19-004}.
%is determined to be $m_H=125.09 \pm 0.21(stat.) \pm 0.11(syst.)\,\GeV$ . 
%This offers a rich field of experimental signatures that can be explored by experiments.

\subsubsection{Production modes} 
The Higgs boson production cross sections depend, among other things, on the center-of-mass energy of the collider.
At the LHC, the four production modes illustrated in \cref{fig:higgsprodfeyn} dominate.
Their corresponding cross sections are shown in \cref{fig:higgsprodxsec}.
% \emph{gluon fusion} (ggF), \emph{vector-boson fusion} (VBF), \emph{Higgs-strahlung} from $W$ or $Z$ boson (VH), and $t\bar{t}$ fusion (ttH).
%- as illustrated in \cref{fig:higgsprodfeyn} with a choice of Feynman diagrams.
%associated production with a $t\bar{t}$ pair (also called \emph{$t\bar{t}$ fusion})

The \emph{gluon fusion} (ggF) process is the leading mechanism at the LHC, characterized by a virtual fermion loop coupling to the Higgs boson and no additional object in the final state (therefore labelled as $pp\rightarrow H$). 
The fermions in the loop are dominated by top quarks because of their high mass and the Higgs boson couples stronger to particles with higher mass. 

%The ggF cross section is calculated with an effective theory to NNNLO in QCD and includes electroweak corrections at NLO precision \cite{Anastasiou:2016cez}.\todo{double-check}
The second-largest contribution to Higgs boson production comes from the \emph{vector-boson fusion} (VBF) process, where two incoming quarks radiate a vector boson that fuse into the Higgs.
The VBF production mode is characterized by the two quarks in the final state that are emitted in the forward directions and have a large invariant mass ($pp\rightarrow qqH$). This provides a distinct signature suitable for determining Higgs boson couplings at the LHC.
%The VBF Two quarks at large radii and with large invariant mass in the final state.
% This feature can be exploited to distinguish between collision events and select the subsequent decays of the Higgs boson. 
% In addition, the VBF production mode provides information about the couplings of the Higgs boson to the $W$ and $Z$ bosons. 
%Observation of the VBF production mode of the Higgs boson was found by a combined measurement of the ATLAS and CMS collaborations\cite{Khachatryan:2016vau}.

The \emph{Higgs-strahlung} process from $W$ or $Z$ bosons ($VH$) has an additional vector boson in the final state besides the Higgs boson ($pp \rightarrow WH$ and $pp \rightarrow ZH$), that provide distinctive detector signatures dependent on the decay mode of the vector bosons. 

The Higgs boson production in association with a top-quark pair ($ttH$) features two top quarks in the final state. It plays an important role in Higgs boson physics, since it allows to directly measure the Yukawa coupling of the top quark. 

Other production modes have smaller contributions but are still relevant in particular when being combined with other channels.
These include the production in association with a $b\bar{b}$ pair ($pp\rightarrow bbH$) or a single-top quark ($pp \rightarrow tH$) that have additional $b$ quarks or a single top quark in the final state. 

One of the most interesting production modes is the double Higgs boson production, as it provides crucial information about the Higgs potential and the Higgs boson self coupling. Double Higgs boson production is driven by the gluon fusion process but has a very low cross section and will therefore only become fully experimentally accessible with data from the High-Luminosity LHC after 2029. 
% difficult to measure due to the relatively low production cross section and the less distinct final state ($pp \rightarrow ttH$).
%  It plays an important role in Higgs boson physics, however, since it allows to directly measure the Yukawa coupling of the top quark. 
%The most recent results of the ATLAS collaboration yield evidence for the $t\bar{t}$-fusion production of the Higgs boson \cite{ATLAS-CONF-2017-077}.

\subsubsection{Decay modes}
Once the Higgs boson is produced, it decays almost instantaneously with a lifetime of about $10^{-22}$ seconds \cite{PDG2020}.
Therefore, only its decay products can be measured.
As discussed in \cref{sec:sm}, the coupling between the Higgs boson and vector bosons is predicted to be proportional to the squared mass of the vector bosons, and the Higgs boson's coupling to fermions is predicted to be proportional to the mass of the fermions. The resulting \emph{branching fractions} of the most relevant decays are displayed in \cref{fig:higgsbr}. 
The decay into a pair of bottom quarks ($H\rightarrow b\bar{b}$) is the leading process followed by the $H\rightarrow WW^*$ decay mode. Since the mass of the Higgs boson is lower than the summed mass of two $\Wpm$ or $Z$ bosons, the decay into vector bosons is suppressed and one of the decaying vector bosons appears as a virtual particle. The decay into other particles occurs significantly less often; some of the decay modes nonetheless produce detector signatures that are very suitable for studying the Higgs boson, as explained below.

\begin{figure}
  \begin{center}
    \subfloat[]{
      % use valgin if pie chart is shown for decays
      \includegraphics[trim=0 100 0 120, width=0.48\textwidth]{figures/theory/Higgs_Prod_XSec_vertLine_fixed.pdf}
      \label{fig:higgsprodxsec}
    }
    % \hspace{-5em}
    % \captionsetup[subfloat]{captionskip=50pt} % space between subfloat caption and image
    % \subfloat[]{
    %   \includegraphics[scale=0.8,valign=t]{figures/theory/h-decay-pie/h-decay-pie.pdf}
    %   \label{fig:higgsbr}
    % }
        \subfloat[]{
            \includegraphics[trim=0 100 0 120, width=0.48\textwidth]{figures/theory/SMHiggsBR_vertLine_fixed.pdf}
            \label{fig:higgsbr}
          }
  \end{center}
  \caption[Higgs boson production cross sections and decay branching fractions.]{(a) Higgs boson production cross sections as a function of the LHC center-of-mass energy and (b) branching fractions of the Higgs boson. The black vertical lines indicate (a) a center-of-mass energy of $\sqrt{s} = 13\,$TeV and (b) the currently measured mass of the Higgs boson including uncertainties by the ATLAS experiment of $m_H = 124.99 \pm 0.19\,\GeV$~\cite{https://doi.org/10.48550/arxiv.2207.00320}. Adapted from Ref.~\cite{deFlorian:2016spz}.}
  % \caption{(a) Higgs boson production cross sections as a function of the LHC center-of-mass energy. The line widths represent the respective theory uncertainties. Taken from Ref.~\cite{deFlorian:2016spz}. (b) branching fractions of the Higgs boson for a mass of 125\,\GeV. Values taken from Ref.~\cite{PDG2020}.}
  %  \label{fig:higgsbr}
\end{figure}

\captionsetup[subfloat]{captionskip=10pt} % space between subfloat caption and image
\begin{figure}
\subfloat[Vector-boson fusion (VBF), $V=W,Z$] {
    \newImageResizeCustom{0.4}{figures/feynman-graphs/Higgs/ProductionModes/VBF.pdf}
}
\subfloat[Gluon fusion (ggF)] {
    \newImageResizeCustom{0.4}{figures/feynman-graphs/Higgs/ProductionModes/ggF.pdf}
} \\
\subfloat[Higgs-strahlung (VH), $V=W,Z$] {
  \newImageResizeCustom{0.4}{figures/feynman-graphs/Higgs/ProductionModes/VH.pdf}
}
\subfloat[$t\bar{t}H$] {
  \newImageResizeCustom{0.4}{figures/feynman-graphs/Higgs/ProductionModes/ttH.pdf}
}
\caption[Feynman diagrams of Higgs boson production.]{Representative Feynman diagrams of the four main production modes of the Higgs boson at the LHC.}
\label{fig:higgsprodfeyn}
\end{figure}
\resetcaptionoffset

\subsection{Experimental sensitivity at the LHC}
\label{subsec:exp-accessibility}
% From PDG
%For a given mH, the sensitivity of a channel depends on the production cross section of the Higgs
% boson, its decay branching fraction, the reconstructed mass resolution, the selection efficiency and
% the level of background in the final state. For a low-mass Higgs boson (110 GeV < mH < 150 GeV)
% for which the SM width would be only a few MeV, five decay channels play an important role
% at the LHC. In the H → γγ and H → ZZ∗ → 4` channels, all final state particles can be
% very precisely measured and the reconstructed mH resolution is excellent (typically 1-2%). While
% the H → W+W− → ` +ν`` 0−ν¯`
% 0 channel has relatively large branching fraction, however, due
% to the presence of neutrinos which are not reconstructed in the final state, the mH resolution,
% obtained through observables sensitive to the Higgs boson mass such as the transverse mass, is poor
% (approximately 20%). The H → b ¯b and the H → τ +τ
% − channels suffer from large backgrounds and
% lead to an intermediate mass resolution of about 10% and 15% respectively.

% PDG
% In order to optimize search sensitivity and also to separate the various Higgs
% boson production modes, ATLAS and CMS split events into several mutually exclusive categories

% For phrasing from PDG
% None of the other production modes have been firmly established by the experiments
% individually. However, the table shows that, for the VBF production mode, the combination had a
% large sensitivity and produced a combined observation of 5.4σ, therefore establishing this process
% with a rate compatible with that expected in the SM.

Physics analyses typically focus on a combination of production and decay mode (referred to as \emph{signal}) to measure the number of Higgs bosons produced. The experimental sensitivity of a particular channel is not only dependent on the respective production cross section and decay branching fraction, but also highly impacted by the signature of the final state. The final state signature determines, for example, how efficient the signal can be reconstructed in the detectors and how well it can be distinguished from other non-Higgs processes considered as \emph{backgrounds}. 
%The different processes are typically first analyzed separately and then combined to obtain the most accurate measurements possible.

%and how well it can be distinguished from other types of events at hadron colliders. 
The $H\rightarrow b\bar{b}$ decay accounts for 58\% of all Higgs boson decays, but the experimental sensitivity to measure this decay mode is limited. The reason is the purely hadronic final state of $H\rightarrow b\bar{b}$ decays. 
This leads to a poor resolution of the mass of the Higgs boson due to the sizable jet energy resolution and also makes it difficult to distinguish the Higgs boson events from QCD multijet events that are extremely abundant at the LHC (see \cref{fig:xsec}).
%from the overwhelming background of QCD events at the LHC. 
The VBF or Higgs-strahlung production modes are therefore typically used for this decay channel where the presence of the additional particles in the final state can be exploited in the event selection.
%It is nonetheless possible to measure this decay by using the VBF or Higgs-strahlung production mode and 
%Using this strategy evidence for the $H\rightarrow b\bar{b}$ decay was recently found in the VH production mode \cite{Aaboud:2017xsd,Sirunyan:2017elk}.\todo{UPDATE}

% The most sensitive channels for the Higgs boson discovery in 2012 were the Higgs boson decays to two photons ($H \to \gamma\gamma$), the $H \to ZZ^* \to llll$, and \HWW\ process.
%Higgs boson initially was discovered in the decay to two photons and the $H \rightarrow ZZ^* \rightarrow llll$ channel \todo{REF}.
The decay into two photons ($H \to \gamma\gamma$) and two $Z$ bosons with a subsequent decay into leptons ($H \to ZZ^* \to llll$) have relatively small branching fractions but exhibit very clean final-state signatures due to the presence of photons and leptons. These decay products can be very precisely measured and allow for a good discrimination of the backgrounds.
In addition, the mass of the Higgs boson can be fully reconstructed in these channels by computing the invariant mass of the reconstructed decay products, providing a well-restricted mass range to measure the signal.
%at a hadron collider, where an overwhelming fraction of events is dominated by QCD effects.
%This is equally important since it allows an excellent selection of signal events and a good control over the background. 
% the largest fraction of events are dominated by QCD effects.
%QCD dominated events
%events at a hadron collider, since the final states are dominated by QCD effects.

The \HWW\ decay also provides a good handle over the backgrounds when selecting leptonically decaying $W$ bosons.
The neutrinos stemming from the $W$ boson decays, however, prevent to fully reconstruct the mass of the Higgs boson, since they cannot be directly detected by the ATLAS experiment.

% Results of the analysis of \HWW\ decays conducted by the ATLAS collaboration and CMS collaboration for data of \RunOne\ can be found in Ref.~\cite{PhysRevD.92.012006} and Ref.~\cite{2013arXiv1312.1129C}, respectively; a corresponding analysis of \RunTwo\ data is given in \cref{sec:ggfanalysis}, focusing on the gluon fusion production-mode.

Events with $H \to \tau\tau $ decays exhibit a similar final state as the \HWW\ process and especially in the case of leptonically decaying $\tau$-leptons allows for an excellent background rejection. 
% htautau paper: https://arxiv.org/abs/2201.08269

The analyses of Higgs boson decays to second generation fermions are challenging, either because of very small branching fractions, as is the case for $H \to \mu\mu$ decays, or because of an overwhelming QCD background, as is the case for $H \to c\bar{c}$ decays.
% 125 considerably more challenging measurements of Higgs boson couplings to second-generation fermions are
% explored via searches for the 𝐻 → 𝜇+𝜇 −
% [32] and, for the first time, 𝐻 → 𝑐𝑐¯ [33] decays.
% From nature:
% The considerably more challenging measurements of Higgs boson couplings to second-generation fermions are explored via searches for the 𝐻 → 𝜇+𝜇−[32] and, for the first time, 𝐻 → 𝑐𝑐¯ [33] decays. Due to the large multijet background, the latter decay mode is currently accessed only via the 𝑊𝐻 and 𝑍𝐻 production. Finally, the inputs to the combination are complemented by the latest direct searches for Higgs boson decays into invisible particles which escape the detector undetected [34, 35].
%Latest cross-section measurements can be found in \todo{REF. Not sure if it's necessary}
%A similar final state as the one in \HWW\ processes, can be exploited when looking at Higgs boson decays into $\tau$-leptons \cite{Aad:2015vsa,Sirunyan:2276465}. 
The Higgs boson decays into $Z\gamma$ ($H \to Z\gamma$) are also very rare and therefore provide limited sensitivity to perform Higgs measurements at the LHC. 
% , where leptonically decaying $Z$ bosons provide the most sensible final state, or $\mu\mu$,  % \cite{Aaboud:2017uhw,Aaboud:2017ojs}.
The final states with gluons ($H \to gg$) make up a considerable fraction of all Higgs decays but are too difficult to differentiate from other events at the LHC due to the overwhelming multijet background and the sizable jet energy resolution. 


\section{Higgs Boson Cross-Section Measurements and Interpretations}
\label{subsec:xsec-measurements}
% From PDG
% For a given mH, the sensitivity of a channel depends on the production cross section of the Higgs boson, its decay branching fraction, the reconstructed mass resolution, the selection efficiency and

% From Peskin/Schroeder
% The experiments that probe the behavior of elementary particles especially in the relativistic regime are scattering experiments. One collides two beams of particles with well dened momenta and observes what comes out The likelihood of any particular final state can be expressed in terms of the cross section, a quantity that is intrinsic to the colliding particles and therefore
% allows comparison of two dierent experiments with dierent b eam sizes and intensities


% From PDG
% In order to optimize search sensitivity and also to separate the various Higgs
% boson production modes, ATLAS and CMS split events into several mutually exclusive categories
The expected cross sections per production mode can be calculated from the SM Lagrangian, once the Higgs boson mass is fixed.
Since the total decay width of the Higgs boson ($\approx 4.1\,\MeV$~\cite{deFlorian:2016spz}) is much smaller than its mass, the narrow-width approximation holds, which allows measuring the Higgs boson cross sections as a product of the production cross section times the branching fractions of the decay modes.
For a given process with final state $f$, the number of observed Higgs boson events is typically expressed in terms of a \emph{signal strength}, defined as 
\begin{equation}
  \label{eq:signal-strength}
  \mu(pp \to H \to f) = \frac{ [ \sigma(pp \to H)  \times \BR(H \to f) ]_\text{meas} } { [ \sigma(pp \to H) \times \BR(H \to f) ]_\text{SM}},
\end{equation}
where the subscript ``meas'' (``SM'') denotes the measured value (prediction of the SM), and $B$ denotes the branching fraction. 

\subsection{Cross-section measurements}
The following briefly outlines the different types of cross-section measurements that are performed in the field of Higgs boson physics. 

% This allows, for example, measuring cross sections in exclusive regions of phase space which improves the resolution with which SM predictions can be probed.
%Differential measurements also allow for easier interpretability of the results, for example in Effective Field Theories (EFT) \todo{REF}.

\subsubsection{Inclusive production cross-section measurements}
Historically, the first measurements of the Higgs boson were based on the total number of Higgs bosons produced per production mode for individual decay channels.
These inclusive production mode cross sections are maximally dependent on theoretical assumptions related to the decay properties of the Higgs boson.
They are typically conducted for the search of a new signal (as done for the Higgs boson discovery in 2012) or to experimentally establish different production modes in the SM.
However, inclusive measurements cannot resolve small deviations from the SM predictions that may occur in regions of phase space where only a small fraction of the total produced Higgs bosons are expected.

\subsubsection{Differential fiducial cross-section measurements}
%As more data has been collected at the LHC in recent years, 
In order to probe the SM predictions for different phase space regions exclusively, Higgs boson production processes are measured differentially in various kinematic and topological variables.
Differential measurements use well-defined phase space regions, known as \emph{fiducial regions}, that allow unfolding the detector effects. 
This enables comparisons between experimental data and theory at generator level and thus minimizes the dependency on theoretical assumptions\footnote{Theoretical assumptions otherwise introduce uncertainties due to detector acceptance effects, as is the case for inclusive cross-section measurements.}.
The unfolding is facilitated by relating the expected number of events at detector level to the corresponding number at generator level. It is therefore favored to use simple discriminants for the signal extraction that are well-defined at both detector- and generator level than to use advanced analysis techniques such as neural networks. 
This limits the sensitivity of differential measurements.
% % using advanced analysis techniques for the signal extraction such as neural networks is discouraged. 
% Due to the rather complex unfolding procedure, the use of advanced analysis techniques such as neural networks is discouraged, which limits the sensitivity of differential measurements.
Another drawback is that it is not easily possible to statistically combine differential measurements with different definitions of the fiducial region.  

% - Least theory dependent
% - Not easily possible to combine measurements, as unfolding procedure very tailored to analysis selection. 

\subsubsection{Simplified template cross section measurements}
The \emph{Simplified Template Cross Sections} (STXS) framework provides a way to increase the experimental sensitivity while still allowing for differential measurements of Higgs boson production.
To achieve this, mutually exclusive kinematic regions of Higgs boson production are defined at generator level. They are known as \emph{STXS bins}. 
The extraction of the signal split in these STXS bins is then performed in reconstructed regions, which are typically aligned with the definitions of the STXS bins. No unfolding of detector effects is performed which allows using sophisticated analysis techniques like neural networks.
The STXS bins are defined based purely on the production mode and kinematics of the Higgs boson, are agnostic to the different Higgs decay channels, and are agreed upon between experiments.
These design choices allow combining Higgs boson measurements of different decay channels as well as experiments, which enables more precise cross-section measurements.
% The measurements of the production mode cross sections are performed in regions of phase space defined at the level of the fully reconstructed collision events. They are defined as similar as possible to the generator-level bins. This allows using sophisticated analysis techniques like neural networks.

Furthermore, the bins are chosen following two main principles: First, the experimental acceptance is aimed at being constant within each bin, which reduces the dependency on theoretical assumptions arising from detector acceptance effects. Second, regions that are expected to be sensitive to physics effects beyond the SM are isolated, so that they can be studied separately.
As the amount of available data increases, the STXS binning evolves in stages, each time increasing the number of bins. 
The data from \RunTwo of the LHC allows measuring cross sections partitioned in the Stage 1.2 STXS scheme, which is shown in Appendix~\ref{app:stxs-measurements-aux}.

\subsection{Coupling-strength measurements and Effective Field Theories}
\label{subsec:coupling-measurements}
Cross-section measurements only probe the production mode of the Higgs boson.
To probe the predicted strengths of the couplings of the Higgs boson to other fundamental particles, the decay processes must also be taken into account.
%A measurement of the Higgs boson's couplings therefore needs to account for both, production and decay processes.
This can be achieved at leading order using the $\kappa$ framework~\cite{LHCHandbookV3}, where the coupling strengths (or simply couplings) are measured by parametrizing the cross sections and branching fractions associated to a particle $j$ in terms of \emph{coupling-strength modifiers}, $\kappa_j^2$.
The cross section times branching fraction in the signal strength of \cref{eq:signal-strength} for a production process $i$ and final state $f$ can then be parametrized as
\begin{equation}
  \label{eq:kappa-parametrization}
  % \left( \sigma \times  \BR \right) (i \to H \to f)  = \kappa_i^2 \times  \kappa_f^2 \times  \sigma_{i}^\mathrm{SM} \times  \frac{\Gamma_f^{\mathrm{SM}}}{\Gamma_H\left(\kappa_i^2, \kappa_f^2\right) }, 
  \left( \sigma \times  \BR \right) (i \to H \to f)  =  \sigma_{i}^\mathrm{SM} \times \BR^\mathrm{SM}(H \to f) \times \frac{\kappa_i^2 \times  \kappa_f^2}{\kappa_H^2}, 
\end{equation}
where $\sigma_{i}^\mathrm{SM}$ and $\BR^\mathrm{SM}(H \to f)$ correspond to the SM expectations. 
The coupling-strength modifiers therefore parametrize deviations from the SM predictions and are unity by definition when the SM is assumed.

The $\kappa$ framework assumes that the data originate from a Higgs boson with a mass of 125\,GeV and the kinematics of both the production and decay are assumed to agree with the SM predictions for the Higgs boson.\footnote{For more details on the assumptions and the parametrization of $\kappa_H$, the reader is referred to \ccite{LHCHandbookV3}.}
The latter assumption, in particular, limits the sensitivity to models beyond the SM that may only affect the SM kinematics. 

An alternative framework to the $\kappa$ framework constists of interpretations of Higgs boson measurements in Effective Field Theories, for example within the framework of Standard Model Effective Field Theory (SMEFT)\footnote{An overview of SMEFT can, for example, be found in \ccite{Brivio_2019}.}.
In SMEFT, the effects of physics beyond the SM at large energy scales $\Lambda$ -- large compared to the Higgs vacuum expectation value ($\Lambda \gg v$) -- are parametrized at low energies, $E \ll \Lambda$, in terms of effective couplings. 
This allows for a theory-independent approach to search for deviations of the SM, and relies on fewer assumptions than the $\kappa$ framework. 

% For the VBF, \HWW process, the modifiers become
% \begin{align}
%   \label{eq:kappa-parametrization-VBF}
%   \kappa_i^2 &= \kappa_\mathrm{VBF}^2 = 0.733 \kappa^2_W + 0.267 \kappa^2_Z, \\
%   \kappa_f^2 &= \kappa_W^2, 
%   % \sigma_{\mathrm{VBF}} \cdot \BR (H \to WW)  = \kappa_\mathrm{VBF}^2 \cdot \kappa_W^2 \cdot \sigma_{\mathrm{VBF}}^\mathrm{SM} \frac{\Gamma_{WW}^\mathrm{SM}}{\Gamma_H\left(\kappa_\mathrm{VBF}^2, \kappa_{W}^2\right) },
% \end{align}
% where the parametrization of $\kappa_\mathrm{VBF}^2$ corresponds to the one used in \ccite{NaturePaper}.
% The latter shows, that a set of assumptions must be made in the $\kappa$-framework, since the couplings cannot be directly accessed. 
% Typically, different parametrizations and scenarios are tested in the couplings measurements. 
% More details left to \ccite{LHCHandbookV3}.
% Because the couplings cannot be directly measured, a set of assumptions must be made in the framework.
% For example, the framework assumes that the data originate from a Higgs boson with a mass of 125\,GeV and its interactions are exactly as predicted by the SM.
% More details on the framework are provided in \ccite{LHCHandbookV3}.
% To this end, the couplings of the Higgs boson to individual particles can be measured by scaling the cross sections and branching fractions for the individual Higgs boson processes in terms of coupling-strength modifiers, $\kappa$. 
%To this end, the $\kappa$ framework~\cite{LHCHandbookV3} parametrizes the signal strengths 
% The couplings of the Higgs boson to individual particles can be measured by parametrizing the cross sections and branching fractions for the individual Higgs boson processes in terms of coupling-strength modifiers, $\kappa$, following the $\kappa$-framework~\cite{LHCHandbookV3}. 
% \subsection{Properties of HWW Decays}
% % THIS MIGHT actually fit in the analysis section
% % as it directly related to what analysis selections we perform
% % THINK ABOUT IT!
% - VBF Higgs -> ww cross section must be exactly the SM value, otherwise VBS xsec converges!
% -> Maybe add this to HWW analysis

\section{Current Experimental Status}
\label{subsec:higgs-exp-status}
% All experimentally accessible decay modes of the Higgs have been observed, which includes $H \to \gamma\gamma$, $H \to ZZ$, $H \to WW$, $H \to \tau\tau$, and $H \to b\bar{b}$. 
After the discovery of a particle consistent with the Higgs boson predicted by the SM in 2012~\cite{HIGG-2012-27,CMS-HIG-12-028}, an era of Higgs boson precision physics has started.
The data from \RunOne\ and \RunTwo\ of the LHC allowed for high-precision measurements of several properties of the Higgs boson such as its mass, width, spin, or parity, as well as Higgs boson production and decay processes. 
At the time of writing, all experimental measurements are consistent with the predictions of the SM.
This section briefly summarizes the current experimental status, mostly focusing on results of the ATLAS collaboration.
% This allows placing the measurements presented in this thesis in a broader context. 
A more comprehensive overview can be found in \ccite{PDG2020}. 
%allowing to place the work presented in this thesis into a broader context. 

\subsection{Higgs boson properties}
% The latest combined measurement of the ATLAS and CMS experiment~\cite{HIGG-2014-14} measures a Higgs boson mass of 
% \begin{equation}
%   m_H = 125.09 \pm 0.21\text{(stat.)} \pm 0.11\text{(syst.)}\,\GeV.
% \end{equation}
The most recent measurements of the Higgs boson mass performed by the ATLAS and CMS collaboration measure a mass of $m_H = 124.99 \pm 0.19\,\GeV$~\cite{https://doi.org/10.48550/arxiv.2207.00320} and $m_H = 125.38 \pm 0.14\,\GeV$~\cite{CMS-HIG-19-004}, respectively.
Measurements of the spin and parity of the Higgs boson confirm the SM predictions of a spin-parity $J^{P} = 0^{+}$ and exclude alternative hypotheses beyond 99.9\% confidence level (CL)~\cite{HIGG-2013-17-witherratum,CMS-HIG-14-018}.
The CP-even hypothesis for the SM Higgs boson has also been probed for several interactions and all measurements are in agreement with the CP-even prediction (see e.g. \ccite{ATLAS-CONF-2022-032,Aad_2020CP,CMS-HIG-17-011}).
The total width of the Higgs boson in the SM is small ($\Gamma_H^{\text{SM}} = 4.1\,$ MeV~\cite{deFlorian:2016spz}) and therefore difficult to measure directly at hadron colliders. Indirect measurements using off-shell production of Higgs bosons result in a measured width of $\Gamma_H = 3.2 ^{+2.4}_{-1.7}\,$ MeV~\cite{https://doi.org/10.48550/arxiv.2202.06923}.

\subsection{Higgs boson production and decay processes}
All major Higgs boson production modes (ggF, VBF, $VH$, $ttH$) and Higgs boson decays to bosons ($H \to WW$, $H \to ZZ$, $H \to \gamma\gamma$) as well as third-generation fermions ($H \to \tau^+\tau^-$, $H \to b\bar{b}$) have been observed at the LHC with significances larger than $5\,\sigma$ above the background expectation~\cite{NaturePaper}.
While precision measurements are now being performed for these channels, rarer Higgs boson processes have not been unambiguously confirmed with the data available ($H \to Z\gamma$; and decays to second-generation fermions $H \to \mu^+\mu^-$, $H \to c\bar{c}$) or remain experimentally out of reach at the LHC ($H \to s\bar{s}$; decays to first-generation fermions $H \to e^+e^-$, $H \to u\bar{u}$, $H \to d\bar{d}$; and decays to gluons $H \to gg$).

A combination of several individual Higgs boson measurements using many of the above-mentioned production and decay modes of the Higgs boson was recently performed by the ATLAS collaboration~\cite{NaturePaper}. 
The combination includes the \HWW\ analysis that is presented in detail in \cref{chap:hww} of this thesis, which is why a summary of the combined measurements is left to \cref{chap:comb}. 
They provide the most precise measurements to date of the production and decay of the Higgs boson and the couplings of the Higgs boson to other fundamental particles. 

Rare decay modes of the Higgs boson begin to emerge from the data of \RunTwo\ of the LHC. 
The search for Higgs decays to a pair of muons of the ATLAS experiment found an excess of $H \to \mu\mu$ events over the background expectation of 2.0 standard deviations, where 1.7 were expected~\cite{HIGG-2019-14}.
The CMS collaboration found evidence for the $H \to \mu\mu$ decay~\cite{CMSHmumuevidence}.
An analysis of $H \to cc$ events yields an observed (expected) upper limit of 26 (31) times the SM prediction~\cite{ATLAS-CONF-2021-021}, and a search for $H \to Z\gamma$ decays results in an observed (expected) upper limit of 3.6 (2.6) times the SM prediction~\cite{HIGG-2018-42}.
An upper limit on the branching fraction of Higgs decays to particles that are invisible to the detector was set to 0.145, where 0.103 was expected~\cite{ATLASInvisible1}.
The cross section of double Higgs boson production is extremely low. The most stringent upper limits by the ATLAS collaboration were set to 4.1 times the SM prediction at 95\% CL using $HH \to b\bar{b}\gamma\gamma$ decays~\cite{ATLAS-CONF-2021-016}.
%Other search to invisible: \cite{Aad_2022}
% From nature
% Even with the current precision of measurements there is room for possible interpretations of the data in terms of new phenomena beyond the SM. 
% From nature
% Finally, the inputs to the combination are complemented by the latest direct searches for Higgs boson
% 129 decays into invisible particles which escape the detector undetected [34, 35].

\subsection{Differential measurements and interpretations in Effective Field Theories}
The \RunTwo\ data of the LHC also allowed measuring differential fiducial cross sections of Higgs boson production. Analyses targeting the $H \to ZZ$ decay~\cite{ATLAS:2020wny} and $H \to \gamma\gamma$ decay~\cite{hgammagammaDiff} measured cross sections for a variety of observables sensitive to the production and decay processes of the Higgs boson, and set constraints on effects beyond the SM.
%(THEORETICALLY there is a combination of the above here: ATLAS-CONF-2022-002)
% Measurements of Higgs boson production cross sections also serve as basis for further interpretation, for example within the framework of Standard Model Effective Field Theory (SMEFT)\footnote{An overview of SMEFT can, for example, be found in \ccite{}.}. In SMEFT, the effects of physics beyond the SM at large energy scales $\Lambda$ -- large compared to the Higgs vacuum expectation value ($\Lambda \gg v$) -- are parametrized at low energies, $E \ll \Lambda$, in terms of effective couplings. 
% This allows for a theory-independent approach to search for deviations of the SM. 

A combination of various Higgs boson cross-section measurements was interpreted in the SMEFT framework by the ATLAS collaboration~\cite{ATLAS-CONF-2020-053}. This also allowed setting constraints on new physics models
As well, the ATLAS collaboration combined the analysis of \HWW\ decays with differential cross-section measurements of $WW^*$ production in order to interpret the measured cross sections in terms of effective couplings~\cite{ATL-PHYS-PUB-2021-010}.
Both of these results pave the way for future interpretations of Higgs boson measurements in Effective Field Theories. %which become especially important in the absence of any direct discovery of new particles. 

% % The latest combined Higgs measurements performed by the ATLAS collaboration of the kinematic properties of Higgs boson production, as well as the Higgs coupling to other particles, are nicely summarized in \ccite{NaturePaper}.

% In addition to the results presented in \ccite{NaturePaper}, Higgs analyses can be interpreted in the framework of Effective Field Theories. 
% Effective field theories parametrise...
% They allow for a general parametrisation of deviations of the SM. 
% %Differential measurements also allow for easier interpretability of the results, for example in Effective Field Theories (EFT).


%%%%%%%%%%%%%%%%%%%
% Maybe not the following!
%%%%%%%%%%%%%%%%%%%

% \subsection{Cross-section measurements of \HWW decays}
% \label{subsec:prev-hwww-cross-section-meas}

% The \HWW process was first observed using data from \RunOne of the LHC~\cite{HIGG-2013-13}, and an analysis of a partial \RunTwo dataset corresponding to 36.1\,\ifb\ reported the most recent ggF and VBF \HWW measurements~\cite{HIGG-2013-13}.
% In the analysis presented here, several improvements compared to the previous \RunTwo results are implemented. Most noteworthy, the discrimination of the VBF signal is performed using a deep neural network (DNN) instead of a boosted decision tree, ggF signal events with two or more jets in the final state are included in the measurement, and measurements of cross sections in the kinematic regions defined by the STXS framework (\emph{STXS measurement}) are reported for the first time for this process.
% The analysis is published in \ccite{HWWPaper} and yields some of the most precise Higgs cross-section meausurements to date.


