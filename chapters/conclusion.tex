\chapter{Conclusion}
\label{chap:conclusion}
% RESOURCES:
% https://www.nature.com/articles/s41567-020-01054-6.pdf
In the search for the fundamental laws of nature, the Higgs boson is a unique tool. 
It sits at the core of the Standard Model (SM), which is known to be incomplete, or merely an approximation of a more fundamental theory. 
The precise measurement of the properties of the Higgs boson allows for testing a broad range of SM predictions and is therefore of paramount interest.
It enables setting strong constraints on physics beyond the SM and may point to signs of new phenomena. % 
% or could point to signs of physics beyond the Standard Model.
%provides a promising path to find possible deviations from the SM. 
% This provides a promising path to the discovery of new phenomena. 
% To test the SM and find possible deviations of its predictions, a precise understanding of all Higgs boson processes is therefore crucial. 
%The precise understanding of Higgs boson processes is therefore a crucial task. 
%OR: The study of the Higgs boson is therefore one of the most important areas of research to test the SM and find its wholes. 
%Ten years after the discovery of the Higgs boson, no experimental signs of deviations of the SM have formed. 
%The physics program at the LHC provides a dataset unprecedented in size and well-suited for the study of Higgs boson processes. 

In this thesis, contributions to this endeavor were presented across different areas that build upon each other. 
First, the measurement of the noise term of the jet energy resolution (JER) was presented. 
This was performed for the first time for particle flow jets and is an important input to the calibration of the JER used in many physics analyses performed with the ATLAS experiment. 
The calibrations are also crucial for the results presented in the second and main part of this work: the measurements of the gluon fusion (ggF) and vector-boson fusion (VBF) production cross sections of the Higgs boson in its decay to a pair of $W$ bosons.

The analysis of \HWW\ decays was carried out using the full dataset collected by the ATLAS experiment during \RunTwo of the LHC, corresponding to 139\,\ifb\ proton-proton collisions at a center-of-mass energy of $\sqrt{s} = 13\,$TeV. 
Compared to the previous results from the ATLAS collaboration~\cite{HIGG-2016-07} that uses a partial \RunTwo dataset, the relative uncertainty on the measurements of the inclusive ggF and VBF production cross sections times branching fraction was reduced by almost 40\% and 60\%, respectively, yielding 
\begin{eqnarray*}
    \sigma_{\mathrm{ggF}} \cdot \mathcal{B}_{H \to WW^{\ast}} &=& 12.0 \pm 1.4~\mathrm{pb}, \,\text{and} \\
    \sigma_{\mathrm{VBF}} \cdot \mathcal{B}_{H \to WW^{\ast}} &=& 0.75\;^{+0.19}_{-0.16}~\mathrm{pb},
\end{eqnarray*}
which is in agreement with the SM predictions.
The measurement of the VBF production mode benefited significantly from the author's work on the development of a deep neural network that distinguishes the VBF, \HWW signal from the backgrounds.
This allowed observing the VBF production mode of the Higgs boson for the first time in the \HWW channel with a significance of $5.8\,\sigma$ above the background expectation, where $6.2\,\sigma$ were expected assuming the SM.
In addition, cross-section measurements of the ggF and VBF production mode have been performed in 11 separate kinematic regions, using the framework of Simplified Template Cross Sections. This probes the Higgs boson production in exclusive kinematic regions that have not been measured before with \HWW decays.
The results more firmly establish the Higgs boson in the SM, as they are all found to be compatible with the SM predictions.

The measurements were used as inputs to combined Higgs boson measurements~\cite{NaturePaper}, where they made an essential contribution. 
In particular, it led in the most precise measurement to date of the coupling of the Higgs boson to vector bosons yielding $\kappa_{V} = 1.035 \pm 0.031$, using the $\kappa$-framework~\cite{LHCHandbookV3}.

The results from the \HWW analysis will continue to be the baseline for future analyses and combinations, as well as reinterpretations such as in Effective Field Theories.
The author also suggested changes to the analysis strategy that could further improve the \HWW cross-section measurements, in particular in the analysis categories with two or more jets.
% using improved analysis strategies as, for example, suggested by the author of this thesis. 
% There is still room to improve the analysis strategies, as studied by the author to further improv the \HWW cross-section measurements.
% For the \HWW analysis, for example, the author suggested alternative analysis strategies that could further improve the \HWW cross-section measurements. 
%, which are expected to make a substantial impact on constraining theories beyond the SM and may open doors to new phenomena.

% \TDinote{}{Will be one of the goals in the future to combine measurements to get the most out of the data}
% Furthermore, the analysis presented was crucial input to combined Higgs boson measurements that combine analyses of multiple Higgs boson processes. In particular, the \HWW analysis is the most important input to the measurement of the coupling of the Higgs boson to vector-bosons, determined to be $\kappa_V = 1.035 \pm 0.031$.
% The data analysis of the full \RunTwo dataset has not been fully completed. 
% More results are imminent, aiming at an even more comprehensive experimental map of Higgs boson processes and other phenomena. 
% Looking ahead, the data analysis of the full \RunTwo dataset has not yet been completed and consolidated results are expected in the coming years.
Looking ahead, the data taking at the LHC is expected to continue with \RunThr at the end of 2022, operating under similar data taking conditions as \RunTwo. 
With more data, as well as more sophisticated analysis strategies, the properties of the Higgs boson can be studied at higher precision and new Higgs boson processes will become experimentally accessible. 
This will allow rare Higgs boson processes to be investigated, for example, the $H \to \mu\mu$ or $H \to Z\gamma$ decays or the coupling of the Higgs boson to itself.
%The latter includes the production of di-Higgs processes, allowing to measure , and other rare Higgs processes. 
% room for new phenomena beyond the SM
Even more potential to measure these key properties and other phenomena is expected from the High-Luminosity LHC and its upgraded experiments after 2030.
%, which are currently estimated to collect about 20 times the amount of data recorded during \RunTwo of the LHC. 
The study of the Higgs boson and its unique properties will continue to be of utmost importance, guiding the particle physics community in the search for the fundamental laws of nature. 

% At the time of writing, in between \RunTwo and \RunThr of the LHC, it would be a miss not to mention that many physicists had hoped for more discoveries of fundamental particles at the LHC than have been reported so far.
% Yet, no experimental signs of deviations of the SM have formed.
% % In a way, this is the very nature of fundamental physics, which is driven by curiosity and can never be too certain of what lies ahead, making it one of the most exciting research fields. 
% The current situation in particle physics, i.e. the discovery of the Higgs boson and the exclusion of the existence of many other fundamental particles that were predicted by theoretical models, most notably the ones predicted by supersymmetric models, requires a close examination of existing theories. This special circumstance can also be considered a unique opportunity to stimulate new groundbreaking ideas.
% %and to revisit the most fundamental principles. 
% The study of the Higgs boson and its unique properties will guide the physics community in this pursuit.
% %, striving to find hints of new phenomena through precision measurements. 
% With a wealth of physics data yet to be collected at the LHC and new research facilities in planning, substantial progress in this area is expected in the years to come. 
% If scientists continue to push beyond the boundaries of technology and collaborative practices, as has been so successfully demonstrated by the LHC program, new insights will be gained that could shed light on the open mysteries of the universe and bring humanity closer to having a complete description of the fundamental laws of nature. 

%could potentially disrupt our entire understanding of the universe and 
% Guided by the data from full \RunTwo of the LHC, it is important to set out new research directions.
% Since no signs of physics beyond the SM have been found to date, it is a crucial task to deepen our understanding of the physics we already know. 
% A prime study is the one of the Higgs boson, which sits at the core of the SM and builds an important building block for many theories beyond the SM.
% This thesis presented the study of Higgs bosons in their decays to $W$ bosons, providing measurements of unprecedented precision as well as resolution. 
% \TDinote{}{Mention machine learning to close the circle with the intro!}
% The analysis will be the baseline for many future combinations and reinterpretations such as in Effective Field Theories. 
% The results were also crucial input to combined Higgs boson measurements that provide a comprehensive study of Higgs boson processes.

% suggest that the SM is merely an approximation of a more fundamental theory. 
% While the physics program at the LHC has been a great success, delivering results with unprecedented precision and constraining many of the theories beyond the SM, ...
% pushing the boundaries of technology and collaborative efforts and leaving no doubt about a huge experimental success.

% Nevertheless, the physics community seems to be slowly growing impatient, and it would be a miss not to mention that several physicists had hoped for the LHC program to provide more answers than it has delivered to date. 
% Several questions remain to be answered and no clear hints on where new physics may hide have been provided.
% In the case of theories beyond the SM such as supersymmetry, for example, it may be deemed disappointing rather than a success that no signs of it being realized in nature have been found so far. 
% In fact, this seems to be how it is often perceived from outside the physics community. 
% Whatever the framing might be, the LHC and its experiments have delivered outstanding results, pushing the boundaries of technology and collaborative efforts and leaving no doubt about a huge experimental success.
% In fact, if looking at physics for what it is, a fundamentally data-driven science, the LHC has done exactly as promised, providing a rich dataset to be explored. 
% From an experimentalist point of view, there is no doubt that the LHC including his experiments were a huge success.
%At the time of writing, 10 years after the Higgs boson discovery, it is therefore a good time to look at physics for what it is, a fundamentally data-driven science.