\chapter{Conclusion}
\label{chap:conclusion}
% RESOURCES:
% https://www.nature.com/articles/s41567-020-01054-6.pdf
In the search for the fundamental laws of nature, the Higgs boson is a unique tool. 
It sits at the core of the Standard Model (SM), which is known to be incomplete, or merely an approximation of a more fundamental theory. 
The precise measurement of all Higgs boson processes allows for testing a broad range of SM predictions and is therefore of paramount interest.
%provides a promising path to find possible deviations from the SM. 
% This provides a promising path to the discovery of new phenomena. 
% To test the SM and find possible deviations of its predictions, a precise understanding of all Higgs boson processes is therefore crucial. 
%The precise understanding of Higgs boson processes is therefore a crucial task. 
%OR: The study of the Higgs boson is therefore one of the most important areas of research to test the SM and find its wholes. 
%Ten years after the discovery of the Higgs boson, no experimental signs of deviations of the SM have formed. 
%The physics program at the LHC provides a dataset unprecedented in size and well-suited for the study of Higgs boson processes. 

This thesis presented measurements of the gluon fusion (ggF) and vector-boson fusion (VBF) production cross sections of the Higgs boson in its decays to a pair of $W$ bosons, using the full \RunTwo dataset of the LHC collected by the ATLAS experiment. 
The results presented contribute to a more precise knowledge of the couplings of the Higgs boson to other fundamental particles and more firmly establish the Higgs boson in the SM. All results are found to be compatible with the SM predictions. 

The measurement of the VBF production mode benefited significantly from the author's work on the development of a deep neural network that discriminates the VBF, \HWW signal from the non-VBF backgrounds. 
The sensitivity of the VBF signal measurement was improved by about 20-30\% compared to the measurements with the previous analysis strategy, which allowed observing the VBF production mode of the Higgs boson for the first time in the \HWW channel with a significance of $5.8\,\sigma$ above the background expectation, where $6.2\,\sigma$ were expected.
Compared to the previous \RunTwo results, the relative uncertainty on the inclusive ggF and VBF production cross-section measurements was reduced by almost 40\% and 60\%, respectively, yielding $\sigma_{\mathrm{ggF}} \cdot \mathcal{B}_{H \to WW^{\ast}} = 12.0 \pm 1.4~\mathrm{pb}$ and $\sigma_{\mathrm{VBF}} \cdot \mathcal{B}_{H \to WW^{\ast}} = 0.75\;^{+0.19}_{-0.16}~\mathrm{pb}$, which is consistent with the SM predictions.
In addition, cross-section measurements of the ggF and VBF production mode split into 11 kinematic regions have been performed for the first time with \HWW decays, using the framework of Simplified Template Cross Sections. 
The results of the \HWW analysis will be the baseline for future analyses and used as inputs for future combinations and reinterpretations such as in Effective Field Theories, all of which are expected to contribute significantly to constraining theories beyond the SM. 

Furthermore, the analysis presented was crucial input to combined Higgs boson measurements that combine analyses of multiple Higgs boson processes. In particular, the \HWW analysis is the most important input to the measurement of the coupling of the Higgs boson to vector-bosons, determined to be $\kappa_V = 1.035 \pm 0.031$.

% The data analysis of the full \RunTwo dataset has not been fully completed. 
% More results are imminent, aiming at an even more comprehensive experimental map of Higgs boson processes and other phenomena. 
% \TDinote{}{Move above sentence lower to close this mini-section. Then go into last paragraph that is already quite nice}
Looking ahead, the data analysis of the full \RunTwo dataset has not yet been completed and consolidated results are expected in the coming years. The data taking at the LHC is then expected to continue with \RunThr at the turn of 2022 and 2023, operating under similar data taking conditions as \RunTwo. 
With more data, the Higgs boson can be studied at higher precision and new Higgs processes will become experimentally accessible. 
This will allow rare Higgs processes to be studied, most notably the coupling of the Higgs boson to itself, which has never been measured before.
%The latter includes the production of di-Higgs processes, allowing to measure , and other rare Higgs processes. 
% room for new phenomena beyond the SM
Even more potential to measure these key properties is expected from the High-Luminosity LHC after 2030, which is currently estimated to deliver about 20 times the amount of data collected during \RunTwo of the LHC. 
The Higgs program will continue to be of great importance and aims at an ever more precise experimental map of the couplings of the Higgs boson to other fundamental particles. 
%This is one of the most promising paths to finding new phenomena beyond the SM that could shed light on the open mysteries. 
%some paper

%\TDinote{}{Maybe be more specific mention couplings to particles that are not measured}
%to test the Standard Model pattern of couplings to elementary particles
% nature paper abstract
% Interactions with
% gluons, photons and 𝑊 and 𝑍 bosons – carriers of all Standard Model forces – are studied in
% detail. Interactions with three third generation matter particles (𝑏, 𝑡, 𝜏) are well measured
% and indications of interactions with second generation (𝜇) are emerging. T
%the LHC including its experiments will undergo a large-scale upgrade to be able to operate at very high-luminosities about \TDinote{200 times}{how much exactly?} larger than in \RunTwo. 
% After \RunThr, the LHC including its experiments will undergo a large-scale upgrade to be able to operate at very high-luminosities about \TDinote{200 times}{how much exactly?} larger than in \RunTwo. 
% This will allow gathering data at a much increased rate, providing a dataset unprecedented in size.
% The dataset is expected to correspond to 3000\ifb which will provide great potential for measuring tiny potential deviations from the SM and discovering new phenomena.

At the time of writing, in between \RunTwo and \RunThr of the LHC, it would be a miss not to mention that many physicists had hoped for more discoveries at the LHC than have been reported so far. 
Yet, no experimental signs of deviations of the SM have formed.
In a way, however, this is the very nature of fundamental physics, which is driven by curiosity and can never be too certain of what lies ahead, making it one of the most exciting research fields. 
With a wealth of physics data yet to be collected at the LHC and new research facilities in planning, substantial progress is expected in the years to come. If scientists continue to push beyond the boundaries of technology and collaboration, as has been so successfully demonstrated by the LHC program, it seems only a matter of time before a new discovery is made that sheds light on the open mysteries and brings humanity another step closer to a complete description of nature. 

%could potentially disrupt our entire understanding of the universe and 
% Guided by the data from full \RunTwo of the LHC, it is important to set out new research directions.
% Since no signs of physics beyond the SM have been found to date, it is a crucial task to deepen our understanding of the physics we already know. 
% A prime study is the one of the Higgs boson, which sits at the core of the SM and builds an important building block for many theories beyond the SM.
% This thesis presented the study of Higgs bosons in their decays to $W$ bosons, providing measurements of unprecedented precision as well as resolution. 
% \TDinote{}{Mention machine learning to close the circle with the intro!}
% The analysis will be the baseline for many future combinations and reinterpretations such as in Effective Field Theories. 
% The results were also crucial input to combined Higgs boson measurements that provide a comprehensive study of Higgs boson processes.

% suggest that the SM is merely an approximation of a more fundamental theory. 
% While the physics program at the LHC has been a great success, delivering results with unprecedented precision and constraining many of the theories beyond the SM, ...
% pushing the boundaries of technology and collaborative efforts and leaving no doubt about a huge experimental success.

% Nevertheless, the physics community seems to be slowly growing impatient, and it would be a miss not to mention that several physicists had hoped for the LHC program to provide more answers than it has delivered to date. 
% Several questions remain to be answered and no clear hints on where new physics may hide have been provided.
% In the case of theories beyond the SM such as supersymmetry, for example, it may be deemed disappointing rather than a success that no signs of it being realized in nature have been found so far. 
% In fact, this seems to be how it is often perceived from outside the physics community. 
% Whatever the framing might be, the LHC and its experiments have delivered outstanding results, pushing the boundaries of technology and collaborative efforts and leaving no doubt about a huge experimental success.
% In fact, if looking at physics for what it is, a fundamentally data-driven science, the LHC has done exactly as promised, providing a rich dataset to be explored. 
% From an experimentalist point of view, there is no doubt that the LHC including his experiments were a huge success.
%At the time of writing, 10 years after the Higgs boson discovery, it is therefore a good time to look at physics for what it is, a fundamentally data-driven science.