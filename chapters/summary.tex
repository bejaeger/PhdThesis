\chapter{Summary}
\label{chap:summary}
% RESOURCES:
% https://www.nature.com/articles/s41567-020-01054-6.pdf
In the search for the fundamental laws of nature, the Higgs boson is a unique tool. 
It sits at the core of the Standard Model (SM). A theory that is known to be incomplete, or merely an approximation of a more fundamental theory. 
Ten years after the discovery of the Higgs boson, no experimental signs of deviations of the SM have formed. 
The precise understanding of Higgs boson processes is therefore a crucial task. 

The physics program at the LHC provides a dataset unprecedented in size and well-suited for the study of Higgs boson processes. 
This thesis presented measurements of the gluon fusion (ggF) and vector-boson fusion (VBF) production cross sections of the Higgs boson in its decays to a pair of $W$ bosons, using the full \RunTwo dataset of the LHC collected by the ATLAS experiment. 
All results are compatible with the SM predictions. 

The measurement of the VBF production mode benefited majorly from the author's work on the development of a deep neural network that discriminates the VBF, \HWWdet signal from the non-VBF backgrounds. 
It allowed the sensitivity of the measurement to be improved by approximately 30\% compared to the previous state of the art based on a boosted decision tree. 
Compared to the previous \RunTwo results that used a dataset smaller by approximately a factor of four, the relative uncertainty on the inclusive ggF and VBF production cross-section measurements was reduced by almost 40\% and 60\%, respectively. 
In addition, differential cross-section measurements of the ggF and VBF production mode have been performed for the first time with \HWW decays, using the framework of Simplified Template Cross Sections. 
The results of the \HWW analysis will be the baseline for future combinations and reinterpretations such as in Effective Field Theories. 

The analysis presented was crucial input to combined Higgs boson measurements using results for multiple Higgs boson processes. 
In particular, it enabled the most precise measurements of the Higgs boson to vector-boson coupling yielding $\kappa_V = 1.035 \pm 0.031$, where the inputs from the \HWWdet analysis made the dominant contribution. 


It is important to note that the analysis of the full \RunTwo dataset has not been completed. 
More results are imminent, aiming at an even more comprehensive experimental map of Higgs boson processes and other phenomena. 

Looking ahead, the data taking at the LHC is expected to continue at the turn of 2022 and 2023 with \RunThr, operating under similar data taking conditions as in \RunTwo. 
With more data the Higgs boson can be studied at even higher precision and new Higgs processes will become experimentally accessible. An example of the latter is the production of di-Higgs processes, allowing to probe the Higgs self coupling, or other rare Higgs processes. 

After \RunThr, the LHC including its experiments will undergo a large-scale upgrade to be able to operate at very high-luminosities about \TDinote{200 times}{how much exactly?} larger than in \RunTwo. 
This will allow gathering data at a much increased rate, providing a dataset unprecedented in size.
The dataset is expected to correspond to 3000\ifb which will provide great potential for measuring tiny potential deviations from the SM and discovering new phenomena.

At the time of writing, in between \RunTwo and \RunThr of the LHC, it would be a miss not to mention that many physicists had hoped for more discoveries at the LHC than have been reported so far. 
In a way, however, this is the very nature of fundamental physics, which is driven by curiosity and never too certain of what lies ahead, making it one of the most exciting research fields. 
With a wealth of physics data yet to be collected at the LHC and new research facilities in planning, it seems only a matter of time before a new discovery is made that will bring humanity closer to a complete description of nature and an understanding of the fundamental laws of the universe.

%could potentially disrupt our entire understanding of the universe and 

% Guided by the data from full \RunTwo of the LHC, it is important to set out new research directions.
% Since no signs of physics beyond the SM have been found to date, it is a crucial task to deepen our understanding of the physics we already know. 
% A prime study is the one of the Higgs boson, which sits at the core of the SM and builds an important building block for many theories beyond the SM.
% This thesis presented the study of Higgs bosons in their decays to $W$ bosons, providing measurements of unprecedented precision as well as resolution. 
% \TDinote{}{Mention machine learning to close the circle with the intro!}
% The analysis will be the baseline for many future combinations and reinterpretations such as in Effective Field Theories. 
% The results were also crucial input to combined Higgs boson measurements that provide a comprehensive study of Higgs boson processes.

\TDinote{}{Mention machine learning to close the circle with the intro!}

% suggest that the SM is merely an approximation of a more fundamental theory. 
% While the physics program at the LHC has been a great success, delivering results with unprecedented precision and constraining many of the theories beyond the SM, ...
% pushing the boundaries of technology and collaborative efforts and leaving no doubt about a huge experimental success.

% Nevertheless, the physics community seems to be slowly growing impatient, and it would be a miss not to mention that several physicists had hoped for the LHC program to provide more answers than it has delivered to date. 
% Several questions remain to be answered and no clear hints on where new physics may hide have been provided.
% In the case of theories beyond the SM such as supersymmetry, for example, it may be deemed disappointing rather than a success that no signs of it being realized in nature have been found so far. 
% In fact, this seems to be how it is often perceived from outside the physics community. 
% Whatever the framing might be, the LHC and its experiments have delivered outstanding results, pushing the boundaries of technology and collaborative efforts and leaving no doubt about a huge experimental success.
% In fact, if looking at physics for what it is, a fundamentally data-driven science, the LHC has done exactly as promised, providing a rich dataset to be explored. 
% From an experimentalist point of view, there is no doubt that the LHC including his experiments were a huge success.
%At the time of writing, 10 years after the Higgs boson discovery, it is therefore a good time to look at physics for what it is, a fundamentally data-driven science.