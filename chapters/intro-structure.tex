\chapter{Introduction}
\label{chap:introduction}


\todo{Maybe want to use Cambria font}
\hl{Thesis Introduction To Come}

\chapter{Introduction}
\label{chap:introduction}

The drive to continuously learn about the world around us seems deeply rooted into humanity.
While kids learn this, others learn this, 
Elementary particle physics took on the challenge to understand the fundamental laws of nature by describing the behavior of the elementary building blocks of matter, the \emph{elementary particles}. 
The current best theory is the Standard Model (SM) of particle physics. The SM has been extremely successful in making predictions that were later confirmed by experimental measurements. The culmination of this success story was the discovery of a new particle in 2012 by the ATLAS and CMS collaborations, as by now the experimental evidence is overwhelming that the observed particle is the Higgs boson that was predicted by the SM almost 50 years earlier. 

Despite this success, the discovery of the Higgs boson was only the beginning of an era rather than an end, as several phenomena cannot be explained by the current theory. 
Dark matter, ...
The Higgs boson plays a special role in the quest for answers. 
It is the only particle with a spin of zero and connected to many of the free parameters (10/19) of the SM. 
This makes it a prime candidate for many theories beyond the SM to suggest slight variations of the Higgs boson physics in order to explain some of the open questions. 

This is especially true in the absence of direct discoveries of new particles. 

Nevertheless, its discovery was only the beginning of an era rather an end, as several phenomena related to the Higgs boson remain to be understood. 

In order to test the predictions by the SM and probe theories beyond it, large accelerators are built that probe the interactions of fundamental particles at high energies. The Large Hadron Collider is the current most powerful accelerator worldwide, and provides high energy collisions between two protons. 
The collision events are measured by large detector machines built around the interaction point. The ATLAS experiment is one such example, built as a multipurpose detector for a variety of physics application. It is especially well-suited to measure the type of detector signals that Higgs bosons produce. 

This thesis presents a measurement of Higgs boson production cross sections in its decay to a pair of $W$ bosons. The \HWW\ decay is the second most likely decay of the Higgs boson and provides an excellent opportunity to study the kinematics of Higgs boson production. 


% Learning From Experience: Measurement of HWW decays to two W bosons with a "learned" discriminator

% The Standard Model (SM) of particle physics is currently the most comprehensive theoretical basis for the study of the fundamental laws of nature.

% - To produce collision events at the highest energies, circular accelerators are built in order to gradually increase the energy of the particles.

% \Minote{}{Mention previous experiments? Tevatron, ...other?}

% The ATLAS experiment is operated by one of the world's largest scientific collaborations. A community of about 3000 scientific authors from 181 institutions and 41 countries~\cite{AtlasCollab} work towards making the most precise measurements of the Standard Model and potentially finding experimental hints for New Physics with the ATLAS detector.

% All experimental measurements show remarkable consistency with the predictions of the SM. 

%Only with a precise knowledge of Higgs processes and an understanding of what exact role the Higgs boson plays in nature, fundamental physics can come closer to answering some of the open fundamental questions.  

% By now the Higgs is firmly established in the SM. 

% This might be good for the thesis introduction and is also repeated below
% Physics beyond the SM is typically referred to as ``New Physics''.
% The quest to find new physics is usually driven by looking for deviations of the SM predictions.
% Therefore, it is important to understand with high precision the physics of the SM itself, in particular the properties of the particles that are predicted to couple to new physics. 
% The Higgs boson is often seen as a portal to new physics, which motivates measuring its properties precisely. 

% From current limitations of SM
% The Higgs boson takes a special role in the SM because it is the only fundamental scalar. Its nature makes it a suitable candidate in many models beyond the SM to predict couplings between the Higgs boson and particles from new physics~\cite{2019BHeinemann}.
% This highly motivates precision measurements of the Higgs boson to be able to detect small deviations from the SM predictions. 




% Simulated $pp$ collision events are therefore generated to compare the data against what is expected given the current knowledge of particle physics.

% RESOURCES:
% https://www.nature.com/articles/s41567-020-01054-6.pdf

% From gianotti
% The Higgs boson as a starting point
% Recent experimental results, in particular
% from the LHC, have radically transformed
% the status of particle physics and form
% the basis for future research directions.
% The discovery of the Higgs boson has
% been a turning point, unveiling a particle
% with unprecedented characteristics and
% shedding new light on a phenomenon
% that has surprising similarities with 
% the way certain materials behave
% as superconductors below a critical
% temperature.
% It was fascinating to realize that the same
% phenomenon operates at cosmic scales,
% and made the early Universe undergo
% a phase transition that transformed the
% nature of empty space. Nevertheless,
% believing that the Higgs boson discovery
% has completed our understanding of this
% complex phenomenon is too simplistic.
% On the contrary, much remains to be
% understood about this very special particle,
% including whether it is an elementary
% or composite object, how it leads to the
% peculiar pattern of quark and lepton masses
% observed, what determines the stability of
% the vacuum and what triggered the phase
% transition in the early Universe.
% These questions are still largely
% unexplored experimentally and raise deep
% conceptual concerns theoretically. That is
% why the ESPP update has identified the
% detailed study of the Higgs boson as the
% most pressing priority for the field. Since
% the Higgs boson discovery in 2012 the
% general-purpose LHC experiments, ATLAS
% and CMS, have made extraordinary
% progress in pinning down the features
% of this particle and, by the end of LHC
% operation in 2038 — thanks to the
% high-luminosity upgrades of the collider
% and the detectors — they should be able to
% measure the Higgs boson properties with
% greatly improved precision.
% To gain even deeper insights into the
% Higgs boson and its role in fundamental
% physics, the ESPP recommends an electron–
% positron collider as the next facility, followed
% by a high-energy proton–proton collider in
% the longer term.

% NIMA on 100 TeV collider
% https://arxiv.org/abs/1511.06495



% Precise understanding of Higgs processes may also reveal other deviations from the SM predictions that could, for example, shed light on the matter-antimatter asymmetry of the universe. \todo{FIND citation or leave out!}

\section*{Outline of the thesis and contributions by the author}
This thesis presents research that has been conducted with the ATLAS collaboration. 
The different chapters as well as the level of detail reflect the author's contributions to the research.
The following provides an outline of the thesis including a summary of the author's contributions.

\Cref{chap:theory} summarizes the theoretical framework of the Standard Model and introduces the concepts required to describe and measure proton-proton collision events. 
% \Cref{chap:higgs} introduces all relevant aspects of Higgs boson physics. First, the importance of Higgs boson measurements is highlighted, followed by a description of the phenomenology of Higgs boson processes and the current experimental status.

\Cref{chap:higgs} summarizes the phenomenology and recent measurements of Higgs boson processes, allowing to place the research presented in this thesis in a broader context. 

\Cref{chap:experiment} presents the experimental setup used to measure Higgs boson events, first describing the LHC and then detailing the various detector systems of the ATLAS experiment.

\Cref{chap:objects} discusses the various algorithms that are used in the ATLAS collaboration to reconstruct and identify physics objects. 

\Cref{chap:calibration} discusses how the energy of jets is calibrated with the ATLAS experiment, focusing on the measurement of the noise term of the jet energy resolution. 
The author carried out all aspects of the noise term measurement and improved the measurement uncertainties by introducing several new methodologies and strategies. As a result of the author's research, this measurement has been successfully performed for the first time for particle-flow jets (published in \ccite{JETM-2018-05}). The calibrations and uncertainties are used in various analyses, and the author implemented the calibration procedure in one of the common analysis software frameworks.
%and \emph{R-scan} jets (jets with a radius of $R=0.2$ and $R=0.6$, publication currently in ATLAS internal review), as a result of the author's studies and research.

\Cref{chap:statistics} introduces the relevant statistical concepts used in the analysis of Higgs boson events.

\Cref{chap:ml} presents the machine learning techniques used to develop a discriminant that separates Higgs boson events from non-Higgs events.

\Cref{chap:hww} is the main part of this thesis and presents the measurement of VBF and ggF Higgs boson production in the \HWW\ decay channel that is published in \ccite{CouplingsPaper}. The author contributed significantly to many aspects of the analysis, focusing on the measurement of the VBF production mode. 
Most notably, the author trained, validated, and incorporated a deep neural network (DNN) to distinguish VBF, \HWW events from other events, which lead to a substantial reduction in measurement uncertainties.
The DNN was used for both the inclusive cross-section measurement and the measurement within the STXS framework.
% This included the development of new strategies to use the DNN output as final discriminant in the statistical analysis. 
% The author also studied different analysis strategies for the cross-section measurements within the STXS framework.
The author developed and extensively studied the statistical model used for both measurements, and introduced several improvements to the treatment of systematic uncertainties.
Furthermore, the author contributed significantly to the development and improvement of the analysis framework as well as the statistical framework. As a result, several aspects of the overall analysis workflow have been vastly improved, which has proved extremely useful to many analyzers inside and outside the \HWW analysis group.
Besides these main contributions, the author frequently contributed to review meetings, performed several smaller optimization and validation studies, is co-author of the internal supporting documentation, and provided final results for the analyses that are published in \ccite{ATLAS-CONF-2021-014,ATLAS-CONF-2020-045,CouplingsPaper}.
\todo{Also mention SFUsMLKit development and histogram smoothing}

\Cref{chap:summary} concludes this thesis by providing a summary and an outlook.


% \begin{itemize}
%     \item HWW Analysis
%           \begin{itemize}
%               \item Framework, migration to R21, setup for full Run2, significant workflow enhancement
%               \item Optimization studies for ggF regions
%               \item Training, testing, incorporation of Deep Neural Network as VBF analysis final discriminant
%               \begin{itemize}
%                   \item Establish binning choice algorithm
%               \end{itemize}
%               \item Statistics Analysis including major improvements in the workflow
%               \item Derivation of Theoretical Uncertainties
%               \item Developed and implemented new treatment of Theoretical Uncertainties (pruning, smoothing)
%               \item STXS Analysis Optimization (CR and SR merging studies)
%               \item Prospect studies derived in cooperation with students about re-definition of HWW 2-jet analysis
%               \item Contributed to several analysis meetings
%           \end{itemize}
%     \item Jet/Etmiss
%           \begin{itemize}
%               \item Derivation of noise term of JER for small-R jets as well as Rscan jets used by the entire collaboration
%               \begin{itemize}
%                   \item Developed new methods to derive Uncertainties
%                   \item Measurement for PFlow jets for the first time
%                   \item Extended to Rscan jets for the first time
%               \end{itemize}
%               \item Studies and comparison of new minimum bias generators as crucial input for new default pile-up generator in ATLAS
%           \end{itemize}
% \end{itemize}

% LUMINOSITY DEFINITION OF MIKE:
% The luminosity measures the number of particles per unit area and time, and together with the probability of interaction (cross section) determines the collision rate. 
% See https://www.notion.so/Mike-Review-of-Experiment-Chapter-ca62b64490f644f8804bc9c34c6de303


1. Epic Intro
- The drive to learn seems deeply rooted into humanity. 
- Learning sits at the core of humanity. 
 Learning the fundamental laws of nature is the challenge elementary particle physics has set itself up for. 
The quest for learning how nature behaves is driven by curiosity only. 

2. Particle physics / SM
The current best theory to describe the behavior of the fundamental building blocks of matter, called elementary particles, is known as the Standard Model. 
It has been hugely successful in describing various experimental measurements correctly. 
The Higgs boson discovery in 2012 can be regarded as the culmination of success the success story. 
It updated the status of particle physics as a whole and largely influences future research directions. 
There open problems and much remains to be understood!

3. Higgs takes special place
The Higgs boson sits at the core of the theory, being connected to 10 out of the 19 free parameters that need to be determined by experimental data.  
This makes it very important to understand all aspects of Higgs boson physics precisely. In particular, the probability of the Higgs boson to be produced, measured in terms of cross sections, or the interactions with other particles.
A comprehensive study of the Higgs boson and all its interactions with other particles is therefore essential. 

4. Experiments (LHC,  ATLAS)
Large particle colliders allow testing SM predictions and looking for new phenomena. 
ATLAS detector

5. Analysis 
This thesis presents a measurement of Higgs boson production cross sections in its decay to a pair of $W$ bosons. 
The HWW decay is the second most likely one and provides an excellent opportunity to study the kinematics of specifically the gluon fusion as well as the vector boson fusion production mode of the Higgs boson.

The success of most physics analyses is determined by two main factors: 
(i) 
- the ability to cleverly select collision events so the amount of signal events that are to be measured is large compared to the other ``background'' events.  
- The ability to distinguish interesting Higgs boson signal events from other types of events.
(ii) the ability to precisely estimate the expected contributions given the selection criteria (query), with as little uncertainty as possible.

6. JER and Higgs cross sections
For the latter, this thesis presents a measurement of the noise term of the jet energy resolution, allowing to select and disentangle events at a finer resolution. 
For the former, this thesis presents the development of a deep neural network that discriminates Higgs boson events produced via vector boson fusion. A supervised learning algorithm is used for a neural net to learn to recognize patterns from labelled example data and separate Higgs events from backgrounds events. These patterns are too complex and multidimensional for a human to find, so machines are taught to learn them for us. 
This allows measuring the amount of VBF, HWW events produced at unprecedented precision. 

7. deviations, STXS, further interpretations. 
The precision measurements of HWW decays can be used for further combinations and interpretations. 


% Learning From Experience: Measurement of HWW decays to two W bosons with a "learned" discriminator

% The Standard Model (SM) of particle physics is currently the most comprehensive theoretical basis for the study of the fundamental laws of nature.

% - To produce collision events at the highest energies, circular accelerators are built in order to gradually increase the energy of the particles.

% \Minote{}{Mention previous experiments? Tevatron, ...other?}

% The ATLAS experiment is operated by one of the world's largest scientific collaborations. A community of about 3000 scientific authors from 181 institutions and 41 countries~\cite{AtlasCollab} work towards making the most precise measurements of the Standard Model and potentially finding experimental hints for New Physics with the ATLAS detector.

% All experimental measurements show remarkable consistency with the predictions of the SM. 

%Only with a precise knowledge of Higgs processes and an understanding of what exact role the Higgs boson plays in nature, fundamental physics can come closer to answering some of the open fundamental questions.  

% By now the Higgs is firmly established in the SM. 

% This might be good for the thesis introduction and is also repeated below
% Physics beyond the SM is typically referred to as ``New Physics''.
% The quest to find new physics is usually driven by looking for deviations of the SM predictions.
% Therefore, it is important to understand with high precision the physics of the SM itself, in particular the properties of the particles that are predicted to couple to new physics. 
% The Higgs boson is often seen as a portal to new physics, which motivates measuring its properties precisely. 

% From current limitations of SM
% The Higgs boson takes a special role in the SM because it is the only fundamental scalar. Its nature makes it a suitable candidate in many models beyond the SM to predict couplings between the Higgs boson and particles from new physics~\cite{2019BHeinemann}.
% This highly motivates precision measurements of the Higgs boson to be able to detect small deviations from the SM predictions. 




% Simulated $pp$ collision events are therefore generated to compare the data against what is expected given the current knowledge of particle physics.

% RESOURCES:
% https://www.nature.com/articles/s41567-020-01054-6.pdf

% From gianotti
% The Higgs boson as a starting point
% Recent experimental results, in particular
% from the LHC, have radically transformed
% the status of particle physics and form
% the basis for future research directions.
% The discovery of the Higgs boson has
% been a turning point, unveiling a particle
% with unprecedented characteristics and
% shedding new light on a phenomenon
% that has surprising similarities with 
% the way certain materials behave
% as superconductors below a critical
% temperature.
% It was fascinating to realize that the same
% phenomenon operates at cosmic scales,
% and made the early Universe undergo
% a phase transition that transformed the
% nature of empty space. Nevertheless,
% believing that the Higgs boson discovery
% has completed our understanding of this
% complex phenomenon is too simplistic.
% On the contrary, much remains to be
% understood about this very special particle,
% including whether it is an elementary
% or composite object, how it leads to the
% peculiar pattern of quark and lepton masses
% observed, what determines the stability of
% the vacuum and what triggered the phase
% transition in the early Universe.
% These questions are still largely
% unexplored experimentally and raise deep
% conceptual concerns theoretically. That is
% why the ESPP update has identified the
% detailed study of the Higgs boson as the
% most pressing priority for the field. Since
% the Higgs boson discovery in 2012 the
% general-purpose LHC experiments, ATLAS
% and CMS, have made extraordinary
% progress in pinning down the features
% of this particle and, by the end of LHC
% operation in 2038 — thanks to the
% high-luminosity upgrades of the collider
% and the detectors — they should be able to
% measure the Higgs boson properties with
% greatly improved precision.
% To gain even deeper insights into the
% Higgs boson and its role in fundamental
% physics, the ESPP recommends an electron–
% positron collider as the next facility, followed
% by a high-energy proton–proton collider in
% the longer term.

% NIMA on 100 TeV collider
% https://arxiv.org/abs/1511.06495



% Precise understanding of Higgs processes may also reveal other deviations from the SM predictions that could, for example, shed light on the matter-antimatter asymmetry of the universe. \todo{FIND citation or leave out!}

\section*{Outline of the thesis and contributions by the author}
This thesis presents research that has been conducted with the ATLAS collaboration. 
The different chapters as well as the level of detail reflect the author's contributions to the research.
The following provides an outline of the thesis including a summary of the author's contributions.

\Cref{chap:theory} summarizes the theoretical framework of the Standard Model and introduces the concepts required to describe and measure proton-proton collision events. 
% \Cref{chap:higgs} introduces all relevant aspects of Higgs boson physics. First, the importance of Higgs boson measurements is highlighted, followed by a description of the phenomenology of Higgs boson processes and the current experimental status.

\Cref{chap:higgs} summarizes the phenomenology and recent measurements of Higgs boson processes, allowing to place the research presented in this thesis in a broader context. 

\Cref{chap:experiment} presents the experimental setup used to measure Higgs boson events, first describing the LHC and then detailing the various detector systems of the ATLAS experiment.

\Cref{chap:objects} discusses the various algorithms that are used in the ATLAS collaboration to reconstruct and identify physics objects. 

\Cref{chap:calibration} discusses how the energy of jets is calibrated with the ATLAS experiment, focusing on the measurement of the noise term of the jet energy resolution. 
The author carried out all aspects of the noise term measurement and improved the measurement uncertainties by introducing several new methodologies and strategies. As a result of the author's research, this measurement has been successfully performed for the first time for particle-flow jets (published in \ccite{JETM-2018-05}). The calibrations and uncertainties are used in various analyses, and the author implemented the calibration procedure in one of the common analysis software frameworks.
%and \emph{R-scan} jets (jets with a radius of $R=0.2$ and $R=0.6$, publication currently in ATLAS internal review), as a result of the author's studies and research.

\Cref{chap:statistics} introduces the relevant statistical concepts used in the analysis of Higgs boson events.

\Cref{chap:ml} presents the machine learning techniques used to develop a discriminant that separates Higgs boson events from non-Higgs events.

\Cref{chap:hww} is the main part of this thesis and presents the measurement of VBF and ggF Higgs boson production in the \HWW\ decay channel that is published in \ccite{CouplingsPaper}. The author contributed significantly to many aspects of the analysis, focusing on the measurement of the VBF production mode. 
Most notably, the author trained, validated, and incorporated a deep neural network (DNN) to distinguish VBF, \HWW events from other events, which lead to a substantial reduction in measurement uncertainties.
The DNN was used for both the inclusive cross-section measurement and the measurement within the STXS framework.
% This included the development of new strategies to use the DNN output as final discriminant in the statistical analysis. 
% The author also studied different analysis strategies for the cross-section measurements within the STXS framework.
The author developed and extensively studied the statistical model used for both measurements, and introduced several improvements to the treatment of systematic uncertainties.
Furthermore, the author contributed significantly to the development and improvement of the analysis framework as well as the statistical framework. As a result, several aspects of the overall analysis workflow have been vastly improved, which has proved extremely useful to many analyzers inside and outside the \HWW analysis group.
Besides these main contributions, the author frequently contributed to review meetings, performed several smaller optimization and validation studies, is co-author of the internal supporting documentation, and provided final results for the analyses that are published in \ccite{ATLAS-CONF-2021-014,ATLAS-CONF-2020-045,CouplingsPaper}.
\todo{Also mention SFUsMLKit development and histogram smoothing}

\Cref{chap:conclusion} concludes this thesis by providing a summary and an outlook.


% \begin{itemize}
%     \item HWW Analysis
%           \begin{itemize}
%               \item Framework, migration to R21, setup for full Run2, significant workflow enhancement
%               \item Optimization studies for ggF regions
%               \item Training, testing, incorporation of Deep Neural Network as VBF analysis final discriminant
%               \begin{itemize}
%                   \item Establish binning choice algorithm
%               \end{itemize}
%               \item Statistics Analysis including major improvements in the workflow
%               \item Derivation of Theoretical Uncertainties
%               \item Developed and implemented new treatment of Theoretical Uncertainties (pruning, smoothing)
%               \item STXS Analysis Optimization (CR and SR merging studies)
%               \item Prospect studies derived in cooperation with students about re-definition of HWW 2-jet analysis
%               \item Contributed to several analysis meetings
%           \end{itemize}
%     \item Jet/Etmiss
%           \begin{itemize}
%               \item Derivation of noise term of JER for small-R jets as well as Rscan jets used by the entire collaboration
%               \begin{itemize}
%                   \item Developed new methods to derive Uncertainties
%                   \item Measurement for PFlow jets for the first time
%                   \item Extended to Rscan jets for the first time
%               \end{itemize}
%               \item Studies and comparison of new minimum bias generators as crucial input for new default pile-up generator in ATLAS
%           \end{itemize}
% \end{itemize}

% LUMINOSITY DEFINITION OF MIKE:
% The luminosity measures the number of particles per unit area and time, and together with the probability of interaction (cross section) determines the collision rate. 
% See https://www.notion.so/Mike-Review-of-Experiment-Chapter-ca62b64490f644f8804bc9c34c6de303
