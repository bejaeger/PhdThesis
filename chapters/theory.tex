\chapter{Theoretical Foundation}
\label{chap:theory}
This chapter presents the theoretical foundations relevant to this thesis.  
The Standard Model (SM) provides the theoretical basis for studying the fundamental particles of nature and the laws that govern them. It was developed over the course of several decades in the second half of the 21$^{\text{th}}$ century through an interplay of theoretical predictions and experimental observations.
% The theoretical framework that forms the basis for describing the interaction between all known fundamental particles is known as the Standard Model (SM) of particle physics.
\Cref{sec:sm} summarizes the mathematical formulation of the Standard Model (SM), placing particular emphasis on the \emph{Higgs mechanism} of \emph{electroweak symmetry breaking} (EWSB) that predicts the existence of the Higgs boson.
\Cref{sec:limitations} addresses the limitations of the SM, which motivates the work presented in this thesis. 
%, which currently provides the best theoretical framework to study the fundamental laws of nature. 
% The SM allows making predictions about the probabilities of various interactions between elementary particles. 
% These predictions can be tested, for example, with high-energy proton-proton ($pp$) collision events.
%<- maybe this in Intro of thesis?
\Cref{sec:anatomy} describes the phenomenology of high-energy proton-proton ($pp$) collision events that are used to test the predictions of the SM.
%and introduces the concepts necessary to describe and measure $pp$ interactions. 

% From CARSTEN:
%This chapter outlines the Lagrange formulation of the Standard Model of Elementary Particle Physics (SM), a successful framework to study the fundamental laws governing the universe. Special attention is devoted to the Brout-Englert-Higgs (BEH) mechanism of electroweak symmetry breaking (EWSB). A more detailed introduction including all aspects of the model can be found in many textbooks [24–28].

% From Master:
%This chapter is intended to give an overview of the theoretical foundations used in elementary particle physics that are relevant for this thesis. The basic mathematical concepts of the Standard Model (SM) of particle physics are described in Section 2.1. The general principles of gauge theories and the mechanism of electroweak-symmetry-breaking leading to the prediction of the Higgs boson are briefly introduced. Section 2.2 summarizes the phenomenology at hadron colliders and introduces the concepts to describe proton-proton interactions. The different production and decay modes of the Higgs boson are finally discussed in Section 2.3 from a phenomenological point of view.

\section{The Standard Model of Particle Physics}
\label{sec:sm}


% From Master's thesis:
The SM of particle physics is a collection of relativistic quantum field theories (QFT) that describe the interactions between all known fundamental particles.
Mathematically, it is a non-abelian gauge theory with the gauge group
\begin{equation}
  \label{eq:sm-gauge-group}
  \text{SU(3)}_C \times \text{SU(2)}_L \times \text{U(1)}_Y,
\end{equation}
the details of which are explained in this section.
% The mathematical formulation SU(3)$_C$ $\times$ SU(2)$_L$ $\times$ U(1)$_Y$ local gauge symmetry that gives rise to the interactions between all known fundamental particles.
%It evolved during the 60’s and 70’s due to a strong interplay between experimental observations and theoretical developments. 
\Cref{subsec:particle-content} provides a high-level overview of the particles and forces that are part of the SM. \Cref{subsec:formalism} briefly outlines the theoretical principles that build the basis for the mathematical description of the SM. \Cref{subsec:qed,subsec:qcd,subsec:ew-model,subsec:ewsymbreaking,subsec:fermion-masses,subsec:final-lagrangian} discuss the mathematical formulation of the SM, including a detailed description of the \emph{Higgs mechanism} of \emph{electroweak symmetry breaking} (EWSB).
The free parameters of the SM are discussed in \cref{subsec:free-pars-sm}, and \cref{subsec:limitations} concludes this section by summarizing the limitations and open questions of the SM.
% that motivate conducting more precise measurements such as the one presented in \cref{chap:hww} of this thesis.
The SM is covered in all its detail in the literature, e.g. \ccite{Peskin:1995ev,Halzen:1984mc,Thomson:2013zua}, which serve as the primary resources for the following descriptions.


% From previous theoretical foundation intro:
% - Section underlines the theoretical underpinnings that manifest the Higgs boson as a special place in the universe
% - Lots of SM parameters are related to Higgs (SM stands and falls with Higgs)
% - The SM also has limitations -> precision measurements of Higgs required
% - Many models of new physics expect to have access to new physics through interactions with Higgs ()


%On a phenomenological level, fundamental physics can be described in terms of elementary particles and forces. 
% - Introduction to QFT and gauge group
% - Lagrangian formalism
% - local gauge invariance: without local gauge invariance:

% From Pich The Standard Model
% Thus, once a given phase convention has been adopted at one reference point x0, the same convention must be taken at all space-time points. This looks very unnatural.

% - highly successful in describing QED
% WHAT HAPPENS IN THIS SECTiION:

% - First overview of particles and forces
% - Formalism
% - QED, QCD, electroweak model
% - Problem: no masses for bosons -> Higgs mechanism
% - Final Lagrangian and parameters to be measured experimentally of the SM
% - Limitations of the SM

\subsection{Particles and Forces}
\label{subsec:particle-content}

\begin{figure}
  \newImageResizeCustom{1}{figures/theory/particles-infographic/particles-infographic.pdf}
  \caption[Overview of particles in the SM.]{Overview of the particles in the SM. Their charge, color, mass, and spin are indicated. More details are given in the text. The graphic is adapted from \ccite{CBurgardParticlesInfographic} and the values taken from \ccite{PDG2020}.}  
  \label{fig:particles-infographic}
\end{figure}
\todo{Maybe put graphic on full page alone to make it bigger}
%In QFT all fundamental fields have an associated particle, that can be thought of as excitations (or vibrations) of the fields.
QFT describes nature in terms of fundamental fields and their interactions. Each field is associated with an elementary particle that can be thought of as a quantum excitation of the underlying field. 
This association makes it possible to describe fundamental physics using elementary particles.
%\footnote{To simplify terminology, the objects are mostly referred to as particles rather than fields in this chapter, but the relationship is important to be kept in mind.} 
% On a \TDnote{phenomenological}{other word?} level, fundamental physics can be described in terms of elementary particles and forces.
% All currently known particles and forces included in the SM are listed in \TDnote{REF}{REF}.
The particle content of the SM is summarized in \cref{fig:particles-infographic}. The particles can be grouped into two main types: 
\emph{fermions} with half-integer spin and \emph{bosons} with integer spin. The interaction between the fermions (sometimes called \emph{matter particles}) are interpreted as exchanges of \emph{gauge bosons} (also called \emph{force carriers}) that mediate the fundamental forces. 
% \emph{fermions} with half-integer spin (sometimes called \emph{matter particles}) and \emph{bosons} with integer spin (known as the \emph{mediators} or \emph{force carriers} of the fundamental forces).

The fermions can be grouped into three generations, each containing two leptons and two quarks. The different types of leptons and quarks are referred to as (quark or lepton) \emph{flavors}.
Quarks have an additional property called \emph{color}, which can take three possible values typically labelled as red, green, and blue\footnote{In fact, also linear combinations of the three colors are realized in nature, as discussed later in this chapter.}.
Additionally, each fermion has a corresponding antiparticle that has the same mass but ``opposite'' internal quantum numbers, that is, oppositely signed charge and anti-color if applicable.

%The interactions between the fermions can be described by exchanges of so-called \emph{gauge bosons}.
The SM includes three of the four known fundamental interactions: The \emph{electromagnetic interaction}, the \emph{strong interaction}, and the \emph{weak interaction}. The electromagnetic and the weak interactions are unified in the \emph{electroweak theory}. The fourth known fundamental interaction, \emph{gravitation}, is not part of the SM but can be fully neglected in present particle physics experiments due to its weak strength.
The gauge boson that mediates the electromagnetic interaction is the massless \emph{photon} that interacts with all charged fermions. Electromagnetic interactions are described by the theory of \emph{quantum electrodynamics} (QED). The strong interaction is transmitted via eight massless gluons, following the rules of \emph{quantum chromodynamics} (QCD). QCD acts on matter particles that carry color, that is, quarks and gluons themselves. 
Due to a property known as \emph{confinement} in QCD\footnote{Confinement arises because of the energy dependence of the strength of the QCD interactions, which is discussed in \cref{subsec:factorisation}.}, quarks cannot be found in isolation. They are confined within hadrons that consist, for example, of a quark and an anti-quark (known as \emph{mesons}) or three quarks (known as \emph{baryons}).
The weak interaction is mediated by three massive gauge bosons, \Wplus, \Wminus, and \Zboson, and acts on all fermions. Similar to gluons, the \Wpm and \Zboson bosons are able to interact with themselves.

The final particle of the SM is the electrically neutral \emph{Higgs boson}. It is the only spin-0 scalar particle and plays a special role in the SM in the mechanism of electroweak symmetry breaking through which the \Wpm and \Zboson bosons gain their masses. A dedicated overview of the physics involving the Higgs boson is provided in \cref{sec:higgs-phen}.


\subsection{Formalism and Principles}
\label{subsec:formalism}
% - Action: S = Int ( Lagrangian ) dt -> EOMs
% - Lagrangian = Int (Lagrange density ) d3x
% -> Lagrange density is commonly referred to as Lagrangian 
% - In QFT, we define EOMs for fields by specifying the Lagrange density (Lagrangian)

% ONE MORE SENTENCE:
% In physics, equations of motion are equations that describe the behavior of a physical system in terms of its motion as a function of time.[1] More specifically, the equations of motion describe the behavior of a physical system as a set of mathematical functions in terms of dynamic variables.

% Particles can be thought of as excitations (or vibrations) of corresponding fundamental fields. 
In order to describe the behaviour of a physical system, the equations of motions can be derived from the action,
\begin{equation}
  \label{eq:action}
  S = \int \Lagrangian d^3x dt,
\end{equation}
by following the action principle.\footnote{The derivation of the equations of motion can be found in any standard textbook on QFT, e.g. in \ccite{Peskin:1995ev}.}
The dynamics of a system are therefore fully specified by the Lagrange density, \Lagrangian. The Lagrange density, often simply denoted \emph{Lagrangian}, generally is a function of the fields $\psi$, and their derivatives $\partial \psi$, so $\Lagrangian \rightarrow \Lagrangian\left(\psi, \partial\psi\right)$. The fields are themselves generally dependent on the space-time coordinate, $\psi \rightarrow \psi(x)$.\footnote{For cleaner notation, these dependencies are not explicitly mentioned throughout this thesis.}

For a given field $\psi$, the principles of QFT demand to include all possible interaction terms in the Lagrangian that are related to the field\footnote{This follows what is sometimes referred to as the ``totalitarian principle'' of quantum mechanics, that satirically states: ``Everything not forbidden is compulsory'', meaning that everything allowed by the laws of physics must actually happen or exist.}.
The number and type of terms that can be included in the SM, however, is strongly constraint by requiring the theory to be invariant under symmetry transformations of the fields following the SM gauge group shown in \cref{eq:sm-gauge-group}.
The SM is governed by the principle of \emph{local gauge invariance}, which implies that the physical content of the theory stays the same when performing transformations of the particle fields,
\begin{equation}
  \psi \rightarrow e^{-iT} \psi,
\end{equation}
independently at every space-time point. Here, $T$ is the generator of a symmetry group which is generally dependent on the space-time coordinate, $T \rightarrow T(x)$. 
\todo{Maybe remove the transformation equation. A bit distracting/random here?}
% Gauge theories require the introduction of so-called gauge fields, that transform in a way so that the theory stays locally gauge invariant.
% Pich
% This is only possible if one adds an extra piece to the Lagrangian, transforming in such a way as to cancel the ∂μθ term in Eq. (6).
% In order to maintain the symmetry, the Lagrangian may need to be manipulated by introducing new fields.
Gauge theories require the addition of so-called gauge fields, which give rise to particle interactions that give rise to the fundamental forces of nature. 
Historically, the first local gauge theory that was established is QED. It follows a U(1) local gauge symmetry and entails the photon as the associated gauge field. 
The SM can be elegantly described by demanding local gauge invariance with the symmetry group as shown in \cref{eq:sm-gauge-group}, and explains the existence of all gauge bosons mentioned in the previous section. 
The fermions, in contrast, are added a priori to the theory and do not follow from fundamental principles.
%The following sections provide an overview of the Lagrangian of the SM, by going through the different symmetry groups. 
The following sections will outline the Lagrange formulation of the SM.
In all equations, natural units are used, that is, $c = \hbar = 1$, which leads to the mass, momentum, and energy of particles being expressed in units of electronvolt (\eV).
Moreover, the Einstein-summation convention is used, that is, terms with equal indices are considered to be summed. If not mentioned otherwise, greek-letter indices take integer values from 1 to 4, while latin-character indices take integer values from 1 to 3.

% - renormalization:
% We consider that the physics is understood up to a given cut-off/scale Lambda.

%The SM is governed by the principle of local gauge invariance and the associated symmetry group is SU(3)$_C$ $\times$ SU(2)$_L$ $\times$ U(1)$_Y$. 

% the next important lesson is that the existence of these symmetries places suck an incredibly strong constraint on what the theory actually is.

% Sean
% QED: demanding all the terms in your Lagrangian being gauge invariant is enforcing the conservation of electric charge gauge
% This is a reflection of Noethers theory. Symmetry conservation -> associated quantity of the U(1) symmetry is charge

% Sean Carrol:
% W and Z bosons are the physical excitations from vibrations in the SU(2) to gauge field
%the existence of these fields giving rise to interactions giving rise to forces of nature comes from the gauge symmetry.
% the next important lesson is that the existence of these symmetries places suck an incredibly strong constraint on what the theory actually is.

% From Master thesis
% An essential part of the mathematical formulation of the SM is based on the postulation of local gauge invariance. It implies, that the physical content of the theory should stay the same when performing certain redefinitions of the particle fields independently at every space-time point. The SM follows a SU(3)×SU(2)L×U(1)Y symmetry group, where SU(3) is the gauge group of QCD and SU(2)L×U(1)Y the corresponding symmetry for the electroweak model (the meaning of the subindizes L and Y will be explained in the following sections). Historically, QED was the first well established gauge theory, following a U(1) symmetry. To demonstrate the principles of a gauge theory, which are the key concepts of the mathematical framework of the SM, the QED Lagrangian will be derived in the following. Subsequently the same principles are applied to describe the main aspects of the more complex theories of QCD and the electroweak model. A description of the mechanism to incorporate masses for the W± and Z bosons via breaking the electroweak symmetry is given thereafter. The following sections follow to a large extend the more detailed descriptions given in Refs. [19–21].


\subsection{Quantum electrodynamics}
\label{subsec:qed}
%The above derivations are shortly recapped: Starting from the Lagrangian of a freely moving relativistic fermion one can impose the requirement of local U(1) gauge invariance. This demands to add a new field $A_\mu$ toghether with an interaction term, which is incorporated in the Lagrangian by introducing a covariant derivative $D_\mu$. Additionally, the kinetic term in \cref{eq:kinetictermqed} needs to be added for the new field, which leads to the final Lagrangian for QED, shown in \cref{eq:Lagrangianqed}.
\todo{Explain w(x) in the equation and does it need to be bold x?}
QED can be regarded as a reflection of an underlying $U(1)$ local gauge symmetry of the complex-valued fermion fields, $\psi_f$, known as \emph{Dirac spinors}.
The Lagrangian that is invariant under $\psi_f \rightarrow e^{i \omega(x)} \psi_f$ transformations can be written as
\begin{equation}
  \mathcal{L}_{\text{QED}} = \sum_f \bar{\psi}_f(i\gamma^\mu D_\mu - m_f)\psi_f - \frac{1}{4}F_{\mu\nu}F^{\mu\nu},
  \label{eq:Lagrangianqed}
\end{equation}
where the sum goes over all electrically charged fermions with masses $m_f$, and $\gamma^\mu$ refers to the four $4 \times 4$ gamma matrices.
Here, $D_\mu$ is the \emph{covariant derivative} defined as
\begin{equation}
  D_\mu = \partial_\mu + ieA_\mu,
\end{equation}
which includes the gauge field $A_\mu$ that is associated with the photon. The requirement of local gauge symmetry prohibits terms quadratic in $A_\mu$, resulting in the prediction of the photon being massless.\footnote{Terms in the Lagrangian that are quadratic in a certain field give rise to particle masses.}
The so-called \emph{field tensor} is defined as
\begin{equation}
  F_{\mu\nu} = \partial_\mu A_\nu - \partial_\nu A_\mu,
\end{equation}
where $\partial_\mu$ are the partial derivatives. This tensor provides an elegant mathematical object representing the electromagnetic field and can also be used, for example, to describe Maxwell's equations of classical electrodynamics.

The symmetry group associated to QED is denoted U(1)$_{\text{QED}}$ and the conserved quantity is the electric charge.\footnote{This is a reflection of Noether's theorem, which states that each symmetry is related to a conserved quantity.}
It should be noted that U(1)$_{\text{QED}}$ is not part of the original symmetry group of the SM. As explained below, U(1)$_{\text{QED}}$ arises from a SU(2)$_L$ $\times$ U(1)$_Y$ symmetry that is spontaneously broken.


\subsection{Quantum chromodynamics}
\label{subsec:qcd}
QCD is a non-abelian gauge theory associated to a SU(3)$_C$ symmetry. The subindex $C$ refers to the color which is the conserved quantity under SU(3)$_C$ transformations.
The locally gauge invariant QCD Lagrangian reads
\begin{equation}
  %  \mathcal{L}_{\text{QCD}} = -\frac{1}{4}G_{\mu\nu}^aG^{a\,\mu\nu} + \bar{q}^i\left( i\gamma^\mu D_\mu-m \right)^j_i q_j,
  \mathcal{L}_{\text{QCD}} = \sum_f \bar{\psi}_f(i\gamma^\mu D_\mu - m_f)\psi_f - \frac{1}{4}G_{\mu\nu}^aG^{\mu\nu}_{a},  \label{eq:lqcd}
\end{equation}
where the sum runs over all quark fields, $\psi_f$, that take the form of spinor triplets. 
\todo{Sum over A in lambdaA GmuA in the equation?}
The covariant derivate is given by
\begin{equation}
    D_\mu = \partial_\mu + i g_s \frac{\lambda^a}{2} G_\mu^a,
\end{equation}
which includes eight gluon fields $G_\mu^a$ (with $a = 1, \ldots, 8$), the strong coupling constant $g_s$, and the Gell-Mann matrices $\lambda^a$ that are the generators of the SU(3)$_C$ group.
The field tensors $G_{\mu\nu}^a$ are defined as
\begin{equation}
  \label{eq:qcd-tensor}
  G_{\mu\nu}^a = \partial_\mu G_\nu^a - \partial_\nu G_\mu^a - g_s f^{abc}G_\mu^b G_\nu^c,
\end{equation}
where $f^{abc}$ are the structure constants of SU(3)$_C$. 
The last term in \cref{eq:qcd-tensor} arises from the non-commuting elements of the SU(3)$_C$ group and gives rise to triple and quartic gluon self-interactions. 
% Triple and quartic self-interactions are part of QCD because gluons themselves carry a combination of a color and anti-color. 
% Invariant under ... transformation, where ... are the group generators
Similar to photons, the gluons are massless because no terms quadratic in the gluon fields are allowed when requiring local gauge invariance. 
%which is similar to the requirement of a massless photon in QED. 



\subsection{The electroweak model}
\label{subsec:ew-model}
  \begin{table}
    \caption[Overview of the fermion content in the electroweak model.]{Overview of the fermion content in the electroweak model with the quantum numbers of the weak isospin $T^3$, hypercharge $Y$ and electric charge $Q$. They are grouped into left-handed doublets and right-handed singlets denoted with the subindex $L$ and $R$ respectively. The down-type quarks $d', s', b'$ are the eigenstates of the electroweak interaction and given by linear combinations of the mass eigenstates $d, s, b$. This mixing is described by the CKM matrix, see text. Since right-handed neutrinos are not undergoing any interaction they do not play a role in the Standard Model and are not listed here.}
    \label{tab:ewfermioncontent}
    \centering
      \begin{tabular}{c |@{}| c c c | c c c}
        \toprule
                                 & \multicolumn{3}{c}{Generation}          & \multicolumn{3}{|c}{Quantum number}                                                                                                            \\
                                 & 1$^{\text{st}}$                         & 2$^{\text{nd}}$                             & 3$^{\text{rd}}$                               & $T^3$          & $Y$            & $Q$            \\
        \midrule
        \multirow{3}{*}{Leptons} & \multirow{2}{*}{$\myvec{\nu_e                                                                                                                                                            \\ e}_L$} & \multirow{2}{*}{$\myvec{\nu_\mu \\ \mu}_L$} & \multirow{2}{*}{$\myvec{\nu_\tau \\ \tau}_L$} & $\frac{1}{2}$  & -1             & 0 \\
                                 &                                         &                                             &                                               & $-\frac{1}{2}$ & -1             & -1             \\
                                 & $e_R$                                   & $\mu_R$                                     & $\tau_R$                                      & 0              & -2             & -1             \\
        \midrule
        \multirow{3}{*}{Quarks}  & \multirow{2}{*}{$\myvec{u                                                                                                                                                                \\ d'}_L$}    & \multirow{2}{*}{$\myvec{c \\ s'}_L$}        & \multirow{2}{*}{$\myvec{t \\ b'}_L$}          & $\frac{1}{2}$  & $\frac{1}{3}$  & $\frac{2}{3}$ \\
                                 &                                         &                                             &                                               & $-\frac{1}{2}$ & $\frac{1}{3}$  & $-\frac{1}{3}$ \\
                                 & $u_R$                                   & $ c_R$                                      & $t_R$                                         & 0              & $\frac{4}{3}$  & $\frac{2}{3}$  \\
                                 & $d_R$                                   & $ s_R$                                      & $b_R$                                         & 0              & $-\frac{2}{3}$ & -$\frac{1}{3}$ \\
        \bottomrule
      \end{tabular}
  \end{table}



% - the chirality can be determined with... right-chiral fermions are singlets under SU(2)L transformations
% - This also means that no right-chiral neutrinos exist in the SM as they don't interact with any of the forces

% "Also called: The Glashow􏰁Weinb erg􏰁Salam Theory of Weak Interactions"
The weak and electromagnetic forces are unified in the electroweak model~\cite{GLASHOW1961579,SALAM1964168,PhysRevLett.19.1264} by imposing local gauge invariance under transformations of the symmetry group
\begin{equation}
  \label{eq:ew-sym-group}
  \text{SU(2)}_L \times \text{U(1)}_Y.
\end{equation}
The electroweak gauge group is based on two major empirical findings:
The first is, that only \emph{left-chiral}\footnote{
  The \emph{chirality} of a fermion can be determined with the projection operators $P_L$ and $P_R$ like
  \begin{equation*}
    \psi       = P_L \psi + P_R \psi = \frac{1}{2} \left( 1 - \gamma^5 \right) \psi + \frac{1}{2} \left( 1 + \gamma^5 \right) \psi = \psi_R + \psi_L,                                \\
  \end{equation*}
  where $\gamma^5 = i\gamma^0\gamma^1\gamma^2\gamma^3$. The chirality becomes identical to the helicity for massless particles.
} (also denoted \emph{left-handed}) fermions interact via the weak interaction, indicated with an $L$ in \cref{eq:ew-sym-group}. An overview of the fermion content is shown in \cref{tab:ewfermioncontent}, that groups the fermions into left-handed doublets, $\psi_L$, and right-handed singlets, $\psi_R$, under SU(2)$_L$ transformations. 
The second finding is that the quarks participating in the electroweak interaction, labelled as $u', d', c'$, are a mixture of the quark mass eigenstates. Their relation is specified by the \emph{Cabibbo–Kobayashi–Maskawa (CKM) matrix} \cite{doi:10.1143/PTP.49.652}, $\pmb{V}$, as
% From Peskin
% The off diagonal terms in Vij allow weak􏲩interaction transitions b e􏲩 tween quark generations.
\begin{equation}
  \begin{pmatrix}
   d' \\
   s' \\
   b'
 \end{pmatrix}
 = 
 \pmb{V} 
 \begin{pmatrix}
   d \\
   u \\
   b
 \end{pmatrix}.
\end{equation}
The CKM matrix is unitary and fully specified with 4 parameters and encodes the strength of the flavor-changing electroweak interactions.
\todo{Maybe add one more sentence describing the rotation that is necessary? see P.Sommer thesis}

\noindent The fermion fields then transform as
\begin{align}
  \psi_L & \rightarrow e^{iY\omega} e^{iT^a\omega^a} \psi_L, \qquad (a = 1, 2, 3) \\
  \psi_R & \rightarrow e^{iY\omega} \psi_R,
\end{align}
under SU(2)$_L$ $\times$ U(1)$_Y$ transformations, where $T^a=\frac{\sigma^a}{2}$ are the Pauli matrices and generators of the SU(2)$_L$ group.
The associated conserved quantity is the \emph{weak isospin} $T$, of which the third component is conserved in weak interactions and given by $T^{(3)} = \pm \frac{1}{2}$ for SU(2)$_L$ doublets and $T^{(3)} = 0$ for SU(2)$_L$ singlets.
The U(1)$_Y$ symmetry is associated to the \emph{hypercharge} $Y$, and cannot be directly associated to the QED gauge group.
The relation to the electromagnetic interaction and the physical electric charge $Q$ is
\begin{equation}
  Q = T^{(3)} + \frac{Y}{2}.
\end{equation}

\noindent The local gauge invariant electroweak Lagrangian can be written as
\begin{equation}
  \mathcal{L}_{\text{EWK}} = \sum_f i\bar{\psi}_{f}\gamma^\mu D_\mu \psi_{f} - \frac{1}{4}W_{\mu\nu}^aW^{\mu\nu}_{a} - \frac{1}{4} B_{\mu\nu}B^{\mu\nu}, 
  \label{eq:lagrangianewk}
\end{equation}
where the sum runs over all fermions $f$, including their left-handed and right-handed counterparts.
%, whose occurrence as left-handed and right-handed particles is explicitly mentioned.
The covariant derivative is defined as
\begin{equation}
  D_\mu = \partial_\mu + igT^aW_\mu^a + ig'\frac{Y}{2}B_\mu 
  \label{eq:covdevewk}
\end{equation}
and includes four gauge fields. The fields $W^a_\mu$ (with a = 1, 2, 3) are the gauge fields of SU(2)$_L$ with associated coupling $g$, and $B_\mu$ is the gauge field of U(1)$_Y$ with coupling $g'$.
The field tensors in \cref{eq:lagrangianewk} are given by
\begin{align}
  W_{\mu\nu}^a & = \partial_\mu W_\nu^a - \partial_\nu W_\mu^a - g \epsilon^{abc} W^b_\mu W^c_\nu, \label{eq:Wtensor} \\
  B_{\mu\nu}   & = \partial_\mu B_\nu - \partial_\nu B_\mu,
\end{align}
where $\epsilon^{abc}$ are the structure constants of SU(2)$_L$. The third term on the right-hand side of \cref{eq:Wtensor} arises because of the non-abelian nature of SU(2)$_L$ and gives rise to triple and quartic self-interactions of the gauge fields $W_{\mu}^a$.
The tensor $B_{\mu\nu}$ has the same structure as the electromagnetic field strength tensor obtained in QED.
\todo{Give details on structure constants? for QCD AND Electroweak model}

\noindent The Lagrangian of \cref{eq:lagrangianewk} describes 4 massless bosons. No terms quadratic in the vector fields are allowed due to the requirement of local gauge invariance. 
Another mechanism is required to explain the existence of the masses of the gauge fields \Wpm and \Zboson.
% and $\gamma$. 


% From experiments one expects two charged bosons $W^\pm$ and two neutral bosons, the $Z$ and the photon.

\subsection{Spontaneous symmetry breaking and the Higgs mechanism}
\label{subsec:ewsymbreaking}
%- Local gauge invariance does not allow adding mass terms
%- The SU(2)$_L$ $\times$ U(1)$_Y$ symmetry is spontaneously broken into U(1)$_\text{QED}$
%- This mechanism is spontaneous symmetry breaking
%- Simply put, the Lagrangian itself maintains the symmetry, but the state of lowest energy is not invariant and breaks the summetry.
% - Could add mass terms disregarding local gauge invariance but this would render theory unrenormalizable
\begin{figure}
  \newImageResizeCustom{0.75}{figures/theory/higgs-potential/higgs-potential.pdf}
  \caption[Two dimensional Higgs potential $V(\phi)$ with $\lambda > 0$ and $\mu^2 < 0$.]{Two dimensional Higgs potential $V(\phi)$ as in \cref{eq:higgspotential} with $\lambda > 0$ and $\mu^2 < 0$. The minimum occurs for different points on the sketched circle in the $(\phi_1, \phi_2)$ plane and has the value given in \cref{eq:higgsminima}.
  }
  \label{fig:higgspotential}
\end{figure}

%naturally appear in the Higgs mechanism explained below.
%An additional mechanism is needed in the SM to explain the finite masses of the weak gauge bosons. 

A principle known as \emph{spontaneous symmetry breaking}, that was first explored in the field of condensed-matter physics, can be applied to elementary particle physics to generate mass terms for the weak gauge bosons without violating local gauge invariance.
This was first formulated by three independent research teams in 1964: Brout and Englert~\cite{PhysRevLett.13.321}, Higgs~\cite{PhysRevLett.13.508,HIGGS1964132}, and Guralnik, Hagan, and Kibble \cite{PhysRevLett.13.585}.\footnote{Their work was inspired by previous advancements in the theory of superconductivity~\cite{PhysRev.108.1175} and specifically P. W. Anderson, who proposed the mechanism of spontaneous symmetry breaking for generating mass terms in a non-relativistic scenario already in 1963~\cite{PhysRev.130.439}.}
Today the mechanism is most often called \emph{Higgs mechanism} or \emph{electroweak symmetry breaking} (ESWB).
The basic principle is to allow the state of lowest energy to violate local gauge invariance while maintaining the gauge symmetry of the Lagrangian itself.

% From Peskin and Schroeder
% - In the theory of sup erconductiv􏰁 ity􏰔 for example􏰔 the Ab elian gauge invariance of electromagnetism is broken by pairs of electrons that condense in the ground state of a metal􏰎
%The mechanism is known as the \emph{Brout-Englert-Higgs mechanism} (or simply \emph{Higgs mechanism}) or also \emph{electroweak symmetry breaking} (ESWB).
%The underlying concept is that the Lagrangian itself maintains the symmetry, but the state of lowest energy is not invariant and breaks the gauge symmetry.
% was published almost simultaneously by three independent groups in 1964: by Robert Brout and François Englert;[3] by Peter Higgs;[4] and by Gerald Guralnik, C. R. Hagen, and Tom Kibble.[5][6][7]

The Higgs mechanism assumes a complex scalar field of the SU(2)$_L$ group,
\begin{equation}
  \phi = \frac{1}{\sqrt{2}} \myvec { \phi_1 + i \phi_3 \\ \phi_2 + i \phi_4},
\end{equation}
and introduces a potential of the form
\begin{equation}
  V(\phi) = \mu^2\phi^\dagger\phi + \lambda \left(\phi^\dagger\phi \right)^2.
  \label{eq:higgspotential}
\end{equation}
The Lagrangian 
\begin{equation}
  \mathcal{L}_{\text{Higgs}} = |D_\mu\phi|^2 - V(\phi), % - \frac{1}{4} W_{\mu\nu}^a W^{a\, \mu\nu} - \frac{1}{4} B_{\mu\nu} B^{\mu\nu},
  \label{eq:lagrangianhiggs}
\end{equation}
where $D_\mu$ is defined as shown in \cref{eq:covdevewk}, is invariant under local SU(2)$_L$ $\times$ U(1)$_Y$ transformations.
The parameters of the potential $V(\phi)$ are specifically chosen to satisfy $\mu^2 < 0$ and $\lambda > 0$.
This choice provides the \emph{Higgs potential} with a characteristic shape, depicted in \cref{fig:higgspotential}, and gives rise to a set of degenerate ground state configurations satisfying
\begin{equation}
  |\phi| = \sqrt{ \frac{\mu^2}{2\lambda} } \equiv \frac{ v }{\sqrt{2}},
  \label{eq:higgsminima}
\end{equation}
where $v$ is the non-zero \emph{vacuum expectation value} of the Higgs field.
%% We find that the Lagrangian for such a field
%% \begin{equation}
%% \end{equation}
%% is invariant under global SU(2) phase transformations
%It is invariant under gauge transformations but not the ground state, which can be found at
%The minimum of the potential can be found on the circle of minima where
%Once a ground state is chosen, the SU(2)$_L$ $\times$ U(1)$_Y$ symmetry becomes spontaneously broken.
% The symmetry is therefore spontaneously broken when a ground state is chosen. 
%The arbitrary choice of a ground state is said to spontaneously break the symmetry. 
Without loss of generality the ground state can be chosen to be
\begin{equation}
  \phi_0 = \frac{1}{\sqrt{2}} \myvec{0 \\ v},
  \label{eq:groundstate}
\end{equation}
that is, $\phi_1 = \phi_2 = \phi_4 = 0$ and $\phi_3 = \frac{v}{\sqrt{2}}$, which spontaneously breaks the SU(2)$_L$ $\times$ U(1)$_Y$ symmetry.
Expanding the field $\phi$ around the minimum to first order in the fields yields
\begin{equation}
  \phi(x) = e^{i\frac{\phi(x)}{v}}\frac{1}{\sqrt{2}} \myvec{ 0 \\ v + h(x) },
    \label{eq:higgsexp}
\end{equation}
where the extra term $e^{i\frac{\phi(x)}{v}}$ describes the fluctuations of the fields $\phi_1, \phi_2, \phi_4$ around the vacuum $\phi_0$.
These fields are known as the \emph{Goldstone bosons} and have no direct physical implications. They can be eliminated from the Lagrangian by choosing an appropriate gauge, the \emph{unitary gauge}, so that 
\begin{equation}
  \phi(x) = \frac{1}{\sqrt{2}} \myvec{ 0 \\ v + h(x) }.
  \label{eq:expanded-groundstate}
\end{equation}
There remains only one real physical field, $h(x)$, which is called the \emph{Higgs field} and is associated with a neutral scalar boson, the \emph{Higgs boson}, labelled $H$.
The mass of the Higgs boson follows by inserting \cref{eq:expanded-groundstate} into \cref{eq:higgs-potential},
\begin{equation}
  m_H = \sqrt{2} \mu = \sqrt{2 \lambda} v.
\end{equation}
Inserting \cref{eq:higgsexp} into \cref{eq:lagrangianhiggs} and using the linear combinations,
\begin{align}
  W_\mu^\pm &= \frac{1}{\sqrt{2}} \left( W_\mu^1 \mp iW_\mu^2 \right),  \\
  Z_\mu &= \cos\theta_\text{w} W_\mu^3 - \sin\theta_\text{w} B_\mu, \\
  A_\mu &= \cos\theta_\text{w} B_\mu - \sin\theta_\text{w} W_\mu^3,
\end{align}
where $\theta_\text{w}$ is the \emph{weak mixing angle} defined as $\sin\theta_\text{w}^2 = \frac{g'^2}{g^2+g'^2}$, leads to the following terms in the Lagrangian
\begin{equation}
  \mathcal{L}_m = \frac{1}{4} \left( v + H \right)^2  \left(g^2 W_\mu^+W^{-\,\mu} + \frac{g^2}{2\cos\theta_\text{w}^2} Z_\mu Z^\mu \right).
  \label{eq:lagrangianmasses}
\end{equation}
One can identify terms quadratic in the physical fields $\Wpm$ and $Z$, generated by the non-vanishing expectation value $v$, giving rise to the masses
\begin{align}
  m_W &= \frac{vg}{2}, \\
  m_Z &= \frac{m_W}{\cos \theta_\text{w}},
  \label{eq:boson-masses}
\end{align}
of the weak gauge bosons.
The mass of the $W$ boson can be related to the Fermi constant, $G_F$, via $m_W = \frac{g}{4 * \sqrt{2}G_F}$. 
No field quadratic in $A_\mu$ appears, which reflects the fact that the photon is massless.

% DIFFERENT Formulation: It is worth while to decompose the Lagrangian into its different pieces:
The Lagrangian in \cref{eq:lagrangianmasses} can be re-written after substituting the masses of the gauge bosons,
\begin{equation}
  \mathcal{L}_{\text{H,boson-coupling}} = m_W^2 W_\mu^+W^{-\,\mu} \left( \frac{2H}{v} + \frac{H^2}{v^2} \right) + \frac{1}{2} m_Z^2 Z_\mu Z^\mu \left( \frac{2H}{v} + \frac{H^2}{v^2} \right),
  \label{eq:higgsbosoncoupling}
\end{equation}
which shows that the interaction between the Higgs boson and the massive gauge bosons is proportional to the square of the mass of the coupled bosons and involves triplet ($V^\dagger VH$) and quartic ($V^\dagger VHH$) couplings.

%and choosing a specific gauge (the unitary gauge) the field $\phi$ can be written as
% As shown below, introducing a complex SU(2)$_L$ doublet does not only give rise to the $\Wpm$ and $Z$ boson masses, but can also be used to construct mass terms for fermions. Hence, the Higgs field couples to all massive elementary particles.
% A more detailed descriptions on experimental Higgs boson physics and an overview of the current state of knowledge is given in \cref{sec:higgsphysics}.
% The parameters of the Higgs potential $\mu^2 = \lambda v^2$ can be fixed for one combination by measuring parameters of the electroweak theory, but the physical Higgs mass cannot be predicted. 
% Moreover, the last two terms in \cref{eq:higgsselfcoupling} predict that the Higgs field couples to itself with cubic and quartic interactions.
% The couplings of the Higgs boson to the weak gauge bosons are already expressed in \cref{eq:lagrangianmasses}.

%While the Higgs boson was discovered experimentally in 2012 \cite{Aad:2012tfa,Chatrchyan:2012xdj}, and is by now seen in several different production and decay modes, the Higgs boson self-couplings are still searched for.


% In summary, adding a complex scalar Higgs field to the electroweak model, together with a potential that exhibits a non-zero vacuum expectation value, provides the ingredients to obtain mass terms for the weak gauge bosons $W^\pm$ and $Z$ when choosing a ground state of the Higgs field which spontaneously breaks the symmetry.
% The three Goldstone bosons that arise can be eliminated from the Lagrangian by using its underlying local gauge symmetry. 
% This results in three bosons acquiring a mass and the appearance of one remaining scalar field $H$. 
% The photon remains massless, because the U(1)$_{\text{QED}}$ symmetry is unbroken. 


\subsection{Fermion masses}
\label{subsec:fermion-masses}
\todo{Maybe add in the beginning that simple inclusion of fermion mass terms is forbidden in electroweak model. As commented out at the end of the EW section.}

The previous section explained how the Higgs mechanism naturally gives rise to mass terms for the $\Wpm$ and $Z$ bosons. The finite fermion masses, however, do not directly follow and the simple inclusion of fermion mass terms is forbidden, given that a term
\begin{equation}
  -m_f \bar{\psi} \psi = -m_f \left( \bar{\psi}_R\psi_L + \bar{\psi}_L\psi_R \right)
  \label{eq:fermionmassterm}
\end{equation}
violates SU(2)$_L$ gauge invariance.
% because $\psi_L$ transforms as a doublet and $\psi_R$ as a singlet. 

The masses of the fermions can be explained with an ad-hoc solution by introducing interaction terms between the left-handed fermion fields and the Higgs field, both appearing as doublets under SU(2)$_L$ transformations.
%This is possible because both the left-handed fermions and the Higgs field appear as doublets under 
For leptons, only electrons, muons, taus, that appear in the lower part of the SU(2)$_L$ doublet (see \cref{tab:ewfermioncontent}), require mass terms, as neutrinos are assumed to be massless in the SM\footnote{See \cref{subsec:limitations} for a brief discussion on neutrino masses.}.
For up-type quarks ($u$, $s$, and $t$ quark) to become massive, the fermion fields are coupled to the charge conjugate of the Higgs field,
\begin{equation}
  \phi^C = \frac{1}{\sqrt{2}} \myvec{v + h(x) \\ 0},
\end{equation}
after choosing a ground state similar to \cref{eq:expanded-groundstate}.
The interactions are known as \emph{Yukawa interactions} and can then be written in the form
\begin{equation}
  \mathcal{L}_\text{Yukawa} = - \sum_{i} Y_l^i \bar{\psi}^{i}_{L} \phi \psi^{i}_{R} - \sum_{ij} Y_{\text{u-type}}^{ij} \bar{\psi}^{i}_{L} \phi \psi^{i}_{\text{u-type},R} + Y_{\text{d-type}}^{ij} \bar{\psi}^{i}_{L} \phi^C \psi^{j}_{\text{d-type}, R} + \text{h.c.}, 
  \label{eq:lyukawa}
\end{equation}
where the first sum runs over all leptons, the second sum over all quarks, and h.c. stands for the Hermitian conjugate of the previous terms.
The left-handed fields are given as $\psi^{i}_{L}$ and include the three lepton- and quark doublets. 
The right-handed lepton fields are labelled as $\psi_R^i$ and the quark fields as $\psi^{i}_{\text{u-type},L}$, $\psi^{i}_{\text{d-type},L}$.
The labels ``u-type'' and ``d-type'' refer, respectively, to the up-type quarks ($u$, $s$, $t$) and down-type quarks ($d$, $c$, $b$). 
The \emph{Yukawa couplings} are labelled as $Y_l^i$ for leptons and as $Y_{\text{u-type}}^{ij}$ and $Y_{\text{d-type}}^{ij}$ for up-type and down-type quarks, respectively. The $Y_{\text{u-type}}^{ij}$ and $Y_{\text{d-type}}^{ij}$ are $3 \times 3$ matrices accounting for the fact that the eigenstates of the weakly interacting quarks do not correspond to their mass eigenstates. 
%A rotation of the quark fields can be performed accounting for the mixing. 
% Mass terms for quarks can be added in a similar but slightly more involved way. The down-type quarks participating in the electroweak interaction (denoted $d'$, $s'$, $b'$) are a mixture of the quark mass eigenstates (denoted $d$, $s$, $b$). 
% For each quark doublet there are two mass terms generated. For lepton doublets only one mass term is added because right-handed neutrinos are not part of the SM. Assuming only one generation of fermions, with quarks $u$ and $d$ and leptons $e$ and $\nu_e$, the Yukawa term can therefore be written as
% \begin{equation}
%   \mathcal{L}_\text{Yukawa} = - Y_{ud} \bar{\psi}_{ud,L} \phi \psi_{u,R} - Y_{ud} \bar{\psi}_{ud,L} \phi \psi_{d,R} - Y_l \bar{\psi}_{e\nu_e,L} \phi \psi_{e,R} + \text{h.c.},
%   \label{eq:lyukawa}
% \end{equation}
% where $\psi_{l}$ stands for lepton fields and $\psi_{q}$ for the quark fields.
The mass terms appear from \cref{eq:lyukawa} once a ground state of $\phi$ is chosen and the Yukawa terms have been diagonalized, resulting in nine independent parameters $Y_f$. 
They take the form
\begin{equation}
  m_f = \frac{Y_f v}{\sqrt{2}},
\end{equation}
which means that the coupling strength between the fermions and the Higgs field is proportional to the mass of the fermions.

%\subsection{Renormalization}
% From modern particle physics book
% As shown by ‘t Hooft, only theories with local gauge invariance are renormalisable, such that the cancellation of all infinities takes place among only a finite number of interaction terms.

\subsection{The Final Standard Model Lagrangian}
\label{subsec:final-lagrangian}
To summarize the previous sections, the final Standard Model Lagrangian is obtained from \cref{eq:lqcd,eq:lagrangianewk,eq:lagrangianhiggs,eq:lyukawa} and can be written as
\begin{align}
  \mathcal{L}_\text{SM} &= \mathcal{L}_\text{QCD} + \mathcal{L}_\text{EWK} + \mathcal{L}_\text{Yukawa} + \mathcal{L}_\text{Higgs} \\
   &= - \frac{1}{4}W_{\mu\nu}^aW^{\mu\nu}_{a} - \frac{1}{4} B_{\mu\nu}B^{\mu\nu} - \frac{1}{4}G_{\mu\nu}^aG^{\mu\nu}_{a} \\
   \label{eq:dirac-term}
   & \quad + \sum_f i \bar{\psi}_f\gamma^\mu D_\mu\psi_f \\
   & \quad - \sum_{i} Y_l^i \bar{\psi}^{i}_{L} \phi \psi^{i}_{R} - \sum_{ij} Y_{\text{u-type}}^{ij} \bar{\psi}^{i}_{L} \phi \psi^{i}_{\text{u-type},R} + Y_{\text{d-type}}^{ij} \bar{\psi}^{i}_{L} \phi^C \psi^{j}_{\text{d-type}, R} + \text{h.c.},  \\
   & \quad + |D_\mu\phi|^2 - \mu^2\phi^\dagger\phi + \lambda \left(\phi^\dagger\phi \right)^2
\end{align}
The covariant derivate includes all gauge fields,
\begin{equation}
  D_\mu = \partial_\mu + i g_s \frac{\lambda^a}{2} G_\mu^a + igT^aW_\mu^a + ig'\frac{Y}{2}B_\mu.
\end{equation}
The sum over $f$ in \cref{eq:dirac-term} includes all fermions, left-handed and right-handed.
\todo{Find one or two more sentences}
% It satisfies a SU(3)$_C$ $\times$ SU(2)$_\text{L}$ $\times$ U(1)$_Y$ gauge symmetry and provides masses to the gauge bosons and fermions via the concept of electroweak symmetry breaking. This is manifested in the couplings of the Higgs field to fermions, which are proportional to the mass of the fermions, and the couplings to the gauge bosons, which are proportional to the mass squared of the gauge bosons.


\subsection{Free parameters of the SM}
\label{subsec:free-pars-sm}
The SM has been very successful in making predictions that were later confirmed by experimental data (see \cref{eq:sm-limitations}). However, the SM has many free parameters that need to be measured experimentally and cannot be derived from theoretical principles.  
In total there are 19 free parameters that can be represented in different ways. The most common ones are summarized below:
\begin{itemize}
  \item 9 fermion masses (or Yukawa couplings)
  \item 2 parameters describing the Higgs field: $\mu$ and $\lambda$ (or $v$ and $m_H$)
  \item 4 parameters to fully specify the CKM matrix, typically parametrized as 3 quark-mixing angles ($\theta_1$, $\theta_2$, $\theta_3$) and a CP-violating phase ($\theta_\delta$).
  \item 3 couplings constants: $\alpha$, $G_f$, $\alpha_s$ (or $g'$, $g_w$, $g_s$)
  \item 1 phase associated to CP violating terms in QCD, $\theta_{\text{QCD}}$
\end{itemize}
In total 14 parameters are associated with the Higgs field, four with the flavor sector, and only three with the gauge interactions. This underlines the special role the Higgs boson plays in the SM.
\todo{Find one or two more sentences}

\subsection{Limitations of the SM}
\label{subsec:limitations}
Despite the extraordinary success of accurately predicting and explaining an enormous amount experimental measurements, the SM has several shortcomings.
\todo{the SM success is also mentioned elsewhere. Align!}
There are several observable phenomona that cannot be explained by the SM as well as many open questions related to theory. 
There are models and theories beyond the SM that attempt to address these shortcomings, but experimental indication that either of them is realised in nature is still pending.
%none of them have made predictions that have been confirmed by experiments to date. 
Notable examples of such theories are supersymmetric theories (see e.g. \ccite{Golfand:1971iw,Volkov:1973ix,Wess:1974tw,Salam:1974ig,Wess:1974jb,Ferrara:1974pu,Fayet:1976et}), dark matter models (e.g. \ccite{Baumgart_2009,Kaplan_2009,Dienes_2012}), or more simple extensions of the SM such as the \emph{two-Higgs-doublet models} \cite{Branco_2012}.
This section provides an overview of some of the biggest unsolved problems of the SM and fundamental physics in general.

% HIGGS and Dark Matter
%https://home.cern/news/news/physics/atlas-probes-dark-matter-using-higgs-boson

% Maybe also read introduction of Deciphering the nature of the Higgs boson!

% From Arnold:
% While several such theories have been developed that based on differing concepts attempt to improve on the shortcomings of the Standard Model, e.g. Supersymmetry or String Theory, experimental indications that either of them is actually realised in Nature are still pending.

%%%%%%%%%%%%%%%%%%%%%%%%%%%%%%%%%%%%%%%%%%%%%%%%%%%%%%%%%%%%%%
% Experimental ones (not sure yet if I should make this distinction)
\subsubsection{Higgs boson}
While several physics problems were solved by introducing the Higgs boson to the SM, other mysteries remain. 
One of the biggest open questions is related to the mass of the Higgs boson itself. 
Assuming that there is no new physics up to the Planck scale, quantum loop corrections to the Higgs boson mass would be extremely large (at the order of the Planck scale). 
This means, that the bare mass of the Higgs boson\footnote{The observable mass of an elementary particle is given by the difference of two terms that are fully independent of each other, the particles bare mass and corrections to it from quantum fluctuations.}
needs to almost perfectly balance the quantum corrections to keep the observable Higgs mass in the \GeV\ range. The precision required for such a balance is extremely high and considered ``unnatural'', meaning that it is highly unlikely that the Higgs mass is so low due to chance alone. 
This is known as the hierarchy problem and is one of the main motivations for the search for new particles with masses in the \TeV\ range at the LHC. Their existence would affect the quantum contributions to the Higgs mass, eliminating the need of ``fine-tuning'' the bare Higgs mass to an incredible precision, as is assumed in the SM. 
\todo{Maybe mention compositenes or 2HDMs?}

\subsubsection{Large number of free parameters}
As discussed in \cref{subsec:free-pars-sm}, the SM has several free parameters that need to be added to the theory ad hoc and cannot be derived from first principles. 
From an experimentalist point of view, this is not a problem per se, but it suggests that there might be a more fundamental theory with fewer parameters that can ultimately explain the relationships between some of the current SM parameters.

% Thomson: 
% [The SM] ... is a model constructed from a number of beauti- ful and profound theoretical ideas put together in a somewhat ad hoc fashion in order to reproduce the experimental data.

\subsubsection{Strong CP problem}
No theoretical principle forbids CP violation in the strong interaction. However, no such process has been found and the strong CP phase, $\theta_{\text{QCD}}$, has been experimentally constraint to be smaller than $10^{-10}$ (inferred from neutron electric dipole moment measurements \cite{Baker_2006}). This is considered another ``fine-tuning'' problem.
% Not sure if I should add that
The so-called Peccei-Quinn theory~\cite{PhysRevLett.38.1440} and modifications of it provide a solution by introducing a scalar field known as axion. However, no experimental evidence of the existence of such a particle has been found to date.
%there is no explanation of why these interactions are so small which makes
% - There is nothing that would forbid CP violation in the strong interaction, however, no CP-violating processes have been observed to date. 
% This is considered a ``fine-tuning'' problem.

\subsubsection{Neutrinos}
Right-handed neutrinos are not part of the SM because they do not couple to any of the other SM particles.
%Neutrino Yukawa interactions (see \cref{subsec:fermion-masses}) are therefore typically not added to the SM. 
However, it is known that neutrinos oscillate \cite{Gonzalez_Garcia_2008}, implying that they have finite masses.
While it is possible to add mass terms from Yukawa interactions (see \cref{subsec:fermion-masses}) for neutrinos ad hoc, without violating any of the theoretical principles, it is not clear exactly which terms to add, as the nature of the neutrinos is not settled\footnote{It is still an open question whether neutrinos are Dirac fermions or Majorana fermions. Majorana fermions are particles that are their own anti-particles, which stands in contrast to Dirac fermions. Different mathematical descriptions are required to describe the different types of particles in the Lagrangian.}. 
Nonetheless, the SM is sometimes described including additional mass terms for the neutrinos, which adds additional seven free parameters (three for the masses and four to describe the mixing between the neutrinos).

%It is possible to add Dirac mass terms by implying the existence of right-handed neutrinos. 
%- One possibility is to add Dirac mass terms, which would add an additional 6 free parameters to the SM. 
%- However, it is not clear if Neutrinos are indeed sterile (Dirac) and unknown why Neutrino masses are so small and if they get their masses from the same process as the other SM particles
% Wikipedia on Majorana:
% A Majorana fermion (/maɪəˈrɑːnə ˈfɛərmiːɒn/[1]), also referred to as a Majorana particle, is a fermion that is its own antiparticle. They were hypothesised by Ettore Majorana in 1937. The term is sometimes used in opposition to a Dirac fermion, which describes fermions that are not their own antiparticles.
% The nature of the neutrinos is not settled—they may turn out to be either Dirac or Majorana fermions.
% From Pich 2012
%The experimental evidence of neutrino oscillations shows that νe, νμ and ντ are also mixtures of mass eigenstates. However, the neu

%%%%%%%%%%%%%%%%%%%%%%%%%%%%%%%%%%%%%%%%%%%%%%%%%%%%%%%%%%%%%%
% Theoretical ones (not sure yet if I should make this distinction)

\subsubsection{Matter-antimatter symmetry}
It is widely believed that the universe was created with equal amounts of matter and antimatter.
However, the currently observable universe consists of significantly more matter than antimatter. 
The SM does not provide a theoretical explanation for this large asymmetry. 

% the SM cannot explain why there is more matter than antimatter in the observable universe. 
% - No mechanism in the SM can explain the asymmetry sufficiently. 
% From Thomson
% However, even if CP violation is observed in neutrino oscillations, it seems quite possible that the CP violation in the Standard Model is insufficient to explain the observed matter–antimatter asymmetry of the Universe.

\subsubsection{Dark matter}
The SM is able to explain the behavior of the currently observable matter in the universe, which makes up about 5\% of the total energy of the universe. 
As known from astrophysics experiments such as measurements of galaxy rotations (see e.g. \ccite{Corbelli_2000}), another 27\% of the universe's energy is made up of another type of matter dubbed \emph{dark matter}. To date, the SM has no viable candidate to explain the nature of dark matter. 

% Another 26\% of the universe's energy is made up of so-called dark matter. 
% - The SM explains the observable mass which makes up about 5\% of the energy of the universe. 
% - The rest of the matter, making up about 26\% is known as dark matter. (known from measurements of orbiting galaxis)
% - The SM cannot explain the nature of dark matter, to date. 
% from Thomson
%velocity distributions of stars as they orbit the galactic centre)

\subsubsection{Dark energy}
About 68\% of the universe's energy constitutes so-called \emph{dark energy}, introduced to explain the acceleration of the expansion of the universe \cite{Riess_1998,Perlmutter_1999}. Dark energy is attributed to the non-zero cosmological constant that appears in Einstein's equation of general relativity and is also sometimes referred to as \emph{vacuum energy}. The SM provides no suitable explanation for the large value of the vacuum energy.
\todo{Reference for GR see gaddaths thesis}

%The SM has no explanation for the large value of the vacuum energy. 
% - Dark energy is the energy needed in order to explain the acceleration of the expansion of the universe. 
% - This is sometimes called vacuum energy and related to the cosmological constant problem. The SM has no explanation for the measured size of the vacuum energy.
% From Thomson
%- dark energy is attributed to a non-zero cosmological constant of Einstein's equations of general relativity

\subsubsection{Gravitation}
The current understanding of gravity is still shaped by Einstein's theory of general relativity. No quantum theory of gravity has been fully developed, which is why it is not part of the SM. 
The interaction strength of gravity is very weak compared to all other fundamental forces, which is why it can be safely neglected in current high energy physics experiment.
% Due to the small strength of gravity, compared to the other fundamental forces, gravity can be fully neglected in current high energy physics experiments. 
% - Quantum theory of gravity not fully developed. Versions that exist predict the graviton, the particle mediating gravity to be a spin 2 particle.
% - Classical theory of gravity, general relativity


% Not sure if I should include this it's kind of weird! I don't understand it, let's say that. Why is it a problem, can't you ALWAYS ask that question? Why is there this particle. Why are there electrons? Not sure if I get that quite right. 
% \subsubsection{The fermion generations}
% The SM has no explanation for why there are three generations of fermions. 



% - Due to these inexplicabilities the Higgs boson (and the fact that it's the least well measured EM particle) is widely considered to be the answer to many other questions in fundamental physics

% Hierarchy Problem from Thomson:
% Just as quantum loop corrections contributed to the W-boson mass (see Section 16.4), quantum loops in the Higgs boson propaga- tor, such as those indicated in Figure 18.3, contribute to the Higgs boson mass. This in itself is not a problem. However, if the Standard Model is part of theory that is valid up to very high mass scales, such as that of a Grand Unified Theory ΛGUT ∼ 1016 GeV or the Planck scale ΛP ∼ 1019 GeV, these corrections become very large. Because of these quantum corrections, which are quadratic in Λ, it is difficult to keep the Higgs mass at the electroweak scale of 102 GeV. This is known as the Hierarchy problem. It can be solved by fine-tuning the new contributions to the Higgs mass such that they tend to cancel to a high degree of precision.

% Why Higgs measurements?
% The experimental study of the Higgs boson at the LHC is undoubtedly one of the most exciting areas in contemporary particle physics. Within the Standard Model, the Higgs boson is unique; it is the only fundamental scalar in the theory. Establish- ing the properties of the Higgs boson such as its spin, parity and branching ratios is essential to understand whether the observed particle is the Standard Model Higgs boson or something more exotic.


\section{Limitations of the SM and Open Questions}
\label{sec:limitations}
Despite the extraordinary success of accurately predicting and explaining an enormous amount experimental measurements, the SM has several shortcomings.
There are several observable phenomona that cannot be explained by the SM as well as several open questions related to theory. 
There are models and theories beyond the SM that attempt to address these shortcomings, but experimental indication that either of them is realised in nature is still pending.
%none of them have made predictions that have been confirmed by experiments to date. 
Notable examples of such theories are supersymmetric theories (see e.g. \ccite{Golfand:1971iw,Volkov:1973ix,Wess:1974tw,Salam:1974ig,Wess:1974jb,Ferrara:1974pu,Fayet:1976et}), dark matter models (e.g. \ccite{Baumgart_2009,Kaplan_2009,Dienes_2012}), or more simple extensions of the SM such as the \emph{two-Higgs-doublet models} \cite{Branco_2012}.
This section provides an overview of some of the biggest unsolved problems of the SM and fundamental physics in general.

% HIGGS and Dark Matter
%https://home.cern/news/news/physics/atlas-probes-dark-matter-using-higgs-boson

% Maybe also read introduction of Deciphering the nature of the Higgs boson!

% From Arnold:
% While several such theories have been developed that based on differing concepts attempt to improve on the shortcomings of the Standard Model, e.g. Supersymmetry or String Theory, experimental indications that either of them is actually realised in Nature are still pending.

%%%%%%%%%%%%%%%%%%%%%%%%%%%%%%%%%%%%%%%%%%%%%%%%%%%%%%%%%%%%%%
% Experimental ones (not sure yet if I should make this distinction)
\subsubsection{Higgs boson}
While several physics problems were solved by introducing the Higgs boson to the SM, other mysteries remain. 
One of the biggest open questions is related to the mass of the Higgs boson itself. 
Assuming that there is no new physics up to the Planck scale, quantum loop corrections to the Higgs boson mass would be extremely large (at the order of the Planck scale). 
This means, that the bare mass of the Higgs boson\footnote{The observable mass of an elementary particle is given by the difference of two terms that are fully independent of each other, the particles bare mass and corrections to it from quantum fluctuations.}
needs to almost perfectly balance the quantum corrections to keep the observable Higgs mass in the \GeV\ range. The precision required for such a balance is extremely high and considered ``unnatural'', meaning that it is highly unlikely that the Higgs mass is so low due to chance alone. 
This is known as the hierarchy problem and is one of the main motivations for the search for new particles with masses in the \TeV\ range at the LHC. Their existence would affect the quantum contributions to the Higgs mass, eliminating the need of ``fine-tuning'' the bare Higgs mass to an incredible precision, as is assumed in the SM. 
\todo{Maybe mention compositenes or 2HDMs?}

\subsubsection{Large number of free parameters}
As discussed in \cref{subsec:final-lagrangian}, the SM has several free parameters that need to be added to the theory ad hoc and cannot be derived from first principles. 
From an experimentalist point of view, this is not a problem per se, but it suggests that there might be a more fundamental theory with fewer parameters that can ultimately explain the relationships between some of the current SM parameters.

% Thomson: 
% [The SM] ... is a model constructed from a number of beauti- ful and profound theoretical ideas put together in a somewhat ad hoc fashion in order to reproduce the experimental data.

\subsubsection{Strong CP problem}
No theoretical principle forbids CP violation in the strong interaction. However, no such process has been found and the strong CP phase, $\theta_{\text{QCD}}$, has been experimentally constraint to be smaller than $10^{-10}$ (inferred from neutron electric dipole moment measurements \cite{Baker_2006}). This is considered another ``fine-tuning'' problem.
% Not sure if I should add that
The so-called Peccei-Quinn theory~\cite{PhysRevLett.38.1440} and modifications of it provide a solution by introducing a scalar field known as axion. However, no experimental evidence of the existence of such a particle has been found to date.
%there is no explanation of why these interactions are so small which makes
% - There is nothing that would forbid CP violation in the strong interaction, however, no CP-violating processes have been observed to date. 
% This is considered a ``fine-tuning'' problem.

\subsubsection{Neutrinos}
Right-handed neutrinos are not part of the SM because they do not couple to any of the other SM particles.
%Neutrino Yukawa interactions (see \cref{subsec:fermion-masses}) are therefore typically not added to the SM. 
However, it is known that neutrinos oscillate \cite{Gonzalez_Garcia_2008}, implying that they have finite masses.
While it is possible to add mass terms from Yukawa interactions (see \cref{subsec:fermion-masses}) for neutrinos ad hoc, without violating any of the theoretical principles, it is not clear exactly which terms to add, as the nature of the neutrinos is not settled\footnote{It is still an open question whether neutrinos are Dirac fermions or Majorana fermions. Majorana fermions are particles that are their own anti-particles, which stands in contrast to Dirac fermions. Different mathematical descriptions are required to describe the different types of particles in the Lagrangian.}. 
Nonetheless, the SM is sometimes described including additional mass terms for the neutrinos, which adds additional seven free parameters (three for the masses and four to describe the mixing between the neutrinos).

%It is possible to add Dirac mass terms by implying the existence of right-handed neutrinos. 
%- One possibility is to add Dirac mass terms, which would add an additional 6 free parameters to the SM. 
%- However, it is not clear if Neutrinos are indeed sterile (Dirac) and unknown why Neutrino masses are so small and if they get their masses from the same process as the other SM particles
% Wikipedia on Majorana:
% A Majorana fermion (/maɪəˈrɑːnə ˈfɛərmiːɒn/[1]), also referred to as a Majorana particle, is a fermion that is its own antiparticle. They were hypothesised by Ettore Majorana in 1937. The term is sometimes used in opposition to a Dirac fermion, which describes fermions that are not their own antiparticles.
% The nature of the neutrinos is not settled—they may turn out to be either Dirac or Majorana fermions.
% From Pich 2012
%The experimental evidence of neutrino oscillations shows that νe, νμ and ντ are also mixtures of mass eigenstates. However, the neu

%%%%%%%%%%%%%%%%%%%%%%%%%%%%%%%%%%%%%%%%%%%%%%%%%%%%%%%%%%%%%%
% Theoretical ones (not sure yet if I should make this distinction)

\subsubsection{Matter-antimatter symmetry}
It is widely believed that the universe was created with equal amounts of matter and antimatter.
However, the currently observable universe consists of significantly more matter than antimatter. 
The SM does not provide a theoretical explanation for this large asymmetry. 

% the SM cannot explain why there is more matter than antimatter in the observable universe. 
% - No mechanism in the SM can explain the asymmetry sufficiently. 
% From Thomson
% However, even if CP violation is observed in neutrino oscillations, it seems quite possible that the CP violation in the Standard Model is insufficient to explain the observed matter–antimatter asymmetry of the Universe.

\subsubsection{Dark matter}
The SM is able to explain the behavior of the currently observable matter in the universe, which makes up about 5\% of the total energy of the universe. 
As known from astrophysics experiments such as measurements of galaxy rotations (see e.g. \ccite{Corbelli_2000}), another 27\% of the universe's energy is made up of another type of matter dubbed \emph{dark matter}. To date, the SM has no viable candidate to explain the nature of dark matter. 

% Another 26\% of the universe's energy is made up of so-called dark matter. 
% - The SM explains the observable mass which makes up about 5\% of the energy of the universe. 
% - The rest of the matter, making up about 26\% is known as dark matter. (known from measurements of orbiting galaxis)
% - The SM cannot explain the nature of dark matter, to date. 
% from Thomson
%velocity distributions of stars as they orbit the galactic centre)

\subsubsection{Dark energy}
About 68\% of the universe's energy constitutes so-called \emph{dark energy}, introduced to explain the acceleration of the expansion of the universe \cite{Riess_1998,Perlmutter_1999}. Dark energy is attributed to the non-zero cosmological constant that appears in Einstein's equation of general relativity and is also sometimes referred to as \emph{vacuum energy}. The SM provides no suitable explanation for the large value of the vacuum energy.
\todo{Reference for GR see gaddaths thesis}

%The SM has no explanation for the large value of the vacuum energy. 
% - Dark energy is the energy needed in order to explain the acceleration of the expansion of the universe. 
% - This is sometimes called vacuum energy and related to the cosmological constant problem. The SM has no explanation for the measured size of the vacuum energy.
% From Thomson
%- dark energy is attributed to a non-zero cosmological constant of Einstein's equations of general relativity

\subsubsection{Gravitation}
The current understanding of gravity is still shaped by Einstein's theory of general relativity. No quantum theory of gravity has been fully developed, which is why it is not part of the SM. 
The interaction strength of gravity is very weak compared to all other fundamental forces, which is why it can be safely neglected in current high energy physics experiment.
% Due to the small strength of gravity, compared to the other fundamental forces, gravity can be fully neglected in current high energy physics experiments. 
% - Quantum theory of gravity not fully developed. Versions that exist predict the graviton, the particle mediating gravity to be a spin 2 particle.
% - Classical theory of gravity, general relativity


% Not sure if I should include this it's kind of weird! I don't understand it, let's say that. Why is it a problem, can't you ALWAYS ask that question? Why is there this particle. Why are there electrons? Not sure if I get that quite right. 
% \subsubsection{The fermion generations}
% The SM has no explanation for why there are three generations of fermions. 



% - Due to these inexplicabilities the Higgs boson (and the fact that it's the least well measured EM particle) is widely considered to be the answer to many other questions in fundamental physics

% Hierarchy Problem from Thomson:
% Just as quantum loop corrections contributed to the W-boson mass (see Section 16.4), quantum loops in the Higgs boson propaga- tor, such as those indicated in Figure 18.3, contribute to the Higgs boson mass. This in itself is not a problem. However, if the Standard Model is part of theory that is valid up to very high mass scales, such as that of a Grand Unified Theory ΛGUT ∼ 1016 GeV or the Planck scale ΛP ∼ 1019 GeV, these corrections become very large. Because of these quantum corrections, which are quadratic in Λ, it is difficult to keep the Higgs mass at the electroweak scale of 102 GeV. This is known as the Hierarchy problem. It can be solved by fine-tuning the new contributions to the Higgs mass such that they tend to cancel to a high degree of precision.

% Why Higgs measurements?
% The experimental study of the Higgs boson at the LHC is undoubtedly one of the most exciting areas in contemporary particle physics. Within the Standard Model, the Higgs boson is unique; it is the only fundamental scalar in the theory. Establish- ing the properties of the Higgs boson such as its spin, parity and branching fractions is essential to understand whether the observed particle is the Standard Model Higgs boson or something more exotic.

% Alternative title
%\section{The Anatomy of Proton-Proton Collision Events}
\section{The Anatomy of Proton-Proton Collision Events}
\label{sec:anatomy}
The theoretical framework described in \cref{sec:sm} allows making predictions about the interactions between the elementary particles. 
These predictions can be tested by studying the outcomes of particle collisions produced by accelerators such as the LHC. 
The LHC accelerates proton beams, which allows reaching very high center-of-mass energies for the collision events. 
The collision events have a complex phenomenology, since protons are not elementary particles.
% The phenomenology of proton-proton collisions ($pp$ collisions) is complex because protons are not elementary particles. 
Protons are compound states of so-called \emph{partons} made up of quarks and gluons that interact with each other according to the laws of QCD.
A correct understanding of QCD processes is therefore essential for the success of physics programs at $pp$ collider experiments.

In this section, the relevant aspects are described to move from QCD calculations to observables that can be measured in collider experiments such as the ATLAS detector.
\Cref{subsec:pp-collision-overview} provides a brief overview of the anatomy of a $pp$ collision event and introduces the key terminology. \Cref{subsec:collision-rates} summarizes basic observables, and \cref{subsec:factorisation} discusses the concepts necessary to compute theoretical particle interaction cross sections.
The theoretical predictions for the observables are quantified using simulated $pp$ collision events to compare them with the experimentally measured data. Details of the ATLAS simulation infrastructure are given in \cref{subsec:event-simulation}. 
% The $pp$ collisions are quantum mechanical by nature and therefore intrinsically probabilistic.
% Therefore, simulated $pp$ collision events generated with the \emph{Monte Carlo} (MC) method are used to compare the data against what is expected given the current knowledge of particle physics.


% The SM expectation of the observables is quantified using simluated $pp$ collision events 
% Later in this section, the procedure to simulate $pp$ collision events is described, 
% including the relevant details of QCD
% This section provides an overview of how to get from QCD calculations to physical observables that can be measured at collider experiments such as the ATLAS detector.

% The necessary computations are complex and affected by many aspects, and are continuously being improved by the theoretical particle physics community.
% This section provides an overview of the physical observables that are defined at hadron colliders, and gives details on the characteristics that need to be considered when simulating the collision events. 

% What is that sentence???
% As we will see we can separate the process of calculating cross sections between quarks and other soft QCD things.

\subsection{Overview of a proton-proton collision event}
\label{subsec:pp-collision-overview}
%A schematic overview of a \emph{hard} $pp$ scattering process is illustrated in \cref{fig:ppcol}.
As illustrated in \cref{fig:ppcol}, a hard $pp$ scattering process is characterized by a parton-parton interaction with large momentum transfer, known as a \emph{hard scatter}, and further low energy processes.
The partons participating in the hard scatter carry a momentum fraction of the proton described by \emph{parton distribution functions} (PDFs).
The hard scatter is accompanied by the so-called \emph{underlying event}, which consists of low energy interactions between the proton remnants that are not taking part in the hard scatter.
The partons in the initial and final states can radiate in the form of \emph{inital-state radiation} (ISR) and \emph{final-state radiation} (FSR), and a single parton can also split into two. Together this is referred to as \emph{parton showering}.
The final-state partons then undergo a process known as \emph{fragmentation} (also called \emph{hadronization}), due to the confining nature of QCD. They end up forming a spray of color-neutral hadrons, which are known as particle \emph{jets}. Many more details on the above-mentioned aspects can be found in standard textbooks on collider physics such as \ccite{Ellis:1991qj}. 

\begin{figure}[h]
  \newImageResizeCustom{0.8}{figures/anatomy/schematic_ppcollision.png}
  \caption[Schematic view of a proton-proton collision.]{Schematic view of a proton-proton collision. Details can be found in the text. Taken from Ref.~\cite{Bhatti2010}. }
  \label{fig:ppcol}
\end{figure}


\subsection{Collision rates}
\label{subsec:collision-rates}
%The whole physics program at collider experiments can more or less be broken down into
The main task of collider experiments is the measurement of event rates. 
%Comparing the event rates measured in data with the predictions from simulations allow for meaningful statistical inferences.
For a given process $p$, the event rate can be written as
\begin{equation}
  \frac{\mathrm{d}N}{\mathrm{d}t} = \sigma_p \mathcal{L},
\end{equation}
where $\sigma_p$ is the cross section for process $p$, and $\mathcal{L}$ the luminosity.
The cross section expresses the likelihood of any particular interaction to occur and can be written as
% PEskin/Schroeder
%The likelihood of any particular final state can be expressed in terms of the cross section, a quantity that is intrinsic to the colliding particles and therefore% allows comparison of two dierent experiments with dierent b eam sizes and intensities
\begin{equation}
  \label{eq:xsec}
  \sigma = \frac{|M|^2}{F} \int \text{d}Q,
\end{equation}
where $M$ is the so-called matrix element, d$Q$ the Lorentz-invariant phase-space factor, and $F$ the Lorentz-invariant flux factor \cite{Halzen:1984mc}. 
The matrix element contains all the dynamical information of the process under investigation, and the square of it represents the probability of going from a certain initial state to a final state. 
The luminosity, $\mathcal{L}$, is a measure of the number of particles per unit area and time and is a crucial performance parameter for particle colliders. It only depends on parameters of the accelerator and is given by
%is the proportionality constant that multiplies the cross section to determine the \emph{event rate} and 
\begin{equation}
  \mathcal{L} = f_rn_b\frac{N_A N_B}{A},
\end{equation}
where $N_A$ and $N_B$ are the numbers of particles in the colliding bunches $A$ and $B$ respectively, $f_r$ is the rotational frequency of the two bunches, $n_b$ the total number of bunches inside the accelerator, and $A$ the area of interaction. Assuming Gaussian shaped beam profiles the area of interaction can be written as $A = 4\pi \sigma_x \sigma_y$, where $\sigma_{x/y}$ are the horizontal and vertical beam widths.
The \emph{integrated luminosity}, 
\begin{equation}
  L_\text{int} = \int \mathcal{L} dt,
\end{equation}
is typically quoted to quantify the size of a dataset collected at a collider experiment.


\subsection{Partonic cross sections and parton distribution functions}
\label{subsec:factorisation}

\begin{figure}
  \newImageResizeCustom{0.60}{figures/anatomy/alphas.png}
  \caption[Summary of measurements of the strong coupling constant $\alpha_s$.]{Summary of measurements of the strong coupling constant $\alpha_s$ as a function of the energy scale $Q$. Taken from \ccite{PDG2020}. }
  \label{fig:alphas}
\end{figure}


% The incoming partons interact according to the laws of QCD.
An important feature of QCD is the energy dependence of the strong coupling constant, $\alpha_s$, as shown in \cref{fig:alphas}.\footnote{The $\alpha_s$ constant is also sometimes referred to as ``running'' coupling constant as it is, in fact, not a constant.} Especially the behavior at low energies where $\alpha_s$ diverges has important consequences, as it requires to separate the treatment of physics at low energies and high energies. 
At high energies, perturbative calculations of the matrix elements can be used because $\alpha_s$ becomes small. This behavior is also known as \emph{asymptotic freedom} and implies that strongly interacting particles behave like free particles at high energies. 
Conversely, the physics at low energies (known as \emph{soft QCD}) can only be described by non-perturbative models because the particles have a strong coupling. This leads to confinement of quarks inside hadrons.

%\footnote{The fact that $\alpha_s$ becomes smaller at high energies is also known as \emph{asymptotic freedom} and implies that strongly interacting particles behave like free particles at high energies. Conversely, at small energies, the particles have a strong coupling. This characteristic of QCD leads to \emph{confinement} of quarks inside hadrons.}

Given these considerations, as well as the fact that protons are compound states of partons, the cross section of a hard-scattering process between two protons, $p_1$ and $p_2$, can be written as~\cite{Ellis:1991qj}
\begin{equation}
  \sigma(p_1p_2 \to Y) = \sum_{i,j} \int_0^1 \mathrm{d}x_1 \int_0^1 \mathrm{d}x_2 f_i(x_1,\mu_F^2) f_j(x_2,\mu_F^2) \hat{\sigma}_{ij \rightarrow Y}(x_1p_1,x_2p_2,\mu_F,\mu_R), 
  \label{eq:hhxsec}
\end{equation}
where $f_i(x,\mu_F^2)$ are the PDFs, and $\hat{\sigma}_{ij \rightarrow Y}$ is the partonic cross section for going from two initial partons $i$ and $j$ to the final state $Y$.
The parameter $\mu_F$, known as the \emph{factorization scale}, marks the boundary between the low energy processes and the perturbative regime. 
All processes with an energy below the value of $\mu_F$ are considered as part of the proton structure and are accounted for within the PDFs. 
This separation between the non-perturbative soft QCD processes and the hard partonic scattering cross section is known as \emph{factorisation theorem}.
% Hence, these non-perturbative soft processes are separated from the hard partonic scattering crosssection, a method that is also known as the \emph{factorization theorem}.
The sum in \cref{eq:hhxsec} runs over all possible combinations of initial state partons $i$ and $j$ inside the protons, and the integral goes over the momentum fractions $x_1$ and $x_2$ of the total proton momenta.

The PDFs are non-perturbative and quantify the probability of observing a parton $i$ in the proton with a momentum fraction $x_1$ of the total momentum of the proton.
They are typically evaluated at the factorization scale $\mu_F$. 
Currently, they cannot be predicted to the required precision from theoretical principles and must therefore be extracted from experimental data collected at dedicated scattering experiments.
Example PDFs are shown in \cref{fig:pdfs} for two different energy scales.
The so-called \emph{DGLAP (Dokshitzer-Gribov-Lipatov-Altarelli-Parisi) evolution equations} \cite{Dokshitzer:1977sg,GRIBOV197178,Altarelli:1977zs} allow transferring the PDFs between the different energy scales, which makes them universally applicable to any process. 
% Different physics groups provide computations of PDF sets; one of the most recent, the MMHT2014 PDF set, can be seen in \cref{fig:pdfs}. \todo{Update pdf set and plot}
%\Cref{fig:pdfs} shows the most recent PDFs from the MMHT2014 PDF set.
%There are different gorups, which provide PDF computations.
%\footnote{There are different groups, which provide PDF calculations. They are discussed in the most recent LHC Run 2 PDF recommendations \cite{Butterworth:2015oua}.}.
%which include a combination of the CT14 \cite{Dulat:2015mca}, MMHT2014 \cite{Harland-Lang:2014zoa} and NNPDF3.0 \cite{Ball:2014uwa} PDF sets.
%($\mathcal{O}(\alpha_s)$) Order of in mathematic form

\begin{figure}
  \newImageResizeCustom{0.95}{figures/anatomy/nnpdf.png}
  \caption[The NNPDF3.1 NNLO parton distribution functions at two different energy scales.]{The NNPDF3.1 NNLO parton distribution functions at two different energy scales, $\mu^2$, and associated 68\% confidence-level uncertainty bands. Taken from Ref.~\cite{2017NNPDF}.}
  \label{fig:pdfs}
\end{figure}

%The partonic cross-sections can be computed using Feynman rules.
%The determination of the matrix element (ME) is done with so called \emph{Feynman rules}, which 
The partonic cross section, $\hat{\sigma}_{ij \rightarrow Y}$, depends on the factorization scale, $\mu_F$, as well as another scale called \emph{renormalization scale}, $\mu_R$.
%Choosing its value sufficiently large, allows calculating the partonic cross section as a perturbation series in the running  coupling $\alpha_s$,
When $\mu_F$ is chosen sufficiently large, the partonic cross sections can be calculated as a perturbation series of the strong coupling $\alpha_s$~\cite{Ellis:1991qj},
\begin{equation}
  \hat{\sigma}_{ij \rightarrow Y} = \alpha^k_S \sum_{n=0}^{m} c^{(n)}\alpha_s^n,
  \label{eq:alphaexp}
\end{equation}
where the coefficients $c^{(n)}$ are functions of both the kinematic variables, $x_1$ and $x_2$, as well as the two scales, $\mu_F$ and $\mu_R$. The cross sections are calculated to fixed order by including all relevant combinations of incoming partons, $i$ and $j$, that may lead to the final state $Y$. Higher order terms are suppressed by higher orders of $\alpha_s$ and can often be neglected.
If the highest order term is linear, $m=0$, the calculation is called \emph{leading order} (LO). If $m=1$ ($m=2$), the calculation is called \emph{next-to-leading order} (NLO) (\emph{next-to-next-to-leading order}, NNLO), and so forth. 
The leading power $k$ in \cref{eq:alphaexp} is determined by the process under consideration. Most processes relevant to this thesis start contributing at $k=0$, while some sub-processes contribute with $k=2$.\footnote{For all processes of interest, cross sections are available at NLO, some of them at NNLO or even higher order. See chapter \cref{sec:data-mc-samples} for details.}
%For the Higgs gluon-fusion cross section there exist NNNLO calculations \cite{Anastasiou:2015ema}.}
The different processes that contribute to the cross sections can be depicted graphically with so-called \emph{Feynman diagrams} (also called \emph{Feynman graphs}). 
With the corresponding set of rules, known as \emph{Feynman rules}, the Feynman diagrams provide a graphical representation of the mathematical expressions found in the Lagrangian and can be used to compute the matrix elements (see \cref{eq:xsec}). 
The procedure is detailed in various textbooks, for example in \ccite{Griffiths:111880}.
% A system of rules was invented by Richard Feynman (known as \emph{Feynman rules}) to compute the matrix elements from the Lagrangians. 
% It makes use of graphical representations of the mathematical expressions, known as \emph{Feynman graphs}. More information on the procedure to compute matrix elements can be found in \todo{Maybe move the whole Feynman graph thingy to the partonic cross section below}.

% From previous section!
% A system of rules was invented by Richard Feynman (known as \emph{Feynman rules}) to compute the matrix elements from the Lagrangians. It makes use of graphical representations of the mathematical expressions, known as \emph{Feynman graphs}. 



When higher order terms are included in the matrix-element calculations, quantum loop corrections and real gluon emissions introduce divergences in the cross sections. 
% When including higher orders in $\alpha_s$ in the matrix-element calculation, loop corrections and real gluon emissions introduce divergences in the cross section. 
%\cref{fig:loops} illustrates these contributions with exemplary Feynman diagrams for different orders in $\alpha_s$.
%($\mathcal{O}(\alpha_s)$ and higher)
%singularities in the cross-section.
The infinities can be absorbed in the SM by a procedure known as \emph{renormalization}, by making the coupling parameters dependent on the renormalization scale, $\mu_R$.\footnote{Theories that allow for such a treatment are known as renormalisable, which is an essential property of the SM.}
The renormalization scale acts as a cut-off energy scale above which the processes that introduce infinities are not contributing anymore. 
The essential property of renormalization is that the physics descriptions below that cut off are independent of the dynamics of the theory above it. 
The drawback is that another arbitrary scale is introduced to the theoretical calculations in addition to the factorization scale.
In general, however, the more orders are included in the cross-section calculations, the less strong is the dependence on $\mu_F$ and $\mu_R$. 
The remaining dependence at fixed order is typically included as a systematic uncertainty in physics measurements.
%There are remaining dependencies at fixed order. Resulting systematic uncertainties are most often quantified by varying the scales over some reasonable range and are incorporated in the analysis.
In practice, the values of $\mu_F$ and $\mu_R$ are often chosen to be identical, with a typical value near the scale of momentum transfer for the hard scattering process under consideration.
Various textbooks, for example \ccite{Peskin:1995ev}, provide detailed discussions on the details of the renormalization procedure.
% From Arnold:
% Thus, in order to deal with the UV and collinear divergences, two non-physical scales are introduced. The dependence of observables on the choice of these scales naturally decreases with increasing accuracy of the calculations; it eventually vanishes when all-orders of the perturbative series are considered. A common choice is μ2 F = μ2 R = Q2, where Q2 represents the hard scale of the process under consideration; e.g. Q2 = M2 for the production of a resonance with mass M, or Q2 = p2 T in the case of the pair-production of massless particles with transverse momentum pT.

%\todo{say more about collinear and soft divergencies? -> Ja wegen infrared safe später beim jet algorithm}
% There exist different approaches for choosing the scale $\mu_R$, at which the coupling $\alpha_s$ is evaluated, known as \emph{renormalization schemes}, but there is no prove of any of them being correct.

\Cref{fig:xsec} shows the $pp$ collision cross sections for various processes, displaying a huge range of cross-section measurements over many orders of magnitudes. 
Given a center-of-mass energy of 13\,\TeV\ and a luminosity of $10^{33}\mathrm{cm^{-1}s^{-1}}$, the production of Higgs bosons, for example, is highly suppressed ($10^{-2}\,\text{events}/s$) compared to the total cross section ($10^8\,\text{events}/s$).
This necessitates a precise knowledge of the experimental setup and measurement resolutions. An example of an energy resolution measurement is discussed in \cref{chap:calibration} of this thesis. 

% An important feature is that the cross sections are not dependent on the renormalization scale, as well as the factorization scale, when infinite orders in $\alpha_s$ would be included. 
% There are remaining dependencies at fixed order. Resulting systematic uncertainties are most often quantified by varying the scales over some reasonable range and are incorporated in the analysis.
%However, it is known that the theoretical error on a quantity, which is calculated to $\mathcal{O(\alpha_s^n}$, is always $\mathcal{O(\alpha_s^{n+1})}$. The remaining theoretical error can be quantified
%MMHT2014 NNLO PDFs

\begin{figure}
  \newImageResizeCustom{0.95}{figures/anatomy/xsecs.pdf}
  \caption[Summary of production cross-sections measurements for $pp$ collisions at different center-of-mass energy.]{
    Summary of production cross-section measurements for $pp$ collisions at different center-of-mass energies, $\sqrt{s}$. Taken from \ccite{ATL-PHYS-PUB-2022-009}.
    }
  \label{fig:xsec}
\end{figure}
% Figure 03a:
% Summary of several Standard Model total and fiducial production cross-section measurements (a) with associated references (b) and (c). The measurements are corrected for branching fractions, compared to the corresponding theoretical expectations. In some cases, the fiducial selection is different between measurements in the same final state for different center-of-mass energies √s, resulting in lower cross section values at higher √s.


%%%%%%%%%%%%%%%%%%%%%%%%%%%%%%%%%%%%%%%%%%%%%%
% NEED SECTION ON:
% Generation of MC events
% Comment from BERND to "Detector simulation" section in experimental chapter:
% I believe this statement "Simulations of the ATLAS detector are required in order to generate full Monte Carlo events” needs to be more precise. Why do we have to simulate collisions in the first place? Probably some discussion about quantum mechanics and its probabilistic nature which makes its way all the way to how we analyze data
\subsection{Event simulation and phenomenology}
\label{subsec:event-simulation}
% From ML paper: https://arxiv.org/pdf/1807.02876.pdf
% Particle discovery relies on the ability to accurately compare the observed detector response data with expectations based on the hypotheses of the Standard Model or models of new physics. While the processes of subatomic
% particle interactions with matter are known, it is intractable to compute the detector response analytically. As
% a result, Monte Carlo simulation tools, such as GEANT [6], have been developed to simulate the propagation
% of particles in detectors to compare with the data. The dedicated CWP on detector simulation [7] discusses the
% challenges of simulations in great detail. This section focuses on the machine learning related aspects

% They are an essential ingredient for extracting physics parameters from the collision data, as described later in this section.

The $pp$ interactions and the interactions of the particles with the detector follow the rules of QFT and are therefore probabilistic by nature.
Simulated $pp$ collision events are therefore generated with the \emph{Monte Carlo} (MC) method to compare the data against what is expected from the SM.
%given the current knowledge of particle physics.
%  are therefore used 
% They are an integral part at hadron colliders 
A simulated event must take into account all aspects of a $pp$ collision, including the matrix element and the non-perturbative regime, in particular the modelling of the hadronization and showering process. In addition, the detector response and geometry needs to be simulated, as well as other nuisances at hadron colliders, such as the underlying event or multiple $pp$ collisions that simultaneously produce detector signals.
The following provides details on the different aspects the simulation needs to cover.

% - MC tools are typically utilized for many of these steps.
% - Parameters of MC are tuned to fit the data, especially for the non-perturbative regime
% - An overview is given in the following

% Dandoy
% A good MC simulation will cover all aspects of particle evolution detailed in Section 2.1, including matrix element calculation, hadronization and showering of final states, and simulation of detector geometry and response.
\paragraph{Matrix element}
Matrix element calculations are performed with MC generators that randomly sample from the SM expectations to generate hard scatter events. 
There are different MC generators that come with varying precision in perturbation theory. The generators used for the \HWW analysis are summarized in \cref{sec:data-mc-samples}.

\paragraph{Parton showering}
Parton showering occurs because there is a finite quantum probability for a single parton to split into two. 
The \emph{QCD splitting functions}~\cite{Altarelli:1977zs} provide the theoretical foundation to describe this process and are used in dedication parton-shower algorithms.
The probability for emitting soft and collinear radiation diverges, respectively, for low energies and small angles. 
Physical observables are therefore chosen such that they are insensitive to these processes and a cut off is applied in the parton-shower algorithms to avoid modelling this radiation explicitly.
% The theoretical foundation for the parton-shower models is provided by the \emph{QCD splitting functions}~\cite{Altarelli:1977zs}.
%There is also the possibility of hard gluon being emitted in the initial or final state, which spoils the measurement of the total transverse momentum of the event.

\paragraph{Matching} 
While the matrix element calculations take into account hard radiation that can be calculated perturbatively, the parton-shower algorithms rely on perturbative approximations of soft and collinear radiation. 
Combining both regimes appropriately without double counting is challenging.
To achieve this, different matching methods that rely on certain assumptions are used in different MC simulations. An overview can be found, for example, in \ccite{QCDIntroSchool}.

\paragraph{Hadronization}
The partons hadronize when their energy decreases and the strong coupling becomes large. There are several phenomenological approaches for modelling this non-perturbative process, one of which is the \emph{Lund-String-Model}~\cite{Andersson:1983ia}. 
It is based on the fact that the force between two strongly interacting particles is constant, implying that the potential increases linearly with increasing distance.
At a certain distance between the partons the potential energy is sufficient to create a quark-antiquark pair out of the vacuum. This procedure is repeated until neutral-colored hadrons are formed.
The Lund-String-Model is used, for example, in the \Pythia event generator. Other generators such as \HERWING\ or \Sherpa make use of an alternative model for hadron formation called \emph{cluster model}. Details can be found in \ccite{Buckley_2011}. 
\paragraph{Decays}
The particles produced in the scattering process or in the process of hadronization are often unstable and further decay. Many of these decays, for example decaying $b$-hadrons, provide useful information about the event. %Dedicated algorithms exist to find the particles which initiate the decay chains.
\paragraph{Underlying event}
The hard scatter is accompanied by many soft processes from the remaining partons in the colliding protons. These interactions produce signals in the detector that overlay the hard scattering process and need to be accounted for in the simulation.
\paragraph{Pile-up}
A nuisance in particular at $pp$ colliders is the overlap of the signals of more than one scattering event in the detector. This is known as \emph{pile-up}. A distinction is drawn between \emph{in-time pile-up}, referring to the amount of multiple interactions per bunch crossing, and \emph{out-of-time pile-up}, referring to overlapping events from previous $pp$ collisions or following bunches due to the long response time of parts of the detector.
Pile-up collisions predominantly consist of inelastic, low energy transfer $pp$ collision events. 
Their impact is typically taken into account by simulating them separately and overlaying them with the generated hard scatter.
\paragraph{Detector simulation}
%\todo{Not clear where to provide details of detector simulation. Here or in the experimental chapter. It might be more meaningful in the hadron collider physics chapter. We could have a small reference to this here from the experiment chapter that might go into more detail about ATLAS. We can keep it general here.}
The exact detector geometry and the detector response to different particles must be simulated.
% using the 
The simulation must consider all detector inefficiencies and the material distribution in the detector, as well as physics properties such as the interaction lengths, the lifetime, and decay rates of the particles.

\paragraph{Truth information}
Information on the event such as the total momentum or the type of the interacting particles are accessible in simulated events but not in data. This extra layer of information, referred to as \emph{truth information}, is crucial for many physics analyses and calibration measurements.

% \subsection{Soft QCD: Minimum bias, pile-up, ...?}
% % Mention in Event Generation subsection, see Master thesis!

% \todo{Not clear if a dedicated chapter is needed here. My preference is NO at the moment! Details can be provided int he dedicated JER chapter to not blow up the theory part.}

% \Rinote{}{Where else should I talk about pile-up!? what is pile-up? HERE! Pile-up reweighting? analysis section. Pile-up as nuisance for jet measurements? -> JER calibration chapter, pile-up in LHC section as something intrinsic to pp collision event?}

% ->  Ruthmann has a nice section about it!

% - I should explain concepts like luminosity blocks  / bunch spacing and stuff in Data Taking Section
% - Then I can explain different pile-up conditions here.
% - This will be valuable to understand the noise term measurement which exactly tries to measure the noise term!
% - Also look back at discussion on skype with Brian about pile-up (actual mu vs average mu and so on)

% Checkout this section for pile-up overlay

% https://indico.cern.ch/event/1003305/contributions/4236702/attachments/2202625/3728039/PileUpTaskForcePandPPlenaryMarch2021.pdf





