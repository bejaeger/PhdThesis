\chapter{Theoretical Foundation}
\label{chap:theory}

\section{The Standard Model of Particle Physics}

\section{Higgs Boson Physics}


\section{The Anatomy of Proton-Proton Collision Events}

Introduce impact parameter cause it's mentioned here \cref{subsec:inner-detector}.


% \section{Particle Detection}
% \subsection{Calorimetry}
% \subsubsection{Electromagnetic showers}
% \subsubsection{Hadronic showers}
% \subsection{Tracking}

% \subsection{Particle colliders}
% \Minote{}{Maybe put this part in a theory, more general section (proton-proton collisions?).}
% A crucial performance indicator of a particle accelerator is its \emph{instantaneous luminosity} $\mathcal{L}$ as it is proportional to the \emph{event rate}, 
% \begin{equation}
%     \frac{\mathrm{d}N}{\mathrm{d}t} = \sigma_{P} \mathcal{L},
% \end{equation}
% for a given physics process $P$ with cross section $\sigma_{P}$.
% The instantaneous luminosity depends on parameters of the accelerator and is given by
% \begin{equation}
%   \mathcal{L} = f_rn_b\frac{N_p^2}{4\pi \sigma_x \sigma_y}, 
% \end{equation}
% when assuming Gaussian shaped beam profiles. Here, $f_r$ denotes the rotational frequency of the two bunches, $n_b$ the total number of bunches inside the accelerator, and $N_p$ the numbers of particles within each colliding bunch. 
% The total number of events after time $T$ is given after integrating with respect to time, 
% \begin{equation}
%   N_{\text{Events}} = \sigma_{P} L_{\text{int}} = \sigma_{P} \int_{0}^{T} \mathcal{L} \mathrm{d}t,
% \end{equation}
% and is a crucial quantity typically measured in high-energy-physics experiments. 
% It becomes clear that increasing the integrated luminosity $L_\text{int}$ is one of the main goals of any collider experiment, as it is directly proportional to the number of expected events of a particular physics process. 

