\chapter{Theoretical Foundation}
\label{chap:theory}

\todo{More general introduction. Move the below to the SM section}

- Section underlines the theoretical underpinnings that manifest the Higgs boson as a special place in the universe

- Lots of SM parameters are related to Higgs (SM stands and falls with Higgs)

- The SM also has limitations -> precision measurements of Higgs required

- Many models of new physics expect to have access to new physics through interactions with Higgs ()

% From CARSTEN:
%This chapter outlines the Lagrange formulation of the Standard Model of Elementary Particle Physics (SM), a successful framework to study the fundamental laws governing the universe. Special attention is devoted to the Brout-Englert-Higgs (BEH) mechanism of electroweak symmetry breaking (EWSB). A more detailed introduction including all aspects of the model can be found in many textbooks [24–28].


\section{The Standard Model of Particle Physics}
\label{sec:sm}


% From Master's thesis:
The SM is a collection of relativistic quantum field theories (QFT) that describe the interactions between all known fundamental particles. 
% It has been extremely successful in describing correctly an enormous amount of experimental measurements. 
% The latest achievement is the discovery of the Higgs boson in 2012.
% It has been extremely successful in making predictions that were later confirmed by experimental measurements. 
% Examples are the observation of the $W$ and $Z$ bosons in 1983 by the UA1 and UA2 collaborations, or the top quark discovery in 1995 by the CDF and D0 experiments. \todo{Maybe references}
% The success story culminated in the discovery of a new particle by the ATLAS and CMS collaborations in 2012~\cite{HIGG-2012-27,CMS-HIG-12-028}.
% The observation of the $W$ and $Z$ bosons by the UA1 and UA2 collaborations in 1983 are just two examples of this.
% The discovery of the long sought-after Higgs boson in 2012 by the ATLAS and CMS collaborations~\cite{Aad:2012tfa,Chatrchyan:2012xdj} marks the highlight of this success story. 
%The success story culminated in the discovery of a new particle by the ATLAS and CMS collaborations in 2012~\cite{HIGG-2012-27,CMS-HIG-12-028}. 
% To date, ten years after this discovery, all experimental measurements show consistency of the observed particle with the properties of the long sought-after Higgs boson predicted by the SM.
The fundamental particles (also called \emph{elementary particles}) are grouped according to their quantum numbers, and their interactions are described by the requirement of local gauge invariance with respect to the gauge group
\begin{equation}
  \label{eq:sm-gauge-group}
  \text{SU(3)}_C \times \text{SU(2)}_L \times \text{U(1)}_Y,
\end{equation}
the details of which are explained in this section.
% The mathematical formulation SU(3)$_C$ $\times$ SU(2)$_L$ $\times$ U(1)$_Y$ local gauge symmetry that gives rise to the interactions between all known fundamental particles.
%It evolved during the 60’s and 70’s due to a strong interplay between experimental observations and theoretical developments. 
\Cref{subsec:particle-content} provides a high-level overview of the particles and forces that are part of the SM. \Cref{subsec:formalism} briefly outlines the theoretical principles that build the basis for the mathematical description of the SM. The remaining sections summarize the Lagrange formulation of the SM.
%, focusing on the \emph{Higgs mechanism} of \emph{electroweak symmetry breaking} (EWSB).
% \Cref{subsec:qed,subsec:qcd,subsec:ew-model,subsec:ewsymbreaking,subsec:fermion-masses,subsec:final-lagrangian} summarize the mathematical formulation of the SM, focusing on the \emph{Higgs mechanism} of \emph{electroweak symmetry breaking} (EWSB).
% The free parameters of the SM are summarized in \cref{subsec:final-lagrangian}, and \cref{sec:limitations} concludes this section by discussing the limitations of the SM.
% that motivate conducting more precise measurements such as the one presented in \cref{chap:hww} of this thesis.
The SM is covered in all its detail in the literature, e.g. \ccite{Peskin:1995ev,Halzen:1984mc,Thomson:2013zua}, which serve as the primary resources for the descriptions in this section.


% From previous theoretical foundation intro:
% - Section underlines the theoretical underpinnings that manifest the Higgs boson as a special place in the universe
% - Lots of SM parameters are related to Higgs (SM stands and falls with Higgs)
% - The SM also has limitations -> precision measurements of Higgs required
% - Many models of new physics expect to have access to new physics through interactions with Higgs ()


%On a phenomenological level, fundamental physics can be described in terms of elementary particles and forces. 
% - Introduction to QFT and gauge group
% - Lagrangian formalism
% - local gauge invariance: without local gauge invariance:

% From Pich The Standard Model
% Thus, once a given phase convention has been adopted at one reference point x0, the same convention must be taken at all space-time points. This looks very unnatural.

% - highly successful in describing QED
% WHAT HAPPENS IN THIS SECTiION:

% - First overview of particles and forces
% - Formalism
% - QED, QCD, electroweak model
% - Problem: no masses for bosons -> Higgs mechanism
% - Final Lagrangian and parameters to be measured experimentally of the SM
% - Limitations of the SM

\subsection{Particles and forces}
\label{subsec:particle-content}

\begin{figure}
  \includegraphics[width=1\textwidth,trim=10 0 10 0]{figures/theory/particles-infographic/particles-infographic.pdf}
  \caption[Overview of particles in the SM.]{Overview of the particles in the SM. Their charge, color, mass, and spin are indicated. More details are given in the text. The graphic is adapted from \ccite{CBurgardParticlesInfographic} and the values taken from \ccite{PDG2020}.}  
  \label{fig:particles-infographic}
\end{figure}
%In QFT all fundamental fields have an associated particle, that can be thought of as excitations (or vibrations) of the fields.
QFT describes nature in terms of fundamental fields and their interactions. Each field is associated with an elementary particle that can be thought of as a quantum excitation of the underlying field. 
This association makes it possible to describe fundamental physics using elementary particles.
%\footnote{To simplify terminology, the objects are mostly referred to as particles rather than fields in this chapter, but the relationship is important to be kept in mind.} 
% On a \TDnote{phenomenological}{other word?} level, fundamental physics can be described in terms of elementary particles and forces.
% All currently known particles and forces included in the SM are listed in \TDnote{REF}{REF}.
The particle content of the SM is summarized in \cref{fig:particles-infographic}. The particles can be grouped into two main types: 
\emph{fermions} with half-integer spin and \emph{bosons} with integer spin. The interaction between the fermions (sometimes called \emph{matter particles}) are interpreted as exchanges of \emph{gauge bosons} (also called \emph{force carriers}) that mediate the fundamental forces. 
% \emph{fermions} with half-integer spin (sometimes called \emph{matter particles}) and \emph{bosons} with integer spin (known as the \emph{mediators} or \emph{force carriers} of the fundamental forces).

The fermions can be grouped into three generations, each containing two leptons and two quarks. The different types of leptons and quarks are referred to as (quark or lepton) \emph{flavors}.
Quarks have an additional property called \emph{color}, which can take three possible values typically labelled as red, green, and blue\footnote{In fact, also linear combinations of the three colors are realized in nature, as discussed later in this chapter.}.
Additionally, each fermion has a corresponding antiparticle that has the same mass but ``opposite'' internal quantum numbers, i.e., oppositely signed charge and anti-color if applicable.

%The interactions between the fermions can be described by exchanges of so-called \emph{gauge bosons}.
The SM includes three of the four known fundamental interactions: The \emph{electromagnetic interaction}, the \emph{strong interaction}, and the \emph{weak interaction}. The electromagnetic and the weak interactions are unified in the \emph{electroweak theory}. The fourth known fundamental interaction, \emph{gravitation}, is not part of the SM but can be fully neglected in present particle physics experiments due to its weak strength.

The gauge boson that mediates the electromagnetic interaction is the massless \emph{photon} that interacts with all charged fermions. Electromagnetic interactions are described by the theory of \emph{quantum electrodynamics} (QED). The strong interaction is transmitted via eight massless gluons, following the rules of \emph{quantum chromodynamics} (QCD). QCD acts on matter particles that carry color, that is, quarks and gluons themselves. 
Due to a property known as \emph{confinement} in QCD\footnote{Confinement arises due to the energy dependence of the strength of the QCD interactions (see \cref{subsec:factorisation}).}, quarks cannot be found in isolation. They are confined within hadrons that consist, for example, of a quark and an anti-quark (known as \emph{mesons}) or three quarks (known as \emph{baryons}).
The weak interaction is mediated by three massive gauge bosons, \Wplus, \Wminus, and \Zboson, and acts on all fermions. Similar to gluons, the \Wpm and \Zboson bosons can interact with themselves.

The final particle of the SM is the electrically neutral \emph{Higgs boson}. It is the only spin-0 scalar particle and plays a special role in the SM in the mechanism of electroweak symmetry breaking through which the \Wpm and \Zboson bosons gain their masses. A dedicated overview of the physics involving the Higgs boson is provided in \cref{chap:higgs}.


\subsection{Formalism and principles}
\label{subsec:formalism}
% - Action: S = Int ( Lagrangian ) dt -> EOMs
% - Lagrangian = Int (Lagrange density ) d3x
% -> Lagrange density is commonly referred to as Lagrangian 
% - In QFT, we define EOMs for fields by specifying the Lagrange density (Lagrangian)

% ONE MORE SENTENCE:
% In physics, equations of motion are equations that describe the behavior of a physical system in terms of its motion as a function of time.[1] More specifically, the equations of motion describe the behavior of a physical system as a set of mathematical functions in terms of dynamic variables.

% Particles can be thought of as excitations (or vibrations) of corresponding fundamental fields. 
In order to describe the behavior of a physical system, the equations of motions can be derived from Lagrange's equations
\begin{equation}
  \label{eq:euler-lagrange}
  \frac{\partial}{\partial x} \left( \frac{\partial \Lagrangian}{\partial \left(\partial \psi / \partial x \right) } \right) - \frac{\partial \Lagrangian}{\partial \psi} = 0,
  %S = \int \Lagrangian d^3x dt,
\end{equation}
where \Lagrangian is the Lagrange density and $\psi$ is a particle field.\footnote{Details on the Lagrange formalism can be found in any standard textbook on QFT, e.g. in \ccite{Peskin:1995ev}.}
The dynamics of a system are therefore fully specified by the Lagrange density, often simply denoted \emph{Lagrangian}. 
The Lagrangian is generally a function of the fields $\psi$, their derivatives $\partial \psi = \frac{\partial \Lagrangian}{\partial x}$, and the space-time coordinate $x$, so $\Lagrangian \to \Lagrangian \left( \psi, \partial \psi, x \right)$. 
The fields are generally dependent on the space-time coordinate, $\psi \rightarrow \psi(x)$.\footnote{For cleaner notation, these dependencies are not explicitly mentioned throughout this thesis.}

For a given field $\psi$, the principles of QFT demand to include all possible interaction terms in the Lagrangian that are related to the field\footnote{This follows what is sometimes referred to as the ``totalitarian principle'' of quantum mechanics, that satirically states: ``Everything not forbidden is compulsory'', meaning that everything allowed by the laws of physics must actually happen or exist.}.
The number and type of terms that can be included in the SM, however, is strongly constraint by requiring the theory to be invariant under symmetry transformations of the fields following the SM gauge group shown in \cref{eq:sm-gauge-group}.
The SM is governed by the principle of \emph{local gauge invariance}, which implies that the physical content of the theory stays the same when performing transformations of the particle fields according to
\begin{equation}
  \psi \rightarrow e^{iT(x)} \psi
\end{equation}
independently at every space-time point, where $T(x)$ is the generator of a certain symmetry group.
% Gauge theories require the introduction of so-called gauge fields, that transform in a way so that the theory stays locally gauge invariant.
% Pich
% This is only possible if one adds an extra piece to the Lagrangian, transforming in such a way as to cancel the ∂μθ term in Eq. (6).
% In order to maintain the symmetry, the Lagrangian may need to be manipulated by introducing new fields.
Local gauge theories require the introduction of \emph{covariant derivates} to make the Lagrangian invariant under local gauge transformations. These covariant derivatives include gauge fields which give rise to particle interactions that give rise to the fundamental forces of nature. 
Historically, QED is the first local gauge theory that was established. It follows a U(1) local gauge symmetry and entails the photon as the associated gauge field. 
The SM can be elegantly described by demanding local gauge invariance with the symmetry group as shown in \cref{eq:sm-gauge-group}, and explains the existence of all gauge bosons mentioned in the previous section. 
The fermions, in contrast, are added a priori to the theory and do not follow from fundamental principles.
%The following sections provide an overview of the Lagrangian of the SM, by going through the different symmetry groups. 
The following sections will outline the Lagrange formulation of the SM.
In all equations, natural units are used, i.e. $c = \hbar = 1$, which leads to the mass, momentum, and energy of particles being expressed in units of electronvolt (\eV).
Moreover, the Einstein-summation convention is used and if not mentioned otherwise, greek-letter indices take integer values from 1 to 4, while latin-character indices take integer values from 1 to 3.

% - renormalization:
% We consider that the physics is understood up to a given cut-off/scale Lambda.

%The SM is governed by the principle of local gauge invariance and the associated symmetry group is SU(3)$_C$ $\times$ SU(2)$_L$ $\times$ U(1)$_Y$. 

% the next important lesson is that the existence of these symmetries places suck an incredibly strong constraint on what the theory actually is.

% Sean
% QED: demanding all the terms in your Lagrangian being gauge invariant is enforcing the conservation of electric charge gauge
% This is a reflection of Noethers theory. Symmetry conservation -> associated quantity of the U(1) symmetry is charge

% Sean Carrol:
% W and Z bosons are the physical excitations from vibrations in the SU(2) to gauge field
%the existence of these fields giving rise to interactions giving rise to forces of nature comes from the gauge symmetry.
% the next important lesson is that the existence of these symmetries places suck an incredibly strong constraint on what the theory actually is.

% From Master thesis
% An essential part of the mathematical formulation of the SM is based on the postulation of local gauge invariance. It implies, that the physical content of the theory should stay the same when performing certain redefinitions of the particle fields independently at every space-time point. The SM follows a SU(3)×SU(2)L×U(1)Y symmetry group, where SU(3) is the gauge group of QCD and SU(2)L×U(1)Y the corresponding symmetry for the electroweak model (the meaning of the subindizes L and Y will be explained in the following sections). Historically, QED was the first well established gauge theory, following a U(1) symmetry. To demonstrate the principles of a gauge theory, which are the key concepts of the mathematical framework of the SM, the QED Lagrangian will be derived in the following. Subsequently the same principles are applied to describe the main aspects of the more complex theories of QCD and the electroweak model. A description of the mechanism to incorporate masses for the W± and Z bosons via breaking the electroweak symmetry is given thereafter. The following sections follow to a large extend the more detailed descriptions given in Refs. [19–21].


\subsection{Quantum electrodynamics}
\label{subsec:qed}
%The above derivations are shortly recapped: Starting from the Lagrangian of a freely moving relativistic fermion one can impose the requirement of local U(1) gauge invariance. This demands to add a new field $A_\mu$ toghether with an interaction term, which is incorporated in the Lagrangian by introducing a covariant derivative $D_\mu$. Additionally, the kinetic term in \cref{eq:kinetictermqed} needs to be added for the new field, which leads to the final Lagrangian for QED, shown in \cref{eq:Lagrangianqed}.
QED can be regarded as a reflection of an underlying $U(1)$ local gauge symmetry of the complex-valued fermion fields, $\psi_f$, known as \emph{Dirac spinors}.
The Lagrangian that is invariant under $\psi_f \rightarrow e^{i \omega(x)} \psi_f$ transformations can be written as
\begin{equation}
  \mathcal{L}_{\text{QED}} = \sum_f \bar{\psi}_f(i\gamma^\mu D_\mu - m_f)\psi_f - \frac{1}{4}F_{\mu\nu}F^{\mu\nu},
  \label{eq:Lagrangianqed}
\end{equation}
where the sum goes over all electrically charged fermions with masses $m_f$, and $\gamma^\mu$ refers to the four $4 \times 4$ gamma matrices.
Here, $D_\mu = \partial_\mu + ieA_\mu$ is the covariant derivative, which includes the gauge field $A_\mu$ which is associated with the photon. The requirement of local gauge symmetry prohibits terms quadratic in $A_\mu$, resulting in the prediction of the photon being massless.\footnote{Terms in the Lagrangian that are quadratic in a certain field give rise to particle masses.}
The so-called \emph{field tensor}, defined as $F_{\mu\nu} = \partial_\mu A_\nu - \partial_\nu A_\mu$, provides an elegant mathematical object representing the electromagnetic field. It can be used, for example, to describe Maxwell's equations of classical electrodynamics.

The symmetry group associated to QED is denoted U(1)$_{\text{QED}}$ and the conserved quantity is the electric charge.\footnote{This is a reflection of Noether's theorem, which states that each symmetry is related to a conserved quantity.}
It should be noted that U(1)$_{\text{QED}}$ is not part of the original symmetry group of the SM. As explained below, U(1)$_{\text{QED}}$ arises from a SU(2)$_L$ $\times$ U(1)$_Y$ symmetry that is spontaneously broken.


\subsection{Quantum chromodynamics}
\label{subsec:qcd}
QCD is a non-Abelian gauge theory associated to a SU(3)$_C$ symmetry. The subindex $C$ refers to the color which is the conserved quantity under SU(3)$_C$ transformations.
The locally gauge invariant QCD Lagrangian reads
\begin{equation}
  %  \mathcal{L}_{\text{QCD}} = -\frac{1}{4}G_{\mu\nu}^aG^{a\,\mu\nu} + \bar{q}^i\left( i\gamma^\mu D_\mu-m \right)^j_i q_j,
  \mathcal{L}_{\text{QCD}} = \sum_f \bar{\psi}_f(i\gamma^\mu D_\mu - m_f)\psi_f - \frac{1}{4}G_{\mu\nu}^aG^{\mu\nu}_{a},  \label{eq:lqcd}
\end{equation}
where the sum runs over all quark fields, $\psi_f$, that take the form of spinor triplets. 
%\todo{Sum over A in lambdaA GmuA in the equation?} <- YEP! Because of self interaction
The covariant derivate is given by
\begin{equation}
    D_\mu = \partial_\mu + i g_s \frac{\lambda^a}{2} G_\mu^a,
\end{equation}
which includes eight gluon fields $G_\mu^a$ (with $a = 1, \ldots, 8$), the strong coupling constant $g_s$, and the Gell-Mann matrices $\lambda^a$ that are the generators of the SU(3)$_C$ group.
The field tensors $G_{\mu\nu}^a$ are defined as
\begin{equation}
  \label{eq:qcd-tensor}
  G_{\mu\nu}^a = \partial_\mu G_\nu^a - \partial_\nu G_\mu^a - g_s f^{abc}G_\mu^b G_\nu^c,
\end{equation}
where $f^{abc}$ are the structure constants of SU(3)$_C$, with $a, b, c = 1, \ldots, 8$. 
The term involving $f^{abc}$ includes an implicit sum over similar latin-character indices. It arises from the non-commuting elements of the SU(3)$_C$ group and gives rise to triple and quartic gluon self-interactions. 
% Triple and quartic self-interactions are part of QCD because gluons themselves carry a combination of a color and anti-color. 
% Invariant under ... transformation, where ... are the group generators
Similar to photons, the gluons are massless because no terms quadratic in the gluon fields are allowed because of the requirement of local gauge invariance. 
%which is similar to the requirement of a massless photon in QED. 



\subsection{The electroweak model}
\label{subsec:ew-model}
% \begin{table}
%   \caption[Overview of the fermion content in the electroweak model.]{Overview of the fermion content in the electroweak model. They are grouped into left-handed SU(2)$_L$ doublets and right-handed singlets denoted with the subindex $L$ and $R$, respectively. The down-type quarks $d', s', b'$ are the eigenstates of the electroweak interaction and given by linear combinations of the mass eigenstates $d, s, b$. This mixing is described by the CKM matrix, see text. Since right-handed neutrinos are not undergoing any interaction in the Standard Model they are not listed here.}
%   \label{tab:ewfermioncontent}
%   \centering
%   \begin{tabular}{c |@{}| c c c | c c c}
%     \toprule
%                              & \multicolumn{3}{c}{Generation} & \multicolumn{3}{|c}{Quantum number}                                                                      \\
%                              & 1$^{\text{st}}$                & 2$^{\text{nd}}$                     & 3$^{\text{rd}}$ & $T^3$          & $Y$            & $Q$            \\
%     \midrule
%     \multirow{3}{*}{Leptons} & \multirow{2}{*}{$\myvec{\nu_e                                                                                                             \\ e}_L$} & \multirow{2}{*}{$\myvec{\nu_\mu \\ \mu}_L$} & \multirow{2}{*}{$\myvec{\nu_\tau \\ \tau}_L$} & $\frac{1}{2}$  & -1             & 0 \\
%                              &                                &                                     &                 & $-\frac{1}{2}$ & -1             & -1             \\
%                              & $e_R$                          & $\mu_R$                             & $\tau_R$        & 0              & -2             & -1             \\
%     \midrule
%     \multirow{3}{*}{Quarks}  & \multirow{2}{*}{$\myvec{u                                                                                                                 \\ d'}_L$}    & \multirow{2}{*}{$\myvec{c \\ s'}_L$}        & \multirow{2}{*}{$\myvec{t \\ b'}_L$}          & $\frac{1}{2}$  & $\frac{1}{3}$  & $\frac{2}{3}$ \\
%                              &                                &                                     &                 & $-\frac{1}{2}$ & $\frac{1}{3}$  & $-\frac{1}{3}$ \\
%                              & $u_R$                          & $ c_R$                              & $t_R$           & 0              & $\frac{4}{3}$  & $\frac{2}{3}$  \\
%                              & $d_R$                          & $ s_R$                              & $b_R$           & 0              & $-\frac{2}{3}$ & -$\frac{1}{3}$ \\
%     \bottomrule
%   \end{tabular}
% \end{table}


\begin{table}[t]
  \caption[Overview of the fermion content in the electroweak model.]{Overview of the fermion content in the electroweak model. They are grouped into left-handed SU(2)$_L$ doublets and right-handed singlets denoted with the subindex $L$ and $R$, respectively. The down-type quarks $d', s', b'$ are the eigenstates of the electroweak interaction and given by linear combinations of the mass eigenstates $d, s, b$. This mixing is described by the CKM matrix, see text. Since right-handed neutrinos are not undergoing any interaction in the Standard Model they are not listed here.}
  \label{tab:ewfermioncontent}
  \centering
  \begin{tabular}{c |@{}| c c c }
    \toprule
                             & \multicolumn{3}{c}{Generation}                                                                       \\
                             & 1$^{\text{st}}$                & 2$^{\text{nd}}$                     & 3$^{\text{rd}}$        \\
    \midrule
    \multirow{3}{*}{Leptons} & \multirow{2}{*}{$\myvec{\nu_e                                                                                                             \\ e}_L$} & \multirow{2}{*}{$\myvec{\nu_\mu \\ \mu}_L$} & \multirow{2}{*}{$\myvec{\nu_\tau \\ \tau}_L$}  \\
                             &                                &                                     &                            \\
                             & $e_R$                          & $\mu_R$                             & $\tau_R$           \\
    \midrule
    \multirow{3}{*}{Quarks}  & \multirow{2}{*}{$\myvec{u                                                                                                                 \\ d'}_L$}    & \multirow{2}{*}{$\myvec{c \\ s'}_L$}        & \multirow{2}{*}{$\myvec{t \\ b'}_L$}       \\
                             &                                &                                     &                 \\
                             & $u_R$                          & $ c_R$                              & $t_R$           \\
                             & $d_R$                          & $ s_R$                              & $b_R$          \\
    \bottomrule
  \end{tabular}
\end{table}

% - the chirality can be determined with... right-chiral fermions are singlets under SU(2)L transformations
% - This also means that no right-chiral neutrinos exist in the SM as they don't interact with any of the forces

% "Also called: The Glashow􏰁Weinb erg􏰁Salam Theory of Weak Interactions"
The weak and electromagnetic forces are unified in the electroweak model~\cite{GLASHOW1961579,SALAM1964168,PhysRevLett.19.1264} by imposing local gauge invariance under transformations of the symmetry group
\begin{equation}
  \label{eq:ew-sym-group}
  \text{SU(2)}_L \times \text{U(1)}_Y.
\end{equation}
The electroweak gauge group is based on two major empirical findings.
The first is, that only \emph{left-chiral}\footnote{
  The \emph{chirality} of a fermion can be determined with the projection operators $P_L$ and $P_R$ like \\
  $\psi = P_L \psi + P_R \psi = \frac{1}{2} \left( 1 - \gamma^5 \right) \psi + \frac{1}{2} \left( 1 + \gamma^5 \right) \psi = \psi_R + \psi_L$, \\
  where $\gamma^5 = i\gamma^0\gamma^1\gamma^2\gamma^3$. The chirality becomes identical to the helicity for massless particles.} (also denoted \emph{left-handed}) fermions interact via the weak interaction, indicated with an $L$ in \cref{eq:ew-sym-group}. An overview of the fermion content is shown in \cref{tab:ewfermioncontent}, that groups the fermions into left-handed doublets, $\psi_L$, and right-handed singlets, $\psi_R$, under SU(2)$_L$ transformations. 
The second finding is that the quarks participating in the electroweak interaction, labelled as $u', d', c'$, are a mixture of the quark mass eigenstates. Their relation is specified by the \emph{Cabibbo–Kobayashi–Maskawa (CKM) matrix} \cite{doi:10.1143/PTP.49.652}, $\pmb{V}$, as
% From Peskin
% The off diagonal terms in Vij allow weak􏲩interaction transitions b e􏲩 tween quark generations.
\begin{equation}
  \begin{pmatrix}
   d' \\
   s' \\
   b'
 \end{pmatrix}
 = 
 \pmb{V} 
 \begin{pmatrix}
   d \\
   s \\
   b
 \end{pmatrix}.
\end{equation}
The CKM matrix is unitary and fully specified with 4 parameters and encodes the strength of the flavor-changing electroweak interactions.
%\todo{Maybe add one more sentence describing the rotation that is necessary? see P.Sommer thesis}

\noindent The fermion fields then transform as
\begin{align}
  \psi_L & \rightarrow e^{iY\omega} e^{iT^a\omega_a} \psi_L, \qquad (a = 1, 2, 3) \\
  \psi_R & \rightarrow e^{iY\omega} \psi_R,
\end{align}
under SU(2)$_L$ $\times$ U(1)$_Y$ transformations, where $T^a=\frac{\sigma^a}{2}$ are the Pauli matrices and generators of the SU(2)$_L$ group.
The associated conserved quantity is the \emph{weak isospin} $T$, of which the third component is conserved in weak interactions and given by $T^{(3)} = \pm \frac{1}{2}$ for SU(2)$_L$ doublets and $T^{(3)} = 0$ for SU(2)$_L$ singlets.
The U(1)$_Y$ symmetry is associated to the \emph{hypercharge} $Y$, and cannot be directly associated to the QED gauge group.
The relation to the electromagnetic interaction and the physical electric charge $Q$ is
\begin{equation}
  Q = T^{(3)} + \frac{Y}{2}.
\end{equation}

\noindent The local gauge invariant electroweak Lagrangian can be written as
\begin{equation}
  \mathcal{L}_{\text{EWK}} = \sum_f i\bar{\psi}_{f}\gamma^\mu D_\mu \psi_{f} - \frac{1}{4}W_{\mu\nu}^aW^{\mu\nu}_{a} - \frac{1}{4} B_{\mu\nu}B^{\mu\nu}, 
  \label{eq:lagrangianewk}
\end{equation}
where the sum runs over all fermions $f$, including their left-handed and right-handed counterparts.
%, whose occurrence as left-handed and right-handed particles is explicitly mentioned.
The covariant derivative is defined as
\begin{equation}
  D_\mu = \partial_\mu + igT^aW_\mu^a + ig'\frac{Y}{2}B_\mu 
  \label{eq:covdevewk}
\end{equation}
and includes four gauge fields. The fields $W^a_\mu$ (with a = 1, 2, 3) are the gauge fields of SU(2)$_L$ with associated coupling $g$, and $B_\mu$ is the gauge field of U(1)$_Y$ with coupling $g'$.
The field tensors in \cref{eq:lagrangianewk} are given by
\begin{align}
  W_{\mu\nu}^a & = \partial_\mu W_\nu^a - \partial_\nu W_\mu^a - g \epsilon^{abc} W^b_\mu W^c_\nu, \label{eq:Wtensor} \\
  B_{\mu\nu}   & = \partial_\mu B_\nu - \partial_\nu B_\mu,
\end{align}
where $\epsilon^{abc}$ are the structure constants of SU(2)$_L$. The third term on the right-hand side of \cref{eq:Wtensor} includes an implicit sum over similar latin-character indices. It arises because of the non-Abelian nature of SU(2)$_L$ and gives rise to triple and quartic self-interactions of the gauge fields $W_{\mu}^a$.
The tensor $B_{\mu\nu}$ has the same structure as the electromagnetic field strength tensor obtained in QED.
%\TDinote{}{Give details on structure constants? for QCD AND Electroweak model}

\noindent The Lagrangian of \cref{eq:lagrangianewk} describes 4 massless bosons. No terms quadratic in the vector fields are allowed due to the requirement of local gauge invariance. 
Another mechanism is required to explain the existence of the masses of the gauge fields \Wpm and \Zboson.
% and $\gamma$. 


% From experiments one expects two charged bosons $W^\pm$ and two neutral bosons, the $Z$ and the photon.

\subsection{Spontaneous symmetry breaking and the Higgs mechanism}
\label{subsec:ewsymbreaking}
%- Local gauge invariance does not allow adding mass terms
%- The SU(2)$_L$ $\times$ U(1)$_Y$ symmetry is spontaneously broken into U(1)$_\text{QED}$
%- This mechanism is spontaneous symmetry breaking
%- Simply put, the Lagrangian itself maintains the symmetry, but the state of lowest energy is not invariant and breaks the summetry.
% - Could add mass terms disregarding local gauge invariance but this would render theory unrenormalizable
\begin{figure}
  \newImageResizeCustom{0.75}{figures/theory/higgs-potential/higgs-potential.pdf}
  \caption{Illustration of the Higgs potential $V(\phi)$ defined in \cref{eq:higgspotential}. The minimum can be found on a circle in the $(\phi_1, \phi_2)$ plane.
  }
  \label{fig:higgspotential}
\end{figure}

%naturally appear in the Higgs mechanism explained below.
%An additional mechanism is needed in the SM to explain the finite masses of the weak gauge bosons. 

A principle known as \emph{spontaneous symmetry breaking}, that was first explored in the field of condensed-matter physics, can be used to generate mass terms for the weak gauge bosons without violating local gauge invariance.
This was first formulated by three independent research teams in 1964: Brout and Englert~\cite{PhysRevLett.13.321}, Higgs~\cite{PhysRevLett.13.508,HIGGS1964132}, and Guralnik, Hagan, and Kibble \cite{PhysRevLett.13.585}.\footnote{Their work was inspired by previous advancements in the theory of superconductivity~\cite{PhysRev.108.1175} and specifically P. W. Anderson, who proposed the mechanism of spontaneous symmetry breaking for generating mass terms in a non-relativistic scenario already in 1963~\cite{PhysRev.130.439}.}
Today the mechanism is most often called \emph{Higgs mechanism} or \emph{electroweak symmetry breaking} (ESWB).
The basic principle is to allow the state of lowest energy to hide local gauge invariance while maintaining the gauge symmetry of the Lagrangian itself.

% From Peskin and Schroeder
% - In the theory of sup erconductiv􏰁 ity􏰔 for example􏰔 the Ab elian gauge invariance of electromagnetism is broken by pairs of electrons that condense in the ground state of a metal􏰎
%The mechanism is known as the \emph{Brout-Englert-Higgs mechanism} (or simply \emph{Higgs mechanism}) or also \emph{electroweak symmetry breaking} (ESWB).
%The underlying concept is that the Lagrangian itself maintains the symmetry, but the state of lowest energy is not invariant and breaks the gauge symmetry.
% was published almost simultaneously by three independent groups in 1964: by Robert Brout and François Englert;[3] by Peter Higgs;[4] and by Gerald Guralnik, C. R. Hagen, and Tom Kibble.[5][6][7]

The Higgs mechanism assumes a complex scalar field of the SU(2)$_L$ group,
\begin{equation}
  \phi = \frac{1}{\sqrt{2}} \myvec { \phi_1 + i \phi_2 \\ \phi_3 + i \phi_4},
\end{equation}
and introduces a potential of the form
\begin{equation}
  V(\phi) = \mu^2\phi^\dagger\phi + \lambda \left(\phi^\dagger\phi \right)^2.
  \label{eq:higgspotential}
\end{equation}
The Lagrangian 
\begin{equation}
  \mathcal{L}_{\text{Higgs}} = |D_\mu\phi|^2 - V(\phi), % - \frac{1}{4} W_{\mu\nu}^a W^{a\, \mu\nu} - \frac{1}{4} B_{\mu\nu} B^{\mu\nu},
  \label{eq:lagrangianhiggs}
\end{equation}
where $D_\mu$ is defined as shown in \cref{eq:covdevewk}, is invariant under local SU(2)$_L$ $\times$ U(1)$_Y$ transformations.
The parameters of the potential $V(\phi)$ are specifically chosen to satisfy $\mu^2 < 0$ and $\lambda > 0$.
This choice provides the \emph{Higgs potential} with a characteristic shape, depicted in \cref{fig:higgspotential}, and gives rise to a set of degenerate ground state configurations satisfying
\begin{equation}
  |\phi| = \sqrt{ \frac{\mu^2}{2\lambda} } \equiv \frac{ v }{\sqrt{2}},
  \label{eq:higgsminima}
\end{equation}
where $v$ is the non-zero \emph{vacuum expectation value} (\emph{vev}) of the Higgs field.
%% We find that the Lagrangian for such a field
%% \begin{equation}
%% \end{equation}
%% is invariant under global SU(2) phase transformations
%It is invariant under gauge transformations but not the ground state, which can be found at
%The minimum of the potential can be found on the circle of minima where
%Once a ground state is chosen, the SU(2)$_L$ $\times$ U(1)$_Y$ symmetry becomes spontaneously broken.
% The symmetry is therefore spontaneously broken when a ground state is chosen. 
%The arbitrary choice of a ground state is said to spontaneously break the symmetry. 
Without loss of generality the ground state can be chosen to be
\begin{equation}
  \phi_0 = \frac{1}{\sqrt{2}} \myvec{0 \\ v},
  \label{eq:groundstate}
\end{equation}
i.e., $\phi_1 = \phi_2 = \phi_4 = 0$ and $\phi_3 = v$, which spontaneously breaks the SU(2)$_L$ $\times$ U(1)$_Y$ symmetry.
Expanding the field $\phi$ around the minimum to first order in the fields yields
\begin{equation}
  \phi(x) = e^{iT^a\frac{\phi_a(x)}{v}}\frac{1}{\sqrt{2}} \myvec{ 0 \\ v + h(x) },
    \label{eq:higgsexp}
\end{equation}
where the extra term $e^{iT^a\frac{\phi_a(x)}{v}}$ describes the fluctuations of the fields $\phi_1, \phi_2, \phi_4$ around the vacuum $\phi_0$.
These fields are known as the \emph{Goldstone bosons} and have no direct physical implications. They can be eliminated from the Lagrangian by choosing an appropriate gauge, the \emph{unitary gauge}, so that 
\begin{equation}
  \phi(x) = \frac{1}{\sqrt{2}} \myvec{ 0 \\ v + h(x) }.
  \label{eq:expanded-groundstate}
\end{equation}
There remains only one real physical field, $h(x)$, which is called the \emph{Higgs field} and is associated with a neutral scalar boson, the \emph{Higgs boson}, labelled $H$.
The mass of the Higgs boson follows by inserting \cref{eq:expanded-groundstate} into \cref{eq:higgspotential},
\begin{equation}
  m_H = \sqrt{2} \mu = \sqrt{2 \lambda} v.
\end{equation}
Inserting \cref{eq:higgsexp} into \cref{eq:lagrangianhiggs} and using the relations
\begin{align}
  W_\mu^\pm &= \frac{1}{\sqrt{2}} \left( W_\mu^1 \mp iW_\mu^2 \right),  \quad \text{and} \\
  \myvec{Z_\mu \\ A_\mu} &= \myvec{ \cos\theta_\text{w} \quad - \sin\theta_\text{w} \\ \cos\theta_\text{w} \quad -\sin\theta_\text{w} } \myvec{ W_\mu^3 \\ B_\mu},  
\end{align}
where $\theta_\text{w}$ is the \emph{weak mixing angle} defined as $\sin\theta_\text{w}^2 = \frac{g'^2}{g^2+g'^2}$, leads to the following terms in the Lagrangian
\begin{equation}
  \mathcal{L}_m = \frac{1}{4} \left( v + H \right)^2  \left(g^2 W_\mu^+W^{-\,\mu} + \frac{g^2}{2\cos\theta_\text{w}^2} Z_\mu Z^\mu \right).
  \label{eq:lagrangianmasses}
\end{equation}
One can identify terms quadratic in the physical fields $\Wpm$ and $Z$, generated by the non-vanishing expectation value $v$, giving rise to the masses
\begin{align}
  m_W &= \frac{vg}{2}, \\
  m_Z &= \frac{m_W}{\cos \theta_\text{w}},
  \label{eq:boson-masses}
\end{align}
of the weak gauge bosons.
The mass of the $W$ boson can be related to the Fermi constant, $G_F$, via $m_W = \frac{g}{4 * \sqrt{2}G_F}$. 
No field quadratic in $A_\mu$ appears, which reflects the fact that the photon is massless.

% DIFFERENT Formulation: It is worth while to decompose the Lagrangian into its different pieces:
The Lagrangian in \cref{eq:lagrangianmasses} can be re-written after substituting the masses of the gauge bosons,
\begin{equation}
  \mathcal{L}_{\text{H,boson-coupling}} = m_W^2 W_\mu^+W^{-\,\mu} \left( \frac{2H}{v} + \frac{H^2}{v^2} \right) + \frac{1}{2} m_Z^2 Z_\mu Z^\mu \left( \frac{2H}{v} + \frac{H^2}{v^2} \right),
  \label{eq:higgsbosoncoupling}
\end{equation}
which predicts that the interaction between the Higgs boson and the massive gauge bosons is proportional to the square of the mass of the coupled bosons and involves triplet ($V^\dagger VH$) and quartic ($V^\dagger VHH$) couplings.

%and choosing a specific gauge (the unitary gauge) the field $\phi$ can be written as
% As shown below, introducing a complex SU(2)$_L$ doublet does not only give rise to the $\Wpm$ and $Z$ boson masses, but can also be used to construct mass terms for fermions. Hence, the Higgs field couples to all massive elementary particles.
% A more detailed descriptions on experimental Higgs boson physics and an overview of the current state of knowledge is given in \cref{sec:higgsphysics}.
% The parameters of the Higgs potential $\mu^2 = \lambda v^2$ can be fixed for one combination by measuring parameters of the electroweak theory, but the physical Higgs mass cannot be predicted. 
% Moreover, the last two terms in \cref{eq:higgsselfcoupling} predict that the Higgs field couples to itself with cubic and quartic interactions.
% The couplings of the Higgs boson to the weak gauge bosons are already expressed in \cref{eq:lagrangianmasses}.
%While the Higgs boson was discovered experimentally in 2012 \cite{Aad:2012tfa,Chatrchyan:2012xdj}, and is by now seen in several different production and decay modes, the Higgs boson self-couplings are still searched for.
% In summary, adding a complex scalar Higgs field to the electroweak model, together with a potential that exhibits a non-zero vacuum expectation value, provides the ingredients to obtain mass terms for the weak gauge bosons $W^\pm$ and $Z$ when choosing a ground state of the Higgs field which spontaneously breaks the symmetry.
% The three Goldstone bosons that arise can be eliminated from the Lagrangian by using its underlying local gauge symmetry. 
% This results in three bosons acquiring a mass and the appearance of one remaining scalar field $H$. 
% The photon remains massless, because the U(1)$_{\text{QED}}$ symmetry is unbroken. 

\subsection{Fermion masses}
\label{subsec:fermion-masses}
The previous section explained how the Higgs mechanism naturally gives rise to mass terms for the $\Wpm$ and $Z$ bosons. 
%The finite fermion masses, however, do not directly follow and the simple inclusion of fermion mass terms is forbidden because they would violate SU(2)$_L$ gauge invariance.
% , given that a term
% \begin{equation}
%   -m_f \bar{\psi} \psi = -m_f \left( \bar{\psi}_R\psi_L + \bar{\psi}_L\psi_R \right)
%   \label{eq:fermionmassterm}
% \end{equation}
% because $\psi_L$ transforms as a doublet and $\psi_R$ as a singlet. 
The masses of the fermions can be explained with an ad-hoc solution by introducing interaction terms between the left-handed fermion fields and the Higgs field, both appearing as doublets under SU(2)$_L$ transformations.
%This is possible because both the left-handed fermions and the Higgs field appear as doublets under 
For leptons, only electrons, muons, taus, that appear in the lower part of the SU(2)$_L$ doublet (see \cref{tab:ewfermioncontent}), require mass terms, as neutrinos are assumed to be massless in the SM\footnote{See \cref{sec:limitations} for a brief discussion on neutrino masses.}.
For up-type quarks ($u$, $s$, and $t$ quark) to become massive, the fermion fields are coupled to the charge conjugate of the Higgs field,
\begin{equation}
  \phi^C = \frac{1}{\sqrt{2}} \myvec{v + h(x) \\ 0},
\end{equation}
after choosing a ground state similar to \cref{eq:expanded-groundstate}.
The interactions are known as \emph{Yukawa interactions} and can then be written in the form
\begin{equation}
  \mathcal{L}_\text{Yukawa} = - \sum_{i} Y_l^i \bar{\psi}^{i}_{L} \phi \psi^{i}_{R} - \sum_{ij} \left( Y_{\text{u-type}}^{ij} \bar{\psi}^{i}_{L} \phi \psi^{i}_{\text{u-type},R} + Y_{\text{d-type}}^{ij} \bar{\psi}^{i}_{L} \phi^C \psi^{j}_{\text{d-type}, R} \right) + \text{h.c.}, 
  \label{eq:lyukawa}
\end{equation}
where the first sum runs over all leptons, the second sum over all quarks, and h.c. stands for the Hermitian conjugate of the previous terms.
The left-handed fields are given as $\psi^{i}_{L}$ and include the three lepton- and quark doublets. 
The right-handed lepton fields are labelled as $\psi_R^i$ and the quark fields as $\psi^{i}_{\text{u-type},L}$, $\psi^{i}_{\text{d-type},L}$.
The labels ``u-type'' and ``d-type'' refer, respectively, to the up-type quarks ($u$, $s$, $t$) and down-type quarks ($d$, $c$, $b$). 
The \emph{Yukawa couplings} are labelled as $Y_l^i$ for leptons and as $Y_{\text{u-type}}^{ij}$ and $Y_{\text{d-type}}^{ij}$ for up-type and down-type quarks, respectively. The $Y_{\text{u-type}}^{ij}$ and $Y_{\text{d-type}}^{ij}$ are $3 \times 3$ matrices accounting for the fact that the eigenstates of the weakly interacting quarks do not correspond to their mass eigenstates. 
%A rotation of the quark fields can be performed accounting for the mixing. 
% Mass terms for quarks can be added in a similar but slightly more involved way. The down-type quarks participating in the electroweak interaction (denoted $d'$, $s'$, $b'$) are a mixture of the quark mass eigenstates (denoted $d$, $s$, $b$). 
% For each quark doublet there are two mass terms generated. For lepton doublets only one mass term is added because right-handed neutrinos are not part of the SM. Assuming only one generation of fermions, with quarks $u$ and $d$ and leptons $e$ and $\nu_e$, the Yukawa term can therefore be written as
% \begin{equation}
%   \mathcal{L}_\text{Yukawa} = - Y_{ud} \bar{\psi}_{ud,L} \phi \psi_{u,R} - Y_{ud} \bar{\psi}_{ud,L} \phi \psi_{d,R} - Y_l \bar{\psi}_{e\nu_e,L} \phi \psi_{e,R} + \text{h.c.},
%   \label{eq:lyukawa}
% \end{equation}
% where $\psi_{l}$ stands for lepton fields and $\psi_{q}$ for the quark fields.

The mass terms appear from \cref{eq:lyukawa} once a ground state of $\phi$ is chosen and the Yukawa terms have been diagonalized, resulting in nine independent parameters $Y_f$. 
They take the form
\begin{equation}
  m_f = \frac{Y_f v}{\sqrt{2}},
\end{equation}
which predicts that the coupling strength between the fermions and the Higgs field is proportional to the mass of the fermions.

%\subsection{Renormalization}
% From modern particle physics book
% As shown by ‘t Hooft, only theories with local gauge invariance are renormalisable, such that the cancellation of all infinities takes place among only a finite number of interaction terms.

\subsection{The final Standard Model Lagrangian and free parameters}
\label{subsec:final-lagrangian}
To summarize the previous sections, the final Standard Model Lagrangian is obtained from \cref{eq:lqcd,eq:lagrangianewk,eq:lagrangianhiggs,eq:lyukawa} and can be written as
\begin{align}
  \mathcal{L}_\text{SM} &= \mathcal{L}_\text{QCD} + \mathcal{L}_\text{EWK} + \mathcal{L}_\text{Yukawa} + \mathcal{L}_\text{Higgs} \\
   &= - \frac{1}{4}W_{\mu\nu}^aW^{\mu\nu}_{a} - \frac{1}{4} B_{\mu\nu}B^{\mu\nu} - \frac{1}{4}G_{\mu\nu}^aG^{\mu\nu}_{a} \\
   \label{eq:dirac-term}
   & \quad + \sum_f i \bar{\psi}_f\gamma^\mu D_\mu\psi_f \\
   & \quad - \sum_{i} Y_l^i \bar{\psi}^{i}_{L} \phi \psi^{i}_{R} - \sum_{ij} \left( Y_{\text{u-type}}^{ij} \bar{\psi}^{i}_{L} \phi \psi^{i}_{\text{u-type},R} + Y_{\text{d-type}}^{ij} \bar{\psi}^{i}_{L} \phi^C \psi^{j}_{\text{d-type}, R} \right) + \text{h.c.},  \\
   & \quad + |D_\mu\phi|^2 - \mu^2\phi^\dagger\phi + \lambda \left(\phi^\dagger\phi \right)^2
\end{align}
The covariant derivate includes all gauge fields,
\begin{equation}
  D_\mu = \partial_\mu + i g_s \frac{\lambda^a}{2} G_\mu^a + igT^aW_\mu^a + ig'\frac{Y}{2}B_\mu.
\end{equation}
The sum over $f$ in \cref{eq:dirac-term} includes all fermions, left-handed and right-handed.
% It satisfies a SU(3)$_C$ $\times$ SU(2)$_\text{L}$ $\times$ U(1)$_Y$ gauge symmetry and provides masses to the gauge bosons and fermions via the concept of electroweak symmetry breaking. This is manifested in the couplings of the Higgs field to fermions, which are proportional to the mass of the fermions, and the couplings to the gauge bosons, which are proportional to the mass squared of the gauge bosons.

% \subsection{Free parameters of the SM}
% \label{subsec:final-lagrangian}
The SM has many free parameters that need to be measured experimentally and cannot be derived from theoretical principles.  
In total there are 19 free parameters that can be represented in different ways. The most common ones are summarized below:
\begin{itemize}
  \item 9 fermion masses (or Yukawa couplings)
  \item 2 parameters describing the Higgs field: $\mu$ and $\lambda$ (or $v$ and $m_H$)
  \item 4 parameters to fully specify the CKM matrix, typically parametrized as 3 quark-mixing angles ($\theta_1$, $\theta_2$, $\theta_3$) and a CP-violating phase ($\theta_\delta$).
  \item 3 couplings constants: $\alpha$, $G_f$, $\alpha_s$ (or $g'$, $g_w$, $g_s$)
  \item 1 phase associated to CP violating terms in QCD, $\theta_{\text{QCD}}$
\end{itemize}
In total 14 parameters are associated with the Higgs field, four with the flavor sector, and only three with the gauge interactions. This underlines the special role the Higgs boson plays in the SM.




% Alternative title
%\section{The Anatomy of Proton-Proton Collision Events}
\section{The Anatomy of Proton-Proton Collision Events}
\label{sec:anatomy}
Introduce impact parameter cause it's mentioned here \cref{subsec:inner-detector}.


% \subsection{Particle colliders}
% \Minote{}{Maybe put this part in a theory, more general section (proton-proton collisions?).}
% A crucial performance indicator of a particle accelerator is its \emph{instantaneous luminosity} $\mathcal{L}$ as it is proportional to the \emph{event rate}, 
% \begin{equation}
%     \frac{\mathrm{d}N}{\mathrm{d}t} = \sigma_{P} \mathcal{L},
% \end{equation}
% for a given physics process $P$ with cross section $\sigma_{P}$.
% The instantaneous luminosity depends on parameters of the accelerator and is given by
% \begin{equation}
%   \mathcal{L} = f_rn_b\frac{N_p^2}{4\pi \sigma_x \sigma_y}, 
% \end{equation}
% when assuming Gaussian shaped beam profiles. Here, $f_r$ denotes the rotational frequency of the two bunches, $n_b$ the total number of bunches inside the accelerator, and $N_p$ the numbers of particles within each colliding bunch. 
% The total number of events after time $T$ is given after integrating with respect to time, 
% \begin{equation}
%   N_{\text{Events}} = \sigma_{P} \intLumi = \sigma_{P} \int_{0}^{T} \mathcal{L} \mathrm{d}t,
% \end{equation}
% and is a crucial quantity typically measured in high-energy-physics experiments. 
% It becomes clear that increasing the integrated luminosity \intLumi is one of the main goals of any collider experiment, as it is directly proportional to the number of expected events of a particular physics process. 

% Lumi defined by Mike:
% The luminosity measures the number of particles per unit area and time, and together with the probability of interaction (cross-section) determines the collision rate.


\subsection{Pile-up}
% Mention in Event Generation subsection, see Master thesis!

% Comment from Bernd:
%Before you can talk about "hard-scatter vertex or from pile-up”, you may have to introduce them. E.g. have a section on Hadron collider physics that goes through these terms?

\Rinote{}{Where else should I talk about pile-up!? what is pile-up? HERE! Pile-up reweighting? analysis section. Pile-up as nuisance for jet measurements? -> JER calibration chapter, pile-up in LHC section as something intrinsic to pp collision event?}

->  Ruthmann has a nice section about it!
From Sommer: "Additional in- elastic, minimum-bias like pp collisions (pile-up) are generated using Pythia8 and overlaid."
Scope:
- I should explain concepts like luminosity blocks  / bunch spacing and stuff in Data Taking Section
- Then I can explain different pile-up conditions here.
- This will be valuable to understand the noise term measurement which exactly tries to measure the noise term!
- Also look back at discussion on skype with Brian about pile-up (actual mu vs average mu and so on)

Checkout this section for pile-up overlay

https://indico.cern.ch/event/1003305/contributions/4236702/attachments/2202625/3728039/PileUpTaskForcePandPPlenaryMarch2021.pdf


%%%%%%%%%%%%%%%%%%%%%%%%%%%%%%%%%%%%%%%%%%%%%%
% NEED SECTION ON:
% Generation of MC events
% Comment from BERND to "Detector simulation" section in experimental chapter::
% I believe this statemtn "Simulations of the ATLAS detector are required in order to generate full Monte Carlo events” needs to be more precise. Why do we have to simulate collisions in the first place? Probably some discussion about quantum mechanics and its probabilistic nature which makes its way all the way to how we analyze data
\subsection{Event Generation}





\section{Higgs Boson Physics}
\label{sec:higgs-phen}
\chapter{Higgs Boson Physics}
\label{chap:higgs}
% The SM has been extremely successful in making predictions that were later confirmed by experimental measurements. Examples are the observation of the $W$ and $Z$ bosons in 1983 by the UA1 and UA2 collaborations, or the top quark discovery in 1995 by the CDF and D0 experiments. \todo{Maybe references}
% The observation of the $W$ and $Z$ bosons by the UA1 and UA2 collaborations in 1983 are just two examples of this.
% The discovery of the long sought-after Higgs boson in 2012 by the ATLAS and CMS collaborations~\cite{Aad:2012tfa,Chatrchyan:2012xdj} marks the highlight of this success story. 
% The success story culminated in the discovery of a new particle by the ATLAS and CMS collaborations in 2012~\cite{HIGG-2012-27,CMS-HIG-12-028}. To date, ten years after this discovery, the experimental evidence is overwhelming that the observed particle is the long sought-after Higgs boson predicted by the SM.

% The particle discovered in 2012~\cite{HIGG-2012-27,CMS-HIG-12-028} is now commonly referred to as the Higgs boson, since all measurements of its properties made so far are consistent with the properties of Higgs boson properties predicted by the SM.

% After the discovery of a new particle in 2012~\cite{HIGG-2012-27,CMS-HIG-12-028}, subsequent measurements showed consistency of the observed particle with the Higgs boson predicted by the SM. 
% However, the uncertainties on many of the measurements are still sizable and much remains to be understood. 
The Higgs boson plays a special role in the SM in many ways.
It is the only fundamental particle in the SM with a spin of zero and is connected to a large fraction of the free parameters (see \cref{subsec:final-lagrangian}). 
% It is responsible for the only interaction that distinguishes between the generations of fermions, and
It is the only boson that is not a consequence of the principle of local gauge invariance, but arises from the introduction of EWSB via a scalar doublet field. 

The unique nature of the Higgs boson makes it a prime candidate for searching for physics beyond the SM and thus relevant to many of the open fundamental questions~\cite{2019BHeinemann}. 
Some dark matter models, for example, predict interactions of the Higgs boson with yet unknown particles that are candidates for dark matter~\cite{Baumgart_2009,Kaplan_2009,Dienes_2012}. 
Other models such as two-Higgs-doublet models~\cite{Branco_2012}\footnote{The main motivation for considering two-Higgs-doublet models is supersymmetry as well as the fact that it introduces new potential sources for CP violation.} hypothesize the existence of additional particles similar to the Higgs boson as well as deviations from the SM predictions for the Higgs boson.
%This would provide new sources of CP violation and
%This is assumed in many supersymmetric models and hypothesizes new sources of CP violation. 

% The Higgs boson is mostly recognized as a result of the Higgs mechanism that provides masses to the $W$ and $Z$ bosons. 
The existence of the Higgs boson also solves a theoretical problem related to $WW$ scattering.
The cross section of the scattering of longitudinally polarized $W$ bosons ($W_LW_L \to W_LW_L$) would diverge at high energies in the absence of the Higgs boson. These divergencies cancel only if the coupling of the Higgs boson to the $W$ bosons is exactly as predicted by the SM.

These, among others, are important reasons for making precision measurements of the Higgs sector to test the predictions of the SM with ever-increasing precision and possibly find deviations from them. 
%This provides stringent constraints on models beyond the SM and may reveal deviations from the SM that could prove to be a gateway to the discovery of new physics. 

%This year (2022) marks the ten years anniversary of the Higgs boson discovery which is celebrated with a paper \cite{NaturePaper} that summarizes all state-of-the art Higgs measurements.
%The paper demonstrates that the observation of the Higgs in 2012 was only the beginning of an era, the era of Higgs precision measurements.
This section provides an overview of the phenomenology of Higgs boson physics. \Cref{subsec:higgschannels} summarizes the different production and decay modes of the Higgs boson at the LHC, and discusses their experimental accessibility. \Cref{subsec:xsec-measurements} outlines different types of Higgs boson measurements performed at the LHC, before \cref{subsec:higgs-exp-status} concludes with a brief overview of the experimental status of Higgs boson physics at the time of writing.
%Many more details on the status of Higgs boson physics can be found in \cref{PDG2020}.

% From PDG:
% All these BSM scenarios can have important effects on the phenomenology of the Higgs boson.


% So far, there is no indication of deviations from the SM. 
% - Properties of the Higgs boson in agreement with the spin-parity JP = 0+ predicted by the SM. 
% - Cross sections all in agreement with the SM predictions. Firmly establishing VBF production in this thesis.
% - There is no indication that the found particle is not the SM. 
%The contrary, the experimental evidence is overwhelming that the measured particle is indeed the particle predicted by the SM.

% Wikipedia
%Since the Higgs field is scalar, the Higgs boson has no spin. The Higgs boson is also its own antiparticle, is CP-even, and has zero electric and color charge.[163]
% The Standard Model spin-parity JP = 0+
% \subsection{The role of the Higgs boson in the SM}
% \todo{Maybe add such a section -> Look for resources first!}

% \subsection{Motivation for Higgs physics}
% The Higgs boson plays a special role in the SM. 
% It is the only fundamental particle with a spin of zero and connected to a large fraction of the free parameters of the SM (see \cref{subsec:final-lagrangian}). 
% It is also the only boson that is not a consequence of the principle of local gauge invariance, but is added in an ad-hoc solution in the mechanism of EWSB. 
% Furthermore, the Higgs boson is a prime candidate for searching for physics beyond the SM and thus relevant to many of the open fundamental questions \cite{2019BHeinemann}.\footnote{This includes, for example, the search for dark matter (e.g. \ccite{Baumgart_2009,Kaplan_2009,Dienes_2012}) by testing models that predict interactions of the Higgs boson with yet unknown particles that are candidates for dark matter; or investigations of the process of EWSB itself by testing two-Higgs-doublet models~\cite{Branco_2012} that predict the existence of additional particles similar to the Higgs boson.}
% %\cite{2019BHeinemann} 

% The above-mentioned provides important reasons for performing precision measurements of the Higgs sector to test the predictions of the SM at an ever greater statistical precision and thus providing stringent constraints on models beyond the SM.
% Only with a precise knowledge of Higgs processes and an understanding of what exact role the Higgs boson plays in nature, fundamental physics can come closer to answering some of the open fundamental questions.  

% Precisely measuring the properties and cross sections of the Higgs boson is needed in order to understand the exact role it plays in nature and thus getting closer to answering some of the open fundamental questions. 

% From gianotti
% o gain even deeper insights into the Higgs boson and its role in fundamental physics

%OR: Precisely measuring the properties and cross sections of the Higgs boson is needed in order to understand the exact role it plays in nature and thus getting closer to answering some of the open fundamental questions. 

% From nature
% The measurements of those production and decay rates probe the strength of the interactions between the Higgs boson and the particles involved. This allows a test of an essential prediction of the SM: that the interaction strengths scale with particle masses.

% provide stringent constraints on many models of new phenomena beyond the Standard Model.


\section{Higgs Boson Production and Decay Modes}
\label{subsec:higgschannels}
The Higgs boson directly couples to all massive particles and therefore has various production and decay modes.
The relative contributions of the various modes (or \emph{channels}) are dependent on the mass of the Higgs boson, which is determined to be $m_H = 125.38 \pm 0.14\,\GeV$ in the most precise measurement to date performed by the CMS collaboration~\cite{CMS-HIG-19-004}.
%is determined to be $m_H=125.09 \pm 0.21(stat.) \pm 0.11(syst.)\,\GeV$ . 
%This offers a rich field of experimental signatures that can be explored by experiments.

\subsubsection{Production modes} 
The Higgs boson production cross sections depend, among other things, on the center-of-mass energy of the collider.
At the LHC, the four production modes illustrated in \cref{fig:higgsprodfeyn} dominate.
Their corresponding cross sections are shown in \cref{fig:higgsprodxsec}.
% \emph{gluon fusion} (ggF), \emph{vector-boson fusion} (VBF), \emph{Higgs-strahlung} from $W$ or $Z$ boson (VH), and $t\bar{t}$ fusion (ttH).
%- as illustrated in \cref{fig:higgsprodfeyn} with a choice of Feynman diagrams.
%associated production with a $t\bar{t}$ pair (also called \emph{$t\bar{t}$ fusion})

The \emph{gluon fusion} (ggF) process is the leading mechanism at the LHC, characterized by a virtual fermion loop coupling to the Higgs boson and no additional object in the final state (therefore labelled as $pp\rightarrow H$). 
The fermions in the loop are dominated by top quarks because of their high mass and the Higgs boson couples stronger to particles with higher mass. 

%The ggF cross section is calculated with an effective theory to NNNLO in QCD and includes electroweak corrections at NLO precision \cite{Anastasiou:2016cez}.\todo{double-check}
The second-largest contribution to Higgs boson production comes from the \emph{vector-boson fusion} (VBF) process, where two incoming quarks radiate a vector boson that fuse into the Higgs.
The VBF production mode is characterized by the two quarks in the final state that are emitted in the forward directions and have a large invariant mass ($pp\rightarrow qqH$). This provides a distinct signature suitable for determining Higgs boson couplings at the LHC.
%The VBF Two quarks at large radii and with large invariant mass in the final state.
% This feature can be exploited to distinguish between collision events and select the subsequent decays of the Higgs boson. 
% In addition, the VBF production mode provides information about the couplings of the Higgs boson to the $W$ and $Z$ bosons. 
%Observation of the VBF production mode of the Higgs boson was found by a combined measurement of the ATLAS and CMS collaborations\cite{Khachatryan:2016vau}.

The \emph{Higgs-strahlung} process from $W$ or $Z$ bosons ($VH$) has an additional vector boson in the final state besides the Higgs boson ($pp \rightarrow WH$ and $pp \rightarrow ZH$), that provide distinctive detector signatures dependent on the decay mode of the vector bosons. 

The Higgs boson production in association with a top-quark pair ($ttH$) features two top quarks in the final state. It plays an important role in Higgs boson physics, since it allows to directly measure the Yukawa coupling of the top quark. 

Other production modes have smaller contributions but are still relevant in particular when being combined with other channels.
These include the production in association with a $b\bar{b}$ pair ($pp\rightarrow bbH$) or a single-top quark ($pp \rightarrow tH$) that have additional $b$ quarks or a single top quark in the final state. 

One of the most interesting production modes is the double Higgs boson production, as it provides crucial information about the Higgs potential and the Higgs boson self coupling. Double Higgs boson production is driven by the gluon fusion process but has a very low cross section and will therefore only become fully experimentally accessible with data from the High-Luminosity LHC after 2029. 
% difficult to measure due to the relatively low production cross section and the less distinct final state ($pp \rightarrow ttH$).
%  It plays an important role in Higgs boson physics, however, since it allows to directly measure the Yukawa coupling of the top quark. 
%The most recent results of the ATLAS collaboration yield evidence for the $t\bar{t}$-fusion production of the Higgs boson \cite{ATLAS-CONF-2017-077}.

\subsubsection{Decay modes}
Once the Higgs boson is produced, it decays almost instantaneously with a lifetime of about $10^{-22}$ seconds \cite{PDG2020}.
Therefore, only its decay products can be measured.
As discussed in \cref{sec:sm}, the coupling between the Higgs boson and vector bosons is predicted to be proportional to the squared mass of the vector bosons, and the Higgs boson's coupling to fermions is predicted to be proportional to the mass of the fermions. The resulting \emph{branching fractions} of the most relevant decays are displayed in \cref{fig:higgsbr}. 
The decay into a pair of bottom quarks ($H\rightarrow b\bar{b}$) is the leading process followed by the $H\rightarrow WW^*$ decay mode. Since the mass of the Higgs boson is lower than the summed mass of two $\Wpm$ or $Z$ bosons, the decay into vector bosons is suppressed and one of the decaying vector bosons appears as a virtual particle. The decay into other particles occurs significantly less often; some of the decay modes nonetheless produce detector signatures that are very suitable for studying the Higgs boson, as explained below.

\begin{figure}
  \begin{center}
    \subfloat[]{
      % use valgin if pie chart is shown for decays
      \includegraphics[trim=0 100 0 120, width=0.48\textwidth]{figures/theory/Higgs_Prod_XSec_vertLine_fixed.pdf}
      \label{fig:higgsprodxsec}
    }
    % \hspace{-5em}
    % \captionsetup[subfloat]{captionskip=50pt} % space between subfloat caption and image
    % \subfloat[]{
    %   \includegraphics[scale=0.8,valign=t]{figures/theory/h-decay-pie/h-decay-pie.pdf}
    %   \label{fig:higgsbr}
    % }
        \subfloat[]{
            \includegraphics[trim=0 100 0 120, width=0.48\textwidth]{figures/theory/SMHiggsBR_vertLine_fixed.pdf}
            \label{fig:higgsbr}
          }
  \end{center}
  \caption[Higgs boson production cross sections and decay branching fractions.]{(a) Higgs boson production cross sections as a function of the LHC center-of-mass energy and (b) branching fractions of the Higgs boson. The black vertical lines indicate (a) a center-of-mass energy of $\sqrt{s} = 13\,$TeV and (b) the currently measured mass of the Higgs boson including uncertainties by the ATLAS experiment of $m_H = 124.99 \pm 0.19\,\GeV$~\cite{https://doi.org/10.48550/arxiv.2207.00320}. Adapted from Ref.~\cite{deFlorian:2016spz}.}
  % \caption{(a) Higgs boson production cross sections as a function of the LHC center-of-mass energy. The line widths represent the respective theory uncertainties. Taken from Ref.~\cite{deFlorian:2016spz}. (b) branching fractions of the Higgs boson for a mass of 125\,\GeV. Values taken from Ref.~\cite{PDG2020}.}
  %  \label{fig:higgsbr}
\end{figure}

\captionsetup[subfloat]{captionskip=10pt} % space between subfloat caption and image
\begin{figure}
\subfloat[Vector-boson fusion (VBF), $V=W,Z$] {
    \newImageResizeCustom{0.4}{figures/feynman-graphs/Higgs/ProductionModes/VBF.pdf}
}
\subfloat[Gluon fusion (ggF)] {
    \newImageResizeCustom{0.4}{figures/feynman-graphs/Higgs/ProductionModes/ggF.pdf}
} \\
\subfloat[Higgs-strahlung (VH), $V=W,Z$] {
  \newImageResizeCustom{0.4}{figures/feynman-graphs/Higgs/ProductionModes/VH.pdf}
}
\subfloat[$t\bar{t}H$] {
  \newImageResizeCustom{0.4}{figures/feynman-graphs/Higgs/ProductionModes/ttH.pdf}
}
\caption[Feynman diagrams of Higgs boson production.]{Representative Feynman diagrams of the four main production modes of the Higgs boson at the LHC.}
\label{fig:higgsprodfeyn}
\end{figure}
\resetcaptionoffset

\subsection{Experimental sensitivity at the LHC}
\label{subsec:exp-accessibility}
% From PDG
%For a given mH, the sensitivity of a channel depends on the production cross section of the Higgs
% boson, its decay branching fraction, the reconstructed mass resolution, the selection efficiency and
% the level of background in the final state. For a low-mass Higgs boson (110 GeV < mH < 150 GeV)
% for which the SM width would be only a few MeV, five decay channels play an important role
% at the LHC. In the H → γγ and H → ZZ∗ → 4` channels, all final state particles can be
% very precisely measured and the reconstructed mH resolution is excellent (typically 1-2%). While
% the H → W+W− → ` +ν`` 0−ν¯`
% 0 channel has relatively large branching fraction, however, due
% to the presence of neutrinos which are not reconstructed in the final state, the mH resolution,
% obtained through observables sensitive to the Higgs boson mass such as the transverse mass, is poor
% (approximately 20%). The H → b ¯b and the H → τ +τ
% − channels suffer from large backgrounds and
% lead to an intermediate mass resolution of about 10% and 15% respectively.

% PDG
% In order to optimize search sensitivity and also to separate the various Higgs
% boson production modes, ATLAS and CMS split events into several mutually exclusive categories

% For phrasing from PDG
% None of the other production modes have been firmly established by the experiments
% individually. However, the table shows that, for the VBF production mode, the combination had a
% large sensitivity and produced a combined observation of 5.4σ, therefore establishing this process
% with a rate compatible with that expected in the SM.

Physics analyses typically focus on a combination of production and decay mode (referred to as \emph{signal}) to measure the number of Higgs bosons produced. The experimental sensitivity of a particular channel is not only dependent on the respective production cross section and decay branching fraction, but also highly impacted by the signature of the final state. The final state signature determines, for example, how efficient the signal can be reconstructed in the detectors and how well it can be distinguished from other non-Higgs processes considered as \emph{backgrounds}. 
%The different processes are typically first analyzed separately and then combined to obtain the most accurate measurements possible.

%and how well it can be distinguished from other types of events at hadron colliders. 
The $H\rightarrow b\bar{b}$ decay accounts for 58\% of all Higgs boson decays, but the experimental sensitivity to measure this decay mode is limited. The reason is the purely hadronic final state of $H\rightarrow b\bar{b}$ decays. 
This leads to a poor resolution of the mass of the Higgs boson due to the sizable jet energy resolution and also makes it difficult to distinguish the Higgs boson events from QCD multijet events that are extremely abundant at the LHC (see \cref{fig:xsec}).
%from the overwhelming background of QCD events at the LHC. 
The VBF or Higgs-strahlung production modes are therefore typically used for this decay channel where the presence of the additional particles in the final state can be exploited in the event selection.
%It is nonetheless possible to measure this decay by using the VBF or Higgs-strahlung production mode and 
%Using this strategy evidence for the $H\rightarrow b\bar{b}$ decay was recently found in the VH production mode \cite{Aaboud:2017xsd,Sirunyan:2017elk}.\todo{UPDATE}

% The most sensitive channels for the Higgs boson discovery in 2012 were the Higgs boson decays to two photons ($H \to \gamma\gamma$), the $H \to ZZ^* \to llll$, and \HWW\ process.
%Higgs boson initially was discovered in the decay to two photons and the $H \rightarrow ZZ^* \rightarrow llll$ channel \todo{REF}.
The decay into two photons ($H \to \gamma\gamma$) and two $Z$ bosons with a subsequent decay into leptons ($H \to ZZ^* \to llll$) have relatively small branching fractions but exhibit very clean final-state signatures due to the presence of photons and leptons. These decay products can be very precisely measured and allow for a good discrimination of the backgrounds.
In addition, the mass of the Higgs boson can be fully reconstructed in these channels by computing the invariant mass of the reconstructed decay products, providing a well-restricted mass range to measure the signal.
%at a hadron collider, where an overwhelming fraction of events is dominated by QCD effects.
%This is equally important since it allows an excellent selection of signal events and a good control over the background. 
% the largest fraction of events are dominated by QCD effects.
%QCD dominated events
%events at a hadron collider, since the final states are dominated by QCD effects.

The \HWW\ decay also provides a good handle over the backgrounds when selecting leptonically decaying $W$ bosons.
The neutrinos stemming from the $W$ boson decays, however, prevent to fully reconstruct the mass of the Higgs boson, since they cannot be directly detected by the ATLAS experiment.

% Results of the analysis of \HWW\ decays conducted by the ATLAS collaboration and CMS collaboration for data of \RunOne\ can be found in Ref.~\cite{PhysRevD.92.012006} and Ref.~\cite{2013arXiv1312.1129C}, respectively; a corresponding analysis of \RunTwo\ data is given in \cref{sec:ggfanalysis}, focusing on the gluon fusion production-mode.

Events with $H \to \tau\tau $ decays exhibit a similar final state as the \HWW\ process and especially in the case of leptonically decaying $\tau$-leptons allows for an excellent background rejection. 
% htautau paper: https://arxiv.org/abs/2201.08269

The analyses of Higgs boson decays to second generation fermions are challenging, either because of very small branching fractions, as is the case for $H \to \mu\mu$ decays, or because of an overwhelming QCD background, as is the case for $H \to c\bar{c}$ decays.
% 125 considerably more challenging measurements of Higgs boson couplings to second-generation fermions are
% explored via searches for the 𝐻 → 𝜇+𝜇 −
% [32] and, for the first time, 𝐻 → 𝑐𝑐¯ [33] decays.
% From nature:
% The considerably more challenging measurements of Higgs boson couplings to second-generation fermions are explored via searches for the 𝐻 → 𝜇+𝜇−[32] and, for the first time, 𝐻 → 𝑐𝑐¯ [33] decays. Due to the large multijet background, the latter decay mode is currently accessed only via the 𝑊𝐻 and 𝑍𝐻 production. Finally, the inputs to the combination are complemented by the latest direct searches for Higgs boson decays into invisible particles which escape the detector undetected [34, 35].
%Latest cross-section measurements can be found in \todo{REF. Not sure if it's necessary}
%A similar final state as the one in \HWW\ processes, can be exploited when looking at Higgs boson decays into $\tau$-leptons \cite{Aad:2015vsa,Sirunyan:2276465}. 
The Higgs boson decays into $Z\gamma$ ($H \to Z\gamma$) are also very rare and therefore provide limited sensitivity to perform Higgs measurements at the LHC. 
% , where leptonically decaying $Z$ bosons provide the most sensible final state, or $\mu\mu$,  % \cite{Aaboud:2017uhw,Aaboud:2017ojs}.
The final states with gluons ($H \to gg$) make up a considerable fraction of all Higgs decays but are too difficult to differentiate from other events at the LHC due to the overwhelming multijet background and the sizable jet energy resolution. 


\section{Higgs Boson Cross-Section Measurements and Interpretations}
\label{subsec:xsec-measurements}
% From PDG
% For a given mH, the sensitivity of a channel depends on the production cross section of the Higgs boson, its decay branching fraction, the reconstructed mass resolution, the selection efficiency and

% From Peskin/Schroeder
% The experiments that probe the behavior of elementary particles especially in the relativistic regime are scattering experiments. One collides two beams of particles with well dened momenta and observes what comes out The likelihood of any particular final state can be expressed in terms of the cross section, a quantity that is intrinsic to the colliding particles and therefore
% allows comparison of two dierent experiments with dierent b eam sizes and intensities


% From PDG
% In order to optimize search sensitivity and also to separate the various Higgs
% boson production modes, ATLAS and CMS split events into several mutually exclusive categories
The expected cross sections per production mode can be calculated from the SM Lagrangian, once the Higgs boson mass is fixed.
Since the total decay width of the Higgs boson ($\approx 4.1\,\MeV$~\cite{deFlorian:2016spz}) is much smaller than its mass, the narrow-width approximation holds, which allows measuring the Higgs boson cross sections as a product of the production cross section times the branching fractions of the decay modes.
For a given process with final state $f$, the number of observed Higgs boson events is typically expressed in terms of a \emph{signal strength}, defined as 
\begin{equation}
  \label{eq:signal-strength}
  \mu(pp \to H \to f) = \frac{ [ \sigma(pp \to H)  \times \BR(H \to f) ]_\text{meas} } { [ \sigma(pp \to H) \times \BR(H \to f) ]_\text{SM}},
\end{equation}
where the subscript ``meas'' (``SM'') denotes the measured value (prediction of the SM), and $B$ denotes the branching fraction. 

\subsection{Cross-section measurements}
The following briefly outlines the different types of cross-section measurements that are performed in the field of Higgs boson physics. 

% This allows, for example, measuring cross sections in exclusive regions of phase space which improves the resolution with which SM predictions can be probed.
%Differential measurements also allow for easier interpretability of the results, for example in Effective Field Theories (EFT) \todo{REF}.

\subsubsection{Inclusive production cross-section measurements}
Historically, the first measurements of the Higgs boson were based on the total number of Higgs bosons produced per production mode for individual decay channels.
These inclusive production mode cross sections are maximally dependent on theoretical assumptions related to the decay properties of the Higgs boson.
They are typically conducted for the search of a new signal (as done for the Higgs boson discovery in 2012) or to experimentally establish different production modes in the SM.
However, inclusive measurements cannot resolve small deviations from the SM predictions that may occur in regions of phase space where only a small fraction of the total produced Higgs bosons are expected.

\subsubsection{Differential fiducial cross-section measurements}
%As more data has been collected at the LHC in recent years, 
In order to probe the SM predictions for different phase space regions exclusively, Higgs boson production processes are measured differentially in various kinematic and topological variables.
Differential measurements use well-defined phase space regions, known as \emph{fiducial regions}, that allow unfolding the detector effects. 
This enables comparisons between experimental data and theory at generator level and thus minimizes the dependency on theoretical assumptions\footnote{Theoretical assumptions otherwise introduce uncertainties due to detector acceptance effects, as is the case for inclusive cross-section measurements.}.
The unfolding is facilitated by relating the expected number of events at detector level to the corresponding number at generator level. It is therefore favored to use simple discriminants for the signal extraction that are well-defined at both detector- and generator level than to use advanced analysis techniques such as neural networks. 
This limits the sensitivity of differential measurements.
% % using advanced analysis techniques for the signal extraction such as neural networks is discouraged. 
% Due to the rather complex unfolding procedure, the use of advanced analysis techniques such as neural networks is discouraged, which limits the sensitivity of differential measurements.
Another drawback is that it is not easily possible to statistically combine differential measurements with different definitions of the fiducial region.  

% - Least theory dependent
% - Not easily possible to combine measurements, as unfolding procedure very tailored to analysis selection. 

\subsubsection{Simplified template cross section measurements}
The \emph{Simplified Template Cross Sections} (STXS) framework provides a way to increase the experimental sensitivity while still allowing for differential measurements of Higgs boson production.
To achieve this, mutually exclusive kinematic regions of Higgs boson production are defined at generator level. They are known as \emph{STXS bins}. 
The extraction of the signal split in these STXS bins is then performed in reconstructed regions, which are typically aligned with the definitions of the STXS bins. No unfolding of detector effects is performed which allows using sophisticated analysis techniques like neural networks.
The STXS bins are defined based purely on the production mode and kinematics of the Higgs boson, are agnostic to the different Higgs decay channels, and are agreed upon between experiments.
These design choices allow combining Higgs boson measurements of different decay channels as well as experiments, which enables more precise cross-section measurements.
% The measurements of the production mode cross sections are performed in regions of phase space defined at the level of the fully reconstructed collision events. They are defined as similar as possible to the generator-level bins. This allows using sophisticated analysis techniques like neural networks.

Furthermore, the bins are chosen following two main principles: First, the experimental acceptance is aimed at being constant within each bin, which reduces the dependency on theoretical assumptions arising from detector acceptance effects. Second, regions that are expected to be sensitive to physics effects beyond the SM are isolated, so that they can be studied separately.
As the amount of available data increases, the STXS binning evolves in stages, each time increasing the number of bins. 
The data from \RunTwo of the LHC allows measuring cross sections partitioned in the Stage 1.2 STXS scheme, which is shown in Appendix~\ref{app:stxs-measurements-aux}.

\subsection{Coupling-strength measurements and Effective Field Theories}
\label{subsec:coupling-measurements}
Cross-section measurements only probe the production mode of the Higgs boson.
To probe the predicted strengths of the couplings of the Higgs boson to other fundamental particles, the decay processes must also be taken into account.
%A measurement of the Higgs boson's couplings therefore needs to account for both, production and decay processes.
This can be achieved at leading order using the $\kappa$ framework~\cite{LHCHandbookV3}, where the coupling strengths (or simply couplings) are measured by parametrizing the cross sections and branching fractions associated to a particle $j$ in terms of \emph{coupling-strength modifiers}, $\kappa_j^2$.
The cross section times branching fraction in the signal strength of \cref{eq:signal-strength} for a production process $i$ and final state $f$ can then be parametrized as
\begin{equation}
  \label{eq:kappa-parametrization}
  % \left( \sigma \times  \BR \right) (i \to H \to f)  = \kappa_i^2 \times  \kappa_f^2 \times  \sigma_{i}^\mathrm{SM} \times  \frac{\Gamma_f^{\mathrm{SM}}}{\Gamma_H\left(\kappa_i^2, \kappa_f^2\right) }, 
  \left( \sigma \times  \BR \right) (i \to H \to f)  =  \sigma_{i}^\mathrm{SM} \times \BR^\mathrm{SM}(H \to f) \times \frac{\kappa_i^2 \times  \kappa_f^2}{\kappa_H^2}, 
\end{equation}
where $\sigma_{i}^\mathrm{SM}$ and $\BR^\mathrm{SM}(H \to f)$ correspond to the SM expectations. 
The coupling-strength modifiers therefore parametrize deviations from the SM predictions and are unity by definition when the SM is assumed.

The $\kappa$ framework assumes that the data originate from a Higgs boson with a mass of 125\,GeV and the kinematics of both the production and decay are assumed to agree with the SM predictions for the Higgs boson.\footnote{For more details on the assumptions and the parametrization of $\kappa_H$, the reader is referred to \ccite{LHCHandbookV3}.}
The latter assumption, in particular, limits the sensitivity to models beyond the SM that may only affect the SM kinematics. 

An alternative framework to the $\kappa$ framework constists of interpretations of Higgs boson measurements in Effective Field Theories, for example within the framework of Standard Model Effective Field Theory (SMEFT)\footnote{An overview of SMEFT can, for example, be found in \ccite{Brivio_2019}.}.
In SMEFT, the effects of physics beyond the SM at large energy scales $\Lambda$ -- large compared to the Higgs vacuum expectation value ($\Lambda \gg v$) -- are parametrized at low energies, $E \ll \Lambda$, in terms of effective couplings. 
This allows for a theory-independent approach to search for deviations of the SM, and relies on fewer assumptions than the $\kappa$ framework. 

% For the VBF, \HWW process, the modifiers become
% \begin{align}
%   \label{eq:kappa-parametrization-VBF}
%   \kappa_i^2 &= \kappa_\mathrm{VBF}^2 = 0.733 \kappa^2_W + 0.267 \kappa^2_Z, \\
%   \kappa_f^2 &= \kappa_W^2, 
%   % \sigma_{\mathrm{VBF}} \cdot \BR (H \to WW)  = \kappa_\mathrm{VBF}^2 \cdot \kappa_W^2 \cdot \sigma_{\mathrm{VBF}}^\mathrm{SM} \frac{\Gamma_{WW}^\mathrm{SM}}{\Gamma_H\left(\kappa_\mathrm{VBF}^2, \kappa_{W}^2\right) },
% \end{align}
% where the parametrization of $\kappa_\mathrm{VBF}^2$ corresponds to the one used in \ccite{NaturePaper}.
% The latter shows, that a set of assumptions must be made in the $\kappa$-framework, since the couplings cannot be directly accessed. 
% Typically, different parametrizations and scenarios are tested in the couplings measurements. 
% More details left to \ccite{LHCHandbookV3}.
% Because the couplings cannot be directly measured, a set of assumptions must be made in the framework.
% For example, the framework assumes that the data originate from a Higgs boson with a mass of 125\,GeV and its interactions are exactly as predicted by the SM.
% More details on the framework are provided in \ccite{LHCHandbookV3}.
% To this end, the couplings of the Higgs boson to individual particles can be measured by scaling the cross sections and branching fractions for the individual Higgs boson processes in terms of coupling-strength modifiers, $\kappa$. 
%To this end, the $\kappa$ framework~\cite{LHCHandbookV3} parametrizes the signal strengths 
% The couplings of the Higgs boson to individual particles can be measured by parametrizing the cross sections and branching fractions for the individual Higgs boson processes in terms of coupling-strength modifiers, $\kappa$, following the $\kappa$-framework~\cite{LHCHandbookV3}. 
% \subsection{Properties of HWW Decays}
% % THIS MIGHT actually fit in the analysis section
% % as it directly related to what analysis selections we perform
% % THINK ABOUT IT!
% - VBF Higgs -> ww cross section must be exactly the SM value, otherwise VBS xsec converges!
% -> Maybe add this to HWW analysis

\section{Current Experimental Status}
\label{subsec:higgs-exp-status}
% All experimentally accessible decay modes of the Higgs have been observed, which includes $H \to \gamma\gamma$, $H \to ZZ$, $H \to WW$, $H \to \tau\tau$, and $H \to b\bar{b}$. 
After the discovery of a particle consistent with the Higgs boson predicted by the SM in 2012~\cite{HIGG-2012-27,CMS-HIG-12-028}, an era of Higgs boson precision physics has started.
The data from \RunOne\ and \RunTwo\ of the LHC allowed for high-precision measurements of several properties of the Higgs boson such as its mass, width, spin, or parity, as well as Higgs boson production and decay processes. 
At the time of writing, all experimental measurements are consistent with the predictions of the SM.
This section briefly summarizes the current experimental status, mostly focusing on results of the ATLAS collaboration.
% This allows placing the measurements presented in this thesis in a broader context. 
A more comprehensive overview can be found in \ccite{PDG2020}. 
%allowing to place the work presented in this thesis into a broader context. 

\subsection{Higgs boson properties}
% The latest combined measurement of the ATLAS and CMS experiment~\cite{HIGG-2014-14} measures a Higgs boson mass of 
% \begin{equation}
%   m_H = 125.09 \pm 0.21\text{(stat.)} \pm 0.11\text{(syst.)}\,\GeV.
% \end{equation}
The most recent measurements of the Higgs boson mass performed by the ATLAS and CMS collaboration measure a mass of $m_H = 124.99 \pm 0.19\,\GeV$~\cite{https://doi.org/10.48550/arxiv.2207.00320} and $m_H = 125.38 \pm 0.14\,\GeV$~\cite{CMS-HIG-19-004}, respectively.
Measurements of the spin and parity of the Higgs boson confirm the SM predictions of a spin-parity $J^{P} = 0^{+}$ and exclude alternative hypotheses beyond 99.9\% confidence level (CL)~\cite{HIGG-2013-17-witherratum,CMS-HIG-14-018}.
The CP-even hypothesis for the SM Higgs boson has also been probed for several interactions and all measurements are in agreement with the CP-even prediction (see e.g. \ccite{ATLAS-CONF-2022-032,Aad_2020CP,CMS-HIG-17-011}).
The total width of the Higgs boson in the SM is small ($\Gamma_H^{\text{SM}} = 4.1\,$ MeV~\cite{deFlorian:2016spz}) and therefore difficult to measure directly at hadron colliders. Indirect measurements using off-shell production of Higgs bosons result in a measured width of $\Gamma_H = 3.2 ^{+2.4}_{-1.7}\,$ MeV~\cite{https://doi.org/10.48550/arxiv.2202.06923}.

\subsection{Higgs boson production and decay processes}
All major Higgs boson production modes (ggF, VBF, $VH$, $ttH$) and Higgs boson decays to bosons ($H \to WW$, $H \to ZZ$, $H \to \gamma\gamma$) as well as third-generation fermions ($H \to \tau^+\tau^-$, $H \to b\bar{b}$) have been observed at the LHC with significances larger than $5\,\sigma$ above the background expectation~\cite{NaturePaper}.
While precision measurements are now being performed for these channels, rarer Higgs boson processes have not been unambiguously confirmed with the data available ($H \to Z\gamma$; and decays to second-generation fermions $H \to \mu^+\mu^-$, $H \to c\bar{c}$) or remain experimentally out of reach at the LHC ($H \to s\bar{s}$; decays to first-generation fermions $H \to e^+e^-$, $H \to u\bar{u}$, $H \to d\bar{d}$; and decays to gluons $H \to gg$).

A combination of several individual Higgs boson measurements using many of the above-mentioned production and decay modes of the Higgs boson was recently performed by the ATLAS collaboration~\cite{NaturePaper}. 
The combination includes the \HWW\ analysis that is presented in detail in \cref{chap:hww} of this thesis, which is why a summary of the combined measurements is left to \cref{chap:comb}. 
They provide the most precise measurements to date of the production and decay of the Higgs boson and the couplings of the Higgs boson to other fundamental particles. 

Rare decay modes of the Higgs boson begin to emerge from the data of \RunTwo\ of the LHC. 
The search for Higgs decays to a pair of muons of the ATLAS experiment found an excess of $H \to \mu\mu$ events over the background expectation of 2.0 standard deviations, where 1.7 were expected~\cite{HIGG-2019-14}.
The CMS collaboration found evidence for the $H \to \mu\mu$ decay~\cite{CMSHmumuevidence}.
An analysis of $H \to cc$ events yields an observed (expected) upper limit of 26 (31) times the SM prediction~\cite{ATLAS-CONF-2021-021}, and a search for $H \to Z\gamma$ decays results in an observed (expected) upper limit of 3.6 (2.6) times the SM prediction~\cite{HIGG-2018-42}.
An upper limit on the branching fraction of Higgs decays to particles that are invisible to the detector was set to 0.145, where 0.103 was expected~\cite{ATLASInvisible1}.
The cross section of double Higgs boson production is extremely low. The most stringent upper limits by the ATLAS collaboration were set to 4.1 times the SM prediction at 95\% CL using $HH \to b\bar{b}\gamma\gamma$ decays~\cite{ATLAS-CONF-2021-016}.
%Other search to invisible: \cite{Aad_2022}
% From nature
% Even with the current precision of measurements there is room for possible interpretations of the data in terms of new phenomena beyond the SM. 
% From nature
% Finally, the inputs to the combination are complemented by the latest direct searches for Higgs boson
% 129 decays into invisible particles which escape the detector undetected [34, 35].

\subsection{Differential measurements and interpretations in Effective Field Theories}
The \RunTwo\ data of the LHC also allowed measuring differential fiducial cross sections of Higgs boson production. Analyses targeting the $H \to ZZ$ decay~\cite{ATLAS:2020wny} and $H \to \gamma\gamma$ decay~\cite{hgammagammaDiff} measured cross sections for a variety of observables sensitive to the production and decay processes of the Higgs boson, and set constraints on effects beyond the SM.
%(THEORETICALLY there is a combination of the above here: ATLAS-CONF-2022-002)
% Measurements of Higgs boson production cross sections also serve as basis for further interpretation, for example within the framework of Standard Model Effective Field Theory (SMEFT)\footnote{An overview of SMEFT can, for example, be found in \ccite{}.}. In SMEFT, the effects of physics beyond the SM at large energy scales $\Lambda$ -- large compared to the Higgs vacuum expectation value ($\Lambda \gg v$) -- are parametrized at low energies, $E \ll \Lambda$, in terms of effective couplings. 
% This allows for a theory-independent approach to search for deviations of the SM. 

A combination of various Higgs boson cross-section measurements was interpreted in the SMEFT framework by the ATLAS collaboration~\cite{ATLAS-CONF-2020-053}. This also allowed setting constraints on new physics models
As well, the ATLAS collaboration combined the analysis of \HWW\ decays with differential cross-section measurements of $WW^*$ production in order to interpret the measured cross sections in terms of effective couplings~\cite{ATL-PHYS-PUB-2021-010}.
Both of these results pave the way for future interpretations of Higgs boson measurements in Effective Field Theories. %which become especially important in the absence of any direct discovery of new particles. 

% % The latest combined Higgs measurements performed by the ATLAS collaboration of the kinematic properties of Higgs boson production, as well as the Higgs coupling to other particles, are nicely summarized in \ccite{NaturePaper}.

% In addition to the results presented in \ccite{NaturePaper}, Higgs analyses can be interpreted in the framework of Effective Field Theories. 
% Effective field theories parametrise...
% They allow for a general parametrisation of deviations of the SM. 
% %Differential measurements also allow for easier interpretability of the results, for example in Effective Field Theories (EFT).


%%%%%%%%%%%%%%%%%%%
% Maybe not the following!
%%%%%%%%%%%%%%%%%%%

% \subsection{Cross-section measurements of \HWW decays}
% \label{subsec:prev-hwww-cross-section-meas}

% The \HWW process was first observed using data from \RunOne of the LHC~\cite{HIGG-2013-13}, and an analysis of a partial \RunTwo dataset corresponding to 36.1\,\ifb\ reported the most recent ggF and VBF \HWW measurements~\cite{HIGG-2013-13}.
% In the analysis presented here, several improvements compared to the previous \RunTwo results are implemented. Most noteworthy, the discrimination of the VBF signal is performed using a deep neural network (DNN) instead of a boosted decision tree, ggF signal events with two or more jets in the final state are included in the measurement, and measurements of cross sections in the kinematic regions defined by the STXS framework (\emph{STXS measurement}) are reported for the first time for this process.
% The analysis is published in \ccite{HWWPaper} and yields some of the most precise Higgs cross-section meausurements to date.




