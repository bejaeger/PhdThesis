
\chapter{Physics Object Reconstruction and Particle Identification}
\label{chap:objects}
Example for reference: As shown in \cref{chap:experiment}.

\section{Tracks and Vertex Reconstruction}
\section{Photons}
\section{Muons}
\section{Electrons}
\section{Jets}

\subsection{Flavor tagging}
\subsection{Jet reconstruction}

\section{Missing Transverse Energy}

\section{Pile-up}
->  Ruthmann has a nice section about it!
From Sommer: "Additional in- elastic, minimum-bias like pp collisions (pile-up) are generated using Pythia8 and overlaid."

Scope:
- I should explain concepts like luminosity blocks  / bunch spacing and stuff in Data Taking Section
- Then I can explain different pile-up conditions here.
- This will be valuable to understand the noise term measurement which exactly tries to measure the noise term!
- Also look back at discussion on skype with Brian about pile-up (actual mu vs average mu and so on)


% Caption from ATLAS (https://twiki.cern.ch/twiki/bin/view/AtlasPublic/LuminosityPublicResultsRun2#Multiple_Year_Collision_Plots)
% Number of Interactions per Crossing
% Shown is the luminosity-weighted distribution of the mean number of interactions per crossing for the 2018 pp collision data at 13 TeV centre-of-mass energy. All data recorded by ATLAS during stable beams is shown, and the integrated luminosity and the mean mu value are given in the figure. The mean number of interactions per crossing corresponds to the mean of the poisson distribution of the number of interactions per crossing calculated for each bunch. It is calculated from the instantaneous per bunch luminosity as μ=Lbunch x σinel / fr where Lbunch is the per bunch instantaneous luminosity, σinel is the inelastic cross section which we take to be 80 mb for 13 TeV collisions, and fr is the LHC revolution frequency. The luminosity shown represents the preliminary 13 TeV luminosity calibration for 2018, released in February 2019, that is based on van-der-Meer beam-separation scans. Data collected by ATLAS for the entire 2018 run through the end of October are shown.

% Total Integrated Luminosity and Data Quality in 2015-2018
% Cumulative luminosity versus time delivered to ATLAS (green), recorded by ATLAS (yellow), and certified to be good quality data (blue) during stable beams for pp collisions at 13 TeV centre-of-mass energy in 2015-2018. The complete pp data sample in 2018 is shown. The delivered luminosity accounts for the luminosity delivered from the start of stable beams until the LHC requests ATLAS to put the detector in a safe standby mode to allow a beam dump or beam studies. The recorded luminosity reflects the DAQ inefficiency, as well as the inefficiency of the so‐ called "warm start": when the stable beam flag is raised, the tracking detectors undergo a ramp of the high-voltage and, for the pixel system, turning on the preamplifiers. The data quality assessment shown corresponds to the All Good efficiency shown in the 2015-2018 Full Dataset DQ tables here. The All Good Data Quality criteria require all reconstructed physics objects to be of good data quality.