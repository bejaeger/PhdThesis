\chapter*{Abstract}
% The discovery of a particle consistent with the Higgs boson of the Standard Model (SM) by the ATLAS and CMS experiments in 2012 was a milestone in the history of particles physics. 
% % maybe mention that by now more data is available
% The Higgs boson is at the core of the SM and related to several of the open fundamental questions that the SM cannot answer. 
% With the increasing number of proton-proton collisions delivered by the LHC and recorded by its experiments, precision measurements of the properties of the Higgs boson including its interactions with other fundamental particles can now be performed. This is crucial to probe the SM predictions and may reveal signs of new physics phenomena, opening doors to an improved understanding of nature. 
% With the increasing amount of proton-proton collisions recorded with the experiments located at the LHC
% This is considered one of the most promising paths to make progress in the search for the fundamental laws of nature.
% After the discovery of a particle consistent with the Higgs boson of the Standard Model (SM) of particle physics by the ATLAS and CMS experiments in 2012, an era of exploration of the properties of the Higgs boson began.
% The Higgs boson is at the core of the SM and is connected to many of the open fundamental questions the SM cannot answer.
The Higgs boson is a unique tool in the search for the fundamental laws of nature, as it is connected to many of the open fundamental questions the Standard Model (SM) of particle physics cannot answer.
Precision measurements of the properties of the Higgs boson, including its interactions with other fundamental particles, provide a powerful tool to test the predictions of the SM and possibly find deviations from them.
% The study of the Higgs boson is crucial for testing the predictions of the Standard Model (SM) of particle physics. 
% The Higgs boson sits at the core of the SM, which is known to be incomplete or merely an approximation of a more fundamental theory. 
% It sits at the core of the SM, which is known to be incomplete, or merely an approximation of a more fundamental theory. 
% For this purpose, the Large Hadron Collider produced proton-proton collisions with a center-of-mass energy of $\sqrt{s} = 13\,$TeV between 2015 and 2018 that were recorded by the ATLAS experiment, providing a dataset corresponding to an integrated luminosity of 139\,\ifb.
% This dataset provides unprecedented opportunities to measure precisely the properties of the Higgs boson including its interactions with other fundamental particles.
% The \HWW\ analysis is part of the Higgs precision program that aims at establishing a precise experimental map of both Higgs boson production and decay processes and the couplings of the Higgs boson to other particles.
% There is a rich phenomenology of final states of Higgs boson events. 
% Measuring all the interactions precisely is one of the major goals of current experimental particle physics. 
% The measurement is performed using proton-proton collision data with a center-of-mass energy of $\sqrt{s} = 13\,$TeV delivered by the LHC between 2015 and 2018. The data were recorded by the ATLAS experiment and correspond to an integrated luminosity of 139\,\ifb. 
% The proton-proton collision data with a center-of-mass energy of $\sqrt{s} = 13\,$TeV delivered by the LHC between 2015 and 2018, recorded by the ATLAS experiment and corresponding to an integrated luminosity of 139\,\ifb, are used to study various combinations of Higgs boson production and decay modes.
% The proton-proton ($pp$) collision data delivered by the Large Hadron Collider (LHC) provide excellent opportunities to study the various Higgs boson production and decay processes.

As part of a broad Higgs boson physics program at the Large Hadron Collider (LHC), this thesis presents cross-section measurements of Higgs boson production via gluon fusion (ggF) and vector-boson fusion (VBF) in decays to $W$ bosons.
The \HWW decay is the second most likely decay of the Higgs boson and, in the VBF production mode, the most sensitive channel to measure the coupling of the Higgs boson to vector bosons at the LHC.
The measurement is based on $pp$ collisions at a center-of-mass energy of $\sqrt{s} = 13\,$TeV recorded by the ATLAS experiment at the LHC between 2015 and 2018, corresponding to an integrated luminosity of 139\,\ifb.
The measurement of the inclusive ggF and VBF cross sections times branching fraction result in $12.0 \pm 1.4~\mathrm{pb}$ and $0.75\;^{+0.19}_{-0.16}~\mathrm{pb}$, respectively.
In addition, Higgs boson production is measured in 11 exclusive kinematic regions. All results are found to be consistent with their corresponding SM predictions.
The analysis is also an important input to combined Higgs boson measurements, which provide some of the most precise measurements of Higgs boson interactions to date and are briefly summarized in this work.

The measurement of the VBF, \HWW process is drastically improved over previous results by the implementation of a binary classifier based on a deep neural network (DNN) that distinguishes the VBF, \HWW signal from other physical processes. 
The development and optimization of the DNN are presented in this thesis. 
% The VBF production mode of the Higgs boson is characterized by two energetic jets emitted in the forward directions of the detector.
% The abundance of jets in proton-proton collisions makes it very important to measure precisely the jet energy resolution (JER) to calibrate the energy of the reconstructed jets.
This thesis also presents the measurement of the jet energy resolution, which is essential for many physics analyses performed with the ATLAS experiment, such as the \HWW analysis, due to the abundance of jets in $pp$ collisions.

% which is essential for many physics analyses performed with the ATLAS experiment, such as the \HWW analysis.

% This thesis also presents the measurement of the effect of multiple proton-proton collisions and detector noise on the JER used to calibrate the energy of reconstructed jets, which is essential for many physics analyses performed with the ATLAS experiment. 

% A precise knowledge of the jet energy resolution (JER) is therefore crucial. The measurement of the effect of multiple $pp$ collisions and detector noise on the JER is presented in this thesis.

% and results in a distinct detector signature that can be distinguished from signatures produced by other physical processes considered as backgrounds. 
% The proton-proton collisions were delivered by the Large Hadron Collider between 2015 and 2018 and recorded by the ATLAS detector, corresponding to an integrated luminosity of 139\,\ifb. 
% H->WW second most likely
% VBF, H->WW most sensitive to Higgs to vector boson coupling
% Cross section results
% xsec measurements in 11 kinematic regions
% All results consistent with the SM, provide input to combined measurements with other Higgs boson analyses. 
