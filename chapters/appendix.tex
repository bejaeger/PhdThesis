\chapter{Development of the VBF DNN}
\label{app:chap:DNN}

\section{Software suite for the development of the VBF DNN}
\label{app:dnn:software-suite}
The software used for the development of the DNN is based on industry-standard state-of-the-art ML libraries. 
The entire workflow is based on Docker images that provide the necessary packages that are all built with a Python frontend. The simulated MC samples provided centrally within the ATLAS collaboration are first transformed in order to remove the ATLAS software dependencies and make them easily useable with open-source ML libraries. The data is then stored in hdf5 format, and handled with the numpy and pandas packages. The training is performed using Keras and TensorFlow. The scikit-learn library is included in the training, and the matplotlib package is used for data visualization. The final DNN model is stored in JSON format and deployed in the \HWW\ analysis using the C++ based LightWeight Tagger Neural Network (lwtnn) package [148].
\todo{REFERENCES!}
\todo{Mention FreeForestML}


\section{Optimization studies for the development of the VBF DNN}
\label{app:dnn:opt-studies}

Different optimization studies for the VBF DNN were performed during the course of the author's PhD thesis.
The following studies were performed with an earlier version of the DNN training setup, including an earlier version of the training data. For this reason, the absolute values of $Z0$(40 bins) are not necessarily comparable with other results presented in this thesis. However, they are perfectly suitable for drawing conclusions when being compared against each other.

Many sets of observables were studied for use as DNN input variables.
As baseline, the set of variables that was used in the BDT-based analysis of the previous iteration of the \HWW\ measurement \cite{HIGG-2016-07} is chosen, comprising in total 8 variables.
The most important comparisons of the performances of models using an increasing number of variables are shown in \cref{tab:input-var-opt}.
Comparing the baseline to the results labelled ``S1'' shows significant improvements. This is an indication that the DNN exploits the correlations between the single $m_{\ell_\alpha j_\beta}$ (with $\alpha, \beta = 1, 2$) and other observables. 
Although the discrimination power of \pTjone and \pTjtwo in linear dimension is limited, they also introduce a significant performance improvement when being included (``S2''). The same is true for the \pTjthree observable (``S3''), which adds information about whether a third jet is present in the event. 
The final variable that was added and showed benefits is \METSig (``S4''). This observable has only recently been developed for use in the ATLAS collaboration \cite{ATLAS-CONF-2018-038} and is a strong discriminant between events with real undetected high-\pT particles and events where the \MET is the result of resolution effects. 
Overall the significance metric $Z0$(40 bins) improves by a maximum of almost 50\% when comparing the baseline with the best performing set.
These results motivate the use of the 15 input variables that correspond to the set labelled ``S4'' in \cref{tab:input-var-opt}.
In addition to the comparisons shown, several other observables were studied but did not lead to further improvements. 
These tests included (i) using the $C_\ell$ observable separately instead of the sum for both leptons, (ii) using \MET instead of \METSig, and (iii) using the \pT of both leptons on top of the other 15 variables.
Furthermore, the performance of DNN models trained with all 15 variables except one was studied. It was seen, that all of the chosen 15 variables provide discrimination power, although dropping highly correlated observables such as \mjj and \dyjj did not show strong performance losses.

The physics-motivated optimization of the input variables was followed by the rather technical optimization of the neural network architecture. 
The performances of different setups are shown in \cref{tab:architecture-opt}. All results are very similar to each other. Among the best performing architectures, the one with the least layers is chosen, which corresponds to the architecture labelled ``A4''. The results were produced with the set of variables labelled ``S3'' in \cref{tab:input-var-opt}

\begin{table}[h]
    \centering
    \small
    \begin{tabular}{ l l | r}
        \toprule
        Identifier & Input Variables                                                                                         & $Z0$(40 bins) \\
        \midrule
        Baseline   & \mjj, \dyjj, \lepetacent, \dphill, \mll, \mT, \pttot, $\sum_{\alpha,\beta=1,2} m_{\ell_\alpha j_\beta}$ & 7.7           \\
        S1         & Baseline w/ separate \mlonejone, \mlonejtwo, \mltwojone, \mltwojtwo,                                    & 8.57          \\
        S2         & S1 + \pTjone, \pTjtwo                                                                                   & 9.55          \\
        S3         & S2 + \pTjthree                                                                                          & 10.92         \\
        S4  & S3 + \METSig                                                                                            & 11.5          \\
        \bottomrule
    \end{tabular}
    \caption{Comparison of the significance metric $Z0$(40 bins) for DNN models trained with different input variables.}
    \label{tab:input-var-opt}
\end{table}

\begin{table}[h]
    \centering
    \small
    \begin{tabular}{ l l | r}
        \toprule
        Identifier & Hidden layers                      & $Z0$(40 bins) \\
        \midrule
        A1         & {32, 16, 8}                        & 10.4          \\
        A2         & {64, 32, 24, 16, 8}                & 10.6          \\
        A3         & {128, 64, 32, 24, 16, 8}           & 10.7          \\
        A4      & {256, 128, 64, 32, 24, 16, 8}      & 10.8          \\
        A5         & {128, 128, 64, 64}                 & 10.5          \\
        A6         & {512, 256, 128, 64, 32, 24, 16, 8} & 10.8          \\
        \bottomrule
    \end{tabular}
    \caption{Comparison of the significance metric $Z0$(40 bins) for DNN models using different architectures.}
    \label{tab:architecture-opt}
\end{table}


\section{Optimization of the EW $WW$ training fraction}
\label{app:sec:ewww-sample-fraction-optimization}
In a previous iteration of the VBF, \HWW\ analysis it was found that theoretical uncertainties related to particular processes dominate the final measurement uncertainties. In particular EW $WW$ processes that mimic the VBF signal signature contributed significantly in the highest DNN bin, which lead to large uncertainties.
This prompted a change in the training procedure that places more emphasis on suppressing the EW $WW$ background by increasing the training fraction of the EW $WW$ sample in the training.
In addition, the performance metric $Z0$(var. bins + syst. unc.) (see \cref{subsec:performance-metrics}) was introduced to take into account theoretical uncertainties on the different processes when comparing different DNN trainings.

The results of this final optimization of the VBF DNN are shown by means of the expected number of events in the highest DNN bin in \cref{fig:bkg-fractions}. The contributions for each process without and with an increased training fraction for the EW $WW$ processes are displayed.
The goal of this optimization is to achieve approximately equal number of expected events in the highest DNN bin for the dominant background processes with large uncertainties (which are ggF, EW $WW$, $WW$, and \ttbar, see \cref{tab:DNNtrainingstats}).
This is expected to prove beneficial when combining the uncertainties in the statistical analysis, compared to a scenario where one of the backgrounds dominates.
%Due to this change, the final measurement uncertainties of the VBF, \HWW\ production cross-section were reduced by roughly 5\%. 
The details of this study that lead to the final choice of the EW $WW$ training fraction are visualized in \cref{fig:ew-fraction-scan}.
It can be seen that the EW $WW$ fraction of the total background in the highest DNN bin decreases as the training fraction used for EW $WW$ processes in the training is increased.
The training with an EW $WW$ training fraction of 0.05 roughly corresponds to the values shown in \cref{fig:bkg-fraction-a}.
In the range of [0.06-0.12], the significance metrics that do not take into account theory uncertainties show a slight downward trend. The $Z0$(var. bins + syst. unc.) metric, on the other hand, shows a very subtle upward trend, indicating a benefit of suppressing the EW $WW$ content. While this metric does not cover all aspects that determine the final measurement uncertainties, it helps to select the model that performs best in the final statistical analysis and therefore provides an improved measure over the more simple metrics that do not account for theory uncertainties.

\begin{table}[h]
    \centering
    \small
    \begin{tabular}{ c  | c}
        \toprule
        Background sample   & $\sigma^\text{rel}_\text{approx}$ \\
        \midrule
        $H_{\mathrm{VBF}}$  & 0.3                               \\
        $H_{\mathrm{ggF}}$  & 0.5                               \\
        $t\bar{t}$          & 0.3                               \\
        $Wt$                & 0.5                               \\
        $WW$ (Strong)       & 0.3                               \\
        $WW$ (EW)           & 0.5                               \\
        $Z/\gamma*$         & 0.25                              \\
        $V\gamma$           & 1                                 \\
        Other $VV$          & 0.12                              \\
        \bottomrule
    \end{tabular}
    \caption{Relative systematic uncertainty $\sigma^\text{rel}_\text{approx}$ assumed on the different processes for constructing a metric.}
    \label{tab:rough-uncertainties}
\end{table}

% Significance Z0:
% hww_syst_unc = {"Vgamma": 1, "otherVV":0.12, "Zjets": 0.25, "WW":0.3, "EWWW": 0.5, "singletop":0.5, "ttbar":0.3, "ggF":0.5, "VBF":0.3}

\begin{figure}[t]
    \subfloat[No optimized training fractions] {
        \includegraphics[width=0.49\textwidth,trim=35 0 0 0]{figures/plots/bkg-fraction-highest-bin/pie-chart-fractions-old.pdf}
        \label{fig:bkg-fractions-a}
    }
    \subfloat[Optimized EW $WW$ training fractions] {
        \includegraphics[width=0.49\textwidth,trim=35 0 0 0]{figures/plots/bkg-fraction-highest-bin/pie-chart-fractions-new.pdf}
        \label{fig:bkg-fractions-b}
    }
    \caption{Fraction of background in highest DNN output bin based on the validation set. }
    \label{fig:bkg-fractions}
\end{figure}

% Plots made with SFUsMLKit on cedar with:
% ./plot.py -c configs/HWW/winningSubmission.cfg --trainingFolderName dropout-02-5-fold-aggressive-lr-schedule-2-fine-scan-ewww-scan-etafix/210805_9_0.12_8226149877761426764
\begin{figure}[t]
    % \subfloat[] {
    %     \newImageResizeCustom{0.47}{figures/plots/sample-fractions/sig_vs_lrate.pdf}
    % }
    % \subfloat[] {
    \newImageResizeCustom{0.5}{figures/plots/sample-fractions/sig_vs_ew_fraction_lr9.pdf}
    % }
    \caption{}
    \label{fig:ew-fraction-scan}
\end{figure}



\FloatBarrier
\section{Distributions of the input variables in bins of the VBF DNN}
\label{app:dnn:input-vars}

\Cref{fig:dnn-inputs-vbf-top1,fig:dnn-inputs-vbf-top2,fig:dnn-inputs-vbf-top3,fig:dnn-inputs-hwwdecay,fig:dnn-inputs-top-sup} show distributions of the DNN input variables with different selections on the DNN output. This shows the effect of the DNN on the variables. 

\captionsetup[subfloat]{captionskip=5pt} % space between subfloat caption and image
\begin{figure}[ht]
    \centering
    \subfloat[$\mjj$]{
        \includegraphics[width=0.32\textwidth]{figures/hww/dnn/blinded/run2-emme-CutVBF_SR-Mjj-lin.pdf} \hfill
        \includegraphics[width=0.32\textwidth]{figures/hww/dnn/blinded/run2-emme-CutVBFSR_DNN25-Mjj-lin.pdf} \hfill
        \includegraphics[width=0.32\textwidth]{figures/hww/dnn/blinded/run2-emme-CutVBFSR_DNN87-Mjj-lin.pdf}
    } \\
    \subfloat[$\dyjj$]{
        \includegraphics[width=0.32\textwidth]{figures/hww/dnn/blinded/run2-emme-CutVBF_SR-DYjj-lin.pdf} \hfill
        \includegraphics[width=0.32\textwidth]{figures/hww/dnn/blinded/run2-emme-CutVBFSR_DNN25-DYjj-lin.pdf} \hfill
        \includegraphics[width=0.32\textwidth]{figures/hww/dnn/blinded/run2-emme-CutVBFSR_DNN87-DYjj-lin.pdf}
    } \\
    \subfloat[$\lepetacent$]{
        \includegraphics[width=0.32\textwidth]{figures/hww/dnn/blinded/run2-emme-CutVBF_SR-contOLV-lin.pdf} \hfill
        \includegraphics[width=0.32\textwidth]{figures/hww/dnn/blinded/run2-emme-CutVBFSR_DNN25-contOLV-lin.pdf} \hfill
        \includegraphics[width=0.32\textwidth]{figures/hww/dnn/blinded/run2-emme-CutVBFSR_DNN87-contOLV-lin.pdf}
    } \\
    {\caption{Distributions of $\dphill$, $\mll$, $\lepetacent$ in the VBF signal region.
        Each row corresponds to one variable with different selections made on the DNN output.
        \label{fig:dnn-inputs-vbf-top1} }}
\end{figure}


\begin{figure}[h]
    \centering
    \subfloat[$\mlonejtwo$]{
        \includegraphics[width=0.32\textwidth]{figures/hww/dnn/blinded/run2-emme-CutVBF_SR-Ml0j0-lin.pdf} \hfill
        \includegraphics[width=0.32\textwidth]{figures/hww/dnn/blinded/run2-emme-CutVBFSR_DNN25-Ml0j0-lin.pdf} \hfill
        \includegraphics[width=0.32\textwidth]{figures/hww/dnn/blinded/run2-emme-CutVBFSR_DNN87-Ml0j0-lin.pdf}
    } \\
    \subfloat[$\mltwojone$]{
        \includegraphics[width=0.32\textwidth]{figures/hww/dnn/blinded/run2-emme-CutVBF_SR-Ml1j0-lin.pdf} \hfill
        \includegraphics[width=0.32\textwidth]{figures/hww/dnn/blinded/run2-emme-CutVBFSR_DNN25-Ml1j0-lin.pdf} \hfill
        \includegraphics[width=0.32\textwidth]{figures/hww/dnn/blinded/run2-emme-CutVBFSR_DNN87-Ml1j0-lin.pdf}
    } \\
    \subfloat[$\mlonejtwo$]{
        \includegraphics[width=0.32\textwidth]{figures/hww/dnn/blinded/run2-emme-CutVBF_SR-Ml0j1-lin.pdf} \hfill
        \includegraphics[width=0.32\textwidth]{figures/hww/dnn/blinded/run2-emme-CutVBFSR_DNN25-Ml0j1-lin.pdf} \hfill
        \includegraphics[width=0.32\textwidth]{figures/hww/dnn/blinded/run2-emme-CutVBFSR_DNN87-Ml0j1-lin.pdf}
    } \\
    \subfloat[$\mltwojtwo$]{
        \includegraphics[width=0.32\textwidth]{figures/hww/dnn/blinded/run2-emme-CutVBF_SR-Ml1j1-lin.pdf} \hfill
        \includegraphics[width=0.32\textwidth]{figures/hww/dnn/blinded/run2-emme-CutVBFSR_DNN25-Ml1j1-lin.pdf} \hfill
        \includegraphics[width=0.32\textwidth]{figures/hww/dnn/blinded/run2-emme-CutVBFSR_DNN87-Ml1j1-lin.pdf}
    } 
    {\caption{Distributions of $\mlonejone$, $\mltwojone$, $\mlonejtwo$, and $\mltwojtwo$ in the VBF signal region.
            Each row shows one variable with different cuts on the DNN output distribution being applied in different columns.
            \label{fig:dnn-inputs-vbf-top2} }}
\end{figure}


\begin{figure}[h]
    \centering
    \subfloat[$\pTjone$]{
        \includegraphics[width=0.32\textwidth]{figures/hww/dnn/blinded/run2-emme-CutVBF_SR-leadJetPt-lin.pdf} \hfill
        \includegraphics[width=0.32\textwidth]{figures/hww/dnn/blinded/run2-emme-CutVBFSR_DNN25-leadJetPt-lin.pdf} \hfill
        \includegraphics[width=0.32\textwidth]{figures/hww/dnn/blinded/run2-emme-CutVBFSR_DNN87-leadJetPt-lin.pdf}
    } \\
    \subfloat[$\pTjtwo$]{
        \includegraphics[width=0.32\textwidth]{figures/hww/dnn/blinded/run2-emme-CutVBF_SR-subleadJetPt-lin.pdf} \hfill
        \includegraphics[width=0.32\textwidth]{figures/hww/dnn/blinded/run2-emme-CutVBFSR_DNN25-subleadJetPt-lin.pdf} \hfill
        \includegraphics[width=0.32\textwidth]{figures/hww/dnn/blinded/run2-emme-CutVBFSR_DNN87-subleadJetPt-lin.pdf}
    } \\
    \subfloat[$\pTjthree$]{
        \includegraphics[width=0.32\textwidth]{figures/hww/dnn/blinded/run2-emme-CutVBF_SR-thirdJetPt-lin.pdf} \hfill
        \includegraphics[width=0.32\textwidth]{figures/hww/dnn/blinded/run2-emme-CutVBFSR_DNN25-thirdJetPt-lin.pdf} \hfill
        \includegraphics[width=0.32\textwidth]{figures/hww/dnn/blinded/run2-emme-CutVBFSR_DNN87-thirdJetPt-lin.pdf}
    } 
    {\caption{Distributions of $\pTjone$, $\pTjtwo$, and $\pTjthree$ in the VBF signal region.
            Each row shows one variable with different cuts on the DNN output distribution being applied in different columns.
            \label{fig:dnn-inputs-vbf-top3}}}
\end{figure}


\begin{figure}[h]
    \centering
    \subfloat[$\dphill$]{
        \includegraphics[width=0.32\textwidth]{figures/hww/dnn/blinded/run2-emme-CutVBF_SR-DPhill-lin.pdf} \hfill
        \includegraphics[width=0.32\textwidth]{figures/hww/dnn/blinded/run2-emme-CutVBFSR_DNN25-DPhill-lin.pdf} \hfill
        \includegraphics[width=0.32\textwidth]{figures/hww/dnn/blinded/run2-emme-CutVBFSR_DNN87-DPhill-lin.pdf}
    } \\
    \subfloat[$\mll$]{
        \includegraphics[width=0.32\textwidth]{figures/hww/dnn/blinded/run2-emme-CutVBF_SR-Mll-lin.pdf}
        \includegraphics[width=0.32\textwidth]{figures/hww/dnn/blinded/run2-emme-CutVBFSR_DNN25-Mll-lin.pdf}
        \includegraphics[width=0.32\textwidth]{figures/hww/dnn/blinded/run2-emme-CutVBFSR_DNN87-Mll-lin.pdf}
    } \\
    \subfloat[$\mT$]{
        \includegraphics[width=0.32\textwidth]{figures/hww/dnn/blinded/run2-emme-CutVBF_SR-MT-lin.pdf} \hfill
        \includegraphics[width=0.32\textwidth]{figures/hww/dnn/blinded/run2-emme-CutVBFSR_DNN25-MT-lin.pdf} \hfill
        \includegraphics[width=0.32\textwidth]{figures/hww/dnn/blinded/run2-emme-CutVBFSR_DNN87-MT-lin.pdf}
    } 
    {\caption{Distributions of $\dphill$, $\mll$, $\mT$ in the VBF signal region.
        Each row corresponds to one variable with different selections made on the DNN output.
        \label{fig:dnn-inputs-hwwdecay} }}
\end{figure}



\begin{figure}[h]
    \centering
    \subfloat[$\pttot$]{
        \includegraphics[width=0.32\textwidth]{figures/hww/dnn/blinded/run2-emme-CutVBF_SR-PtTot-lin.pdf} \hfill
        \includegraphics[width=0.32\textwidth]{figures/hww/dnn/blinded/run2-emme-CutVBFSR_DNN25-PtTot-lin.pdf} \hfill
        \includegraphics[width=0.32\textwidth]{figures/hww/dnn/blinded/run2-emme-CutVBFSR_DNN87-PtTot-lin.pdf}
    } \\
    \subfloat[$\METSig$]{
        \includegraphics[width=0.32\textwidth]{figures/hww/dnn/blinded/run2-emme-CutVBF_SR-METSig_broad-lin.pdf} \hfill
        \includegraphics[width=0.32\textwidth]{figures/hww/dnn/blinded/run2-emme-CutVBFSR_DNN25-METSig_broad-lin.pdf} \hfill
        \includegraphics[width=0.32\textwidth]{figures/hww/dnn/blinded/run2-emme-CutVBFSR_DNN87-METSig_broad-lin.pdf}
    } \\
    {\caption{Distributions of $\pttot$ and $\METSig$ in the VBF signal region.
            Each row shows one variable with different cuts on the DNN output distribution being applied in different columns.
            \label{fig:dnn-inputs-top-sup} }}
\end{figure}




\FloatBarrier
\chapter{Auxiliary material for \HWW analysis}

\FloatBarrier
\section{Cutflows}
\label{app:cutflows}

\begin{landscape}
    \thispagestyle{empty} 
    \begin{table}
        \subfloat[]{
        \resizebox{\textwidth}{!}{
        %%% created on Wed Jun 15 01:04:43 2022 from TQSampleFolder 'samples' with TQLibrary UNKNOWN compiled with GCC 8.3.0 against ROOT 6.18/04
\providecommand{\xmark}{{\sffamily \bfseries X}}
\providecommand\rotatecell[2]{\rotatebox[origin=c]{#1}{#2}}
\begin{tabular}{ r || r  r  r | r  r  r  r  r  r | r  r }
\ensuremath{\sqrt{s}=13\,}TeV, 139 fb$^{-1}$ & $H_{VBF}$ & $H_{ggF}$ & Other $H$ & Top & $WW$ (Strong) & $WW$ (EW) & $Z/\gamma^{*}$ & Mis-Id & Other $VV$($V$) & Total Bkg & Data\tabularnewline
\hline
Preselection & \ensuremath{597.44\pm 0.74} & \ensuremath{5626.30\pm 7.79} & \ensuremath{2098.36\pm 2.72} & \ensuremath{1158767.99\pm 236.35} & \ensuremath{126397.14\pm 115.16} & \ensuremath{1067.67\pm 2.02} & \ensuremath{255778.21\pm 429.05} & \ensuremath{32969.04\pm 243.45} & \ensuremath{20099.62\pm 103.68} & \ensuremath{1600705.98\pm 568.59} & \ensuremath{1585575}\tabularnewline
\hline
\TwoJet & \ensuremath{366.95\pm 0.58} & \ensuremath{1061.65\pm 3.08} & \ensuremath{769.79\pm 1.39} & \ensuremath{916426.43\pm 203.00} & \ensuremath{24253.86\pm 30.33} & \ensuremath{879.26\pm 1.83} & \ensuremath{26050.68\pm 106.57} & \ensuremath{11627.40\pm 161.67} & \ensuremath{5216.69\pm 49.46} & \ensuremath{985515.98\pm 286.50} & \ensuremath{975285}\tabularnewline
$\Nbjetsub=0$ & \ensuremath{324.06\pm 0.54} & \ensuremath{914.40\pm 2.86} & \ensuremath{445.15\pm 1.16} & \ensuremath{64086.49\pm 57.56} & \ensuremath{20989.02\pm 28.72} & \ensuremath{761.52\pm 1.69} & \ensuremath{21857.05\pm 100.95} & \ensuremath{3836.65\pm 68.22} & \ensuremath{4115.48\pm 45.38} & \ensuremath{116560.61\pm 145.10} & \ensuremath{109428}\tabularnewline
$\mtt\,{<}\,m_Z - 25~\GeV$ & \ensuremath{278.43\pm 0.50} & \ensuremath{798.62\pm 2.68} & \ensuremath{198.24\pm 0.76} & \ensuremath{39059.60\pm 44.87} & \ensuremath{12209.30\pm 22.70} & \ensuremath{391.10\pm 1.21} & \ensuremath{7756.85\pm 70.31} & \ensuremath{2378.72\pm 51.42} & \ensuremath{2206.51\pm 36.12} & \ensuremath{64800.70\pm 106.91} & \ensuremath{61311}\tabularnewline
$\mjj > 120\ \GeV$ & \ensuremath{254.30\pm 0.48} & \ensuremath{488.43\pm 2.07} & \ensuremath{97.17\pm 0.53} & \ensuremath{26982.66\pm 36.93} & \ensuremath{8209.18\pm 17.75} & \ensuremath{335.95\pm 1.12} & \ensuremath{5105.96\pm 56.93} & \ensuremath{1386.09\pm 40.53} & \ensuremath{1494.00\pm 26.94} & \ensuremath{44002.26\pm 85.41} & \ensuremath{41466}\tabularnewline
CJV & \ensuremath{235.21\pm 0.46} & \ensuremath{416.07\pm 1.92} & \ensuremath{76.90\pm 0.48} & \ensuremath{21266.00\pm 33.12} & \ensuremath{6822.28\pm 16.77} & \ensuremath{291.77\pm 1.05} & \ensuremath{4334.52\pm 54.34} & \ensuremath{1092.99\pm 36.55} & \ensuremath{1221.76\pm 24.36} & \ensuremath{35445.39\pm 79.15} & \ensuremath{33802}\tabularnewline
OLV & \ensuremath{203.70\pm 0.43} & \ensuremath{201.86\pm 1.36} & \ensuremath{31.60\pm 0.31} & \ensuremath{7697.63\pm 20.03} & \ensuremath{2186.05\pm 10.73} & \ensuremath{196.98\pm 0.86} & \ensuremath{1686.28\pm 42.33} & \ensuremath{355.21\pm 21.82} & \ensuremath{437.44\pm 14.94} & \ensuremath{12761.45\pm 54.87} & \ensuremath{12189}\tabularnewline
\hline
VBF SR, DNN output $\leq 0.25$ & \ensuremath{38.94\pm 0.19} & \ensuremath{120.57\pm 1.05} & \ensuremath{27.93\pm 0.29} & \ensuremath{7361.63\pm 19.54} & \ensuremath{2026.09\pm 10.26} & \ensuremath{168.46\pm 0.80} & \ensuremath{1548.30\pm 37.24} & \ensuremath{303.00\pm 20.79} & \ensuremath{381.25\pm 13.71} & \ensuremath{11909.30\pm 49.96} & \ensuremath{11281}\tabularnewline
VBF SR, DNN output $> 0.25$ & \ensuremath{164.76\pm 0.39} & \ensuremath{81.29\pm 0.86} & \ensuremath{3.67\pm 0.10} & \ensuremath{336.00\pm 4.39} & \ensuremath{159.96\pm 3.14} & \ensuremath{28.52\pm 0.33} & \ensuremath{137.98\pm 20.13} & \ensuremath{52.21\pm 6.62} & \ensuremath{56.19\pm 5.94} & \ensuremath{852.14\pm 22.68} & \ensuremath{908}\tabularnewline
VBF SR, DNN output $> 0.87$ & \ensuremath{31.97\pm 0.17} & \ensuremath{2.81\pm 0.16} & \ensuremath{0.04\pm 0.01} & \ensuremath{3.06\pm 0.47} & \ensuremath{2.12\pm 0.29} & \ensuremath{2.50\pm 0.10} & \ensuremath{0.72\pm 0.27} & \ensuremath{1.81\pm 0.72} & \ensuremath{0.68\pm 0.15} & \ensuremath{13.71\pm 0.98} & \ensuremath{38}\tabularnewline
\hline
\hline
VBF \Zgamma CR & \ensuremath{17.29\pm 0.13} & \ensuremath{14.78\pm 0.36} & \ensuremath{19.30\pm 0.21} & \ensuremath{277.83\pm 3.77} & \ensuremath{82.53\pm 1.80} & \ensuremath{6.40\pm 0.15} & \ensuremath{1678.23\pm 17.93} & \ensuremath{11.62\pm 8.88} & \ensuremath{79.65\pm 6.42} & \ensuremath{2151.04\pm 21.43} & \ensuremath{2114}\tabularnewline
\hline
\hline
VBF Top CR & \ensuremath{22.49\pm 0.15} & \ensuremath{28.48\pm 0.52} & \ensuremath{10.95\pm 0.16} & \ensuremath{40218.20\pm 44.11} & \ensuremath{270.73\pm 3.47} & \ensuremath{20.50\pm 0.29} & \ensuremath{264.39\pm 10.19} & \ensuremath{404.69\pm 33.38} & \ensuremath{89.68\pm 6.86} & \ensuremath{41296.67\pm 56.77} & \ensuremath{41112}\tabularnewline
\hline
\hline
VBF $WW$ VR & \ensuremath{16.86\pm 0.12} & \ensuremath{42.90\pm 0.62} & \ensuremath{17.88\pm 0.23} & \ensuremath{8499.91\pm 21.78} & \ensuremath{5347.94\pm 13.44} & \ensuremath{339.27\pm 1.13} & \ensuremath{247.23\pm 13.36} & \ensuremath{429.50\pm 21.39} & \ensuremath{621.07\pm 12.85} & \ensuremath{15527.81\pm 38.18} & \ensuremath{14452}
\end{tabular}

        }
        % \caption{VBF Cutflow.}
        % \label{tab:VBF-cutflow}
        } \\
        \vspace{20pt}
        \subfloat[]{
        \resizebox{\textwidth}{!}{
        \input{figures/hww/cutflows/run2-emme-VBF-cutflow-processes-paper-ggF-cutflow-cuts-2jet.tex}
        }
        }
        \caption{(a) VBF (b) ggF 2-jet cutflow.}
        \label{app:tab:cutflows}
    \end{table}   
\end{landscape}

\FloatBarrier
\section{Post-fit Distributions of the DNN input variables}
\label{app:post-fit-distributions}
\TDinote{}{MAYBE!?}


\FloatBarrier
\section{DNN distributions in STXS SRs}
\Cref{fig:dnn-output-vbf-stxs} shows the distribution of the DNN output in the top-quark and \Ztautau CRs used in the VBF, STXS analysis. 

\captionsetup[subfloat]{captionskip=5pt} % space between subfloat caption and image
\begin{figure}[t]
    \centering
    \subfloat[{\tiny $350 \leq \mjj < 700~\GeV, \pTH < 200~\GeV$}]{
        \includegraphics[width=0.32\textwidth]{figures/220605-Thesis/topcr/stxs/plots/run2-emme-CutVBF_TopControl_2jet_MJJ_350_700_PTH_0_200-DNNoutputG_FitBin1-log.pdf} \hfill
    }
    \subfloat[{\tiny $\mjj \geq 700~\GeV, \pTH < 200~\GeV$}]{
        \includegraphics[width=0.32\textwidth]{figures/220605-Thesis/topcr/stxs/plots/run2-emme-CutVBF_TopControl_2jet_MJJ_GT700_PTH_0_200-DNNoutputG_FitBin1-log.pdf} \hfill
    }
    \subfloat[{\tiny $\mjj \geq 350~\GeV, \pTH \geq 200~\GeV$}]{
        \includegraphics[width=0.32\textwidth]{figures/220605-Thesis/topcr/stxs/plots/run2-emme-CutVBF_TopControl_2jet_MJJ_GT350_PTH_GT200-DNNoutputG_FitBin1-log.pdf} \hfill
    }  \\
    \subfloat[{\tiny $350 \leq \mjj < 700~\GeV, \pTH < 200~\GeV$}]{
        \includegraphics[width=0.32\textwidth]{figures/220605-Thesis/zttcr/stxs/plots/run2-emme-CutVBF_ZtautauControl_2jet_MJJ_350_700_PTH_0_200-DNNoutputG_FitBin1-log.pdf} \hfill
    }
    \subfloat[{\tiny $\mjj \geq 700~\GeV, \pTH < 200~\GeV$}]{
        \includegraphics[width=0.32\textwidth]{figures/220605-Thesis/zttcr/stxs/plots/run2-emme-CutVBF_ZtautauControl_2jet_MJJ_GT700_PTH_0_200-DNNoutputG_FitBin1-log.pdf} \hfill
    }
    \subfloat[{\tiny $\mjj \geq 350~\GeV, \pTH \geq 200~\GeV$}]{
        \includegraphics[width=0.32\textwidth]{figures/220605-Thesis/zttcr/stxs/plots/run2-emme-CutVBF_ZtautauControl_2jet_MJJ_GT350_PTH_GT200-DNNoutputG_FitBin1-log.pdf} \hfill
    }
    {\caption{Distribution of the DNN output in the (a)-(c) top-quark CRs and (d)-(f) \Ztautau CRs used in the VBF, STXS analysis.
            \label{fig:dnn-output-vbf-stx} }}
\end{figure}
\captionsetup[subfloat]{captionskip=7pt} % space between subfloat caption and image


\FloatBarrier
\section{Event Selections in the ggF \ZeroJet and \OneJet Category}
\label{app:event-selection-ggf}

\begin{figure}[!h]
    \centering
    \subfloat[]{
      \includegraphics[width=0.49\textwidth]{\paperfiguredir/ggF/ZeroJet/CutGGF_Ptll_0jet-Mll}
      \label{fig:ggf:Plots:selections-a}
    }
    \subfloat[]{
      \includegraphics[width=0.49\textwidth]{\paperfiguredir/ggF/ZeroJet/CutGGF_bVeto_0jet-DPhill}
      \label{fig:ggf:Plots:selections-b}
    } \\
    \subfloat[]{
      \includegraphics[width=0.49\textwidth]{\paperfiguredir/ggF/OneJet/CutGGF_ZttVeto_1jet-Mll}
      \label{fig:ggf:Plots:selections-c}
    }
    \subfloat[]{
      \includegraphics[width=0.49\textwidth]{\paperfiguredir/ggF/OneJet/CutGGF_TopoMll_1jet-DPhill}
      \label{fig:ggf:Plots:selections-d}
    }
  \caption{
  Distributions of (a) \mll\ and (b) \dphill\ in the \ZeroJet category as well as (c) \mll\ and (d) \dphill\ in the \OneJet category, after the preselection and background rejection steps, and also after the selection on \mll\ for the \dphill plots. The dashed lines indicate where the selection on the observable is made. The distributions are normalized to
  their nominal yields, before the final fit to all SRs and CRs (pre-fit normalizations). The hatched band shows the normalization component of the total pre-fit uncertainty, assuming SM Higgs boson production. The bottom panels show the normalized distributions for the signal and backgrounds, from which it can be inferred which background processes are primarily removed by the indicated selections.
  Figure and caption taken from \cref{HWWPaper}. 
  \label{fig:ggf:Plots:selections}
  }
  \end{figure}
  
\FloatBarrier
\section{Post-fit distributions in CRs}
\label{app:post-fit-cr-dists}
\TDinote{Add MT and DNN output plots post fit in CRs}


\FloatBarrier
\section{Fake factor method to estimate background from lepton misidentifications}
\label{app:fake-factor-method}
This section describes how the background from lepton misidentifications is estimated using the fake factor method.

\subsubsection{Control sample definition}
The control sample is distinguished from the nominal selection by using an alternative lepton selection.
One of the two prompt lepton candidates is required to fail the full identification criteria defined in \cref{sec:object-selection} but satisfies a looser set of criteria, referred to as \emph{anti-identified} lepton.
%\Cref{tab:leptonID} summarizes the lepton selections for fully identified and anti-identified leptons. 
%Paper
%They are estimated using a data-driven technique8 where 464 control samples are established in which all nominal selections are applied with the exception that one of 465 the two lepton candidates fails to meet the full identification criteria defined in Section 4, but satisfies a 466 looser set of identification criteria (referred to as an anti-identified lepton).
% \begin{table}
%     \scalebox{0.77}{
%         \begin{tabular}{c|c||c|c}
%             \toprule
%             % \dbline
%             \multicolumn{2}{c||}{Electron}                                       & \multicolumn{2}{c}{Muon}                                                                                                      \\
%             identified                                                           & anti-identified                                  & identified                          & anti-identified                      \\
%             \midrule
%             % \sgline
%             \multicolumn{2}{c||}{$\pt  > 15 \mathrm{GeV}$}                     & \multicolumn{2}{c}{$\pt   > 15 \mathrm{GeV}$ }                                                                              \\
%             \multicolumn{2}{c||}{$|\eta|  <2.47$,excluding  $1.37<|\eta|< 1.52$} & \multicolumn{2}{c}{$|\eta|  <2.5$}                                                                                            \\
%             \multicolumn{2}{c||}{$|z_{0}\sin\theta|<0.5$ mm }                    & \multicolumn{2}{c}{$|z_{0}\sin\theta|<0.5$ mm }                                                                               \\
%             \multicolumn{2}{c||}{$|d_{0}|/\sigma(d_{0})<5$ }                     & $|d_{0}|/\sigma(d_{0}) <  3$                     & $|d_{0}|/\sigma(d_{0}) <  15$                                              \\
%             Pass LHTight if $\pt < 25 \;\mathrm{GeV}$                            & \multirow{3}{*}{Pass LHLoose}                    & \multirow{3}{*}{Pass Quality Tight} & \multirow{3}{*}{Pass Quality Medium} \\
%             Pass LHMedium if $\pt > 25 \;\mathrm{GeV}$                           &                                                  &                                     &                                      \\
%             Pass FCTight isolation                                               &                                                  & Pass FCTight isolation              &                                      \\
%             \multicolumn{2}{c||}{ {\scshape Author} $= 1$}                       &                                                  &                                                                            \\
%                                                                                  & Veto against identified electron                 &                                     & Veto against identified muon         \\
%             % \dbline
%             \bottomrule
%         \end{tabular}
%     }
%     \caption{Selection criteria for fully identified and anti-identified leptons.}
%     \label{tab:leptonID}
% \end{table}

\subsection{Extrapolation to analysis regions}
The contributions of backgrounds with misidentified leptons (Mis-Id background) are estimated by subtracting the  expected number of events in the control sample with the expected contribution from processes with two true, prompt leptons, and then multiplying it with the corresponding fake factors.
The full expression can be written as,
\begin{equation}
    \label{eq:fake-estimation}
    N_{\text{id,id}}^{\text{Mis-Id}} = FF \left( N_{\text{id,\sout{id}}}^{\text{data}} - N_{\text{id,\sout{id}}}^{\text{2 true prompt}} \right) - FF_1FF_2 \left( N_{\text{\sout{id},\sout{id}}}^{\text{data}} - N_{\text{\sout{id},\sout{id}}}^{\text{2 true prompt}} \right),
\end{equation}
where $N_{\text{id,\sout{id}}}^{\text{data}}$ and $N_{\text{\sout{id},\sout{id}}}^{\text{data}}$ are, respectively, the number of expected events with one or two anti-identified leptons measured in the control data sample, and $N_{\text{id,\sout{id}}}^{\text{2 true prompt}}$ and $N_{\text{\sout{id},\sout{id}}}^{\text{2 true prompt}}$ are, respectively, the number of expected events with two true, prompt leptons measured in simulated event samples.
The fake factors $FF$ and $FF_{1/2}$ are applied according to which leptons are anti-identified, that is, whether it is the electron, the muon, or both.
They are dependent on the \pT of the anti-identified lepton and, in the case of a misidentified muon, $\abseta$, i.e. $FF \to FF(\pT, \abseta)$.
The second term on the right-hand side of \cref{eq:fake-estimation} provides a correction for events where both of the leptons are misidentified, a contribution that is double-counted in the first term. A rigorous mathematical derivation of \cref{eq:fake-estimation} goes beyond the scope of this thesis and the interested reader is referred to \ccite{HIGG-2013-13,HIGG-2016-07}.
\TDinote{}{Refer Konstis thesis.}

\subsection{Derivation of nominal fake factors in \Zjets-enriched sample}
\label{subsubsec:zjets-ffs}
The fake factors are estimated using a data sample enriched in \Zjets events and a three-lepton selection (referred to as \Zjets selection), that targets events with two leptons originating from the $Z$ boson decay and a jet that serves as candidate for being misidentified as a lepton (referred to as Mis-Id candidate).
%A three-lepton selection, referred to as \Zjets selection, is therefore applied. 
All leptons must satisfy \ptGT{15}\,GeV.
The two leptons of opposite charge and same flavor that qualify as $Z$ decay particles are chosen to be the ones that have an invariant mass close the $Z$ boson mass. If these two leptons are electrons (muons), a requirement of $80 < m_{ee} < 110\,$GeV ($70 < m_{\mu\mu} < 110\,$GeV) is applied.
The two leptons are required to be fully identified except of the isolation requirement that is loosened to the ``Loose'' working point selections and the identification working point that is modified to ``Loose'' for electrons and ``Medium'' for muons.
\todo{Find out about efficiencies? Otherwise statements of loose/medium/... are pretty useless.}
Relaxing these criteria allows for a better statistical precision in the fake factor derivation.
In addition, at least one of the two leptons must be associated to the online object that triggered the single-lepton trigger used in the analysis as described in \cref{sec:data-mc-samples}.
If more than one pair of leptons satisfies these requirements, the pair with the invariant mass closest to the $Z$ boson mass is chosen.
The respective other lepton in the event serves as the Mis-Id candidate.
The fake factor is then defined as the ratio of events, where this Mis-Id candidate is fully identified, $N_{\text{id}}^{\text{data}}$, and anti-identified, $N_{\text{\sout{id}}}^{\text{data}}$, each time subtracting the expected number of events with three true, prompt leptons that are not originating from \Zjets processes,
\begin{equation}
    \label{eq:ff}
    FF = \frac{ N_{\text{id}}^{\text{data}} - N_{\text{id}}^{\text{non-\Zjets}}}{N_{\text{\sout{id}}}^{\text{data}} - N_{\text{\sout{id}}}^{\text{non-\Zjets}}}.
\end{equation}
The events with three true, prompt leptons, $N_{\text{id/\sout{id}}}^{\text{non-Zjets}}$, are estimated using simulated samples and are dominated by leptonically decaying $WZ$ production.
To reduce the contamination of this process, a requirement of $m_T^W = \sqrt{2E^{\textrm{miss}}_\textrm{T} E^{\text{Mis-Id cand.}}_\textrm{T} (1-\cos\phi)} < 50\;\textrm{GeV}$ is additionally applied in the \Zjets selection.
The resulting fake factors are summarized in \cref{tab:ZjetsFF-uncertainties}.
The fake factors are derived separately for electrons and muons. The electron fake factors are split into four different \pT bins and are measured inclusively for the entire \abseta range.
The muon fake factors are also derived in four \pT bins and two \abseta regions \absetaST{1.05} and \absetaGT{1.05}. The value of the last \pT bin with \ptGT{50}\,GeV is derived by an extrapolation from the respective previous bin, due to the limited statistical precision at high \pT.

\subsection{Derivation of triggered fake factors in dijet-enriched sample}
The events in the control sample are selected based on one of three trigger decisions: The dilepton trigger is fired, only one single-lepton trigger is fired by the fully identified lepton, or only one single-lepton trigger is fired by the anti-identified lepton.
For events falling into the latter category, the trigger requirements are more stringent than the definition of the anti-identified lepton, leading to a bias in the event yield.
Although the number of events in that category accounts for only $\mathcal{O}(1\%$) or less and the impact on the analysis is therefore small, specific so-called \emph{triggered fake factors} subject to the same bias are derived for these types of events. This is achieved by considering a data sample where the anti-identified lepton fires the single-lepton trigger.
    Applying such a requirement in addition to the \Zjets selection, however, leads to an insufficient event yield to derive reasonable fake factors. The triggered fake factors are therefore derived in a sample enriched in dijet events.
    The dijet selection requires each event to have at least one Mis-Id candidate with \ptGT{15}~\,~GeV and one jet with \ptGT{25}~\,~GeV. To ensure a back-to-back topology between the lepton and the jet, the two objects must have an angular separation $\Delta \phi(\text{lep}, \text{jet}) > 2.5$.
    In order to suppress the contamination from non-dijet processes, mostly coming from \Wjets processes, the events are required to have $\MET < 30$\,GeV and the transverse mass between the lepton and the missing transverse energy must satisfy $m_T < 60\,$GeV.
    The fake factors are then derived analogous to \cref{eq:ff}, but subtracting non-dijet events in the numerator and denominator.
    % The overall number of events that are subject to a trigger bias is very small.
    % \begin{equation}
    %     \label{eq:ff-dijet}
    %     FF^{\text{dijet}} = \frac{ N_{\text{id}}^{\text{data}} - N_{\text{id}}^{\text{non-dijet}}}{N_{\text{\sout{id}}}^{\text{data}} - N_{\text{\sout{id}}}^{\text{non-dijet}}}.
    % \end{equation}


    \subsection{Corrections and uncertainties}
    Uncertainties in the estimation of the Mis-Id background in the analysis are account for via uncertainties in the fake factors.
    The fake factor uncertainties can be grouped into three main components: statistical uncertainties, uncertainties in the normalization of non-\Zjets processes that are subtracted in \cref{eq:ff}, and uncertainties due to the different flavor composition of jets in \Zjets processes compared to \Wjets processes.
    The latter uncertainty arises due to a correction that must be applied to the fake factors derived in \Zjets events.
    Each uncertainty component and correction is discussed below, and their impact is summarized in \cref{tab:ZjetsFF-uncertainties}.


    \subsubsection{Statistical uncertainty}
    The statistical uncertainties refer to the uncertainties in the data sample used to derive the fake factors as shown in \cref{eq:ff}.

    \subsubsection{$EW$ subtraction uncertainty}
    The contribution of non-\Zjets processes subtracted in \cref{eq:ff} is primarily due to $WZ$ processes but also $ZZ$ and $Z\gamma$ processes can enter the \Zjets selection.
    Theoretical uncertainties in the normalization of these processes are derived and propagated, leading to systematic uncertainties in the fake factor.
    In order to reduce the impact of theoretical uncertainties associated to the $WZ$ process, the normalization of the $WZ$ contribution is extracted from a control region. The CR is defined by inverting the $m_T^W$ selection to $m_T^W > 50\,$GeV and requiring the Mis-Id candidate to pass the full identification criteria. This region is very pure in $WZ$ events so that a precise normalization factor of $0.99 \pm 0.01$ can be extracted and applied in the \Zjets selection. The impact of the uncertainties affecting similarly the normalization in the \Zjets selection and the $WZ$ CR selection can thus be reduced.

% Uncertainties associated to the $WZ$ process and their impact on the fake factors are estimated with simulated samples and propagated to the final analysis results.
% Together they are referred to as the $EW$ subtraction uncertainty.

\subsubsection{Correction factor and flavor composition uncertainty}
The nominal fake factors are derived in \Zjets candidate events, because of the ability to define a region very pure in \Zjets events with little contamination from other processes. However, the dominant contribution to the Mis-Id background in this analysis comes from \Wjets processes.
Because the flavor of the quarks that give rise to the jets in \Zjets processes differs on average from that in \Wjets processes, and the quark flavor is expected to affect the misidentification rate, a \emph{correction factor} (CF) is applied to the nominal fake factors.
The CF is given as the ratio of fake factors derived in simulated \Wjets events and \Zjets events,
\begin{equation}
    CF = \frac{FF^{\Wjets}}{FF^{\Zjets}},
\end{equation}
and applied as a multiplicative factor to the fake factor,
\begin{equation}
    FF^{\Wjets} = FF^{\Zjets} \times CF.
\end{equation}
The same selections are applied to the simulated samples as in the three-lepton \Zjets selection and the CF is derived separately for misidentified electrons and muons.
The systematic uncertainties associated to the CF are derived by using an alternative MC generator and summed in quadrature with the statistical uncertainty in the simulated \Wjets and \Zjets sample.


\begin{table}[ht]
    \begin{center}
        % This file was created on 2020-06-04 06:11:19 by the script makeSummaryTable.py.
\begin{tabular}{c|c|c|c|c|c}
    \toprule
    Kinematic Region                        & Nominal & Statistical & $EW$ Subtraction & Flavour composition & Total \\
    ($|\eta|$ and $\pt$ range)              &       &             &                &                    &       \\
    \midrule
    \multicolumn{1}{l|}{\textbf{Electron}} &       &             &                &                    &       \\
    \multicolumn{1}{l|}{$0<|\eta|<2.5$}   &       &             &                &                    &       \\
    \multicolumn{1}{r|}{$15-20$\,GeV}    & 0.076 & 6.2         & 3.9            & 7.5                & 10    \\
    \multicolumn{1}{r|}{$20-25$\,GeV}    & 0.086 & 11          & 8.1            & 31                 & 34    \\
    \multicolumn{1}{r|}{$25-35$\,GeV}    & 0.14  & 13          & 14             & 7.4                & 20    \\
    \multicolumn{1}{r|}{$> 35$\,GeV} & 0.21  & 12          & 25             & 21                 & 35    \\
    \midrule
    \multicolumn{1}{l|}{\textbf{Muon}}     &       &             &                &                    &       \\
    \multicolumn{1}{l|}{$0<|\eta|<1.05$}  &       &             &                &                    &       \\
    \multicolumn{1}{r|}{$15-20$\,GeV}    & 0.042 & 8.4         & 7.1            & 8.1                & 14    \\
    \multicolumn{1}{r|}{$20-25$\,GeV}    & 0.017 & 35          & 34             & 11                 & 50    \\
    \multicolumn{1}{r|}{$25-50$\,GeV}    & 0.026 & 39          & 58             & 11                 & 71    \\
    \multicolumn{1}{r|}{$50-\infty$\,GeV}  & 0.043 & 47          & 58             & 11                 & 75    \\
    \multicolumn{1}{l|}{$1.05<|\eta|<2.5$}  &       &             &                &                    &       \\
    \multicolumn{1}{r|}{$15-20$\,GeV}    & 0.060 & 6.7         & 5.3            & 8.1                & 12    \\
    \multicolumn{1}{r|}{$20-25$\,GeV}    & 0.042 & 17          & 14             & 11                 & 25    \\
    \multicolumn{1}{r|}{$25-50$\,GeV}    & 0.065 & 19          & 28             & 11                 & 36    \\
    \multicolumn{1}{r|}{$> 50$\,GeV}  & 0.11  & 32          & 28             & 11                 & 44    \\
    \bottomrule
    \end{tabular}
    
    \end{center}
    \caption{Summary of the fake factors from the $Z$+jets estimate with uncertainties. All uncertainties are quoted in percent on the nominal value. More information on the different uncertainty components is given in the text.
        % \textit{Value} denotes the nominal fake factor value. \textit{Statistical} denotes the statistical uncertainties on the fake factors including the extrapolation uncertainty for the highest muon \pT bin. \textit{EW Subtraction} denotes the uncertainty due to the electroweak backgrounds that enter the $Z$+jets fake factor estimate. \textit{Sample Composition} denotes the uncertainty that accounts for differences in fake factors between $Z$+jets and $W$+jets processes, and includes both statistical and systematic uncertainty on the correction factors. The column \textit{Total} sums all individual contributions in quadrature to give an overview of the total uncertainty of the fake factor. In the fit, however, the EW subtraction and sample composition uncertainties are treated as correlated between different bins, while the statistical uncertainty is uncorrelated.
    }
    \label{tab:ZjetsFF-uncertainties}
    % Nominal values, statistical uncertainties and EWSUBTR uncertainties produced on tag kon_improveZjetsFFStats_v2
\end{table}


\FloatBarrier
\section{STXS Measurement}
\label{app:stxs-measurement}
\begin{table}[htp]
    \caption{
    Best-fit values and uncertainties for the production cross section times $\hww$ branching fraction $({\sigma_i \cdot \mathcal{B}_{H \to WW^{\ast}}})$ in each STXS bin.
    }
    \begin{center}
      \small
      \renewcommand{\arraystretch}{1.5}
      \scalebox{0.90}{
        \begin{tabular}{l|rccccc|S[table-format=5,table-number-alignment=right]@{$\,\pm\,$}
  S[table-format=3,table-number-alignment=left]}
\toprule
\multirow{2}{*}{STXS bin $({\sigma_i \cdot \mathcal{B}_{H \to WW^{\ast}}})$} & \multicolumn{1}{c}{Value} & \multicolumn{5}{c|}{ Uncertainty [fb]} & \multicolumn{2}{c}{SM prediction} \\
&  \multicolumn{1}{c}{[fb]} &     Total                 & Stat.                       & Exp.\ Syst.                  & Sig.\ Theo.                  &    Bkg.\ Theo.    & \multicolumn{2}{c}{[fb]}   \\
\midrule
\makecell[l]{\noalign{\vskip 2mm} \ggHZeroJ  \\ {\scriptsize \ggHZeroJMath}}                  & $7100$                    & $^{+ 950}_{-910}$           & $^{+480}_{-470}$          & $^{+570}_{-530}$            & $^{+320}_{-260}$            & $^{+490}_{-480}$          & 5870  & 390           \\ [0.4cm]
\makecell[l]{\noalign{\vskip 1mm}\ggHOneJVLowPt  \\ {\scriptsize \ggHOneJVLowPtMath}}          & $1140$                    & $^{+ 800}_{-820}$           & $^{+420}_{-410}$          & $^{+380}_{-380}$            & $^{+80\phantom{0}}_{-70}$   & $^{+570}_{-600}$          & 1400 &  190            \\ [0.4cm]
\makecell[l]{\ggHOneJLowPt  \\ {\scriptsize \ggHOneJLowPtMath}}                 & $540$                     & $^{+ 470}_{-470}$           & $^{+310}_{-310}$          & $^{+230}_{-230}$            & $^{+42\phantom{0}}_{-47}$   & $^{+270}_{-280}$          & 970  & 150             \\ [0.4cm]
\makecell[l]{\ggHOneJMedPt  \\ {\scriptsize \ggHOneJMedPtMath}}                 & $230$                     & $^{+ 130}_{-120}$           & $^{+100}_{-100}$ & $^{+60\phantom{0}}_{-60}$   & $^{+10\phantom{0}}_{-10}$   & $^{+50\phantom{0}}_{-50}$ & 160  & 30             \\ [0.4cm]
\makecell[l]{\ggHTwoJ  \\ {\scriptsize \ggHTwoJMath}}                     & $1610$                    & $^{+ 900}_{-890}$           & $^{+440}_{-440}$          & $^{+430}_{-420}$            & $^{+300}_{-150}$   & $^{+640}_{-650}$          & 1010  & 220             \\ [0.4cm]
\makecell[l]{\ggHHighPt  \\ {\scriptsize \ggHHighPtMath}}                  & $260$                     & $^{+ 100}_{-100}$           & $^{+80\phantom{0}}_{-80}$ & $^{+40\phantom{0}}_{-40}$   & $^{+40\phantom{0}}_{-20}$   & $^{+40\phantom{0}}_{-40}$ & 122 & 31             \\ [0.4cm]
\makecell[l]{\qqHLowMjj \\ {\scriptsize \qqHLowMjjMath}}                  & $6$                     & $^{+ 63\phantom{0}}_{-62}$  & $^{+46\phantom{0}}_{-42}$ & $^{+31\phantom{0}}_{-34}$   & $^{+11\phantom{0}}_{-14}$   & $^{+24\phantom{0}}_{-26}$ & 109 & 7             \\ [0.4cm]
\makecell[l]{\qqHMedMjj \\ {\scriptsize \qqHMedMjjMath}}                   & $31$                      & $^{+ 35\phantom{0}}_{-33}$  & $^{+30\phantom{0}}_{-27}$ & $^{+15\phantom{0}}_{-14}$   & $^{+8\phantom{0}}_{-7}$    & $^{+11\phantom{0}}_{-10}$ & 56 & 4             \\ [0.4cm]
\makecell[l]{\qqHHighMjj \\ {\scriptsize \qqHHighMjjMath}}                 & $60$                      & $^{+ 26\phantom{0}}_{-23}$  & $^{+23\phantom{0}}_{-21}$ & $^{+7\phantom{00}}_{-7}$    & $^{+9\phantom{00}}_{-5}$    & $^{+5\phantom{00}}_{-5}$  & 51 & 4            \\ [0.4cm]
\makecell[l]{\qqHVHighMjj \\ {\scriptsize \qqHVHighMjjMath}}          & $57$                      & $^{+ 20\phantom{0}}_{-18}$  & $^{+18\phantom{0}}_{-17}$ & $^{+5\phantom{00}}_{-5}$    & $^{+3\phantom{00}}_{-3}$    & $^{+4\phantom{00}}_{-4}$  & 50 & 4             \\ [0.4cm]
\makecell[l]{\qqHHighPt \\ {\scriptsize   \qqHHighPtMath}}                & $37$                      & $^{+ 16\phantom{0}}_{-14}$  & $^{+14\phantom{0}}_{-13}$ & $^{+4\phantom{00}}_{-3}$    & $^{+4\phantom{00}}_{-3}$    & $^{+3\phantom{00}}_{-3}$  & 32 & 1            \\ [0.4cm]
\bottomrule
\end{tabular}

      }
    \end{center}
    \label{tab:STXS-XSecs}
  \end{table}

\FloatBarrier
\section{STXS Stage-1.2 Framework}
\label{app:stxs-measurements-aux}

\Cref{fig:stxs:stage-12-definition} shows the definitions of the fiducial kinematic regions in the Stage-1.2 STXS framework, including an indication of the merged regions measured in the \HWW analysis. 

\begin{figure}[h]
    \centering
    \subfloat[]{
        \newImageResizeCustom{0.9}{figures/hww/stxs/ggFSTXS-measured.pdf}
    } \\
    \subfloat[]{
        \newImageResizeCustom{0.9}{figures/hww/stxs/VBFSTXS-measured.pdf}
    } \\
    {\caption{Definition of fiducial kinematic regions in the Stage-1.2 STXS framework for (a) $ggH$ and (b) EW $qqH$ production. The colored boxes indicate the merged regions measured in the \HWW analysis.
            \label{fig:stxs:stage-12-definition} }}
\end{figure}


\chapter{Improvements of the \TwoJet Categories using Multiclass Classification}
\label{app:multi-class-2jet-strategy}

% %-----------------------------------------------------------------------
% %-----------------------------------------------------------------------
% \chapter{Mismodelling of data at the jet constituent level}
% \label{app:constituents-mismodelling}

% The reason is the challenge to model pile-up correctly, as it is impacted by non-perturbative effects.

% The MC samples therefore rely on theoretical models which parameters are \emph{tuned} to correctly describe the data in as many variables as possible.

% The effect is constant with pile-up and independent of the energy of the jet. The jet calibration can thereforegreatly reduce the effects of this mismodelling.

% \TDinote{Checkout notes for pile-up task force meetings}

% %-----------------------------------------------------------------------
% %-----------------------------------------------------------------------
% \chapter{Noise Term Measurement for \Rscan Jets}
% \label{app:noise-term-rscan}


% \section{Cluster Weighting}
% An alternative approach to correct for energy losses is the so-called \emph{local hadronic cell weighting} (LCW), which applies energy corrections already at cluster level.
% This approach is used for different jet definitions, one of which is described in \cref{app:noise-term-rscan}
% Different variables can be defined to characterize a topo-cluster based on its shape and other properties. These observables, known as \emph{cluster moments}, are used to extract information about the hadronic signal content in a given cluster which in turn is used to correct the energy to the \emph{LCW scale}. More information can be found in \ccite{PERF-2014-07}.

% \section{Definition and Calibration of \Rscan jets}

% \section{Changes to Noise Term Measurement}

% - Additional uncertainty from mu=0 fit range

% - Additional uncertainty based on the difference between different parametrisations


% \section{Results}
