

- This work would not have been possible without the support of countless people.
- HEP is collaborative, you can't manage to do anything on your own.

\begin{itemize}
    \item Supervisors: Bernd, expertise, trust, for support to travel, pitching new ideas, encouraging to apply for grants, competitions, .... Mike. expertise, support, ...
    \item Eric: for most hilarious/inspiring/... meeting announcements
    \item SFU students, in particular, Konstantin for great discussions, Steven Metalconcerts
    \item Freiburg students, Karsten Koeneke
    \item Brian, Manu, Chris Boehm
    \item ATLAS groups and everybody at CERN: the brightest people I have ever met were people at CERN.
    \item In particular the Jet/Etmiss group. Incredible workshops at the HCW. Tae Hyoun Park
    \item HWW group: Karsten, Benedict, Carsten, Yun-Ju
    \item HWW task force: Hayden, Robin, Konstantin, Federica. Incredible weeks before deadline
    \item (segway with pandemic, sitting at home) Vancouver one of my best decisions in life, thank you to Chenyi for sending me along this ride and broadcasting to people outside of physics how cool physics is
    \item Friends in VAN: Patrick, Karam, Olivia
    \item All friends in Van. Sorry for not naming anyone I am afraid I would miss someone. Being one of the most relaxed person that exists on earth: Bassel. Parties: Anthony, Annissa.
    \item Friends in GER: Like no time has passed. For visiting me in Vancouver, Belek, Zaum. 
    \item Andy Rive Leute for sharing the music taste and inspiring me regularly with new jams. 
    \item I want to thank Melina, countless hours on the phone, making life feel easy, never stops to surprise me, keeping me grounded, her love and appreciation, for reminding me to slow down and taking time off, amazing holidays, being a safe haven, supporting me...
    \item I want to thank my brother
    \item My parents and family. 
\end{itemize}

- This work would not have been possible without the support of countless people.

The end of my PhD studies marks the 10 years anniversary of my academic career (so almost exactly one third of my lifetime!), during which I had the pleasure to interact and work with truly amazing people and make lasting connections and friends.
I feel incredibly grateful and lucky that I have been able to pursue the things I enjoy, and am very aware that this would not have been possible without the support and love of countless people. 
I want to take this opportunity to thank some of those special people explicitly.
If you are not mentioned in the following, rest assured that I am still deeply grateful for our encounters.
% The end of my PhD studies marks the 10 years anniversary of my academic life (so almost exactly one third of my lifetime!) during which I had the pleasure to interact and work with truly amazing people and make lasting connections and friends.
% I want to take this chance to say thank you to every single one of them. If you are not mentioned explicitly in the following, rest assured that I am still grateful for our encounters. Similar to the work in HEP, which is highly collaborative by nature with individuals working on a small piece of the greater picture, I think about life similarly, in that I believe that every little conversation, every small input from another person, influences us in some way, if not noticeably it does so subconsciously. This makes us the persons we become at the place we end up at. 
% I couldn't be more grateful for the place I ended up at this moment. 
% can nudge you in unexpected directions 
% - HEP is collaborative, you can't manage to do anything on your own.

First, I would like to thank my supervisor, Bernd Stelzer, for the unique opportunity to join SFU in 2018 and the support and guidance in all aspects of a PhD student's life. 
I am very grateful for the freedom I was granted during my studies, while always knowing that the door is open to get his expert opinion on any scientific matter.
His reminders to think about the big picture and the encouragement to pitch new ideas, his support to participate in physics workshops and conferences, and his unique approaches to the scientific endeavor shaped me as a researcher. 
% and his support to participate in physics workshops and conferences 
% Living a scientific culture that sees the most benefits when researchers have the freedom to pursue what they like and let curiosity guide the way. 

I would also like to thank Mike Vetterli, who I worked with closely during my first year at SFU, for his great supervision and support.
I benefitted greatly from his immense expertise, and learning about his rigorous approaches to scientific problems made me a better researcher. 

I would also like to thank Dugan O'Neil, the third member of my supervisory committee, for his support and feedback throughout the years. 

For almost the entire five years at SFU, I have had the chance to work closely with Konstantin Lehmann, who shared almost the same timeline for his PhD studies. Sharing ideas, discussing complex problems, and working toward the same goals were not only immensely helpful, but also truly enjoyable. I am very grateful that we have become friends, and I am looking forward for our next encounter. 

I cannot leave unmentioned Eric, who started as a Postdoc in Bernd's group. Thank you for writing the most refreshing meeting announcements, sharing your knowledge, and demonstrating how delightful collaboration can be.

I would also like to thank the broader SFU physics group, for inspiring chats during lunch, for great social events, and, of course, to my fellow office colleagues for always opening the office door when I forgot my keys again. 
I wished we would have been able to live a normal university and office culture for longer than the pandemic allowed. 
I would also like to thank explicitly Steven Large, a fellow student who became a true friend. Spending time with you has been a delight, and thank you for having the best music taste. 

It is not too far from SFU to UBC (depending on the person you ask, of course!), so I would also like to acknowledge the connections and friends I have made at UBC, made possible by joint meetings of Bernd's and his brother Oliver Stelzer's group. 
Thank you for living an open scientific culture where people are able to share ideas and opinions, as well as making lasting connections. 
In particular, I would like to thank Robin Hayes, a fellow student, colleague, and friend. It was truly inspiring to work together with her. Her thorough approaches, tireless work ethics, and curiosity to get to the bottom of anything, have motivated me over the years and made work a pleasure. I wish you all the best for your new challenges, and hope to see you soon, at Kitsilano beach or somewhere else. 

Before my PhD studies at SFU, I studied five years of Bachelor and Master studies in Freiburg. 
The Freiburg HEP group, especially Karl Jakobs' group, got me passionate about particle physics and laid the foundation for my research life.
In particular, I would like to thank Karsten Köneke, my supervisor during my Master studies.
From him, I learned most of what it takes to be a researcher in a large collaboration like the ATLAS experiment, and I learned to love it.   

At Freiburg University, I also made lasting connections and friends. Thank you Brian Moser, Manuel Guth, and Christopher Böhm for funny, intense, and truly inspiring discussions about science and life over the years.
I am especially grateful to Brian Moser, the times I shared an office with him were among my favorite during my studies. Thanks for helping me understand countless scientific problems, always sharing your insights and knowledge about the most pressing topics in particle physics, and for being a true inspiration in life.


% \item In particular the Jet/Etmiss group. Incredible workshops at the HCW. Tae Hyoun Park
% \item HWW group: Karsten, Benedict, Carsten, Yun-Ju
% \item HWW task force: Hayden, Robin, Konstantin, Federica. Incredible weeks before deadline

% Segway: Got to know a lot of people via the ATLAS collaboration. Countless!

% Because of the enormous team work necessary to make valuable research in HEP
% Valuable research in HEP is not possible without collaboration. 

I would also like to thank the fellow researchers that I have met through working in two different subgroups of the ATLAS collaboration. 
The Jet/ETmiss subgroup offered an especially pleasant working environment where everyone tries to support each other. I had an amazing time at two of the Hadronic Calibration Workshops, which I am sure will remain a prime example of fruitful discussion and collaboration for me in the years to come. 
I have been enjoying working in the \HWW analysis subgroup for about 6 years (sometimes being more, sometimes less active). Thank you to all conveners that organized and managed the group, and aligned the group's work with the big picture goals: Frank Filthaut, Kathrin Becker, Jonas Strandberg, Claudia Bertella, Kristin Lohwasser, Yun-Ju Lu, Benedict Tobias Winter, Carsten Burgard. 
I owe a particular thanks to Benedict, who did an amazing job guiding the group through a long and intense review process for the analysis described in this thesis. 
Many thanks also to Carsten Burgard and Ralf Gugel, who are the pillars of the analysis software framework, for tirelessly supporting people in software-related questions, and also helping me significantly in my first years in the \HWW subgroup. 
I cannot leave unmentioned David Shope, who gave me incredible guidance especially in my first years in the ATLAS collaboration, and as analysis contact and paper editor went through a lot of the hard work necessary to publish our analysis.
%My appreciation for git, and thus my ability to collaborate, would not be at the current level without his help. 
It was an incredible pleasure for me to work closely together with a number of students in the group: Federica Pasquali, Hayden Smith, Tae Hyoun Park, Robin Hayes, Konstantin Lehmann. 
I will not forget the weeks before the deadlines in Summer 2020 and February/March 2021 to prepare our analysis for publication.
This might have been the most efficient teamwork I will ever experience, as we worked day and night shifts in Europe and Canada, undoubtedly working toward the same goals.
And yes it was tough, but mostly it was a joy being part of such an amazing research team.

I am also grateful for having had the opportunity of spending in total almost a year at CERN, attending several workshops and conferences, and also staying for longer periods of time. 
I had the pleasure to meet and work with the brightest people I have ever met, and I am deeply grateful for having been part of this amazing community. 
It has been an honor to feel like a small puzzle piece of the big picture.
% CERN remains the most inspiring place I have ever been.
% \item ATLAS groups and everybody at CERN: the brightest people I have ever met were people at CERN.
% individuals working on a small piece of the greater picture. Truly grateful for all encounters at CERN, talking to some of the brightest people I have ever met. 
% Work in HEP, which is highly collaborative by nature with individuals working on a small piece of the greater picture. Truly grateful for all encounters at CERN, talking to some of the brightest people I have ever met. 

Besides the people I have met through research in particle physics, I would like to thank many of the people and friends that have been part of my life in the past years, beginning with the people I got to know in Vancouver. 
Patrick Mayerhofer, I cannot overstate how grateful I am that our paths crossed at SFU. My experience in Vancouver would have been much different and much less enjoyable without sharing it with you. 
The same goes for Karam Elabd, you have influenced me in many ways and got me passionate about things I did not even know existed before meeting you.
The both of you are a true inspiration in life that I do not want to miss. 



Also want to thank roommates.

To everyone, I wish you all the best for your future. Thank you for having been or still being part of my journey. 

% Maybe in Deutsch!
Lastly, I would like to express my deep thankfulness to my parents and family, for everything they provide me, their support and advisory in every situation in life. 
I feel incredibly lucky to have always had a safe haven like the one you provide, and that you supported my dreams, even if it means we are separated for more than 8000 kilometers. Rest assured, that I 