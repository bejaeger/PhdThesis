

- This work would not have been possible without the support of countless people.
- HEP is collaborative, you can't manage to do anything on your own.

\begin{itemize}
    \item Supervisors: Bernd, expertise, trust, for support to travel, pitching new ideas, encouraging to apply for grants, competitions, .... Mike. expertise, support, ...
    \item Eric: for most hilarious/inspiring/... meeting announcements
    \item SFU students, in particular, Konstantin for great discussions
    \item Freiburg students, Karsten Koeneke
    \item Brian, Manu, Chris Boehm
    \item ATLAS groups and everybody at CERN: the brightest people I have ever met were people at CERN.
    \item In particular the Jet/Etmiss group. Incredible workshops at the HCW. Tae Hyoun Park
    \item HWW group: Karsten, Benedict, Carsten, Yun-Ju
    \item HWW task force: Hayden, Robin, Konstantin, Federica. Incredible weeks before deadline
    \item (segway with pandemic, sitting at home) Vancouver one of my best decisions in life, thank you to Chenyi for sending me along this ride and broadcasting to people outside of physics how cool physics is
    \item Friends in VAN: Patrick, Karam, Olivia
    \item All friends in Van. Sorry for not naming anyone I am afraid I would miss someone. Being one of the most relaxed person that exists on earth: Bassel. Parties: Anthony, Annissa.
    \item Friends in GER: Like no time has passed. For visiting me in Vancouver, Belek, Zaum. 
    \item Andy Rive Leute for sharing the music taste and inspiring me regularly with new jams. 
    \item I want to thank Melina, countless hours on the phone, making life feel easy, never stops to surprise me, her love and appreciation, for reminding me to slow down and taking time off, amazing holidays
    \item I want to thank my brother
    \item My parents and family. 
\end{itemize}

- This work would not have been possible without the support of countless people.

I feel incredibly grateful and lucky that I have been able to pursue the things I like the most and am very aware that this would not have been possible without the support of countless people.
The end of my PhD studies marks the 10 years anniversary of my academic life (so almost exactly one third of my lifetime!) during which I had the pleasure to interact and work with truly amazing people and make lasting connections and friends.
I want to take this chance to say thank you to every single one of them. If you are not mentioned explicitely in the following, rest assured that I am still grateful for our encounters. Similar to the work in HEP, that is highly collaborative by nature and individuals fill out a tiny space of the larger picture, I approach life being aware that every little conversation, every small input from another person, influences you in noticeably or subconscuously, and makes you the person you become at the place you are at. I couldn't be more grateful for the place I ended up at this moment. 

can nudge you in unexpected directions 
- HEP is collaborative, you can't manage to do anything on your own.

I would like to thank my supervisor Bernd Stelzer, for his great expertise and advice in any scientific question, his encouragement to think big and pitch new ideas, his support and encouragement to participate in physics workshops and conferences, and in general the great support for every aspect of a PhD students life. I am very thankful for the freedom I was granted during my studies to embark in the projects I am most curious about. 
Living a scientific culture that sees the most benefits when researchers have the freedom to pursue what they like and let curiosity guide the way. 

I also would like to thank my second supervisor Mike Vetterli, who I had the pleasure to work closely with during my first year at SFU. 
The rigorous scientific method, immense expertise, support

Eric, competency, most refreshing meeting announcement especially during difficult times during the pandemic.
Physics students at SFU for inspiring chats during lunch and, in particular, my fellow HEP researchers at SFU for great conversations about both life and science, and, of course, always opening the office door when I forgot my keys again. I wished we would have been able to live a normal office culture for longer than the pandemic allowed. 
I would like to thank Konstantin. 

My PhD studies at SFU were followed by 5 years at Freiburg, which laid the basics of my research life.
