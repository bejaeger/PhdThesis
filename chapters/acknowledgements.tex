% - This work would not have been possible without the support of countless people.
% - HEP is collaborative, you can't manage to do anything on your own.
% \begin{itemize}
%     \item Supervisors: Bernd, expertise, trust, for support to travel, pitching new ideas, encouraging to apply for grants, competitions, .... Mike. expertise, support, ...
%     \item Eric: for most hilarious/inspiring/... meeting announcements
%     \item SFU students, in particular, Konstantin for great discussions, Steven Metalconcerts
%     \item Freiburg students, Karsten Koeneke
%     \item Brian, Manu, Chris Boehm
%     \item ATLAS groups and everybody at CERN: the brightest people I have ever met were people at CERN.
%     \item In particular the Jet/Etmiss group. Incredible workshops at the HCW. Tae Hyoun Park
%     \item HWW group: Karsten, Benedict, Carsten, Yun-Ju
%     \item HWW task force: Hayden, Robin, Konstantin, Federica. Incredible weeks before deadline
%     \item (segway with pandemic, sitting at home) Vancouver one of my best decisions in life, thank you to Chenyi for sending me along this ride and broadcasting to people outside of physics how cool physics is
%     \item Friends in VAN: Patrick, Karam, Olivia
%     \item All friends in Van. Sorry for not naming anyone I am afraid I would miss someone. Being one of the most relaxed person that exists on earth: Bassel. Parties: Anthony, Annissa.
%     \item Friends in GER: Like no time has passed. For visiting me in Vancouver, Belek, Zaum. 
%     \item Andy Rive Leute for sharing the music taste and inspiring me regularly with new jams. 
%     \item I want to thank Melina, countless hours on the phone, making life feel easy, never stops to surprise me, keeping me grounded, her love and appreciation, for reminding me to slow down and taking time off, amazing holidays, being a safe haven, supporting me...
%     \item I want to thank my brother
%     \item My parents and family. 
% \end{itemize}
The end of my PhD studies marks the 10 years anniversary of my academic career -- which is almost exactly a third of my life --, during which I had the pleasure to work and interact with truly amazing people and make lasting connections and friends.
I feel incredibly grateful and lucky that I have been able to pursue the things I enjoy, and am very aware that this would not have been possible without the support and love of countless people. 
I would like to take this opportunity to thank some of those special people explicitly.
These include colleagues, fellow researchers, friends I have met through university and research, and also friends outside of physics who have supported me and made my life incredibly enjoyable. 
If you are not mentioned here, rest assured that I am still very grateful for all the encounters we have had.
% The end of my PhD studies marks the 10 years anniversary of my academic life (so almost exactly one third of my lifetime!) during which I had the pleasure to interact and work with truly amazing people and make lasting connections and friends.
% I want to take this chance to say thank you to every single one of them. If you are not mentioned explicitly in the following, rest assured that I am still grateful for our encounters. Similar to the work in HEP, which is highly collaborative by nature with individuals working on a small piece of the greater picture, I think about life similarly, in that I believe that every little conversation, every small input from another person, influences us in some way, if not noticeably it does so subconsciously. This makes us the persons we become at the place we end up at. 
% I couldn't be more grateful for the place I ended up at this moment. 
% can nudge you in unexpected directions 
% - HEP is collaborative, you can't manage to do anything on your own.

First, I would like to thank my supervisor, Bernd Stelzer, for the unique opportunity to join SFU in 2018 and the support and guidance over the years in all aspects of a PhD student's life. 
I am grateful for the freedom I was granted during my studies, while always knowing that the door is open to get your expert opinion on any scientific matter.
Your reminders to think about the big picture, your support to participate in physics workshops and conferences, and your unique approaches to the scientific endeavor shaped me as a researcher. 
% and his support to participate in physics workshops and conferences 
% Living a scientific culture that sees the most benefits when researchers have the freedom to pursue what they like and let curiosity guide the way. 

I would also like to thank Mike Vetterli for his great supervision and support especially in my first year at SFU. 
I benefitted greatly from your immense expertise, and learning about your rigorous approaches to scientific problems made me a better researcher. 

I would also like to thank Dugan O'Neil, the third member of my supervisory committee, for your support and feedback throughout the years. 

For almost the entire five years at SFU, I have had the chance to work closely with Konstantin Lehmann and am very grateful that we have become good friends. Sharing ideas, discussing complex problems, and working toward the same goals were not only immensely helpful with you, but also truly enjoyable. 
% , and I am looking forward for our next encounter. 

I cannot leave unmentioned Eric Drechsler. Thank you for writing the most refreshing meeting announcements, sharing your knowledge, and demonstrating how delightful collaboration can be.

I would also like to thank the broader SFU physics group, for inspiring chats during lunch, for great social events, and, of course, to my fellow office colleagues for great discussions about HEP.
% I wished we would have been able to live a normal university and office culture for longer than the pandemic allowed. 
I owe an explicit thanks to Steven Large. Sharing the PhD experience with you has been a delight, and thank you for having the best music taste. 

It is not too far from SFU to UBC (depending on who you ask), so I would like to acknowledge the connections and friends I have made at UBC, made possible by the joint meetings of Bernd's and his brother Oliver Stelzer's group. 
Thank you for fostering a science culture where everyone is welcome to share their ideas and opinions. 
In particular, I would like to thank Robin Hayes, a fellow student, colleague, and friend from whom I have benefited greatly. You were never too tired to do what needed to be done, and your thorough approaches and curiosity to get to the bottom of everything have motivated me over the years and made working with you a pleasure. 
%I wish you all the best for your new challenges, and hope to see you soon, at Kitsilano beach or somewhere else. 

Prior to my PhD studies at SFU, I completed five years of bachelor's and master's studies in Freiburg. 
The Freiburg HEP group, especially Karl Jakobs' group, got me passionate about particle physics and laid the foundation for my research life.
In particular, I would like to thank Karsten Köneke for the great supervision during my master's studies.
From him, I learned most of what it takes to be a researcher in a large collaboration like the ATLAS experiment, which helped me a lot, especially at the beginning of my PhD studies.

At Freiburg University, I also made lasting connections and friends. Thank you Brian Moser, Manuel Guth, and Christopher Böhm for funny, intense, and inspiring discussions about science and life over the years.
I am especially grateful to Brian Moser. The time I shared an office with you was one of my favorite times during my academic life. Thanks for helping me understand countless scientific problems, always sharing your insights and knowledge about the most pressing topics in particle physics, and for being a true inspiration in life.

% \item In particular the Jet/Etmiss group. Incredible workshops at the HCW. Tae Hyoun Park
% \item HWW group: Karsten, Benedict, Carsten, Yun-Ju
% \item HWW task force: Hayden, Robin, Konstantin, Federica. Incredible weeks before deadline
% Segway: Got to know a lot of people via the ATLAS collaboration. Countless!
% Because of the enormous team work necessary to make valuable research in HEP
% Valuable research in HEP is not possible without collaboration. 
I also owe a big thanks to the fellow researchers I have met through working in two different subgroups of the ATLAS collaboration. 

The Jet/ETmiss subgroup offered an especially pleasant working environment, with everyone trying to support each other. I had an amazing time at two of the Hadronic Calibration Workshops, which I am sure will remain prime examples of fruitful discussion and collaboration for me in the years to come. 

I have been involved in the \HWW analysis subgroup for about 6 years.
Thank you to all conveners that organized and managed the group, and aligned the group's work with the big picture goals: Frank Filthaut, Kathrin Becker, Jonas Strandberg, Claudia Bertella, Kristin Lohwasser, Yun-Ju Lu, Benedict Tobias Winter, and Carsten Burgard. 
I owe a particular thanks to Benedict, who did an amazing job guiding the group through a long and intense review process for the analysis described in this thesis. 
Many thanks also to Carsten Burgard and Ralf Gugel, who are the pillars of the analysis software framework, for tirelessly supporting people in software-related questions, and also helping me immensely in my first years in the \HWW subgroup. 
I cannot leave unmentioned David ``Git'' Shope, who gave me incredible guidance especially in my first years in the ATLAS collaboration, and as analysis contact and paper editor went through a lot of the hard work necessary to publish our analysis. I do not think it is an exaggeration to say that your appreciation for Git version control made me think differently about software and collaboration. 
%My appreciation for git, and thus my ability to collaborate, would not be at the current level without his help. 

It was also an incredible pleasure for me to work closely with a number of fellow PhD students in the \HWW subgroup: 
Federica Pasquali, Hayden Smith, Tae Hyoun Park, Robin Hayes, Konstantin Lehmann. 
I will never forget the weeks leading up to the Summer 2020 and February/March 2021 deadlines to prepare our analysis for publication. We did it!
This might have been the most efficient teamwork I will ever experience as we worked in day and night shifts in Europe and Canada. 
Much of it was hard work, yes, but mostly it was a joy being part of such an amazing research team that relentlessly worked toward the same goals. It was an honor to have had the opportunity to experience something like that. 

I am also grateful for having had the opportunity of spending in total almost a year at CERN, to attend several workshops and conferences, and also to call it my office environment for longer periods of time. 
I had the pleasure to meet and work with the brightest people I have ever met, and I am deeply grateful for having been part of such an amazing community. 
%It has been an honor to feel like a small puzzle piece of the big picture.
% CERN remains the most inspiring place I have ever been.
% \item ATLAS groups and everybody at CERN: the brightest people I have ever met were people at CERN.
% individuals working on a small piece of the greater picture. Truly grateful for all encounters at CERN, talking to some of the brightest people I have ever met. 
% Work in HEP, which is highly collaborative by nature with individuals working on a small piece of the greater picture. Truly grateful for all encounters at CERN, talking to some of the brightest people I have ever met. 

Besides the people I have met through research in particle physics, I would like to thank many of the people and friends that have been part of my life in the past years, beginning with the people I got to know in Vancouver. 

Patrick Mayerhofer, I cannot overstate how grateful I am that our paths crossed at SFU. My experience in Vancouver would have been much different and likely much less enjoyable without the opportunity to share it with you.
The same goes for Karam Elabd, you have influenced me in many ways and got me passionate about things I did not even know existed before I met you. The both of are a true inspiration in life.
Olivia Aguiar, thanks for all the great conversations and support in the ups and downs of life. 
Thanks also to my ex-roommate, Bassel Tarhini, for letting me learn about a new way of thinking about life. 
Anissa, Anthony, Abbas, Mimzy, everyone I have met through SFU and elsewhere in Vancouver, and of course my current teammates at Narps FC, thanks for making my time in Vancouver outside of physics interesting and enjoyable. 
Many thanks also to Chenyi Yue for giving me the push I needed to consider Vancouver for my PhD studies in the first place. 

Since moving to Canada, I have been able to keep in touch with most of my friends from Germany. 
I am immensely grateful to call myself part of a special group of friends who have not broken apart over the years but were able to keep in touch and collect and share amazing memories.
Ich glaube das ist wirklich etwas Besonderes und ich weiß das unglaublich zu schätzen. Danke ihr Hobelfritzen: 
Anton Koch, Dominik Rockenberger, Jonas Belke, Lukas Weniger, Lukas Zaum, Manuel Jäger, Martin Wehr, Niclas Braun, Nils Erley, Philipp Rassbach, Richard Combé, Simon Neuhaus.
I owe a special thanks to Jonas Belke and Lukas Zaum, who visited me in Vancouver, who I can always count on, and who never hesitate to challenge me in whatever new sports they come up with. 
A special thanks to my brother, Manuel Jäger, for always making it feel like no time has passed when we see each other. You are very dear to my heart. 

A very special place in my life takes Melina Frietsch. 
I cannot put into words how lucky I feel that we met.
% and that we have been able to share so many amazing moments together in the past years.
Thank you for always supporting me, being my safe haven, and being patient with me. 
%I am in debt to you.
You keep surprising me and make life together with you a wonderful journey.
%I cannot stress enough how excited I am that you are moving to Vancouver, and we are starting a new chapter in life together.
% your love and appreciation means everything to me. 
%I did not know how enjoyable video chats can be before talking to you
% Your love, appreciation
% I feel like I am in debt with you. 
% I would like to thank Melina, countless hours on the phone, making life feel easy, never stop to surprise me, keeping me grounded, her love and appreciation, for reminding me to slow down and taking time off, amazing holidays, being a safe haven, supporting me...

% Maybe in Deutsch!
Finally, I would like to express my deep gratitude to my parents and family for everything they provide me, for their support and advice in every situation in life. 
Danke, dass ihr meine Träume unterstützt, auch wenn das bedeutet, dass ich mehr als 8000 Kilometer entfernt von euch bin. 
Ich bin unglaublich dankbar darüber, dass ich immer auf euch zählen kann. 

\paragraph{}\mbox{}\\
To all of you, I wish you all the best for your future!
Thank you for being part of my journey. 