

- This work would not have been possible without the support of countless people.
- HEP is collaborative, you can't manage to do anything on your own.

\begin{itemize}
    \item Supervisors: Bernd, expertise, trust, for support to travel, pitching new ideas, encouraging to apply for grants, competitions, .... Mike. expertise, support, ...
    \item Eric: for most hilarious/inspiring/... meeting announcements
    \item SFU students, in particular, Konstantin for great discussions, Steven Metalconcerts
    \item Freiburg students, Karsten Koeneke
    \item Brian, Manu, Chris Boehm
    \item ATLAS groups and everybody at CERN: the brightest people I have ever met were people at CERN.
    \item In particular the Jet/Etmiss group. Incredible workshops at the HCW. Tae Hyoun Park
    \item HWW group: Karsten, Benedict, Carsten, Yun-Ju
    \item HWW task force: Hayden, Robin, Konstantin, Federica. Incredible weeks before deadline
    \item (segway with pandemic, sitting at home) Vancouver one of my best decisions in life, thank you to Chenyi for sending me along this ride and broadcasting to people outside of physics how cool physics is
    \item Friends in VAN: Patrick, Karam, Olivia
    \item All friends in Van. Sorry for not naming anyone I am afraid I would miss someone. Being one of the most relaxed person that exists on earth: Bassel. Parties: Anthony, Annissa.
    \item Friends in GER: Like no time has passed. For visiting me in Vancouver, Belek, Zaum. 
    \item Andy Rive Leute for sharing the music taste and inspiring me regularly with new jams. 
    \item I want to thank Melina, countless hours on the phone, making life feel easy, never stops to surprise me, keeping me grounded, her love and appreciation, for reminding me to slow down and taking time off, amazing holidays, being a safe haven, supporting me...
    \item I want to thank my brother
    \item My parents and family. 
\end{itemize}

- This work would not have been possible without the support of countless people.

The end of my PhD studies marks the 10 years anniversary of my academic career (so almost exactly one third of my lifetime!), during which I had the pleasure to interact and work with truly amazing people and make lasting connections and friends.
I feel incredibly grateful and lucky that I have been able to pursue the things I enjoy, and am very aware that this would not have been possible without the support and love of countless people. 
I want to take this opportunity to thank some of those special people explicitly.
If you are not mentioned in the following, rest assured that I am still deeply grateful for our encounters.
% The end of my PhD studies marks the 10 years anniversary of my academic life (so almost exactly one third of my lifetime!) during which I had the pleasure to interact and work with truly amazing people and make lasting connections and friends.
% I want to take this chance to say thank you to every single one of them. If you are not mentioned explicitly in the following, rest assured that I am still grateful for our encounters. Similar to the work in HEP, which is highly collaborative by nature with individuals working on a small piece of the greater picture, I think about life similarly, in that I believe that every little conversation, every small input from another person, influences us in some way, if not noticeably it does so subconsciously. This makes us the persons we become at the place we end up at. 
% I couldn't be more grateful for the place I ended up at this moment. 
% can nudge you in unexpected directions 
% - HEP is collaborative, you can't manage to do anything on your own.

First, I would like to thank my supervisor, Bernd Stelzer, for the unique opportunity to join SFU in 2018 and the support and guidance in all aspects of a PhD student's life. 
I am very grateful for the freedom I was granted during my studies, while always knowing that the door is open to get his expert opinion on any scientific matter.
His reminders to think about the big picture and the encouragement to pitch new ideas, his support to participate in physics workshops and conferences, and his unique approaches to the scientific endeavor shaped me as a researcher. 
% and his support to participate in physics workshops and conferences 
% Living a scientific culture that sees the most benefits when researchers have the freedom to pursue what they like and let curiosity guide the way. 

I would also like to thank Mike Vetterli, who I worked with closely during my first year at SFU, for his great supervision and support.
I benefitted greatly from his immense expertise, and learning about his rigorous approaches to scientific problems made me a better researcher. 

I would also like to thank Dugan O'Neil, the third member of my supervisory committee, for his support and feedback throughout the years. 

For almost the entire five years at SFU, I have had the chance to work closely with Konstantin Lehmann, who shared almost the same timeline for his PhD studies. Sharing ideas, discussing complex problems, and working toward the same goals were not only immensely helpful, but also truly enjoyable. I am very grateful that we have become friends, and I am looking forward for our next encounter. 

I cannot leave unmentioned Eric, who started as a Postdoc in Bernd's group. Thank you for writing the most refreshing meeting announcements, sharing your knowledge, and demonstrating how delightful collaboration can be.

I would also like to thank the broader SFU physics group, for inspiring chats during lunch, for great social events, and, of course, to my fellow office colleagues for always opening the office door when I forgot my keys again. 
I wished we would have been able to live a normal university and office culture for longer than the pandemic allowed. 
I would also like to thank explicitly Steven Large, a fellow student who became a true friend. Spending time with you has been a delight, and thank you for having the best music taste. 

It is not too far from SFU to UBC (depending on the person you ask, of course!), so I would also like to acknowledge the connections and friends I have made at UBC, made possible by joint meetings of Bernd's and his brother Oliver Stelzer's group. 
Thank you for living an open scientific culture where people are able to share ideas and opinions, as well as making lasting connections. 
In particular, I would like to thank Robin Hayes, a fellow student, colleague, and friend. It was truly inspiring to work together with you. Your thorough approaches, tireless work ethics, and curiosity to get to the bottom of anything, have motivated me over the years and made work a pleasure. I wish you all the best for your new challenges, and hope to see you soon, at Kitsilano beach or somewhere else. 

Before my PhD studies at SFU, I studied five years of Bachelor and Master studies in Freiburg. 
The Freiburg HEP group made particle physics my passion, and laid the foundations for my research life.


Valuable research in HEP is not possible without collaboration. 

individuals working on a small piece of the greater picture. Truly grateful for all encounters at CERN, talking to some of the brightest people I have ever met. 

% Work in HEP, which is highly collaborative by nature with individuals working on a small piece of the greater picture. Truly grateful for all encounters at CERN, talking to some of the brightest people I have ever met. 


To everyone, I wish you all the best for your future. Thank you for having been or still being part of my journey. 
