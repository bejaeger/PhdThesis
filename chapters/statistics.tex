\chapter{Statistical Analysis Techniques}
\label{chap:statistics}
Collider experiments such as the ATLAS detector at the LHC, simply put, are \emph{counting experiments}.
Sophisticated statistical methods are used to extract physics parameters such as cross sections from the counting experiments while accounting for all underlying uncertainties. 
% to express the counting measurements in terms of physics parameters such as cross sections and to allow for a full consideration of the underlying uncertainties. 
This chapter briefly outlines the statistical concepts used in the cross-section measurement of \HWW\ decays presented in \cref{chap:hww}.
The \HWW analysis relies on comparisons between the number of events measured in data and their corresponding SM expectation obtained from MC simulated samples. The expectation value is a complicated quantity and depends on many parameters. This is discussed in \cref{sec:exp-value}.
\Cref{sec:likelihood} then describes how the MC simulations are used to construct a statistical template that is fit to the actual data using a maximum likelihood estimation. This allows the extraction of the physics parameters from the fitted MC template.
The procedure to test different signal hypotheses with frequentist methods, for example to claim the discovery of a new signal, is outlined in \cref{sec:hypothesis-testing}.
This chapter is based to a large extent on \ccite{Cowan:2010js}.
% The counting is of the number of collision events in different phase space regions. 
% The number of events as measured in data can be compared with the value predicted by theory.
% Given the total amount of collision events that were produced, as measured by the integrated luminosity, the probabilities of the occurrence of different physics processes can be measured, known as cross sections.
% To measure the cross sections, as well as make other statistical inferences, sophisticated statistical analyses are required. 
% The following explains the statistical setup used for the work presented in this thesis.

%%%%%%%%%%%%%%%%%%%%%%%%%%%%%%%%%%%%
% Expectation value
\section{Expectation Value}
\label{sec:exp-value}
%While in data, no information of the actual underlying physics process that give rise to certain detector signals is available, MC simulations can be used to infer the expected contributions of different processes.
% per measurement region.  
%The measurement regions typically constitute different bins of a distribution of an observable, but can also include  different phase space regions to measure certain properties from the data. 
%The measurement regions can be different bins of a distribution as well as different phase space regions selected according to kinematic and topological requirements.
The physical processes that contribute to a particular analysis are typically divided into signal processes and background processes, with the exact distinction depending on the goals of the statistical analysis.
Given a binned data distribution of a particular observable with $n_i$ data events per bin $i$, the corresponding expectation value can be written as
\begin{equation}
    \text{E}[n_i] = \sum_{\sigma \in \pmb{\sigma}} \mu_\sigma s_{\sigma,i}(\pmb{\theta}_s, \pmb{\gamma}) + \sum_{\beta \in \pmb{\beta}} \nu_\beta b_{\beta,i}(\pmb{\theta}_b, \pmb{\gamma}),
\end{equation}
where the first sum is over all signal categories $\sigma \in \pmb{\sigma}$, each with an associated expected number of events $s_{\sigma, i}$, and the second sum is over all background categories $\beta \in \pmb{\beta}$ with expected number of background events $b_{\beta,i}$. 
%The multipliers $\mu_\sigma$ ($\nu_\beta$) scale the amount of signal (background) and are called \emph{signal strengths} (\emph{normalization factors}). 
The multipliers $\mu_\sigma$ scale the amount of signal events of category $\sigma$ and are known as \emph{signal strengths}. 
The number of signal categories depends on the specifics of the analysis. The multipliers $\nu_\beta$ have the same effect, from the statistical point of view, and scale the amount of background events of background category $\beta$. They are referred to as \emph{normalization factors} (NFs). 
The expected number of events, $s$ and $b$, depend on a set of so-called \emph{nuisance parameters} (NPs), $\pmb{\theta}_s$, $\pmb{\theta}_b$, and $\pmb{\gamma}$. These parameters encode the effects of systematic uncertainties ($\pmb{\theta}_s$ and $\pmb{\theta}_b$) as well as statistical uncertainties ($\pmb{\gamma}$). In the following, all NPs, including the NFs, are collectively referred to as $\pmb{\theta} = \{\pmb{\theta}_s, \pmb{\theta}_b, \pmb{\gamma}, \pmb{\nu}\}$. 
% and are fit simultaneously with the signal strengths and NFs. 

% Check out master thesis

%%%%%%%%%%%%%%%%%%%%%%%%%%%%%%%%%
% Maximum Likelihood Method
\section{The Likelihood Function}
\label{sec:likelihood}
%in order to find the values of the POIs that make the observed data most likely under the assumed statistical model. 
%To this end, the MC simulations serve to construct a template that is fit to data so that the observed data is most likely under the assumed statistical model. \Mnote{}{Maybe rephrase cause this is close to wikipedia}
The statistical model is based on a likelihood function, or simply \emph{likelihood}, that consists of products of probability density functions (PDFs). The full likelihood can be decomposed into three different terms,
% The statistical model is based on a template constructed with MC simulations that is fit to the data. 
% The full likelihood can be written as
\begin{equation}
    \label{eq:likelihood-three-terms}
    \Likelihood = \Lmeas \times \Lsyst \times \Lstat,
\end{equation}
that are all multiplied by each other.

The first term, \Lmeas, measures the probability of measuring $n_i$ data events when $E[n_i]$ are expected assuming Poisson probabilities, 
\begin{equation}
    \Lmeas = \prod_{i \in \text{bins}} \text{Pois}(n_i | E[n_i] ),
\end{equation}
where the product goes over all phase space regions and histogram bins of an observable (together simply called bins) included in the measurement. The histogram bins correspond to a binned distribution of an observable, for example, in a signal-sensitive phase space region that discriminates well between signal and background. The other phase space regions are typically selected such that they are pure in specific background categories. They serve as subsidiary measurements and allow determining the normalization of certain background categories from the data.

The second term, \Lsyst, encodes the effect of the systematic uncertainties in the likelihood. Most commonly standard Gaussian functions are used,
\begin{equation}
    \Lsyst = \prod_{\theta \in \{ \pmb{\theta}_s,\pmb{\theta}_b \}} \text{Gauss}(0 | \theta, 1),
\end{equation}
where the NPs are normalized and scaled, such that they have a mean of zero and a unit standard deviation.
The effect of most NPs is modelled by varied templates -- i.e. varied values of $\text{E}[n_i]$ -- corresponding to the $\pm 1$ variation of the respective NP. This is done by either propagating the effect of a previously measured uncertainty through the analysis -- as is typically the case for uncertainties related to the experiment -- or -- for the majority of uncertainties related to theory -- by comparing different theoretical models with each other and defining one standard deviation ad-hoc from the collection of available theoretical models.
To provide a continuous description of the effect of each NP, piece-wise linear interpolation is used.
The effect on $\text{E}[n_i]$ of a single source of uncertainty is typically encoded in terms of two components: the first varies the event yields in each region corresponding to the uncertainty in the overall normalization in that region (\emph{normalization uncertainty}), and the second, if applicable, varies the event yield in each histogram bin corresponding to the uncertainty in the shape of the distribution (\emph{shape uncertainty}).

The third and last term in \cref{eq:likelihood-three-terms}, \Lstat, accounts for the finite statistical precision of the MC template. For each bin, one uncertainty component is considered in the MC template, following the light version of the Barlow and Beeston procedure~\cite{BARLOW1993219}. 
The statistical uncertainties encoded with the NPs $\pmb{\gamma}$ enter the likelihood as
\begin{equation}
    \Lstat = \prod_{i \in \text{bins}} \text{Gauss}(\beta_i | \gamma_i, \sqrt{\gamma_i \beta_i}),    
\end{equation}
where $\beta_i$ corresponds to $1 / \sigma_{\text{rel}, i}$, with $\sigma_{\text{rel}, i}$ being the relative statistical uncertainty in bin $i$. 

% - Explain three terms: 
%     1. Extended Maximum Likelihood (see paper: https://www.sciencedirect.com/science/article/abs/pii/0168900290913348?via%3Dihub)
%         - Signal regions
%         - subsidiary measurements
%     2. Syst Likelihood with Gaussian constraints
%     3. Stats likelihood 

%The best-fit values of the parameters $\pmb{\theta}$, $\mu$, $\nu$ in the likelihood, that is the values that make the observed data most likely under the assumed statistical model, 

\section{Maximum Likelihood Estimation}
The values of the parameters $\{\pmb{\mu}, \pmb{\theta} \}$ that make the observed data most likely under the assumed statistical model, known as \emph{best-fit values}, are obtained by fitting them simultaneously using a maximum likelihood method.
%The parameters of the likelihood, $\{\pmb{\mu}, \pmb{\theta} \}$, are fit simultaneously using a maximum likelihood method. 
%This cannot be computed
%The values of the parameters $\{\pmb{\theta}, \mu, \nu\}$, that make the observed data most likely under the assumed statistical model are known as \emph{best-fit values} and labelled as $ \{\hat{\pmb{\theta}}, \hat{\pmb{\mu}}, \hat{\pmb{\nu}} \}$.
In practice, the best-fit values, labelled as $\{\hat{\pmb{\mu}}, \hat{\pmb{\theta}} \}$, are found by minimizing the negative logarithm of the likelihood,
\begin{equation}
    \label{eq:best-fit-values}
    \{\hat{\pmb{\mu}}, \hat{\pmb{\theta}} \} = \text{arg} \min_{\{\pmb{\mu}, \pmb{\theta}\}} \left( - \ln (\Likelihood) \right),
\end{equation}
as it is numerically more stable. This is done using numerical methods implemented in the MINUIT package~\cite{James:1975dr}.
%using numeric procedures. 
The fitted $\hat{\pmb{\mu}}$ can then be directly translated into physics quantities such as cross sections by comparing them to the values that were assumed in the original MC template.

% For the background categories that can be measured well in a region pure in events of that category, the NFs are typically freely floating in the statistical analysis. For the other background categories the values are fixed to $\nu_\beta = 1$. 

%Typically, these are the \emph{parameters of interest} (POIs) in the statistical analysis.

%%%%%%%%%%%%%%%%%%%%%%%%%%%%%%%%%%%%
% Hypothesis testing
\section{Hypothesis Testing}
\label{sec:hypothesis-testing}
To test the compatibility of the data with a hypothesized value of $\mu$ for a single signal category, the following test statistic is considered,
\begin{equation}
    t_\mu = -2 \ln \lambda(\mu),
\end{equation}
where $\lambda(\mu)$ is the \emph{likelihood ratio} defined as
\begin{equation}
    \label{eq:likelihoodratio}
    \lambda(\mu) = \frac{ \Likelihood \left( \mu, \hat{\hat{\pmb{\theta}}} \right) } { \Likelihood \left( \hat{\mu} , \hat{\pmb{\theta}} \right) }.
\end{equation}
Here, the values $\hat{\hat{\pmb{\theta}}}$ correspond to the NP values that minimize the negative logarithm of the likelihood for a given value of the signal strength $\mu$, and $\hat{\pmb{\theta}}$ are defined as shown in \cref{eq:best-fit-values}. 
According to the Neyman-Pearson lemma~\cite{10.2307/91247}, the likelihood ratio is the most powerful statistical test to reject a certain null-hypothesis against an alternative hypothesis.

The agreement between data and the hypothesized value of $\mu$ is quantified with the $p$-value, which indicates the probability of finding data that is equally or less compatible than the observed data, assuming that the null hypothesis is correct.
The $p$-value can be computed as
\begin{equation}
    \label{eq:p-value-discovery}
    p_{\mu_h} = \int_{t_{\mu_h}, \text{obs}}^{\infty} f \left( t_{\mu_h} | \mu = \mu_h \right) dt_{\mu_h},
\end{equation}
where $f \left( t_{\mu_h} | \mu=\mu_h \right)$ is the probability density function of the test statistic assuming the hypothesized value $\mu=\mu_h$.
The probability density function is generally not known but can be sampled from MC simulations. 
% \begin{equation}
%     \label{eq:p-value-discovery}
%     p_0 = \int_{q_0, \text{obs}}^{\infty} f \left( q_0 | \mu = 0 \right) dq_0,
% \end{equation}
% $f \left( q_0 | \mu = 0 \right)$

To establish the discovery of a signal, the null hypothesis is defined as the background-only hypothesis with $\mu = 0$, and the test statistic $q_0$ can be used, constructed as
\begin{equation}
    q_0 =
     \begin{cases}
       \lambda(0) & \text{for } \hat{\mu} \ge 0, \\
       0 & \text{for } \hat{\mu} < 0,  \\
     \end{cases}
     \label{eq:q0}
  \end{equation}
where $q_0$ is set to zero for negative values of $\mu$, as they are unphysical.
The $p$-value then becomes
\begin{equation}
    \label{eq:p-value-discovery}
    p_0 = \int_{q_0, \text{obs}}^{\infty} f \left( q_0 | \mu = 0 \right) dq_0,
\end{equation}
with $f \left( q_0 | \mu = 0 \right)$ generally unknown. In the large sample limit, $f$ follows a $\chi^2$ distribution and the $p$-value can be written as
\begin{equation}
    \label{eq:p-value-approx}
    p_0 = 1 - \Phi( \sqrt{q_0}),
\end{equation}
where $\Phi$ is the Gaussian cumulative distribution function. 
In HEP, the $p$-value is often converted to Gaussian standard deviations, labelled as $Z$, by using the relation
\begin{equation}
    \label{eq:Z0}
    Z = \Phi^{-1} ( 1 - p_0) \overset{\ref{eq:p-value-approx}}{=} \sqrt{q_0},
\end{equation}
where $\Phi^{-1}$ is the inverse of the Gaussian cumulative distribution function. 
This value is referred to as \emph{discovery significance} in the following. 

It is useful to find a simple approximation of the discovery significance. 
This can be constructed by considering a simple counting experiment with one measurement region and disregarding all uncertainties. In this case, the likelihood function can be written as
\begin{equation}
  \Likelihood(\mu) = \frac{(\mu s +b)^n}{n!} e^{-(\mu s+b)},
\end{equation}
where the expected number of signal and total background events, $s$ and $b$, respectively, are assumed to be known without uncertainty.
By using \cref{eq:Z0}, a simple approximation for the discovery significance $Z$ can be found,
% Assuming the nominal signal hypothesis with $\mu = 1$ one can replace the value $n$ by its Asimov value $s+b$ and can approximate the median significance, by using the definition of $q_0$ from \cref{eq:q0}, to obtain
%By asimov data set the value $n$ can be replaces by $s+b$
%% \begin{equation}
%%   Z = \sqrt{q_0} = \sqrt{-2 \text{ln} \frac {L(0)}{L(\hat{\mu})}} = \sqrt{2 \left( n \text{ln}\left( \frac{n}{b} \right) + b - n \right)} = \sqrt{2 \left( \left(s + b\right) \text{ln}\left(1 + \frac{s}{b}\right) - s\right)}.
%% \end{equation}
\begin{equation}
    \label{eq:discovery-significance}
  Z0(s, b) = \sqrt{2 \left( \left(s + b\right) \text{ln}\left(1 + \frac{s}{b}\right) - s\right)},
\end{equation}
which is valid for large enough $b$.
A more sophisticated approximation that also accounts for uncertainties on the expected number of background events is given by
\begin{equation}
    \label{eq:simple-sign}
    Z0(s, b, \sigma) = 
    \begin{cases}
    + \sqrt{ 2 \left( n \ln \left[ \frac{n \left( b + \sigma^2 \right)}{b^2 + n \sigma^2} \right] - \frac{b^2}{\sigma^2} \ln \left[ 1 + \frac{\sigma^2 \left(n - b \right) }{b \left( b + \sigma ^2 \right) } \right] \right)   } & \text{if } n \geq b, \\
    - \sqrt{ 2 \left( n \ln \left[ \frac{n \left( b + \sigma^2 \right)}{b^2 + n \sigma^2} \right] - \frac{b^2}{\sigma^2} \ln \left[ 1 + \frac{\sigma^2 \left(n - b \right) }{b \left( b + \sigma ^2 \right) } \right] \right)   } & \text{if } n < b, \\
\end{cases}
\end{equation}
where $\sigma$ is the uncertainty on the expected number of background events, $b$, and $n$ corresponds to the sum of the expected number of signal and background events, $n = s+b$.
More information on how this formula is derived can be found in \ccite{ATL-PHYS-PUB-2020-025}.


% From master thesis
% The common re-quirement for a signal to claim discovery is a minimum significance of Z = 5, which corresponds to p = 2.87 · 10−7.

