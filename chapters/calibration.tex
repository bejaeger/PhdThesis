\chapter{Measurement of the Noise Term of the Jet Energy Resolution}
\label{chap:calibration}

- Jets are complicated

- Both the non-compensating nature and the differences between electromagnetic and hadronic showers pose great challenges for calibrating the energy correctly.

- To correct for the loss of energy in a hadron shower, dedicated calibrations are performed typically at the level of the fully reconstructed objects.


\section{Detector Response to Jets}
- As mentioned in \cref{chap:objects}, reconstructed clusters are calibrated at the EM scale. 
- Hadronic showers cannot be captured and some signals are lost (non-compensating). 
%Most secondary hadrons produced in the nuclear interactions are charged or neutral pions.
- A fraction of the energy of the particles participating in a hadronic shower cannot be captured due to several reasons:
\begin{itemize}
    \item Showering particles can transfer energy to atomic nuclei that are broken up during the interaction or remain in excited states.
    \item Showering particles can decay to non-interacting secondaries (for example in $\pi \rightarrow \mu \nu_\mu$) that escape the detector.
    \item Neutrons can be produced that undergo mostly elastic scattering and can thus escape the detector.
\end{itemize}
This phenomenon is known as \emph{non-compensating}.

Other factors contribute to energy losses:
- Dead material
- clustering inefficiencies
- non-accounted deposits: jet algorithm have a finite radius -> maybe there are deposits outside that cone coming from the original quark

- The e/h ratio, where e is the response to electromagnetic and h the response to hadronic interactions, is an important quantity 


While the charged pions further interact hadronically, the neutral pions decay almost instantaneously into a pair of photons, that are more directly absorbed in the calorimeters. This results in a fraction of the hadronic shower being of electromagnetic nature.

R = e*fem(E) + h * (1 - fem(E)E)



\section{Jet Energy Corrections}

- Dandoy: "corrections derived from MC simluation and from data, the latter is referred to as in-situ calibraitions."

- Stages of calibration. 
- Define different "scales"

-> reconstructed jets (EMTopo, pflow, rscan) -> constituent scale

-> pile-up subtracted scale

-> JES scale

-> GSC scale (only affects resolution, not JES on average! (discuss?))
% From DANDOY: The GSC shifts the energy of individual jets while maintaining the mean energy response derived in the previous jet energy scale calibration.    

-> in-situ calibration (only affects data! can keep this short as it does not affect JER measurements)

\Minote{}{Could make a table for the above and provide use cases for each scale}


An alternative approach to correct for energy losses is the so-called \emph{local hadronic cell weighting} (LCW), which applies energy corrections already at cluster level. Different variables can be defined to characterize a topo-cluster based on its shape and other properties. These observables, known as \emph{cluster moments}, are used to extract information about the hadronic signal content in a given cluster which in turn is used to correct the energy to the \emph{LCW scale}. More information can be found in \ccite{PERF-2014-07}.
\Minote{}{Maybe put entire Rscan studies in a nice appendix? Explaining new jet finding, LCW cluster usage, JES, results? -> I think that would make a lot of sense as I don't know too much about these jets!}

\begin{figure}
    \newImageResize{figures/calibration/jes-calibration.pdf}
    \caption{Summary of the different stages of the jet energy calibration. Each correction is applied to the four momentum of the jet. Taken from \ccite{JETM-2018-05}}
    \label{fig:jes-calibration}
\end{figure}



\section{Jet Energy Resolution}


\subsection{Contributing Effects and Parametrisation}
% From Karl Jakobs Slides
% For EM Calorimeter:
- Shower fluctuations: number of particles produced in a shower -> 1 / sqrt(E)
- Photo-electron statistics: inefficiencies converting photons to electrical signals -> 1/sqrt(E)
- Shower leakage: out of radius of jet or out of detector  -> constant 
- sampling fluctuations: Number of low-energy electrons crossing active layers -> 1 / sqrt(E)

smaller effects:
- track length fluctuations
- landau fluctuations: asymptotic energy loss distribution for thin active layers is laundau instead of gaussian. 

More effects from hadronic interactions:
- Fluctuations in EM fraction
- Fluctuations in neutron component
- Fluctuations in invisible energy
- Fluctuations in binding energy losses and nuclear excitation losses
- Fluctuations in the number of heavily ionizing particles

- Full N,S,C Formula for JER


\subsection{Determination of the jet energy resolution in data}

- determined for jets at JES

- MC JES from fitting Gaussian to truth response Edet_JES / Etruth in bins of Etruth

- JER scales with JES!!!

- Explain Dijet balance method but keep this very brief and forward to paper "
-> But keep plot to see improvement!

- Noise term measured separately -> point to next section


\section{Measurement of the Noise Term}

- const scale noise with random cones scaled to JES 
- adding electronic noise from 

- Ingredients (put this in table):

- zero-bias data sample with 2017 pile-up (for random cones)
- zero-bias MC sample with 2017 pile-up -> for randon-cones non-closure uncertainty

- standard dijet sample with 2017 pile-up (-> for JES scaling)
- dijet sample without pile-up overlaid  (-> for electronic noise AND JES scaling)





\subsection{The Random Cones Method}

- zero bias data
- random cones

\begin{figure}
    \subfloat[] {
        \newImageResizeHalf{figures/calibration/zero-bias-event-deposits.pdf}
    }
    \subfloat[] {
        \newImageResizeHalf{figures/calibration/random-cones-difference.pdf}
    } \\
    \subfloat[] {
        \newImageResizeHalf{figures/calibration/const-noise-vs-eta.pdf}
    }
    \subfloat[] {
        \newImageResizeHalf{figures/calibration/const-noise-vs-mu.pdf}
    }
    \caption{Taken from \ccite{SinglePublicPlotZeroBiasDeposits,PublicPlotsJER}.}
    \label{fig:const-scale-noise}
\end{figure}


\begin{figure}
    \subfloat[] {
        \newImageResizeCustom{0.5}{figures/calibration/pile-up-jer-vs-pt.pdf}
    }
    \caption{Taken from \ccite{SinglePublicPlotZeroBiasDeposits}.}
    \label{fig:pile-up-jer-vs-pt}
\end{figure}


\subsection{Extracting the electronic noise from Monte Carlo}
- MC JER from sample with NO pile-up overlaid!


\subsection{Systematic uncertainties}


\begin{figure}
    \subfloat[] {
        \newImageResizeCustom{0.5}{figures/calibration/random-cones-non-closure.pdf}
    }
    \caption{Taken from \ccite{JETM-2018-05}.}
    \label{fig:non-closure}
\end{figure}


\subsection{Noise Term Results}
\subsubsection{Calorimeter jets}
\subsubsection{Particle-flow jets}

\begin{figure}
    \subfloat[] {
        \newImageResizeCustom{0.7}{figures/calibration/noise-term-results-pflow.pdf}
    }
    \caption{Taken from \ccite{JETM-2018-05}.}
    \label{fig:noise-term-results-pflow}
\end{figure}



\subsubsection{Results for R-scan Jets}
\Rinote{Probably I will put this in the backup}


\section{Jet Energy Resolution Combination}

- Input is dijet measurement and noise term

\begin{figure}
    \subfloat[] {
        \newImageResizeHalf{figures/calibration/jer-combination-vs-pt.pdf}
    }
    \subfloat[] {
        \newImageResizeHalf{figures/calibration/jer-combination-vs-eta.pdf}
    } \\
    \subfloat[] {
        \newImageResizeHalf{figures/calibration/jer-unc-vs-pt.pdf}
    }
    \subfloat[] {
        \newImageResizeHalf{figures/calibration/jer-unc-vs-eta.pdf}
    }
    \caption{Taken from \ccite{PublicPlotsJER}.}
    \label{fig:jer-combination-results}
\end{figure}



\section{Future improvements}

- split in mu bins!
- Factoring in GSC
- Different fit model