\chapter{Measurement of the Noise Term of the Jet Energy Resolution}
\label{chap:calibration}

\section{Jet Calibration Procedure}
The so-called \emph{local hadronic cell weighting} makes corrections already at cluster level by using information from the shape and internal signal distribution of a given cluster\footnote{Many variables can be defined to characterize a topo-cluster. These observables known as \emph{cluster moments} are used to extract information about the hadronic signal content in a given cluster which in turn drives the decision about what energy correction to apply. More information can be found in \ccite{PERF-2014-07}}.

- Also look at Schouten's thesis who has a very good explanation for these effects

The more widely adopted strategy in ATLAS is to correct for the energy after the full objects have been reconstructed.




\subsection{Jet energy scale}
Great section on "Jet Corrections" in Schouten's thesis.

\subsection{Jet energy resolution}
\section{Methodology of the Noise Term Measurement}
\subsection{The Random Cones Method}
\subsection{Extracting the electronic noise from Monte Carlo}
\subsection{Systematic uncertainties}
\subsection{Jet energy resolution combination}
\section{Results for Small-Radius Jets}
\subsection{Calorimeter jets}
\subsection{Particle-flow jets}
\section{Results for R-scan Jets}
