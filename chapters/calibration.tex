\chapter{Measurement of the Noise Term of the Jet Energy Resolution}
\label{chap:calibration}

- Jets are complicated

- Both the non-compensating nature and the differences between electromagnetic and hadronic showers pose great challenges for calibrating the energy correctly.

- To correct for the loss of energy in a hadron shower, dedicated calibrations are performed typically at the level of the fully reconstructed objects.

- "Introduce JES and JER abbreviation"
- The JER describes the precision with which the energy of a jet can be measured.

- A precise knowledge is crucial in order to be able to correct the JER in MC such that it matches the one found in data.

- The following presents the calibrations for \antikt jets with a radius of $R = 0.4$. They are treated as massless four-vectors which means that their energy is equivalent to their momentum\footnote{The terms energy and momentum are thus often used interchangeably throughout this thesis}.
%- Paper: paper presents the strategy used for the determination of the jet energy scale (JES) and resolution (JER) by the ATLAS experiment and its implementation as it pertains to the analysis of data from Run 2 of the LHC.

%- This publication focuses on calibrating jets reconstructed with the anti-𝑘𝑡 [1] algorithm with radius parameter 𝑅 = 0.4.


\section{Detector Response to Jets}
The response of the calorimeter to traversing particles describes the fraction of the particle's original energy that can be measured. The jet response is thus the average detector response to the showering particles that form the jet.
As mentioned in \cref{chap:objects}, the calorimeter cells are calibrated at the EM scale which means that EM interacting particles like photons and electrons are measured with a response of $e = 1$. The energy of hadrons cannot be fully captured and some fraction of the energy does not contribute to the calorimeter signals, resulting in a hadron response of $h < 1$. This is known as \emph{non-compensating} nature.
A hadron shower produced by a jet consists of both components:
An EM component, that occurs because neutral pions decay into pairs of photons which creates EM sub-showers, and a non-EM component, which includes all hadronic interactions.
The jet response can thus be written as
\begin{equation}
    R(E) = e*\fEM(E) + h * \left(1 - \fEM(E)\right),
\end{equation}
where \fEM is the fraction of the jet that interacts electromagnetically. This fraction increases as the energy of the jet increases, because the number of possibilities for producing neutral pions that halt the hadronic decay chain increases. The uncalibrated calorimeter response is therefore \emph{non-linear} with increasing jet energy.

Several physics effects contribute to the non-compensating nature of the detector:
\begin{itemize}
    \item Showering particles transfer energy to atomic nuclei that are broken up during the interaction or remain in excited states. The energy is lost in the binding energy or the excited states of the nuclei.
    \item Showering particles decay to non-interacting secondaries (for example in $\pi \rightarrow \mu \nu_\mu$ decays) that escape the detector.
    \item The production of neutrons that mostly undergo elastic scattering and thus escape the detector.
\end{itemize}
Beside these effects, other factors related to the experiment and reconstruction contribute to potential energy mismeasurements:
\begin{itemize}
    \item Non-functional (\emph{dead}) detector material.
    \item Inefficiencies in the topo-clustering algorithm.
    \item Energy deposits that are not taking into account due to the finite radius of the jet algorithm.
\end{itemize}

To determine the jet response in the experiment, the reconstructed jet \pT is compared to the \pT of a well-calibrated or well-known \emph{reference object}. The reference object may constitute a well-measured physics object or system in data, as explained in \cref{subsec:insitu-calibration}, or a particle-level truth jet, when the jet response is studied in MC.
In the latter case, the truth jets are geometrically matched to the reconstructed jets by requiring them to be within a radius of $\Delta R = 0.3$.
The so-called \emph{truth response} is then defined as the ratio of the reconstructed jet \pT and the \pT of the matched truth jet,
\begin{equation}
    R_{\text{truth}} =  \frac{\pTreco}{\pTtruth},
\end{equation}
which is a useful quantity for calibrating the jet energy, explained in the next section.
%When the energy of the reconstructed jet is calibrated so that $\pTreco = \pTcalib$, the mean jet response is what is considered to be the jet energy scale (JES) and the width of the jet response constitutes the jet energy resolution (JER). The mean and width are determined from a Gaussian fit to the core of the response distribution, as shown in \cref{fig:truth-response}. 

\FloatBarrier
\begin{figure}[t]
    \newImageResizeHalf{figures/calibration/truth-response.pdf}
    \caption{Jet truth response.}
    \label{fig:truth-response}
\end{figure}


\section{Jet Energy Corrections}
\label{sec:jes-calibration}
The goal of the jet energy correction is to correct for the non-linearity as well as the non-compensating nature and bring the average jet response to unity.
To this end, a calibration procedure is performed that consists of several stages, each defining a particular \emph{energy scale}. The different steps are illustrated in \cref{tab:jes-calibration} and briefly explained in the following. More details can be found in \cref{JETM-2018-05}.

%The different steps use either MC simulation or data and are also applied to either MC or data, depending on the correction.
\subsection{Pile-up correction}
After the jet is reconstructed, the jet \pTreco is first corrected on an even-by-event basis to on average remove the energy coming from pile-up interactions. This is done in two steps: First, a correction based on the \pT density of the pile-up, $\rho$, and the jet area, $A$, is applied, and then a residual correction removes any residual \pT dependence on the number of primary vertices (to correct for in-time pile-up contributions), $N_\text{PV}$, and the average pile-up (to correct for out-of-time pile-up contributions), $\mu$.
The $\rho$ observable is estimated by the median \pT density of jets, calculated with jets reconstructed with the \kt algorithm\footnote{The \kt algorithm works similar to the \antikt algorithm (see \cref{subsubsec:jet-algorithm}) but with different definitions of the distance measures:
    \begin{align*}
        d_{ij} & = \text{min}\left(p_{T_i}^2,p_{T_j}^2\right) \frac{\Delta R_{ij}^2}{R^2}, \\
        d_{iB} & = p_{T_i}^2.
        \label{eq:kt-distances}
    \end{align*}
}
with radius parameter $R = 0.4$ and positive-energy topo-clusters at \absetaST{2} as input. The jet area is determined using a procedure known as ghost association, where particles are treated as four-vectors of infinitesimal magnitude, known as \emph{ghost particles}, during the jet reconstruction and assigned to the jet with which they are clustered. The area of a jet is calculated from the fraction of ghost particles associated to that jet.
The residual correction is based on a comparison with geometrically matched truth jets. The energy after all pile-up corrections, referred to as the \emph{pile-up subtracted (energy) scale}, can be written as
\begin{equation}
    \pTPUsub = \pTreco - \rho \times A - \alpha \times \left( N_{\text{PV}} - 1\right) - \beta \times \mu,
\end{equation}
where $\alpha$ and $\beta$ are derived from fits to the \pT dependence on $N_{\text{PV}}$ and $\mu$, respectively.

\subsection{MC-based correction to the particle level}
What follows is a purely MC-based calibration of the entire four-momentum of the jet. The energy and direction of the reconstructed jets is corrected to on average match the one of the associated truth jets. This is done by considering the average jet response, defined by the mean of a Gaussian fit to the core of the $\pTPUsub / \pTtrue$ response distribution. The average response is parametrized as a function of \pTPUsub and the \pT calibration factor (also denoted \emph{JES factor} throughout this thesis) then derived using a numerical inversion procedure~\cite{PERF-2011-03}. A similar procedure is used to correct biases in the reconstructed jet $\eta$, as detailed in \cref{JETM-2018-05}.
Jets calibrated with the MC-based corrections are considered to be at the \emph{JES scale}.

\subsection{Global sequential calibration}
The jet response is dependent on many factors related to the shower development in the calorimeter, such as the shower shape, the energy distribution and flavour of the showering particles, or the particle composition.
Differences are especially observed between quark-initiated and gluon-initiated jets.
% and leads to an increased JER.
To reduce the impact these effects have on the JER, six (five) observables for PFlow jets (EMTopo jets) that model these effects are identified in the \emph{Global sequential calibration} (GSC). The jet truth response is parametrized as a function of these observables and a numerical inversion procedure~\cite{PERF-2011-03} is performed for each of them in sequence.
These corrections reduce the JER while leaving the mean energy response unchanged.
Jets that have the GSC calibration applied are considered to be at the PFlow+JES+GSC scale (or EMTopo+JES+GSC scale).
%- GSC scale (only affects resolution, not JES on average! (discuss?))
% From DANDOY: The GSC shifts the energy of individual jets while maintaining the mean energy response derived in the previous jet energy scale calibration.    

\subsection{\textbf{\emph{In~situ}} calibration}
\label{subsec:insitu-calibration}
A final calibration step accounts for the residual differences between the jet response in data and MC, caused by imperfections in both the detector simulations and physics modelling. This calibration, known as \insitu calibration, is derived in data and only applied to data.
Different event topologies are exploited that allow to measure the jet response with respect to a well-calibrated reference object. The \insitu response is calculated in data and MC and the final correction factor is given by the following double ratio:
\begin{equation}
    c = \frac{R_{\insitu}^{\text{data}}}{R_{\insitu}^{\text{MC}}} = \frac{ \pTcalibData / \pTrefData}{\pTcalibMC / \pTrefMC}.
\end{equation}
The double ratio is parametrized as a function of \pTref and the final calibration obtained again with a numerical inversion procedure~\cite{PERF-2011-03}.

\Minote{}{Could make a table for the above and provide use cases for each scale. Maybe just replace the figure from the paper, which really isn't all that great!}

%- Definition of truth jets reconstructed using stable final-state particles
%Truth jets are reconstructed using stable final-state particles and exclude muons, neutrinos, and particles from pile-up interactions. Truth jets are selected with the same 𝑝T > 7 GeV and |𝜂| < 4.5 thresholds as EMtopo and PFlow jets, and are geometrically matched to those jets using the angular distance Δ𝑅 with the requirement Δ𝑅 < 0.3.


\begin{table}
    \newImageResize{figures/calibration/jes-calibration.pdf}
    \caption{Summary of the different stages of the jet energy calibration. Each correction is applied to the four momentum of the jet. Taken from \ccite{JETM-2018-05}}
    \label{tab:jes-calibration}
\end{table}


\section{Jet Energy Resolution}
Multiple physics and detector effects contribute to a sizeable JER. They scale differently with increasing energy of the jet. For calorimeter-based measurements, the effects can be grouped into three categories captured by three parameters: the \emph{noise term} ($N$), the \emph{stochastic term} ($S$), and the \emph{constant term} ($C$).

The noise term captures resolution effects from energy deposits stemming from pile-up interactions and other noise such as electronic noise. These contributions are approximately independent of the jet energy. A detailed discussion of the noise term is provided in \cref{sec:noise-term-meas}.

The stochastic term is dominated by fluctuations related to the shower development in the calorimeter, whose impact on the JER scale with $\sqrt{\ETjet}$:
%The measurements of a calorimeter are subject to the following fluctuations:
\begin{itemize}
    \item Shower fluctuations: fluctuations in the number of particles produced in the shower and in the EM fraction, \fEM.
    \item Sampling fluctuations: fluctuations in the number of ionising particles crossing the active layers of the sampling calorimeter.
    \item Photo-electron statistics: inefficiencies converting photons to electrical signals in the hadronic tile calorimeter.
    \item Fluctuations in the invisible energy (neutron production, binding energy losses, nuclear excitation losses).
    \item Fluctuations in the number of heavily ionising particles, such as alpha particles.
\end{itemize}
The constant term captures the effect of shower leakage for example from energy depositions outside the jet radius and detector inefficiencies. These effects scale with \ETjet. \footnote{The reason why this term is referred to as the constant term is because typically the relative JER is quoted, see \cref{eq:jer-parametrisation}}

Given these considerations, the relative JER can be parametrised as follows:
\begin{equation}
    \label{eq:jer-parametrisation}
    \frac{\sigma_{p_{\rm T}}}{p_{\rm T}} = \frac{N}{p_{\rm T}} \oplus \frac{S}{\sqrt{p_{\rm T}}} \oplus C.
\end{equation}
which is denoted as \emph{N,S,C parametrisation} throughout this thesis.

% % From Karl Jakobs Slides
% smaller effects:
% - track length fluctuations
% - landau fluctuations: asymptotic energy loss distribution for thin active layers is laundau instead of gaussian.

It should be noted, that this parametrisation is based not only on the physics considerations mentioned above, it is also based on empirical evidence from MC studies. Especially the jet calibration, most notably the GSC, has a non-trivial impact on the JER, which makes it unfeasible to derive a formula for the JER from first principle.
Moreover, the aforementioned physics considerations are only considering calorimeter-type measurements. In the case of PFlow jets, even though some of the aforementioned effects are mitigated - because parts of the calorimeter measurements are replaced by more precise measurements from the ID - additional contributions to the JER need to be considered. The additional contributions are dominated by the imperfect cell subtraction in the PFlow algorithm (see \cref{subsec:pflow-algorithm}), an effect captured in a term called \emph{confusion term}. The confusion term quantifies the amount of incorrectly subtracted energy and is mostly relevant at medium jet \pT, which is detailed in \ccite{PERF-2015-09}.
Despite these limitations of the N,S,C parametrisation for PFlow jets, it empirically proves to be an adequate formula to represent the PFlow jet energy and is therefore also used in the noise term measurement presented in \cref{sec:noise-term-meas}.




\subsection{Estimation of the jet energy resolution in Monte Carlo}
The width of the truth response using fully calibrated jets \TDinote{(JES+GSC)}{Check if this is indeed the standard} divided by the mean,
\begin{equation}
    \mcjer = \sigma \left( \frac{\pTcalib}{\pTtruth} \right) / \mu \left( \frac{\pTcalib}{\pTtruth} \right),
\end{equation}
constitutes the standard definition of the JER in MC (MC JER) simulation.
The width and mean are taken from a Gaussian fit to the core of the distribution, as shown in \cref{fig:truth-response}. The division by the mean corrects for the residual non-closure of the jet calibration, which is sizeable at low \pT\footnote{After the JES calibration, the response diverges from unity by a maximum of about $5\% (3\%, 1\%)$ at $\pTtrue = 20 (30, 50)\,\GeV$~\cite{JETM-2018-05}.}.

\Cref{fig:jer-parametrisation} shows an illustration of the effect of the different terms as a function of jet \pT, with typical values for the three parameters, N, S, and C. The noise term is dominant at low \pT ($\pT \lesssim 40\,\GeV$), the stochastic term at medium \pT ($40 \lesssim \pT \lesssim 1000\,\GeV$), and the constant term at high \pT ($\pT \gtrsim 1000\,\GeV$).

\TDinote{}{Show example of PFlow JER fit?}
\TDinote{}{Then show breakdown of N, S, C terms? Beside it?}
\TDinote{}{Maybe show EMTopo JER and PFlow JER?}


\FloatBarrier
\begin{figure}[t]
    \subfloat[] {
        \label{fig:jer-parametrisation-a}
        \newImageResizeHalf{figures/calibration/NoiseTerm_withPileupResolution_EMPFlowJets_mc16d_00eta07_nominalFit.pdf}
    
    } 
    \subfloat[] {
        \label{fig:jer-parametrisation-b}
        \newImageResizeHalf{figures/calibration/Breakdown-of-JER-parameters-MC16dMCJER.pdf}    
    }
    \caption{MC JER and Illustration of JER parametrisation breakdown.}
    \label{fig:jer-parametrisation}
\end{figure}

\FloatBarrier
\begin{figure}[t]
    \newImageResize{figures/calibration/flow-chart-jer.png}
    \caption{Overview of the different ingredients to perform the jet energy resolution measurement.}
    \label{fig:flow-chart-jer}
\end{figure}


\subsection{Determination of the jet energy resolution in data}
Similar to the \insitu jet calibration, the JER can be derived in data by considering physics processes with a jet recoiling against a well-measured reference object. For \RunTwo, the ATLAS collaboration uses well-defined dijet systems in a method known as \emph{dijet balance} to measure the JER.
In events with clean signatures of two initial state quarks and two final state quarks ($2 \rightarrow 2$), the transverse momenta of the jets are expected to balance each other (they are known to be \emph{back-to-back}), so that any deviation from an exact balance can be attributed to resolution effects.

The dijet-balance method selects jets measured in a well-calibrated detector region ($0.2 \leq \absetadet \leq 0.7$) as reference object. An asymmetry is defined to compare the \pT of the reference jet, labelled as \pTref, with the momentum of the \emph{probe jet} for which the resolution is to be measured. The probe jet is allowed to be anywhere in the detector within \absetadetST{4.5}.
The JER is then extracted in different bins of \pTref and \etadet by an iterative fitting procedure. For more details on how the asymmetry is defined, and the JER extracted, the interested reader is referred to \ccite{JETM-2018-05}. Uncertainties on the dijet JER measurement mostly stem from the presence of additional radiation in the events which spoil the perfect back-to-back signature, biases due to the event selection, and uncertainties from the JES.
The resolution of the JER measured in dijet data events compared to the JER in dijet MC is shown in \cref{fig:insitu-jer-dijet-only}.
As can be seen, the uncertainties become large at low \pT, and the dijet JER measurement is not able to probe the JER to values much lower than $\pT = 50\,\GeV$, which is the regime at which the noise term becomes dominant.\footnote{JER measurements that use different event topologies such as \Zjets events are able to provide more precise results also at lower \pT. For the JER measurement presented in this thesis, they were not included due to other complexities. A detailed discussion about this goes beyond the scope of this thesis. However, the independent derivation of the noise term as presented in \cref{sec:noise-term-meas} is still useful, even if other JER measurements are included.}
The noise term is therefore derived independently, to reduce the JER uncertainties at low \pT. This method is explained in the following section.

\FloatBarrier
\begin{figure}
    \newImageResizeHalf{figures/calibration/insitu-jer-dijet-only.pdf}
    \caption{Dijet JER. Taken from \ccite{JETM-2018-05}.}
    \label{fig:insitu-jer-dijet-only}
\end{figure}


\section{Measurement of the Noise Term}
\label{sec:noise-term-meas}
% - As can be seen in the previous section, the measurement of the JER at low \pT has large uncertainties.
% - Therefore, an independent measurement of the noise term is important to reduce the uncertainties at low \pT.
- The noise term is composed of two terms that can be treated independently. The noise due to pile-up (denoted as \emph{pile-up noise}), \Npileup, and the residual noise that is present without any collision events, \Nmuzero. The latter contribution is expected to be largely dominated by \emph{electronic noise} and is therefore denoted as electronic noise in the following.

- Both terms are derived separately.

- The pile-up noise term is derived with a dataset without any bias on the selection of the events (zero-bias data), which loosely described corresponds to a dataset with ``pile-up only'' which can be considered to be overlaid with any $pp$ collision event.

In a method referred to as \emph{random cones method}, the energy fluctuations expected due to pile-up within a cone of radius \Rrandomcone are estimated.
%the energy found within two cones of radius \Rrandomcone is summed and compared to each other. This provides a measure of the fluctuations due to pile-up.

The electronic noise is extracted from an MC sample that has no pile-up overlaid.

In order to estimate systematic uncertainties on the noise term meausrement, two more datasets are used: A minimum bias MC sample and a dijet MC sample with pile-up corresponding to the 2017 data taking conditions.

- A summary of the datasets and their usage is provided in \cref{tab:noise-term-samples}.


% \subsection{Dataset and Monte Carlo samples}
% - Ingredients (put this in table):
% - zero-bias data sample with 2017 pile-up (for random cones)
% - zero-bias MC sample with 2017 pile-up -> for randon-cones non-closure uncertainty
% - standard dijet sample with 2017 pile-up (-> for JES scaling)
% - dijet sample without pile-up overlaid  (-> for electronic noise AND JES scaling)

% - Mininum bias sample: ATLAS will use the Minimum Bias Trigger Scintillators (MBTS)
% - described in page 136 of ATLAS experiment. BUT we don't use minbias data! We use zero bias data! and min bias MC!
% From a proceeding: https://cds.cern.ch/record/1639610/files/ATL-PHYS-PROC-2013-346.pdf
%We trigger the minimum bias events with dedicated scintillators called minimum bias trigger
%scintillators (MBTS) and we reconstruct the tracks with the inner detector only. The trigger
%scintillation counters are mounted on LAr endcap cryostats covering the radial dimension of
%the inner detector. 
% - Zero bias sample: A random trigger at Level-1 accepts all types of inelastic events zero bias.
% - Dijet sample
% - from random reddit thread: simulated minbias events would be events where something visible ends up inside the detector and zero bias would be where a bunch crossing happened but there was no requirement for something to be in the detector.

\begin{table}[t]
    \centering
    \begin{tabular}{l l l }
        \toprule
        Dataset / MC Sample & Pile-up condition             & Usage                                      \\
        \midrule
        Zero-bias data      & Recorded in 2017              & random cones                               \\
        Mininum-bias MC     & corresponding to 2017 pile-up & random cones uncertainty                   \\
        Dijet MC            & corresponding to 2017 pile-up & random cones uncertainty, JES scaling      \\
        Dijet MC            & without pile-up               & noise at $\mu=0$, uncertainty, JES scaling \\
        \bottomrule
    \end{tabular}
    \caption{
        Samples considered in the JER noise term measurement.}
    \label{tab:noise-term-samples}
\end{table}


\subsection{The Random Cones Method}
\label{subsec:random-cones-method}

%the energy found within two cones of radius \Rrandomcone is summed and compared to each other. This provides a measure of the fluctuations due to pile-up.
The random cones method can be used to measure the noise expected from pile-up in zero-bias data. The momentum distribution of particle flow objects of a single zero-bias event is shown in \cref{fig:random-cones-balance-a}.
The method is based on the assumption that the pile-up activity is spherically symmetric and independent of the hard scatter.
For each zero-bias event, the energy of the jet constituents (either particle flow objects or topo-clusters) found within two non-overlapping circular cones of radius \Rrandomcone is summed.
The random cones are placed at random $\phi$ and at random \abseta within a restricted range.
The distribution of the difference of the energy within the cones, $\Delta \pTrandomcone = p_{\text{T}}^{\text{Rcone1}} - p_{\text{T}}^\text{Rcone2}$, when sampled over many events, provides a measure of the expected energy fluctuations to which a jet with $R = \Rrandomcone$ is subjected.
% The width of the resulting distribution of the \pTrandomcone difference, $\Delta \pTrandomcone$, when performing this for many events, 
The distribution is shown in \cref{fig:random-cones-balance-b} for random cones with radius $\Rrandomcone = 0.4$ within the central region of the detector (\absetaST{0.7}). The width of the distribution, labelled as $\sigmaRC$, is estimated by taking the \SI{68}{\percent} \emph{confidence interval} (CI), due to the non-Gaussian behavior.
The noise term due to pile-up at the energy scale of the constituents, \Npileupconstscale, is determined by
\begin{equation}
    \Npileupconstscale = \frac{\sigmaRC}{2\sqrt{2}},
\end{equation}
where \sigmaRC is divided by 2 to obtain the half-width and by $\sqrt{2}$ to obtain the fluctuations corresponding to just a single random cone~\cite{JETM-2018-05}.
The constituent scale noise as measured in data and MC is shown in \Cref{fig:const-scale-noise-results} as a function of \abseta and $\mu$.
The difference between data and MC is observed to be significant ($\approx 20\,\%$). This discrepancy is confirmed in other studies of data recorded by the ATLAS experiment. The author contributed various studies to further understand this behavior, which is summarized in \cref{app:constituents-mismodelling}. 
Since the effect is constant with pile-up and independent of the energy of the jet, the jet calibration can greatly minimize the impact of this mismodelling. 

%As \pTrandomcone corresponds to the sum of energies at the scale of the jet constituents, \Npileup corresponds to the noise at the scale of the constituents.

The constituent scale noise needs to be scaled to the JES scale, which requires the same calibration factor to be applied as in the MC-based jet calibration to the particle level (see \cref{sec:jes-calibration}).
Due to technical reasons\footnote{The JES scale factor is not available in bins of \pTtruth which is required for the random cones method}, this scaling is derived independently by using a dijet sample (see \cref{tab:noise-term-samples}). The scale factors applied to \Npileupconstscale are determined by taking the mean of the distribution of the ratio $\pTcalib / \pTPUsub$ in bins of \pTtruth. Truth jets with $\pT > 7\,\GeV$ are matched geometrically ($\Delta R < 0.3$) to the reconstructed jets for this purpose.
The resulting pile-up contribution to the relative JER in the central region of the detector (\absetaST{0.7}) is shown in \cref{fig:pile-up-jer-vs-pt}.
The noise due to pile-up is significantly lower in the case of PFlow jets than EMTopo jets, reflecting the superior resolution of track-based measurement at low \pT compared to calorimeter-based measurements.


\subsubsection{Quantifying the pile-up noise}
- Scaling of const scale to JES
- Different options to do that
- Fits N / pT 
- Pile-up dependent noise term 
- GSC complication

Different considerations need to be made when quantifying the pile-up noise at JES scale, \Npileup. A straightforward approach is to fit $N/\pT$ to the distribution shown in \cref{fig:pile-up-jer-vs-pt}. This, however, assumes that the JES scaling is independent of the \pT of the jet, which is not the case due to the non-linearity of the calorimeter. The JES scale factor is about XX for jets with $\pT = 20\,\GeV$ and XX for jets with $\pT = 40\,\GeV$.
This means that the pile-up noise is amplified more at lower \pT by multiplying the jet \pT with a larger JES factor. If this were to be considered, the noise term needs to be extended by a \pT-dependent term, $\Npileup = \NpileuppT$.

- Studies of MC JER showed that the pT dependence of the noise term is mitigated by the GSC calibration. The GSC uses on global jet properties that cannot be defined on a constituent level as would be needed for the random cones method. 

- Disadvantage of deriving something at constituent scale that is manipulated at the global level.

- The noise due to pile-up at JES scale, \Npileup, is finally quantified by fitting $N/\pT$ to this distribution and extracting $N$.


Since the calorimeter granularity changes across $\eta$ (see \cref{subsec:calorimeter}), \Npileup is determined in several bins of \absetadet. 
The results are shown in TABLE XX. \TDinote{Result tables!}


\subsubsection{Closure test}
A closure test can be performed for the random cones method by comparing the MC JER in a sample with and without pile-up. The quadratic difference of the two resolutions is expected to reflect the contributions to the JER from pile-up. Figure \cref{non-closure} shows a comparison of the MC JER extracted in the different samples, as well the quadratic difference that is to be compared to the noise from the random cones as extracted from MC that is also displayed. The non-closure is quantified by comparing \Npileup as extracted from MC, to the parameter $N$ that is extracted from a fit of the functional form $N/\pT \oplus S/\sqrt{\pT}$ to the quadratic difference. The stochastic term is included because it is non-negligible in the quadratic difference, due to the fact that the energy deposits from pile-up are also subject to stochastic fluctuations.

\FloatBarrier
\begin{figure}[t]
    \subfloat[] {
        \newImageResizeHalf{figures/calibration/zero-bias-event-deposits.pdf}
        \label{fig:random-cones-balance-a}
    }
    \subfloat[] {
        \newImageResizeHalf{figures/calibration/random-cones-difference.pdf}
        \label{fig:random-cones-balance-b}
    }
    \caption{(a) Sum of transverse momenta of neutral and selected charged particle flow objects in an area of $\Delta \eta \times \Delta \phi = 0.2 \times 0.2$ from a zero-bias event from the 2017 dataset. (b) Difference of the transverse momentum within two random cones with radius $R = 0.4$ using neutral and charged particle flow objects for the zero-bias dataset recorded in 2017.
        Previously published in \ccite{SinglePublicPlotZeroBiasDeposits,PublicPlotsJER}.}
    \label{fig:random-cones-balance}
\end{figure}

\begin{figure}[t]
    \subfloat[] {
        \newImageResizeHalf{figures/calibration/const-noise-vs-eta.pdf}
    }
    \subfloat[] {
        \newImageResizeHalf{figures/calibration/const-noise-vs-mu.pdf}
    }
    \caption{Constituent-scale pile-up noise in \antikt $R=0.4$ jets (a) as a function of \absetadet and (b) as a function of $\mu$. Results are derived using the random cones method with either neutral and selected charged particle flow objects or topo-clusters at the electromagnetic scale as inputs. (a)
        Previously published in \ccite{PublicPlotsJER}.}
    \label{fig:const-scale-noise-results}
\end{figure}

\begin{figure}[t]
    \newImageResizeCustom{0.5}{figures/calibration/RandomCones_noise_over_eta_mc16d_MuDependenceStudy.pdf}
    % from /Users/bj/offline_cernbox/AuthorshipQualification/Figures/180901/results/random-cones/RandomCones_noise_over_eta_mc16d_MuDependenceStudy.pdf
\caption{Constituent-scale pile-up noise in \antikt $R=0.4$ jets as a function of \absetadet for a low pile-up ($\mu < 35$) and high pile-up ($\mu > 35$). Results are derived using the random cones method with either neutral and selected charged particle flow objects or topo-clusters at the electromagnetic scale as inputs.}
\label{fig:const-scale-noise-pileup-dependence}
\end{figure}



\begin{figure}[t]
        \newImageResizeCustom{0.6}{figures/calibration/pile-up-jer-vs-pt.pdf}
    \caption{The expected contribution to the jet energy resolution from pile-up extracted from 2017 data as a function of particle-jet \pT for \antikt jets with $R = 0.4$ in the central region of the detector $\absetadet < 0.7$. Previously published in \ccite{PublicPlotsJER}.}
    \label{fig:pile-up-jer-vs-pt}
\end{figure}



\begin{figure}[t]
        \newImageResizeCustom{0.5}{figures/calibration/random-cones-non-closure.pdf}
    \caption{Comparison between the pile-up noise term \Npileup determined using the random cone method (black solid circles) and the expectation from MC simulation (orange squares) as extracted from the difference in quadrature of MC simulation with (red downward triangles) and without (blue upward triangles) pile-up. Results are shown at the PFlow+JES energy scale for jets in the central region of the detector $\absetadet < 0.7$. Previously published in \ccite{JETM-2018-05}.}
    \label{fig:non-closure}
\end{figure}


\subsection{Extracting the electronic noise from Monte Carlo}
The noise derived with the random cones method does not capture contributions to the JER from effects other than pile-up, because of the effective noise reduction in the 420 topo-clustering algorithm. A finite \Nmuzero is still expected in the presence of a hard scatter event.
For example, when considering a cell that receives signals that are just not enough to exceed the necessary energy thresholds for it to be included in topo-clustering algorithm, additional electronic noise could be the deciding factor for the cell to exceed that threshold and eventually be considered in the jet reconstruction.
The amount of noise that this causes is estimated by extracting the parameter $N$ from a fit with the $N, S, C$ parametrisation to the standard MC JER derived in a dedicated MC sample that has no pile-up overlaid.
The results are shown in Table XX.


\subsection{Systematic uncertainties and results}
The uncertainties of the assumptions included in the method need to be propagated to the final results. To this end, a variety of uncertainty sources need to be considered by varying different quantities. 

The dominant uncertainty on \Npileup is the uncertainty of the method itself derived from the closure test described in \cref{subsec:random-cones-method}. The following uncertainties on \Npileup are considered:
\begin{itemize}
    \item \Npileup non-closure:
          The noise extracted from the quadratic difference between the MC JER derived in a sample with and without pile-up is compared to the noise from the random cones method in MC, as shown in \cref{fig:non-closure}. The difference is taken as an uncertainty.
    \item Definition of \sigmaRC:
          The definition of \sigmaRC is varied by using the 87\% CI divided by 1.5 and 50\% CI multiplied 1.5 instead of the 68\% CI.
    \item JES conversion factor: the JES conversion factor is varied by using the mean of a Gaussian fit instead of the mean of $\pTcalib / \pTPUsub$.
\end{itemize}

Systematics on the electronic noise, \Nmuzero, include the following:
\begin{itemize}
    \item MC vs data: JER measurements in 2010 without pile-up~\cite{PERF-2011-04} show that the noise term between data and MC is known to the level of 20\% uncertainty. This uncertainty is applied to \Nmuzero.
    \item Fit extraction uncertainty: the value of \Nmuzero is extracted from a fit to MC JER with the $N, S, C$ parametrisation. This parametrisation is known to be sub-optimal. As well, the three parameters are highly correlated which adds additional uncertainty on the extracted value, $N$. Therefore, an alternative method to extract the noise at $\mu=0$ is performed: the dijet sample with pile-up is split into several bins of pile-up ($\mu$) and the MC JER is derived in each of those bins. The parameter $N$ is again extracted from a fit with the $N, S, C$ parametrisation and the value of $N$ extrapolated to $\mu=0$. The difference between the extrapolated value and the value extracted from the sample without pile-up is taken as a systematic uncertainty.
\end{itemize}

Additional uncertainties are propagated for both components, \Npileup and \Nmuzero, but found to be negligible. These include variations of fit ranges as well as the errors on the parameters from the fits itself. Also, statistical uncertainties are found to be negligible. 

The uncertainties are propagated to \Nfull, by varying them separately and observing the changes in \Nfull. The propagated uncertainty on \Nfull is then symmetrized and used in the JER combination. 
The results, including a breakdown of uncertainties, are shown in \cref{fig:noise-term-results-pflow} for PFlow jets.

% From Paper: The systematic uncertainties enter the combined JER fit unsymmetrized in 𝜂 but are symmetrized during the statistical combination, and so the one-sided components are symmetrized in Figure 28 to illustrate their final contribution to the total uncertainty.

%\begin{table}[t]
    \centering
    \begin{tabular}{p{0.08\textwidth} | p{0.3\textwidth} | p{0.6\textwidth} }
        \toprule
                                  & Uncertainty                & Estimation                                                                     \\
        \midrule
        \multirow{2}{*}{\Npileup} & Non-closure                & Quadratic difference between MC JER derived in sample with and without pile-up \\
                                  & Const scale noise          & corresponding to 2017 pile-up                                                  \\
        \midrule
        \multirow{2}{*}{\Nmuzero} & Variations of fit function & varying fit function by                                                        \\
                                  & Fit error                  & Error from fit                                                                 \\
        \bottomrule
    \end{tabular}
    \caption{
        Uncertainties underlying the noise term measurement.}
    \label{tab:noise-term-uncertainties}
\end{table}


The uncertainties become very large at high $\eta$ which is where the detector granularity becomes significantly worse.

- It is expected that threshold effects will have increased impact on the JER, so that the assumptions in the method become less valid. 
%that the contribution of pile-up to the JER is independent of the presence of a hard scatter becomes less valid. The closure uncertainty covers this effect.

- electronic noise also more important at high pT

- Add EMTopo jets comparison

- R-scan results presented in \cref{app:noise-term-rscan}.



\begin{table}[t]
    \centering
    \subfloat[PFlow+JES jets] {
        \label{tab:prelim-jer-2018-noise-term-inputs-a}
        \begin{tabular}{ l | c  c  c }
            %                                    & \multicolumn{3}{|c}{MC16d, EMPFlowJets}\tabularnewline
            \toprule
                                    & $N_{\text{pile-up}}$ & $N_{\mu = 0}$    & $N_{\text{full}}$\tabularnewline
            \midrule
            $0.0 < |\etadet| < 0.7$ & 2.29 $\pm$ 0.38      & -0.00 $\pm$ 0.15 & 2.29 $\pm$ 0.38\tabularnewline
            \hline
            $0.7 < |\etadet| < 1.3$ & 2.42 $\pm$ 0.51      & 0.00 $\pm$ 0.21  & 2.42 $\pm$ 0.51\tabularnewline
            \hline
            $1.3 < |\etadet| < 1.8$ & 2.34 $\pm$ 1.22      & 0.61 $\pm$ 1.03  & 2.42 $\pm$ 1.21\tabularnewline
            \hline
            $1.8 < |\etadet| < 2.5$ & 2.01 $\pm$ 1.31      & 1.48 $\pm$ 1.51  & 2.49 $\pm$ 1.38\tabularnewline
            \hline
            $2.5 < |\etadet| < 3.2$ & 2.56 $\pm$ 2.99      & 2.02 $\pm$ 1.33  & 3.26 $\pm$ 2.49\tabularnewline
            \hline
            $3.2 < |\etadet| < 3.5$ & 3.64 $\pm$ 2.26      & 3.52 $\pm$ 1.59  & 5.07 $\pm$ 1.97\tabularnewline
            \hline
            $3.5 < |\etadet| < 4.5$ & 3.86 $\pm$ 1.27      & -0.00 $\pm$ 2.37 & 3.86 $\pm$ 1.27\tabularnewline
            \bottomrule
        \end{tabular}
    } \\
    \subfloat[EM+JES jets]{
        \label{tab:prelim-jer-2018-noise-term-inputs-b}
        \begin{tabular}{ l | c  c  c }
            \toprule
                                    & $N_{\text{pile-up}}$ & $N_{\mu = 0}$   & $N_{\text{full}}$\tabularnewline
            \midrule
            $0.0 < |\etadet| < 0.7$ & 3.86 $\pm$ 0.43      & 2.58 $\pm$ 1.07 & 4.64 $\pm$ 0.70\tabularnewline
            \hline
            $0.7 < |\etadet| < 1.3$ & 4.12 $\pm$ 0.63      & 2.98 $\pm$ 0.86 & 5.09 $\pm$ 0.72\tabularnewline
            \hline
            $1.3 < |\etadet| < 1.8$ & 3.76 $\pm$ 0.56      & 3.87 $\pm$ 0.96 & 5.40 $\pm$ 0.79\tabularnewline
            \hline
            $1.8 < |\etadet| < 2.5$ & 2.69 $\pm$ 0.41      & 2.85 $\pm$ 0.83 & 3.91 $\pm$ 0.66\tabularnewline
            \hline
            $2.5 < |\etadet| < 3.2$ & 2.55 $\pm$ 2.61      & 2.33 $\pm$ 0.71 & 3.46 $\pm$ 1.99\tabularnewline
            \hline
            $3.2 < |\etadet| < 3.5$ & 3.48 $\pm$ 1.93      & 4.70 $\pm$ 1.14 & 5.85 $\pm$ 1.47\tabularnewline
            \hline
            $3.5 < |\etadet| < 4.5$ & 3.76 $\pm$ 1.25      & 1.64 $\pm$ 1.67 & 4.10 $\pm$ 1.32\tabularnewline
            \bottomrule
        \end{tabular}}
    \caption{Results of the noise term measurement for (a) PFlow+JES jets and (b) EM+JES jets, including a breakdown of the full noise term into the pile-up noise, \Npileup, and the electronic noise, \Nmuzero. These results were used in the JER combination for \RunTwo performed in 2018. }
    \label{tab:prelim-jer-2018-noise-term-inputs}
\end{table}
\begin{table}[t]
    \centering
    \subfloat[]{
    \label{tab:improved-noise-term-inputs-a}
    \begin{tabular}{ l | c  c  c }
                             & \multicolumn{3}{|c}{MC16d, EMTopoJets}\tabularnewline
        \hline
                             & $N_{\text{pile-up}}$                                  & $N_{\mu = 0}$   & $N_{\text{full}}$\tabularnewline
        \hline
        $0.0 < |\eta| < 0.7$ & 3.86 $\pm$ 0.43                                       & 2.51 $\pm$ 0.52 & 4.61 $\pm$ 0.46\tabularnewline
        \hline
        $0.7 < |\eta| < 1.3$ & 4.16 $\pm$ 0.51                                       & 2.94 $\pm$ 0.62 & 5.09 $\pm$ 0.55\tabularnewline
        \hline
        $1.3 < |\eta| < 1.8$ & 3.83 $\pm$ 0.66                                       & 3.88 $\pm$ 0.84 & 5.46 $\pm$ 0.75\tabularnewline
        \hline
        $1.8 < |\eta| < 2.5$ & 2.69 $\pm$ 0.76                                       & 2.76 $\pm$ 0.57 & 3.85 $\pm$ 0.67\tabularnewline
        \hline
        $2.5 < |\eta| < 3.2$ & 2.52 $\pm$ 2.67                                       & 2.22 $\pm$ 0.45 & 3.36 $\pm$ 2.03\tabularnewline
        \hline
        $3.2 < |\eta| < 3.5$ & 3.41 $\pm$ 1.94                                       & 4.38 $\pm$ 1.04 & 5.55 $\pm$ 1.45\tabularnewline
        \hline
        $3.5 < |\eta| < 4.5$ & 3.68 $\pm$ 1.17                                       & 1.51 $\pm$ 0.44 & 3.98 $\pm$ 1.10
    \end{tabular}} \\
    
    \subfloat[] {
        \label{tab:improved-noise-term-inputs-b}

        \begin{tabular}{ l | c  c  c }
                                 & \multicolumn{3}{|c}{MC16d, EMPFlowJets}\tabularnewline
            \hline
                                 & $N_{\text{pile-up}}$                                   & $N_{\mu = 0}$   & $N_{\text{full}}$\tabularnewline
            \hline
            $0.0 < |\eta| < 0.7$ & 2.29 $\pm$ 0.54                                        & 0.00 $\pm$ 0.06 & 2.29 $\pm$ 0.54\tabularnewline
            \hline
            $0.7 < |\eta| < 1.3$ & 2.46 $\pm$ 0.78                                        & 0.00 $\pm$ 0.09 & 2.46 $\pm$ 0.78\tabularnewline
            \hline
            $1.3 < |\eta| < 1.8$ & 2.36 $\pm$ 1.13                                        & 0.85 $\pm$ 0.58 & 2.51 $\pm$ 1.08\tabularnewline
            \hline
            $1.8 < |\eta| < 2.5$ & 1.94 $\pm$ 1.55                                        & 1.26 $\pm$ 0.57 & 2.32 $\pm$ 1.34\tabularnewline
            \hline
            $2.5 < |\eta| < 3.2$ & 2.47 $\pm$ 2.88                                        & 1.96 $\pm$ 0.49 & 3.15 $\pm$ 2.28\tabularnewline
            \hline
            $3.2 < |\eta| < 3.5$ & 3.57 $\pm$ 2.33                                        & 3.16 $\pm$ 0.97 & 4.77 $\pm$ 1.86\tabularnewline
            \hline
            $3.5 < |\eta| < 4.5$ & 3.81 $\pm$ 1.25                                        & 0.00 $\pm$ 1.94 & 3.81 $\pm$ 1.25
        \end{tabular}
    }    
    \caption{
        Improved noise term inputs with a single \pT independent noise term.}
    \label{tab:improved-noise-term-inputs}
\end{table}
\begin{table}[t]
    \centering
    \subfloat[]{
    \label{tab:pT-dependent-noise-term-inputs-a}
    \begin{tabular}{ l | c  c  c }
        & \multicolumn{3}{|c}{MC16d, EMTopoJets}\tabularnewline
       \hline
        & $N_{\text{pile-up}}$ & $N_{\mu = 0}$ & $N_{\text{full}}$\tabularnewline
       \hline
       $0.0 < |\eta| < 0.7$ & 4.75 $\pm$ 1.13 & 2.51 $\pm$ 0.52 & 5.38 $\pm$ 1.03\tabularnewline
       \hline
       $0.7 < |\eta| < 1.3$ & 5.42 $\pm$ 1.42 & 2.94 $\pm$ 0.62 & 6.17 $\pm$ 1.28\tabularnewline
       \hline
       $1.3 < |\eta| < 1.8$ & 5.17 $\pm$ 1.19 & 3.89 $\pm$ 0.84 & 6.47 $\pm$ 1.07\tabularnewline
       \hline
       $1.8 < |\eta| < 2.5$ & 3.31 $\pm$ 0.49 & 2.76 $\pm$ 0.57 & 4.31 $\pm$ 0.53\tabularnewline
       \hline
       $2.5 < |\eta| < 3.2$ & 3.03 $\pm$ 2.34 & 2.22 $\pm$ 0.45 & 3.76 $\pm$ 1.90\tabularnewline
       \hline
       $3.2 < |\eta| < 3.5$ & 4.10 $\pm$ 1.50 & 4.21 $\pm$ 0.90 & 5.88 $\pm$ 1.23\tabularnewline
       \hline
       $3.5 < |\eta| < 4.5$ & 4.18 $\pm$ 0.92 & 1.15 $\pm$ 1.20 & 4.33 $\pm$ 0.94
       \end{tabular}} \\
    \subfloat[] {
        \label{tab:pT-dependent-noise-term-inputs-b}
        \begin{tabular}{ l | c  c  c }
            & \multicolumn{3}{|c}{MC16d, EMPFlowJets}\tabularnewline
           \hline
            & $N_{\text{pile-up}}$ & $N_{\mu = 0}$ & $N_{\text{full}}$\tabularnewline
           \hline
           $0.0 < |\eta| < 0.7$ & 2.50 $\pm$ 0.40 & 0.00 $\pm$ 0.04 & 2.50 $\pm$ 0.40\tabularnewline
           \hline
           $0.7 < |\eta| < 1.3$ & 2.73 $\pm$ 0.59 & 0.00 $\pm$ 0.09 & 2.73 $\pm$ 0.59\tabularnewline
           \hline
           $1.3 < |\eta| < 1.8$ & 2.66 $\pm$ 0.93 & 0.92 $\pm$ 0.42 & 2.82 $\pm$ 0.89\tabularnewline
           \hline
           $1.8 < |\eta| < 2.5$ & 2.25 $\pm$ 1.31 & 1.26 $\pm$ 0.57 & 2.58 $\pm$ 1.17\tabularnewline
           \hline
           $2.5 < |\eta| < 3.2$ & 3.05 $\pm$ 2.47 & 1.98 $\pm$ 0.44 & 3.63 $\pm$ 2.08\tabularnewline
           \hline
           $3.2 < |\eta| < 3.5$ & 4.46 $\pm$ 1.68 & 2.65 $\pm$ 0.66 & 5.19 $\pm$ 1.48\tabularnewline
           \hline
           $3.5 < |\eta| < 4.5$ & 4.47 $\pm$ 0.92 & 0.00 $\pm$ 1.66 & 4.47 $\pm$ 0.92
           \end{tabular}
    }
    \caption{
        Noise term results extracted with a \pT dependent noise term and evaluated at $\pT = 30\,\GeV$.}
    \label{tab:pT-dependent-noise-term-inputs}
\end{table}

\begin{figure}
    \subfloat[] {
        \newImageResizeCustom{0.6}{figures/calibration/noise-term-results-pflow.pdf}
    }
    \caption{Noise term of the jet energy resolution (JER) and its uncertainties as a function of \abseta. Previously published in \ccite{JETM-2018-05}.}
    \label{fig:noise-term-results-pflow}
\end{figure}



\section{Jet Energy Resolution Combination}

The combination is performed by a fit to the dijet measurement with the functional form described in \cref{eq:jer-parametrisation} and the noise term fixed to the values found in the noise term measurement (see \cref{sec:noise-term-meas}).

The resulting combined JER of PFlow jets is shown in \cref{fig:jer-combination-incl-noise-term-a}. The effect of the noise term is illustrated in the form of a (pink) line and and uncertainty band, that corresponds to $N / \pT$ taken from the noise term measurement. The dijet results are shown as (black) data points.
The JER taken from MC (black dotted line) can be compared to the combined \insitu measurement (blue line). The latter shows significantly smaller uncertainties than the dijet measurement or noise term measurement separately. \TDnote{This is due to the fact that the combined measurement performs a simultaneous fit to all $\eta$ regions which leads to some uncertainties in the fit being constraint which is not reflected in the single measurements.}{Double-check this hypothesis.}

The uncertainties of the noise term measurement are propagated to the final results by varying the value of $N$ for each uncertainty source separately and repeating the fit. The varied $N$ corresponds to the result that is obtained from the noise term measurement when the respective uncertainty is varied by $1\sigma$.
% Mention correlation scheme? Therefore I have to explain the combination a bit more!
An eigenvalue decomposition is performed to reduce the final number of nuisance parameters. This is done similarly for the JES calibration and a more detailed explanation can be found in \ccite{JETM-2018-05}.

- A comparison of the JER between PFlow and EMTopo jets is shown in \cref{fig:jer-combination-results}.

- Noticeable is the improvement of PFlow jets at low \pT.

- Above around 100 pT PFlow reconstruction is almost identical to EM jets, just the pile-up subtraction is performed differently, so that PFlow+JES jets are still different from EMTopo+JES jets.



\FloatBarrier
\begin{figure}
    \subfloat[] {
        \newImageResizeHalf{figures/calibration/combination-incl-noise-term-contribution.pdf}
        \label{fig:jer-combination-incl-noise-term-a}
    }
    \subfloat[] {
        \newImageResizeHalf{figures/calibration/combination-unc-incl-noise-term-contribution.pdf}
        \label{fig:jer-combination-incl-noise-term-b}
    }
    \caption{(a) The relative jet energy resolution as a function of \pT for fully calibrated PFlow+JES jets. The error bars on points indicate the total uncertainties on the derivation of the relative resolution in dijet events, adding in quadrature statistical and systematic components. The expectation from Monte Carlo simulation is compared with the relative resolution as evaluated in data through the combination of the dijet balance and random cone techniques. (b) Absolute uncertainty on the relative jet energy resolution as a function of \pTjet. Uncertainties from the two \insitu measurements and from the data/MC simulation difference are shown separately. Taken from \ccite{JETM-2018-05}.}
    \label{fig:jer-combination-incl-noise-term}
\end{figure}



\begin{figure}
    \subfloat[] {
        \newImageResizeHalf{figures/calibration/jer-combination-vs-pt.pdf}
    }
    \subfloat[] {
        \newImageResizeHalf{figures/calibration/jer-combination-vs-eta.pdf}
    }
    \caption{The relative jet energy resolution for fully calibrated PFlow+JES jets (blue curve) and EM+JES jets (green curve) (a) as a function of \pTjet and (b) as a function of $\eta$. Taken from \ccite{JETM-2018-05}.}
    \label{fig:jer-combination-results}
\end{figure}


\Minote{}{Not sure if I would like to show the uncertainty plot! Can raise a lot of discussions}

\FloatBarrier
\begin{figure}[t]
    \subfloat[] {
        \newImageResizeHalf{figures/calibration/jer-unc-vs-pt.pdf}
    }
    \subfloat[] {
        \newImageResizeHalf{figures/calibration/jer-unc-vs-eta.pdf}
    }
    \caption{Fractional jet energy resolution systematic uncertainty summed across all components for \antikt $R = 0.4$ jets (a) as a function of jet \pTjet at $\eta = 0.2$ and (b) as a function of $\eta$ at $\pTjet = 30\,\GeV$. The total JER uncertainty is shown for both EM+JES and PFlow+JES jets. Taken from \ccite{JETM-2018-05}.}
    \label{fig:jer-combination-uncertainties}
\end{figure}



\section{Application of JER in Physics Analyses}
The measured JER is used to improve comparisons between MC simulations and data by both, correcting the nominal JER of the MC simulations to on average match the one found in data, and propagating systematic uncertainties of the JER measurements.
To this end, a procedure known as \emph{smearing} is applied in which the transverse momenta of the jets are randomly varied according to a Gaussian function with a certain width $\sigma_{\text{smear}}$.

The nominal correction of the JER is only applied in regions of \pT and $\eta$ where the resolution in MC simulation is smaller than in data. In regions where the resolution in MC simulation is larger than in data, no nominal smearing is applied, but the full JER difference is taken as an additional systematic uncertainty.\footnote{Ideally, the JER of the MC simulation should be decreased (``unsmeared'') in scenarios like these. This, however, is technically non-trivial and computing intensive, as it requires the comparison with truth information that is not always accessible in the MC simulation samples. Therefore, a conservative uncertainty is applied instead. This is planned to be revisited for future analyses.}

The systematic uncertainties are propagated by applying a Gaussian smearing with width
\begin{equation}
    \sigma_{\text{smear}} =  \left( \sigma_{\text{nom}} - |\sigma_{\text{variation}}|   \right)^2  - \sigma_{\text{nom}},
\end{equation}
where $\sigma_{\text{nom}}$ is the nominal value of the JER and $\sigma_{\text{variation}}$ corresponds to the JER after one-standard-deviation variation of a certain systematic uncertainty.



% From paper
% Also JER systematic uncertainties are propagated through physics analyses by smearing jets according to a Gaussian function with width 𝜎smear.


\section{Future Improvements}

- split in mu bins!
- Factoring in GSC (hmm...I have to think about that again!)
- Different JER parametrisation for PFlow


- Maybe put that in an appendix.