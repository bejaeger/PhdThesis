
\chapter{Physics Object Reconstruction and Particle Identification}
\label{chap:objects}

- Goal is to infer physics objects from the raw detector signals so that they can be used in physics analyses. 

- physics objects are described with four-vectors.

The combined physics objects in an event then provide information to associate the detector signals to a particular hard scatter process. 

- Carsten: "algorithms under constant development as they greatly influence the efficiency and performance of the reconstruction process".
- Arnold: "This raw data is then processed in several steps by various, sophisticated reconstruction and identification algorithms implemented in the ATHENA framework [170] in order to finally identify the measured signals with physics objects, such as electrons or jets, defined by relatively few parameters."

- Valente: "These four-vectors are generally known as the physics objects of the collision as they represent the quarks, leptons and gauge bosons resulting from the p-p collisions of the LHC."

- four-momentum reconstruction

- Ruthmann: "The event and object reconstruction chain is defined and implemented in the Athena software framework [75]."

- Dickinson talks about final state and the associated particles -> maybe a good idea? Good segway to explain that tau lepton reconstruction is not explained in the thesis

- Reconstruction subject to experimental uncertainties which will be treated as sytematic uncertainties in the analysis presented in later chapters.

- energy calibration are also mentioned.

- What follows is....more detail given on clustering and jet reconstruction

\section{Inner Detector Tracks and Vertex Reconstruction}
The charged-particle hits measured in the ID are used to build tracks that are input to further reconstruction algorithms. 
The track reconstruction uses a sequence of pattern recognition algorithms that are described in much more detail in \ccite{Cornelissen:1020106,Cornelissen_2008}. The sequences primarily consist of an \emph{inside-out} algorithm and an \emph{outside-in} algorithm. They complement each other in the tasks to find tracks associated to the hard-scatter interaction (inside-out algorithm), and tracks stemming from secondary decay vertices in the ID as well as tracks from photon conversions (outside-in algorithm).

The \emph{inside-out} algorithm is seeded by three hits in the silicon detectors. An initial track collection is formed based on a window search and a combinatorial Kalman filter~\cite{fruhwirth_application_1987}. Two more steps follow: first, ambiguities in the initial track collection are resolved; second, the tracks are extended to hits in the TRT without manipulating the original tracks found from the silicon detectors measurements.

The \emph{outside-in} algorithm starts with a global pattern recognition from all TRT hits. The identified segments are then traced back into the silicon detectors, excluding any hits already used in the inside-out stage.

The track reconstruction efficiency, defined as the fraction of charged particles with $\pT > 400\,MeV$ and \absetaST{2.5} that are matched to a reconstructed track, has only a small dependence on the number of pile-up interactions and is well above \SI{70}{\percent} for a range \absetaST{2} and pile-up conditions of $\mu = 41$.~\cite{ATLAS-CONF-2012-042}. \Mnote{}{Not sure if I should mention this here. Should I then also mention the PV reco efficiency? Don't know yet. Probably not? Cause of this written in Sommers thesis: "The efficiencies shown in Fig. 3.1b is obtained in simulated events that also contain collisions with low momentum transfer. The efficiency to reconstruct the vertices of the interactions studied in this thesis is close to 100 percent". Also track reconstruction probs not useful as it does not reflect the high pT muons}

The reconstructed tracks are then used to determine the positions of the collision vertices, known as \emph{primary vertices} (PV). To this end, an \emph{iterative vertex finding} approach \cite{PERF-2015-01} starts with selecting tracks that both satisfy certain quality criteria and are compatible with the interaction region. A vertex seed is chosen based on the selected tracks' density which is input to an \emph{adaptive vertex fitting} algorithm \ccite{0954-3899-34-12-N01}. 
This algorithm iteratively (re-)fits tracks and seeds new vertex candidates if a track is incompatible with the currently assigned vertex until either all tracks are associated to a vertex or no further vertex candidates can be built, which requires at least two associated tracks.
%This algorithm iteratively (re-)fits tracks including the constraint that they originate from that vertex, and seeds a new vertex candidate if they are incompatible. This is repeated until either all track are associated to a vertex or no further vertex candidates can be built, which requires at least two associated tracks.

Once the PVs are found, the \emph{hard-scatter vertex} is chosen to be the one with the largest sum of all squared transverse momenta of tracks that are associated to that vertex. The other PVs are treated as \emph{pile-up vertices}.


% - "reference comes from a reference given in the latest electron reco publication (CERN-EP-2019-145) where a reference is given to the tracking: \ccite{PERF-2015-08}."
% - long reference: \ccite{Cornelissen:1020106}
% - shorter summary in journal: \ccite{Cornelissen_2008}
% - in thesis I used this reference \ccite{ATLAS-CONF-2012-042} that describes the algorithms briefly


\section{Calorimeter Clustering}
\ccite{PERF-2014-07,ATL-LARG-PUB-2008-002}
Each cell of the calorimeter system is read out and can be interpreted as a massless four-vector. In a first reconstruction step, these cells are grouped together to form so-called \emph{topological clusters} or \emph{topo-clusters}. Topo-clusters are used as input to other reconstruction algorithms to form jets, electrons, hadronically decaying $\tau$ leptons, or missing transverse energy. 
The cells are calibrated at the electromagnetic scale (EM scale) so that the energy of electromagnetic interacting particles is correctly taken into account\footnote{The calibration of calorimeter cells is derived in dedicated test-beam measurements described for example in \cref{Abat_2010}}. The calorimeter does not fully account for the energy deposited by hadronic showers, however, which is known as a \emph{non-compensating} nature.
To correct for the loss of energy, dedicated calibrations are performed, which are discussed in more detail in \cref{chap:calibration}.
As the calorimeter cells are subject to significant noise contributions, the main goal of the \emph{topo-clustering algorithm} is to enhance the signal to noise ratio. The total noise can be described as
\begin{equation}
    \sigma_{\text{noise}} = \sqrt{ \left( \electronicnoise  \right)^2  + \left( \pileupnoise  \right)^2 },
\end{equation}
where \electronicnoise is the electronic noise, and \pileupnoise is the noise from pile-up and varies for different data taking conditions. 

\begin{figure}
    \subfloat[electronic noise]
{    \newImageResizeHalf{figures/reconstruction/cells-electronic-noise.png}
}    \subfloat[total noise]
{    \newImageResizeHalf{figures/reconstruction/cells-total-noise.png}
}
\caption{Noise in calorimeter cells from (a) electronic noise at $\mu=0$ and (b) total noise including pile-up noise at instantaneous luminosities of $\mathcal{L} = \SI{e34}{\per\cm\per\s}$ corresponding to peak pile-up conditions of around $\mu\approx20$. Taken from \ccite{ATL-LARG-PUB-2008-002}.}
\end{figure}
% From https://atlas.web.cern.ch/Atlas/GROUPS/PHYSICS/PAPERS/PERF-2014-07/
% Probs get it from here cause other one is 50ns bunch crossing...
% ATL-LARG-PUB-2008-002 -> Should find out what pile-up is expected with design lumi. Design maximum pile-up is 20!


Topo-clusters are formed based on the cell-energy significance, $z_{\text{cell}}$, defined as the level of signal, $E_{\text{cell}}$, above the total noise, 
\begin{equation}
    z_{\text{cell}} = \frac{E_{\text{cell}}}{\sigma_{\text{noise}}}. 
\end{equation}
The algorithm consists of three steps defined by three parameters, $S$, $N$, and $P$:

\begin{enumerate}
    \item A seed calorimeter cell is found with 
    \item All neighboring cells satisfying $blah$ are added
    \item Finally, all neighboring cells with $blah$ are added
\end{enumerate}

The parameters that have been adopted for the ATLAS calorimeters are $S = 4$, $N = 2$, $P = 0$, so that the algorithm is sometimes referred to as ``420 clustering''. 

- the initial set of identified clusters undergoes a step of splitting

- PLOT: add plots of topocluster formation as in Valente's thesis?




\section{Jets}
Jets are the experimental manifestation of quarks and gluons.
They appear as collimated sprays of particles in the detector, originating from the hadronisation of quarks and gluons, and are abundant at the LHC. Almost all physics analyses need to consider jets.

This section goes over the most important aspects of jet reconstruction in ATLAS, starting with a description of the evolution of a jet in the detector. 
A useful characteristic of a jet is its origin, in particular, whether the jet is more likely to stem from the hard-scatter vertex or a pile-up vertex. The strategy to identify so-called \emph{pile-up jets} is discussed further below.
The information about which hadron a jet originates from, specifically whether a $b$-hadron was involved, is another important metric to characterize and classify different experimental signatures. A description of this process known as \emph{flavour tagging} concludes this section.
A description of how the energy of the reconstructed jets is calibrated to match the energy of the initial quark or gluon is left to \cref{chap:calibration}. 

\subsubsection{Jet definition}
\Minote{}{Maybe move this to the main intro to jets before the outline is explained.}
- parton from collision event
- parton showering
    - infrared and collinear divergences
- hadronization
- calorimeter measurements

As a jet is the closest signature of a quark or gluon that can be measured experimentally, special care needs to be taken in its definition to facilitate meaningful comparisons between experiment and theory.
In particular, the jet definition needs to be insensitive to two theoretical divergences:

- Two main considerations need to be thought about:
    a: what kind of measurements are used for defining a jet -> jet constituents
    b: how to group these measurements together? 

- A jet is entirely defined by the choice of constituents and the algorithm to group them together.


\subsubsection{Jet constituents}
- In recent years ATLAS has moved from purely calorimeter-based jets as a standard jet collection to one that uses information from both, the calorimeter and the ID. The latter jets are known as particle flow jets and have two basic ingredients: charged particle flow objects (CPFOs) and neutral particle flow objects (NPFOs). 
- describe the goal of particle flow: to optimally exploit the detector. 

\Minote{}{Think about dedicated subsection for difference between calo only jet and pflow? Or add pflow here?}


\subsubsection{Jet finding}
The jet finding algorithm widely adopted in ATLAS is the so-called \emph{anti-$k_T$} algorithm~\cite{Cacciari:2008gp}. The algorithm is both infrared and collinear safe and yielded the best results in various benchmark physics analyses. \TDnote{}{maybe need to add reference?}
It takes a set of input objects and in a first step computes the values of two distinct distance metrics for all their combinations.
The first type of metric is defined as the distance between two objects $i$ and $j$, 
\begin{equation}
    d_{ij} = \text{min}\left(\frac{1}{p_{T_i}^2},\frac{1}{p_{T_j}^2}\right) \frac{\Delta R_{ij}^2}{R^2},
    \label{eq:dij}
\end{equation}
the second type is given by the distance between object $i$ and the beam, 
    \begin{equation}
        d_{iB} = \frac{1}{p_{T_i}^2}.
\end{equation}
The algorithm finds the minimum of all the computed values and proceeds as follows:
\begin{itemize}
    \item If the smallest distance is of type $d_{ij}$, the objects $i$ and $j$ are combined and the algorithm starts again with computing all distances including the one for the newly combined object.
    \item If the smallest distance is of type $d_{iB}$, the object $i$ is considered as a jet and is removed from the list of objects. The algorithm then proceeds with assessing the next minimum.
\end{itemize}
The procedure is repeated until no objects are left and only jets remain. 
The parameter $R$ in \cref{eq:dij} controls the radius and entirely defines the algorithm. 
In the \HWW analysis presented in \cref{chap:hww}, a radius parameter of $R = 0.4$ is used. The jet calibration measurements presented in \cref{chap:calibration} consider both, jets with $R = 0.4$ and so-called $R$-scan jets with $R = 0.2$ or $R = 0.6$.

The inputs to the \emph{anti-$k_T$} algorithm are typically used also in other object reconstruction algorithms such as in the reconstruction for electrons and photons (see \cref{sec:electron-photon-reconstruction}). 
To avoid double consideration of detector signals in the event reconstruction a dedicated procedure known as \emph{overlap removal} is performed in physics analyses that resolves these ambiguities. The detailed procedure relevant for the work in this thesis is described in the appropriate analysis chapter in \cref{subsec:overlap-removal}.
\Rinote{}{Need to make sure that this reflects the correct understanding of overlap removal}



\subsection{Pile-up jet identification}
Pile-up jets are jets with an origin other than the hard scatter. They are a nuisance to physics analyses as they can spoil the measurements when not being identified correctly. 
\Cref{fig:pile-up-jets-illustration} illustrates the two categories of pile-up jets\footnote{It should be noted, that this distinction as well as \cref{fig:pile-up-jets-illustration} is mainly used for the purpose of a convenient description. In the actual experiment with many pile-up interactions this boundary is blurred, as every jet also has contributions from pile-up effects.}
\emph{Stochastic pile-up}, reconstructed from energy deposits randomly clustered in certain regions, and \emph{QCD pile-up jets}, originating from pile-up vertices.

\TDinote{}{Potential question on this: How many stochastic and qcd pile-up jets are expected? What dominates?}

- To identify these types of jets a multivariate discriminant called \emph{Jet Vertex Tagger} (JVT)~\cite{ATLAS-CONF-2014-018} is used. 
- The JVT uses a combination of two track-based variables to construct a two-dimensional likelihood that provides the relative probability for a given jet to stem from the hard scatter. 

- Physics analyses typically select jets with a minimum JVT threshold for pile-up suppression.
- In the work presented in this thesis, this value is set to JVT $> 0.59$. 
- Calibrations are provided by analyzing data from $Z (\rightarrow \mu\mu)$ candidate events. 
- The JVT achieves a hard scatter efficiency that is stable with increasing number of pile-up interactions. 
- For pile-up conditions of around $\mu = 30$ a hard scatter jet efficiency of \SI{90}{\percent} is achieved for pile-up jet rates of \SI{1}{\percent}.
- More information can be found in \ccite{ATLAS-CONF-2014-018}.

\TDinote{}{Checkout JVT extension to 120 GeV}


\begin{figure}
    \newImageResize{pile-up-jets-illustration.pdf}
    \caption{Illustration of an event containing a hard-scatter jet, a pile-up jet from QCD, and a stochastic pile-up jet. Taken from \ccite{PERF-2016-06}.}
    \label{fig:pile-up-jets-illustration}
\end{figure}

\subsection{Flavour tagging}
Different detector signals give information about the likelihood that a given jet originates from a certain quark flavour. This in turn is crucial input to classify different hard scatter events.
A typical distinction is made between jets originating from $b$-quarks ($b$-jets), $c$-quarks and light-quarks\footnote{Light-jets refer to jets originating from $u$-, $d$-, $s$-quarks, or gluons.}, whereas the identification of $b$-jets is particularly important for the work presented in this thesis.

A neural network that uses a number of low-level taggers as input used to discriminate 
The low-level taggers primarily use the information from tracks reconstructed in the ID and associated to jets. 

% From master thesis:
% Especially the identification of b-jets, also referred to as b-jet tagging or abbreviated b-tagging throughout this thesis, is important for many analyses and achieved with the precise track and vertex information of the ID. The mean lifetime of b-hadrons is sufficiently long - about 1.5 ps - to allow them to fly a measurable distance in the calorimeter - typically several millimeter - before decaying and producing secondary vertices. These secondary vertices can be tagged with dedicated algorithms

- describe DL1r

% From CONF note:
Jets with $\pT > 20 GeV$ and $|\eta| < 2.5$ containing $b$-hadrons ($b$-jets) are identified using a neural network discriminant based on a number of lower level taggers which utilise relevant quantities such as the associated track impact parameters and information from secondary vertices. A working point is adopted that has an average 85\% $b$-jet tagging efficiency, as estimated from simulated ttbar events \ccite{ATL-PHYS-PUB-2017-013,FTAG-2018-01}.

the latter reference is the ttbar calibration


\section{Electron and Photon Reconstruction}
\label{sec:electron-photon-reconstruction}
- topo-clusters are seeds to finding superclusters. 
- superclusters good to recover energy lost in bremsstrahlung photons or from electrons imitted by photon conversion
See here: \ccite{EGAM-2018-01}


\section{Muon Reconstruction}
Muons leave traces in the ID and the MS and deposit little energy in the calorimeters. 
Different strategies exist to reconstruct muons, which differ in which detector subsystems are used.
In the work presented in this thesis, muons are reconstructed from a global fit of matching ID tracks and initially separately reconstructed MS tracks and also taking into account the energy losses in the calorimeters. 

- MS tracks are reconstructed starting from identified straight-line track segments. 
- Describe MS track finding

To select high-quality muon candidates several quality criteria are imposed. 
The requirements are based on the number of hits in the ID and MS, the compatibility between the ID and MS tracks, and track fit properties~\cite{MUON-2018-03}.

- Reconstruction efficiencies with tag-and-probe method also using \Jpsimumu and \Zmumu events~\cite{MUON-2018-03}. (Mention this?)

The muon momentum scale and resolution are calibrated using \Jpsimumu and \Zmumu events by exploiting the known mass of the $Z$ boson~\cite{PERF-2015-10}. 

%( From master thesis)
% To calibrate the momentum scale of muons, dimuon decays of the Z boson or other known mass resonances are exploited. The identification efficiency and the trigger efficiency of the muon trigger system is measured using almost pure Z → μμ or J/Ψ → μμ samples with a tag-and-probe method [70].


% (Rom Master thesis)
% Different muons are categorized in Loose, Medium and Tight with increasing purity, based on the
% 4.4 Lepton Isolation 29
%  quality of reconstruction and identification criteria. Additionally there is a selection especially suited for High-pT muons [70].
% To calibrate the momentum scale of muons, dimuon decays of the Z boson or other resonances, where the mass is known, are considered. The identification efficiency and the trigger efficiency of the muon trigger system is measured using almost pure Z → μμ or J/Ψ → μμ samples with a tag-and-probe method [70].


% From CONF
% For muons, a quality-based identification method [25] is employed, selecting the “Tight” working point with an efficiency of ∼95% so as to maximise the sample purity. The impact parameter requirements are |𝑧0 sin𝜃| < 0.5 mm and |𝑑0|/𝜎𝑑0 < 5 (3) for electrons (muons)2. Leptons are required to be isolated from other activity in the event using maximum thresholds on both energy (using close-by clusters in the calorimeter) and 𝑝T (using close-by tracks). At least one of the offline reconstructed leptons must be matched to an online object that triggered the recording of the event. In the case where the 𝑒–𝜇 trigger is solely responsible for the recording of the event, each lepton must correspond to one of the trigger objects. This trigger matching scheme also requires the 𝑝T of the lepton to be at least 1 GeV above the trigger level threshold.

\section{Lepton Isolation}


\section{Missing Transverse Energy}



\section{Pile-up}
->  Ruthmann has a nice section about it!
From Sommer: "Additional in- elastic, minimum-bias like pp collisions (pile-up) are generated using Pythia8 and overlaid."
Scope:
- I should explain concepts like luminosity blocks  / bunch spacing and stuff in Data Taking Section
- Then I can explain different pile-up conditions here.
- This will be valuable to understand the noise term measurement which exactly tries to measure the noise term!
- Also look back at discussion on skype with Brian about pile-up (actual mu vs average mu and so on)
