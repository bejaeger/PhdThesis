\chapter{Summary of Combined Measurements of Higgs Boson Interactions}
% \chapter{$H\rightarrow W^{\pm}W^{\mp^*}$ Cross-Sections Measurements}
% \chapter{Measurements of $H\rightarrow W^{\pm}W^{\mp^*}$ Cross sections}
\label{chap:comb}

The analysis of \HWW decays presented in \cref{chap:hww} is combined with other measurements of Higgs boson production and decays. This is crucial to probe every aspect of the rich phenomenology of Higgs boson physics (see \cref{chap:higgs}).
The results from the ATLAS collaboration are published in \ccite{NaturePaper} and as the title of the paper states establish a ``detailed map of Higgs boson interactions [\ldots] ten years after the discovery''. 
Together with similar measurements performed by the CMS collaboration~\cite{CMSNaturePaper} they represent the most precise and comprehensive measurements of the properties of the Higgs boson to date.
This chapter summarizes the results of the ATLAS collaboration, highlighting the impact of the \HWW analysis.
All measurements are found to be consistent with the SM expectations, thereby setting strong constraints on couplings to new particles beyond the SM.  

\section{Input Measurements and Fit Procedure}
The combined measurement is performed using the results of the analyses of various Higgs boson decay modes.
This includes the diboson decay modes: \HZZ, $H \to \gamma\gamma$, \HWW, and $H \to Z\gamma$; as well as the fermion decay modes: $H \to b\bar{b}$, $H \to \tau\bar{\tau}$, $H \to \mu\mu$, and $H \to c\bar{c}$. 
Different Higgs boson production modes are considered for each analysis including ggF, VBF, $VH$, $t\bar{t}H$, $tH$, and $bbH$. 
Most input measurements use the full set of \RunTwo data recorded by the ATLAS experiment at the LHC, corresponding to 139\ifb.

The combined measurements are performed by fitting a combined likelihood formed by the product of the likelihood function of each of the input measurements. 
The systematic uncertainties affecting multiple measurements are treated coherently in the combined fit to take into account the correlations between the NPs. 
Several combined fits are performed, testing different scenarios that differ, for example, in the definition of signal strengths in the likelihood.

% \section{Global Signal Strength Measurement}
% %%%%%%%%%%%%%%%%%%%%%%%%%%%%%%%%%%%%%%%

\section{Cross-Section Measurements}
When assuming that all production and decay processes scale with the same signal strength $\mu$, the fully inclusive Higgs boson signal strength is measured to be 
\begin{equation*}
   \mu =1.05 \pm 0.06 = 1.05\pm 0.03\, (\text{stat.})\, \pm 0.03\, (\text{exp.})\, \pm 0.04\, (\text{sig.\ th.})\, \pm 0.02\, (\text{bkg.\ th.}).
\end{equation*}

%%%%%%%%%%%%%%%%%%%%%%%%%%%%%%%%%%%%%%%
% Prod mode cross section
The cross sections of the different Higgs boson production modes times branching fraction are measured for specific combinations of production and decay processes. 
The results of a combined fit are shown in \cref{fig:prod-per-channel} and reveal the varying precision with which different Higgs boson processes have been measured to date. 
The contribution from the \HWW\ analysis is among the most important for the measurement of the ggF and VBF production modes.
%As can be seen, 
\begin{figure}
  \newImageResizeCustom{1}{figures/theory/higgs-measurements/fig4.pdf}
  \caption{Ratio of observed rate to predicted SM event rate for different combinations of
  Higgs boson production and decay processes. The horizontal bar on each point denotes the 68\% confidence interval. The narrow grey bands indicate the theory uncertainties in the SM cross section times the branching
fraction predictions. The $p$-value for compatibility of the measurement and the SM prediction is
72\%. Figure and caption taken from \ccite{NaturePaper}.}
  \label{fig:prod-per-channel}
\end{figure}
%%%%%%%%%%%%%%%%%%%%%%%%%%%%%%%%%%%%%%%%%%
% Decay fractions
% \todo{Maybe putting the decay fraction results is not necessary here!}
% The Higgs decay branching fractions can be measured by fixing the production mode cross section to the respective SM expectation and assuming that there are no non-SM decays. The results can be seen in \cref{fig:br-per-channel}.

%%%%%%%%%%%%%%%%%%%%%%%%%%%%%%%%%%%%%%%%%%
% - Latest STXS combination (from nature)
The kinematic properties of Higgs boson production are measured following the Stage 1.2 STXS scheme. 
The results are shown in \cref{fig:stxs-stage12}. 
The classification of Higgs boson production into 5 classes ($t\bar{t}H$, $tH$, $qq\to Hqq$, $pp\to VH$, and $gg\to H$) is detailed in \ccite{NaturePaper}. 
In particular, the measurements of $qq\to Hqq$ production at large \mjj benefit significantly from the input of the \HWW analysis. 
\TDinote{}{Messungen ergaenzen sich nicely gegenseitig.}
% From nature paper
% Higgs boson production is first classified according to the nature of the initial state and the associated particles, the latter including the decay products of $W$ and $Z$ bosons if they are present. These classes are: $t\bar{t}H$ and $tH$ processes; $qq'\to Hqq'$ processes, with contributions from both VBF and quark-initiated $VH$ (where $V=W, Z$) production with a hadronic decay of the vector boson; $pp\to VH$ production with a leptonic decay of the vector boson ($V(\ell\ell,\ell\nu)H$), including $gg\to ZH \to \ell\ell H$ production; and finally the ggF process combined with $gg\to ZH \to q\bar{q}H$ production to form a single $gg\to H$ process. The contribution of the $b\bar{b}H$ production process is taken into account as a $1\%$~\cite{YR4} increase of the \ggtoH\ yield in each kinematic region, since the acceptances for both processes are similar for all input analyses~\cite{YR4}.

\begin{figure}
  \newImageResizeCustom{1}{figures/theory/higgs-measurements/fig8.pdf}
  \caption{Observed and predicted Higgs boson production cross sections in different
  kinematic regions. The vertical bar on each point denotes the 68\% confidence interval. The $p$-value for compatibility of the combined measurement and the SM prediction is 94\%. Kinematic regions are defined separately for each production process, based on the jet multiplicity, the transverse momentum of the Higgs ($p_{\textrm{T}}^H$) and vector bosons ($p_{\textrm{T}}^W$ and $p_{\textrm{T}}^Z$) and the two-jet invariant mass ($m_{jj}$).
The `VH-enriched' and `VBF-enriched' regions with the respective requirements of $m_{jj}\in[60, 120)$ \GeV\ and $m_{jj}\notin[60,120)$ \GeV\ are enhanced in signal events from $VH$ and VBF productions, respectively. Figure and caption taken from \ccite{NaturePaper}.
  }
  \label{fig:stxs-stage12}
\end{figure}

\section{Higgs Boson Coupling Strenghts}
% The couplings of the Higgs boson to individual particles can be measured by parametrizing the cross sections and branching fractions for the individual Higgs boson processes in terms of coupling-strength modifiers, $\kappa$, following the $\kappa$ framework~\cite{LHCHandbookV3}. 
% For a production (decay) via the coupling to a given particle $p$, the modifier $\kappa_p$ is defined as $\kappa_p^2 = \sigma_p / \sigma_p^\text{SM}$ ($\kappa_p^2 = \Gamma_p/\Gamma_p^\text{SM}$), where $\Gamma_p$ is the partial decay width into a pair of particles $p$.\cite{NaturePaper}
%Different statistical models with varying assumptions that are detailed in \cite{NaturePaper} may be considered. 
The couplings of the Higgs boson to individual particles are measured for different scenarios following the $\kappa$ framework introduced in \cref{subsec:coupling-measurements}.
Assuming that there are only SM processes that interact exactly as predicted, and independently measuring coupling-strength modifiers for all included particles ($\kappa_W$, $\kappa_Z$, $\kappa_t$, $\kappa_b$, $\kappa_c$, $\kappa_\tau$, and $\kappa_\mu$), the results can be visualized as shown in \cref{fig:h-couplings}. 
This finds good consistency of the internal structure of the SM, but at the same time reveals that the uncertainties on the measurements are still sizable. 
The $\kappa_W$ modifier is measured with relative uncertainties of about 5-10\%, depending on the model assumed. 
This measurement is largely driven by the VBF, \HWW analysis, because it is sensitive to $\kappa_W$ in both the production and decay mode.
The VBF production mode is parametrized as $\kappa_\mathrm{VBF}^2 = 0.733 \kappa^2_W + 0.267 \kappa^2_Z$ in the combined measurement.

If the measurement is performed using only two independent coupling-strength modifiers, one for the vector bosons, $\kappa_V = \kappa_W = \kappa_Z$, and one for all fermions, $\kappa_F$, the results are $\kappa_V = 1.035 \pm 0.031$ and $\kappa_F = 0.95 \pm 0.05$. 
%and Higgs boson to fermion coupling, $\kappa_{F}$, are measured with a relative uncertainty of the order of 10\% and 20\%, respectively. 
% The VBF, \HWW analysis provides the most sensitive input to measure both $\kappa_W$ and $\kappa_V$, since the $HW$ vertex appears twice in the diagram. The VBF production mode enters the parametrization in the $\kappa$ framework (\cref{eq:kappa-parametrization}) as $\kappa_\mathrm{VBF}^2 = 0.733 \kappa^2_W + 0.267 \kappa^2_Z$~\cite{NaturePaper}. 
\begin{figure}
  \newImageResizeCustom{0.7}{figures/theory/higgs-measurements/fig6_paper.pdf}
  \caption{
    Reduced Higgs boson coupling strength modifiers and their uncertainties. They are defined as $\kappa_F \cdot m_F / \text{vev}$ for fermions
($F=t,b,\tau,\mu$) and $\sqrt{\kappa_V}\cdot m_V/\text{vev}$ for vector bosons as a
function of their masses $m_F$ and $m_V$. Two fit scenarios with $\kappa_c =
\kappa_t$ (colored circle markers), or $\kappa_c$ left free-floating in the fit (grey
cross markers) are shown. Loop-induced processes are assumed to have the SM structure, and Higgs boson decays to non-SM particles are not allowed. The vertical bar on each point denotes the 68\% confidence interval. The $p$-value for compatibility of the combined measurement and the SM prediction are 56\% and 65\% for the respective scenarios. The lower panel shows the values of the coupling strength modifiers. The grey arrow points in the direction of the best-fit value and the corresponding grey uncertainty bar extends beyond the lower panel range. Figure and caption taken from \ccite{NaturePaper}.}
  \label{fig:h-couplings}
\end{figure}
