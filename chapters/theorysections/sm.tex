

% From Master's thesis:
The SM of particle physics is a collection of relativistic quantum field theories (QFT) and describes the interactions between all known fundamental particles.
%It evolved during the 60’s and 70’s due to a strong interplay between experimental observations and theoretical developments. 

- Introduction to QFT and gauge group

- Lagrangian formalism
- local gauge invariance: without local gauge invariance:
% From Pich The Standard Model
% Thus, once a given phase convention has been adopted at one reference point x0, the same convention must be taken at all space-time points. This looks very unnatural.
- highly successful in describing QED

\subsection{Particles and Forces}
\label{sec:particle-content}
%In QFT all fundamental fields have an associated particle, that can be thought of as excitations (or vibrations) of the fields.
On a \TDnote{phenomenological}{other word?} level, fundamental physics can be described in terms of elementary particles and forces.

The SM includes three of the four known fundamental interactions: The \emph{electromagnetic interaction} -- described by the theory of \emph{quantum electrodynamics} (QED) --, the \emph{strong interaction} -- using \emph{quantum chromodynamics} (QCD) [15] -- and the \emph{weak interaction}. The electromagnetic and the weak interaction are unified in the \emph{electroweak model} [16–18]. The fourth known fundamental interaction, \emph{gravitation}, is not part of the SM but can be fully neglected in present particle physics experiments due to its weak strength.

% All currently known particles and forces included in the SM are listed in \TDnote{REF}{REF}.
The elementary particles included in the SM are listed in \TDnote{REF}{REF}. They can be grouped into two main types: \emph{fermions}, that are sometimes referred to as \emph{matter particles} carrying half-integer spin, and \emph{bosons}, that are the \emph{mediators} (or \emph{force carriers}) of the fundamental forces carrying integer spin.

The fermions can be further divided into a set of six leptons and six quarks, each of which can be classified into three generations.
Their interactions with each other can be described by exchanges of so-called \emph{gauge bosons}.
Every fermion has a corresponding antiparticle that has the same mass and spin but opposite internal quantum numbers and opposite \emph{chirality}.
Quarks have an additional property called \emph{colour}, which can take three possible values or any linear combinations of these values.

The boson that mediates the electromagnetic interaction is the massless \emph{photon} that interacts with all charged fermions.
The strong interaction is transmitted via eight massless gluons and acts only on quarks and gluons themselves.
Due to the property of \emph{confinement} in QCD, quarks cannot be found in isolation. They are confined within hadrons that consist, for example, of a quark and an anti-quark (known as \emph{mesons}) or three quarks (known as \emph{baryons}).

The massive gauge bosons, \Wplus, \Wminus, and \Zboson, are the mediators of the weak interaction which acts on all fermions. As well, the \Wmp and \Zboson undergo self-interactions.

The final particle of the SM is the electrically neutral \emph{Higgs boson}. It is the only spin-0 scalar particle and plays a special role in the SM in the mechanism of \emph{electroweak symmetry breaking}, which gives rise to the masses of the \Wpm and \Zboson bosons. More details about Higgs boson physics are provided in \TDnote{REF}{REF}.


\subsection{Formalism and Principles}
% - Action: S = Int ( Lagrangian ) dt -> EOMs
% - Lagrangian = Int (Lagrange density ) d3x
% -> Lagrange density is commonly referred to as Lagrangian 
% - In QFT, we define EOMs for fields by specifying the Lagrange density (Lagrangian)

% Particles can be thought of as excitations (or vibrations) of corresponding fundamental fields. 
QFT describes nature in terms of fundamental fields. Each field has an associated particle that can be thought of as excitations (or vibrations) of the fields.\footnote{To simplify terminology, the objects are mostly referred to as particles rather than fields in this chapter, but the relationship is important to be kept in mind.}

The equations of motions of a given physical system can be derived from the action,
\begin{equation}
  \label{eq:action}
  S = \int \Lagrangian \quad d^3x dt,
\end{equation}
by following the action principle.\footnote{The derivation can be found in any standard text QFT such as REF (Peskin schroeder?).}.
In QFT, the dynamics of a system are therefore expressed in terms of the Lagrange densities (often just denoted \emph{Lagrangian}), \Lagrangian. 

The Lagrangian is a function of the fields, $\psi$, and their derivatives, $\partial \psi$, that is $\Lagrangian \rightarrow \Lagrangian\left(\psi, \partial\psi)\right)$. The fields are themselves dependent on the space-time coordinate, $\psi \rightarrow \psi(x)$. 
For a given field, $\psi$, the principles of QFT demand to include all possible terms in the Lagrangian related to the field. 
The number and type of terms that can be included, however, is strongly constraint by requiring the theory to be invariant under certain symmetry transformations of the fields. 
The SM is governed by the principle of \emph{local gauge invariance}. 

It implies, that the physical content of the theory should stay the same when performing certain re-definitions of the particle fields independently at every space-time point.
The structure of the Lagrangian is therefore determined by requiring invariance under field transformations of the sort,
\begin{equation}
  \psi \rightarrow e^{-i\omega(x)} \psi,
\end{equation}
where $\omega(x)$ is the generator of a certain symmetry group and dependent on the space-time coordinate. 
In order to maintain the symmetry, the Lagrangian may need to be manipulated by introducing new fields.
These fields are known as gauge fields (or gauge bosons) and give rise to particle interactions that give rise to the fundamental forces of nature. 
Historically, the first local gauge theory that was established was QED. It follows a U(1) local gauge symmetry and entails the photon as the associated gauge field. 
The SM is a reflection of a SU(3)$_C$ $\times$ SU(2)$_L$ $\times$ U(1)$_Y$ local gauge symmetry that implies the existence of the gauge bosons described in the previous section.

The following sections break down the full Lagrangian of the SM in different parts. 

Throughout this thesis, natural units are used, that is $c = \hbar = 1$. This means that the mass, momentum and energy are all given in units of electronvolt (\eV).
- Terms with equal indices are considered to be summed (Einstein-summation convention) and greek letters go from 1-4 and latin letters from 1-3 if not mentioned otherwise.

- renormalization:
We consider that the physics is understood up to a given cut-off/scale Lambda.

%The SM is governed by the principle of local gauge invariance and the associated symmetry group is SU(3)$_C$ $\times$ SU(2)$_L$ $\times$ U(1)$_Y$. 

% the next important lesson is that the existence of these symmetries places suck an incredibly strong constraint on what the theory actually is.

% Sean
% QED: demanding all the terms in your Lagrangian being gauge invariant is enforcing the conservation of electric charge gauge
% This is a reflection of Noethers theory. Symmetry conservation -> associated quantity of the U(1) symmetry is charge


% Sean Carrol:
% W and Z bosons are the physical excitations from vibrations in the SU(2) to gauge field
%the existence of these fields giving rise to interactions giving rise to forces of nature comes from the gauge symmetry.
% the next important lesson is that the existence of these symmetries places suck an incredibly strong constraint on what the theory actually is.

% From Master thesis
% An essential part of the mathematical formulation of the SM is based on the postulation of local gauge invariance. It implies, that the physical content of the theory should stay the same when performing certain redefinitions of the particle fields independently at every space-time point. The SM follows a SU(3)×SU(2)L×U(1)Y symmetry group, where SU(3) is the gauge group of QCD and SU(2)L×U(1)Y the corresponding symmetry for the electroweak model (the meaning of the subindizes L and Y will be explained in the following sections). Historically, QED was the first well established gauge theory, following a U(1) symmetry. To demonstrate the principles of a gauge theory, which are the key concepts of the mathematical framework of the SM, the QED Lagrangian will be derived in the following. Subsequently the same principles are applied to describe the main aspects of the more complex theories of QCD and the electroweak model. A description of the mechanism to incorporate masses for the W± and Z bosons via breaking the electroweak symmetry is given thereafter. The following sections follow to a large extend the more detailed descriptions given in Refs. [19–21].


\subsection{Quantum electrodynamics}
%The above derivations are shortly recapped: Starting from the Lagrangian of a freely moving relativistic fermion one can impose the requirement of local U(1) gauge invariance. This demands to add a new field $A_\mu$ toghether with an interaction term, which is incorporated in the Lagrangian by introducing a covariant derivative $D_\mu$. Additionally, the kinetic term in \cref{eq:kinetictermqed} needs to be added for the new field, which leads to the final Lagrangian for QED, shown in \cref{eq:Lagrangianqed}.
QED is the reflection of an underlying $U(1)$ local gauge symmetry of the complex-valued fermion fields, $\psi_f$, known as \emph{Dirac spinors}.
The Lagrangian that is invariant under $\psi_f \rightarrow e^{i \omega(x)} \psi_f$ transformations can be written as
\begin{equation}
  \mathcal{L}_{\text{QED}} = \sum_f \bar{\psi_f}(i\gamma^\mu D_\mu - m_f)\psi_f - \frac{1}{4}F_{\mu\nu}F^{\mu\nu},
  \label{eq:Lagrangianqed}
\end{equation}
where the sum goes over all electrically charged fermions with mass $m_f$.
Here, $D_\mu$ is the \emph{covariant derivative}, defined as
\begin{equation}
  D_\mu = \partial_\mu + ieA_\mu,
\end{equation}
which includes the gauge field $A_\mu$, associated with the photon. The requirement of local gauge symmetry prohibits terms quadratic in $A_\mu$, which reflects the fact that the photon is massless.\footnote{Terms in the Lagrangian that are quadratic in a certain field give rise to particle masses.}
The so-called \emph{field tensors} are defined as
\begin{equation}
  F_{\mu\nu} = \partial_\mu A_\nu - \partial_\nu A_\mu,
\end{equation}
where $\gamma^\mu$ refers to the four $4x4$ gamma matrices, and $\partial_\mu$ are the partial derivatives.

According to Noether's theorem\TDnote{REF}{REF} each symmetry of a system results in a conserved quantity.
The symmetry group associated to QED is denoted U(1)$_{\text{QED}}$ and the conserved quantity is the electric charge.
It should be noted that U(1)$_{\text{QED}}$ is not part of the original symmetry group of the SM. As explained below, U(1)$_{\text{QED}}$ arises from a SU(2)$_L$ $\times$ U(1)$_Y$ symmetry that is spontaneously broken.


\subsection{Quantum chromodynamics}
QCD is a non-abelian gauge theory associated to the SU(3)$_C$ symmetry. The subindex $C$ refers to the colour and is the conserved quantity under SU(3)$_C$ transformations.
The local gauge invariant QCD Lagrangian reads
\begin{equation}
  %  \mathcal{L}_{\text{QCD}} = -\frac{1}{4}G_{\mu\nu}^aG^{a\,\mu\nu} + \bar{q}^i\left( i\gamma^\mu D_\mu-m \right)^j_i q_j,
  \mathcal{L}_{\text{QCD}} = \sum_f \bar{\psi_f}(i\gamma^\mu D_\mu - m_f)\psi_f - \frac{1}{4}G_{\mu\nu}^aG^{a\,\mu\nu},  \label{eq:lqcd}
\end{equation}
where the sum runs over all quark fields, $\psi_f$, where the fields are represented as spinor triplet. 
The covariant derivate is given by
\begin{equation}
    D_\mu = \partial_\mu + i g_s \frac{\lambda^a}{2} G_\mu^a,
\end{equation}
which includes the eight gluon fields $G_\mu^a$ (with $a = 1, \ldots, 8$), as well as the strong coupling constant, $g_s$, and the Gell-Mann matrices, $\lambda^a$, which are the generators of the SU(3)$_C$ group.
The field tensors, $G_{\mu\nu}^a$, are defined as
\begin{equation}
  \label{eq:qcd-tensor}
  G_{\mu\nu}^a = \partial_\mu G_\nu^a - \partial_\nu G_\mu^a - g_s f^{abc}G_\mu^b G_\nu^c,
\end{equation}
where $f^{abc}$ are the structure constants of SU(3)$_C$. 
% Invariant under ... transformation, where ... are the group generators

The gluons are massless, as no term quadratic in the gluon fields are allowed when requiring local gauge invariance. Other than photons, however, gluons can interact with each other, a feature that arises from the non-commuting elements of the SU(3)$_C$ group. Triple and quartic self-interactions between gluons are therefore part of QCD because gluons themselves carry a combination of colour and anti-colour. 
%which is similar to the requirement of a massless photon in QED. 



\subsection{The electroweak model}
% - the chirality can be determined with... right-chiral fermions are singlets under SU(2)L transformations
% - This also means that no right-chiral neutrinos exist in the SM as they don't interact with any of the forces

% "Also called: The Glashow􏰁Weinb erg􏰁Salam Theory of Weak Interactions"
The weak and electromagnetic force are unified in the electroweak model developed by Glashow, Salam, and Weinberg \TDnote{REF}{REF}.
The unification is based on imposing local gauge invariance under transformations of the symmetry group
\begin{equation}
  \text{SU(2)}_L \times \text{U(1)}_Y.
\end{equation}
The subindex $L$ reflects the empirical finding that only \emph{left-chiral}\footnote{
  The \emph{chirality} of a fermion can be determined with the projection operators $P_L$ and $P_R$ like
  \begin{equation*}
    \psi       = P_L \psi + P_R \psi = \frac{1}{2} \left( 1 - \gamma^5 \right) \psi + \frac{1}{2} \left( 1 + \gamma^5 \right) \psi = \psi_R + \psi_L,                                \\
  \end{equation*}
  where $\gamma^5 = i\gamma^0\gamma^1\gamma^2\gamma^3$. The chirality becomes identical to the helicity for massless particles.
} (also denoted \emph{left-handed}) fermions interact via the weak interaction.
The fermions are therefore grouped into left-handed doublets and right-handed singlets under SU(2)$_L$ transformations, denoted $\psi_L$ and $\psi_R$, respectively. They transform as
\begin{align}
  \psi_L & \rightarrow e^{iY\omega} e^{iT^a\omega^a} \psi_L \\
  \psi_R & \rightarrow e^{iY\omega} \psi_R,
\end{align}
where $T^a = \frac{\sigma^a}{2}$ (with $a = 1, 2, 3$) are the generators of the SU(2)$_L$ group that are the Pauli matrices, $\frac{\sigma^a}{2}$, for SU(2)$_L$ doublets and zero for SU(2)$_L$ singlets. 
The associated conserved quantity is the \emph{weak isospin}, $T$, that is $T^3 = \pm \frac{1}{2}$ for SU(2)$_L$ doublets and $T^3 = 0$ for SU(2)$_L$ singlets.
The U(1)$_Y$ symmetry is associated to the \emph{hypercharge}, $Y$, and can not be directly associated to the QED gauge group.
The relation to the electromagnetic interaction and the physical electric charge $Q$ is
\begin{equation}
  Q = T^3 + \frac{Y}{2}.
\end{equation}
The local gauge invariant electroweak Lagrangian can be written as
\begin{equation}
  \mathcal{L}_{\text{EWK}} = \sum_f i\bar{\psi}_{f, L}\gamma^\mu D_\mu \psi_{f, L} + \sum_f i\bar{\psi}_{f, R}\gamma^\mu D_\mu \psi_{f, R} - \frac{1}{4}W_{\mu\nu}^aW^{a\,\mu\nu} - \frac{1}{4} B_{\mu\nu}B^{\mu\nu}, 
  \label{eq:lagrangianewk}
\end{equation}
where the sum runs over all fermions, $f$, whose occurrence as left-handed and right-handed particles is explicitly mentioned.
The covariant derivative is defined as
\begin{equation}
  D_\mu = \partial_\mu + igW_\mu^aT^a + ig'B_\mu \frac{Y}{2}
  \label{eq:covdevewk}
\end{equation}
and includes four gauge fields. The fields $W^a_\mu = W^1_\mu, W^1_\mu, W^3_\mu$ are the gauge fields of SU(2)$_L$ with associated coupling $g$ and $B_\mu$ is the gauge field of U(1)$_Y$ with coupling $g'$.
The field tensors are given by
\begin{align}
  W_{\mu\nu}^a & = \partial_\mu W_\nu^a - \partial_\nu W_\mu^a - g \epsilon^{abc} W^b_\mu W^c_\nu, \label{eq:Wtensor} \\
  B_{\mu\nu}   & = \partial_\mu B_\nu - \partial_\nu B_\mu,
\end{align}
where $\epsilon^{abc}$ are the structure constants of SU(2)$_L$. The third term on the right-hand side of \cref{eq:Wtensor} arises because of the non-abelian nature of SU(2)$_L$ and gives rise to triple and quartic self-interactions between the gauge fields $W_{\mu}^a$.
The tensor $B_{\mu\nu}$ has the same structure as the electromagnetic field strength tensor obtained in QED.

The Lagrangian in \cref{eq:lagrangianewk} describes 4 massless bosons. No terms quadratic in the vector fields are allowed due to the requirement of local gauge invariance. Moreover, also the simple inclusion of fermion mass terms is forbidden, given that a term
\begin{equation}
  -m_f \bar{\psi} \psi = -m_f \left( \bar{\psi_R}\psi_L + \bar{\psi_L}\psi_R \right)
  \label{eq:fermionmassterm}
\end{equation}
violates SU(2)$_L$ gauge invariance, since $\psi_L$ transforms as a doublet and $\psi_R$ as a singlet.

The physical fields, \Wpm, \Zboson, $\gamma$ naturally appear in the Higgs mechanism explained below.

% From experiments one expects two charged bosons $W^\pm$ and two neutral bosons, the $Z$ and the photon.


An overview of the fermion content of the electroweak model is shown in \cref{tab:ewfermioncontent}. Quarks undergoing the electroweak interaction are present as a mixture of the quark mass eigenstates. This phenomenon is described by the \emph{Cabibbo–Kobayashi–Maskawa (CKM) mechanism} \cite{doi:10.1143/PTP.49.652}.









\subsection{Spontaneous symmetry breaking and the Higgs mechanism}
\label{sec:ewsymbreaking}
%- Local gauge invariance does not allow adding mass terms
%- The SU(2)$_L$ $\times$ U(1)$_Y$ symmetry is spontaneously broken into U(1)$_\text{QED}$
%- This mechanism is spontaneous symmetry breaking
%- Simply put, the Lagrangian itself maintains the symmetry, but the state of lowest energy is not invariant and breaks the summetry.

% - Could add mass terms disregarding local gauge invariance but this would render theory unrenormalizable





\begin{figure}
  \newImageScale{2.5}{figures/theory/higgspotential.jpg}
  \caption[Two dimensional Higgs potential $V(\Phi)$ with $\lambda > 0$ and $\mu^2 < 0$.]{Two dimensional Higgs potential $V(\Phi)$ as in \cref{eq:higgspotential} with $\lambda > 0$ and $\mu^2 < 0$. The minimum occurs for different points on the sketched circle in the $(\phi_1, \phi_2)$ plane and has the value given in \cref{eq:higgsminima}. Taken from Ref.~\cite{Halzen:1984mc}.
  }
  \label{fig:higgspotential}
\end{figure}

An additional mechanism is needed in the SM to explain the finite masses of the weak gauge bosons. 
A principle known as \emph{spontaneous symmetry breaking}, that has been first explored in the field of condensed-matter physics, \todo{REF, Anderson?} can therefore be applied to the SM to generate the necessary mass terms without violating local gauge invariance.
% From Peskin and Schroeder
% - In the theory of sup erconductiv􏰁 ity􏰔 for example􏰔 the Ab elian gauge invariance of electromagnetism is broken by pairs of electrons that condense in the ground state of a metal􏰎
This was observed by three independent research teams in 1960\todo{DATE}: Brout and Englert \cite{PhysRevLett.13.321}, Peter Higgs \cite{PhysRevLett.13.508,HIGGS1964132} and Guralnik, Hagan and Kibble \cite{PhysRevLett.13.585}.
The mechanism is known as the \emph{Brout-Englert-Higgs mechanism} (or simply \emph{Higgs mechanism}) or also \emph{electroweak symmetry breaking} (ESWB).
%The underlying concept is that the Lagrangian itself maintains the symmetry, but the state of lowest energy is not invariant and breaks the gauge symmetry.
The basic principle is to allow the state of lowest energy to violate gauge invariance while maintaining the gauge symmetry of the Lagrangian itself.

% was published almost simultaneously by three independent groups in 1964: by Robert Brout and François Englert;[3] by Peter Higgs;[4] and by Gerald Guralnik, C. R. Hagen, and Tom Kibble.[5][6][7]

The Higgs mechanism assumes a complex scalar field of the SU(2)$_L$ group (known as \emph{Higgs field})
\begin{equation}
  \Phi = \myvec{ \Phi_\alpha \\ \Phi_\beta } = \frac{1}{\sqrt{2}} \myvec { \Phi_1 + i \Phi_3 \\ \Phi_2 + i \Phi_4}
\end{equation}
and introduces a potential of the form
\begin{equation}
  V(\Phi) = \mu^2\Phi^\dagger\Phi + \lambda \left| \Phi^\dagger\Phi \right|^2, 
  \label{eq:higgspotential}
\end{equation}
where the parameters are specifically chosen to satisfy $\mu^2 < 0$ and $\lambda > 0$.
This is the so-called \emph{Higgs potential} and is illustrated in \cref{fig:higgspotential}.
The Lagrangian 
\begin{equation}
  \mathcal{L}_{\text{Higgs}} = |D_\mu\Phi|^2 - V(\Phi), % - \frac{1}{4} W_{\mu\nu}^a W^{a\, \mu\nu} - \frac{1}{4} B_{\mu\nu} B^{\mu\nu},
  \label{eq:lagrangianhiggs}
\end{equation}
with $D_\mu$ defined as shown in \cref{eq:covdevewk}, is invariant under local SU(2)$_L$ $\times$ U(1)$_Y$ transformations.
%% We find that the Lagrangian for such a field
%% \begin{equation}
%% \end{equation}
%% is invariant under global SU(2) phase transformations
%It is invariant under gauge transformations but not the ground state, which can be found at

The constraints on $\mu$ and $\lambda$ in the Higgs potential give rise to a non-zero \emph{vacuum expectation value}, $v$, of the Higgs field.
The minimum of the potential can be found on the circle of minima where
\begin{equation}
  |\Phi| = \sqrt{ \frac{\mu^2}{2\lambda} } \equiv \frac{ v }{\sqrt{2}}.
  \label{eq:higgsminima}
\end{equation}
The symmetry is spontaneously broken when a ground state is chosen. 
%The arbitrary choice of a ground state is said to spontaneously break the symmetry. 

Without loss of generality the ground state can be chosen to be
\begin{equation}
  \Phi_0 = \frac{1}{\sqrt{2}} \myvec{0 \\ v},
  \label{eq:groundstate}
\end{equation}
that is $\Phi_1 = \Phi_2 = \Phi_4 = 0$ and $\Phi_3 = \frac{v}{\sqrt{2}}$, 
and expand the field $\Phi$ around it
\begin{equation}
  \Phi(x) = e^{i\frac{\phi(x)}{v}}\frac{1}{\sqrt{2}} \myvec{ 0 \\ v + H(x) }.
    \label{eq:higgsexp}
\end{equation}
The extra term of $e^{i\frac{\phi(x)}{v}}$ describes the fluctuations of the fields $\Phi_1, \Phi_2, \Phi_4$ from the vacuum $\Phi_0$.
Plugging \cref{eq:higgsexp} in the locally SU(2) invariant Lagrangian in \cref{eq:lagrangianhiggs} one can choose a specific gauge (the unitary gauge), where the three fields $\Phi_{1/2/4}$, that is $\phi(x)$, are not contained anymore. These fields are known as the \emph{Goldstone bosons} and have no physical implication. As shown in the following, they will reappear in the longitudinal polarization of the gauge bosons, which makes the gauge bosons become massive.
With these considerations the expanded field $\Phi$ can be reduced to
\begin{equation}
  \Phi(x) = \frac{1}{\sqrt{2}} \myvec{ 0 \\ v + H }
\end{equation}
%The generation of masses can be seen when
and inserted into the Lagrangian in \cref{eq:lagrangianhiggs}, which leads to the following kinetic term
\begin{align}
  |D_\mu\Phi|^2 & \supset \frac{1}{8} \left| \left( gW_\mu^a \sigma^a+g'B_\mu Y \right) \myvec{ 0 \\ v + H} \right|^2  \label{eq:lagrangianmasspre} \\
  % &= \frac{1}{8} \left| \left( g \myvec{ W_\mu^3 & W_\mu^1 - i W_\mu^2 \\ W_\mu^1 + i W_\mu^2 & W_\mu^3  } + g' B_\mu Y \right)  \myvec{ 0 \\ v + H } \right|^2 \\
  &= \frac{ \left( v + H \right) ^2}{8} \left[ g^2 \left( W_\mu^1 - i W_\mu^2 \right) \left( W_\mu^1 + i W_\mu^2 \right) + \left| g W_\mu^3 + g' B_\mu Y \right|^2 \right].
  \label{eq:lagrangianmass}
\end{align}
The gauge fields mix and one can redefine them to obtain the physical fields with the following linear combinations:
\begin{align}
  W_\mu^\pm &= \frac{1}{\sqrt{2}} \left( W_\mu^1 \mp iW_\mu^2 \right) \\
  Z_\mu &= \cos\theta_\text{w} W_\mu^3 - \sin\theta_\text{w} B_\mu \\
  A_\mu &= \cos\theta_\text{w} B_\mu - \sin\theta_\text{w} W_\mu^3.
\end{align}
Here $\theta_\text{w}$ is the so called \emph{weak mixing angle} with the definition $\sin\theta_\text{w}^2 = \frac{g'^2}{g^2+g'^2}$.
The following expression arises in the Lagrangian %quadratic in the gauge fields arise:
\begin{equation}
  \mathcal{L}_M = \frac{1}{4} \left( v + H \right)^2  \left(g^2 W_\mu^+W^{-\,\mu} + \frac{g^2}{2\cos\theta_\text{w}^2} Z_\mu Z^\mu \right),
  \label{eq:lagrangianmasses}
\end{equation}
where the vacuum expectation value generates terms quadratic in the fields $W_\mu^\pm$ and $Z_\mu$. Identifying these terms with $M_W^2 W_\mu^+W^{-\,\mu}$ and $\frac{1}{2} M_Z^2 Z_\mu Z^\mu$ the $W^\pm$ and $Z$ bosons have acquired the masses
\begin{equation}
  M_W = \frac{v g}{2}, \qquad  
  M_Z = \frac{v g}{2\cos\theta_\text{w}} = \frac{M_W}{\cos\theta_\text{w}}.
\end{equation}
No field quadratic in $A_\mu$ appears, which represents the fact that the photon is massless.

In addition, the Lagrangian shown in \cref{eq:lagrangianhiggs} comprises the expression
\begin{equation}
  \mathcal{L}_{\text{H,boson-coupling}} = M_W^2 W_\mu^+W^{-\,\mu} \left( \frac{2H}{v} + \frac{H^2}{v^2} \right) + \frac{1}{2} M_Z^2 Z_\mu Z^\mu \left( \frac{2H}{v} + \frac{H^2}{v^2} \right),
  \label{eq:higgsbosoncoupling}
\end{equation}
which shows that the Higgs interactions are proportional to the mass squared of the coupled boson and that there exist triplet ($V^\dagger VH$) and quartic ($V^\dagger VHH$) interaction terms.

%and choosing a specific gauge (the unitary gauge) the field $\Phi$ can be written as


The predicted scalar field $H$ is known as the \emph{Higgs boson} and has a mass of 
\begin{equation}
  M_H = \sqrt{2} \mu = \sqrt{2 \lambda} v.
\end{equation}
As explained below, the Higgs boson also provides masses to fermions and thus couples to all massive elementary particles.

% The parameters of the Higgs potential $\mu^2 = \lambda v^2$ can be fixed for one combination by measuring parameters of the electroweak theory, but the physical Higgs mass can not be predicted. 
% Moreover, the last two terms in \cref{eq:higgsselfcoupling} predict that the Higgs field couples to itself with cubic and quartic interactions.
% The couplings of the Higgs boson to the weak gauge bosons are already expressed in \cref{eq:lagrangianmasses}.

%While the Higgs boson was discovered experimentally in 2012 \cite{Aad:2012tfa,Chatrchyan:2012xdj}, and is by now seen in several different production and decay modes, the Higgs boson self-couplings are still searched for.
A more detailed descriptions on experimental Higgs boson physics and an overview of the current state of knowledge is given in \cref{sec:higgsphysics}.




In summary, adding a complex scalar Higgs field to the electroweak model, together with a potential that exhibits a non-zero vacuum expectation value, provides the ingredients to obtain mass terms for the weak gauge bosons $W^\pm$ and $Z$ when choosing a ground state of the Higgs field which spontaneously breaks the symmetry.
The three Goldstone bosons that arise can be eliminated from the Lagrangian by using its underlying local gauge symmetry. 
This results in three bosons acquiring a mass and the appearance of one remaining scalar field $H$. 
The photon remains massless, because the U(1)$_{\text{QED}}$ symmetry is unbroken. 






\subsection{Fermion masses}
The masses of the fermions can be explained with the same Higgs field, $\Phi$, by introducing interaction terms between the fermion fields and the Higgs field.
This is possible because both the left-handed fermions and the Higgs field appear as doublets under SU(2)$_L$ transformations.
These interactions are known as \emph{Yukawa interactions} and have the form
\begin{equation}
  \mathcal{L}_\text{Yukawa} = - Y_f \left[ \bar{\psi_L} \Phi \psi_R + \bar{\psi_R} \Phi^\dagger \psi_L \right],
  \label{eq:lyukawa}
\end{equation}
where $Y_f$ denotes the \emph{Yukawa coupling} of fermion $f$ to the Higgs field.

\todo{QUARK yukawa couplings}

Given \cref{eq:lyukawa}, the following mass terms are obtained after spontaneous symmetry breaking:
\begin{equation}
  m_f = \frac{Y_f v}{\sqrt{2}}.
\end{equation}
The coupling strength between the fermions and the Higgs field is therefore proportional to the mass of the fermions.

%\subsection{Renormalization}
% From modern particle physics book
% As shown by ‘t Hooft, only theories with local gauge invariance are renormalisable, such that the cancellation of all infinities takes place among only a finite number of interaction terms.

\subsection{The Final Standard Model Lagrangian}
To summarize the previous sections, the final Standard Model Lagrangian is obtained from \cref{eq:lqcd,eq:lagrangianewk,eq:lagrangianhiggs,eq:lyukawa} and can be written as
\begin{equation}
  \mathcal{L}_\text{SM} = \mathcal{L}_\text{QCD} + \mathcal{L}_\text{EWK} + \mathcal{L}_\text{Higgs} + \mathcal{L}_\text{Yukawa}.
\end{equation}
The covariant derivate includes all gauge fields and is defined as
\begin{equation}
  D_\mu = \partial_\mu + i g_s \frac{\lambda^a}{2} G_\mu^a + igW_\mu^aT^a + ig'B_\mu \frac{Y}{2}.
\end{equation}

% It satisfies a SU(3)$_C$ $\times$ SU(2)$_\text{L}$ $\times$ U(1)$_Y$ gauge symmetry and provides masses to the gauge bosons and fermions via the concept of electroweak symmetry breaking. This is manifested in the couplings of the Higgs field to fermions, which are proportional to the mass of the fermions, and the couplings to the gauge bosons, which are proportional to the mass squared of the gauge bosons.


\subsection{Free parameters of the SM}
The SM has been very successful in making predictions that were later confirmed by experimental data (see \cref{eq:sm-limitations}). However, the SM has many free parameters that need to be measured experimentally and cannot be derived from theoretical principles.  
In total there are 19 free parameters that can be represented with different parametrizations:
\begin{itemize}
  \item 9 fermion masses (or Yukawa couplings)
  \item 2 parameters describing the Higgs field: $\mu$ and $\lambda$ or $v$ and $m_H$
  \item four mixing angles of the CKM matrix
  \item 3 couplings constants: $\alpha$, $G_f$, $\alpha_s$ (or $g'$, $g_w$, $g_s$)
\end{itemize}

In total 14 parameters are associated with the Higgs field, four with the flavour sector, and only three with the gauge interactions. This underlines the fact that the Higgs boson plays a major role in the SM.


\subsection{Limitations of the SM}


\subsubsection{Gravitation}

\subsubsection{Higgs boson}

\subsubsection{Dark matter}

\subsubsection{Neutrinos}
Right-handed neutrinos are not part of the SM and therefore masses terms cannot be added. However, from neutrino oscillation experiments it was found that neutrinos have masses.

Extensions of the SM suggest to add mass terms which would add an additional 6 free parameters to the SM





