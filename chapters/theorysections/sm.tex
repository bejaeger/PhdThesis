

% From Master's thesis:
The SM of particle physics is a collection of relativistic quantum field theories (QFT) and describes the interactions between all known fundamental particles. 
%It evolved during the 60’s and 70’s due to a strong interplay between experimental observations and theoretical developments. 

- Introduction to QFT and gauge group

- Lagrangian formalism
- local gauge invariance: without local gauge invariance:
% From Pich The Standard Model
% Thus, once a given phase convention has been adopted at one reference point x0, the same convention must be taken at all space-time points. This looks very unnatural.
- highly successful in describing QED

\subsection{Particles and Forces}
\label{sec:particle-content}
%In QFT all fundamental fields have an associated particle, that can be thought of as excitations (or vibrations) of the fields.
On a \TDnote{phenomenological}{other word?} level, fundamental physics can be described in terms of elementary particles and forces. 

The SM includes three of the four known fundamental interactions: The \emph{electromagnetic interaction} -- described by the theory of \emph{quantum electrodynamics} (QED) --, the \emph{strong interaction} -- using \emph{quantum chromodynamics} (QCD) [15] -- and the \emph{weak interaction}. The electromagnetic and the weak interaction are unified in the \emph{electroweak model} [16–18]. The fourth known fundamental interaction, \emph{gravitation}, is not part of the SM but can be fully neglected in present particle physics experiments due to its weak strength.

% All currently known particles and forces included in the SM are listed in \TDnote{REF}{REF}.
The elementary particles included in the SM are listed in \TDnote{REF}{REF}. They can be grouped into two main types: \emph{fermions}, that are sometimes referred to as \emph{matter particles} carrying half-integer spin, and \emph{bosons}, that are the \emph{mediators} (or \emph{force carriers}) of the fundamental forces carrying integer spin. 

The fermions can be further divided into a set of six leptons and six quarks, each of which can be classified into three generations. 
Their interactions with each other can be described by exchanges of so-called \emph{gauge bosons}. 
Every fermion has a corresponding antiparticle that has the same mass and spin but opposite internal quantum numbers and opposite \emph{chirality}. 
Quarks have an additional property called \emph{colour}, which can take three possible values or any linear combinations of these values.

The boson that mediates the electromagnetic interaction is the massless \emph{photon} that interacts with all charged fermions. 
The strong interaction is transmitted via eight massless gluons and acts only on quarks and gluons themselves. 
Due to the property of \emph{confinement} in QCD, quarks cannot be found in isolation. They are confined within hadrons that consist, for example, of a quark and an antiquark (known as \emph{mesons}) or three quarks (known as \emph{baryons}). 

The massive gauge bosons, \Wplus, \Wminus, and \Zboson, are the mediators of the weak interaction which acts on all fermions. As well, the \Wmp and \Zboson undergo self-interactions. 

The final particle of the SM is the electrically neutral \emph{Higgs boson}. It is the only spin-0 scalar particle and plays a special role in the SM in the mechanism of \emph{electroweak symmetry breaking}, which gives rise to the masses of the \Wpm and \Zboson bosons. More details about Higgs boson physics are provided in \TDnote{REF}{REF}.


\subsection{Formalism and Principles} 
% Particles can be thought of as excitations (or vibrations) of corresponding fundamental fields. 
QFT describes nature in terms of fundamental fields. Each field has an associated particle that can be thought of as excitations (or vibrations) of the fields.\footnote{To simplify terminology, the objects are mostly referred to as particles rather than fields in this paper, but the relationship should be kept in mind.}

The equations of motions of a given physical system can be derived from the action, 
\begin{equation}
    \label{eq:action}
    S = \int \Lagrangian \quad dt d^3x, 
\end{equation}
by following the action principle.\footnote{The derivation can be found in any standard text QFT such as REF (Peskin schroeder?).}.
The dynamics of a system in QFT are therefore expressed in terms of the Lagrange densities (often just \emph{Lagrangian}), \Lagrangian. 

- gauge theories!

- Lagrangian depends on fields and derivates of fields (w.r.t. time or space): L (phi(x), del phi(x))

The Lagrangians that are introduced is constructed to be invariant under local gauge transformations, 
\begin{equation}
  \psi \rightarrow e^{-i\omega(x)} \psi,
\end{equation}
where $\omega \rightarrow \omega(x)$ is dependent on the point $x$ in spacetime. 


The SM is a reflection of an underlying symmetry group of the participating quantum fields. 
The SM follows a SU(3)$_C$ $\times$ SU(2)$_L$ $\times$ U(1)$_Y$ symmetry group. 

% Sean
% QED: demanding all the terms in your Lagrangian being gauge invariant is enforcing the conservation of electric charge gauge
% This is a reflection of Noethers theory. Symmetry conservation -> associated quantity of the U(1) symmetry is charge

% - Action: S = Int ( Lagrangian ) dt -> EOMs
% - Lagrangian = Int (Lagrange density ) d3x
% -> Lagrange density is commonly referred to as Lagrangian 
% - In QFT, we define EOMs for fields by specifying the Lagrange density (Lagrangian)

% Sean Carrol:
% W and Z bosons are the physical excitations from vibrations in the SU(2) to gauge field
%the existence of these fields giving rise to interactions giving rise to forces of nature comes from the gauge symmetry.
% the next important lesson is that the existence of these symmetries places suck an incredibly strong constraint on what the theory actually is.

% An essential part of the mathematical formulation of the SM is based on the postulation of local gauge invariance. It implies, that the physical content of the theory should stay the same when performing certain redefinitions of the particle fields independently at every space-time point. The SM follows a SU(3)×SU(2)L×U(1)Y symmetry group, where SU(3) is the gauge group of QCD and SU(2)L×U(1)Y the corresponding symmetry for the electroweak model (the meaning of the subindizes L and Y will be explained in the following sections). Historically, QED was the first well established gauge theory, following a U(1) symmetry. To demonstrate the principles of a gauge theory, which are the key concepts of the mathematical framework of the SM, the QED Lagrangian will be derived in the following. Subsequently the same principles are applied to describe the main aspects of the more complex theories of QCD and the electroweak model. A description of the mechanism to incorporate masses for the W± and Z bosons via breaking the electroweak symmetry is given thereafter. The following sections follow to a large extend the more detailed descriptions given in Refs. [19–21].

- renormalization:
We consider that the physics is understood up to a given cut-off/scale Lambda. 

Throughout this thesis, natural units are used, that is $c = \hbar = 1$. This means that the mass, momentum and energy are all given in units of electronvolt (\eV).


\subsection{Quantum electrodynamics}
%The above derivations are shortly recapped: Starting from the Lagrangian of a freely moving relativistic fermion one can impose the requirement of local U(1) gauge invariance. This demands to add a new field $A_\mu$ toghether with an interaction term, which is incorporated in the Lagrangian by introducing a covariant derivative $D_\mu$. Additionally, the kinetic term in \cref{eq:kinetictermqed} needs to be added for the new field, which leads to the final Lagrangian for QED, shown in \cref{eq:Lagrangianqed}.

QED is the reflection of an underlying $U(1)$ local gauge symmetry of the complex-valued fermion fields, $\psi_f$, known as \emph{Dirac spinors}.
The Lagrangian that is invariant under $\psi_f \rightarrow e^{i \omega(x)} \psi_f$ transformations can be written as
\begin{equation}
  \mathcal{L}_{\text{QED}} = \sum_f \bar{\psi_f}(i\gamma^\mu D_\mu - m_f)\psi_f - \frac{1}{4}F_{\mu\nu}F^{\mu\nu},
  \label{eq:Lagrangianqed}
\end{equation}
where the sum goes over all electrically charged fermions with mass $m_f$. 
Here, $D_\mu$ is the \emph{covariant derivative}, defined as
\begin{equation}
  D_\mu = \partial_\mu + ieA_\mu,  
\end{equation}
which includes the gauge field $A_\mu$, associated with the photon. The requirement of local gauge symmetry prohibits terms quadratic in $A_\mu$, which reflects the fact that the photon is massless.\footnote{Terms in the Lagrangian that are quadratic in a certain field give rise to particle masses.}
The so-called \emph{field tensors} are defined as 
\begin{equation}
  F_{\mu\nu} = \partial_\mu A_\nu - \partial_\nu A_\mu, 
\end{equation}
where $\gamma^\mu$ refers to the four $4x4$ gamma matrices, and $\partial_\mu$ are the partial derivatives. 

According to Noether's theorem\TDnote{REF}{REF} each symmetry of a system results in a conserved quantity. 
The symmetry group associated to QED is denoted U(1)$_{\text{QED}}$ and the conserved quantity is the electric charge.
It should be noted that U(1)$_{\text{QED}}$ is not part of the original symmetry group of the SM. As explained below, U(1)$_{\text{QED}}$ arises from a SU(2)$_L$ $\times$ U(1)$_Y$ symmetry that is spontaneously broken.


\subsection{Quantum chromodynamics}
QCD is a non-abelian gauge theory associated to the SU(3)$_c$ symmetry, where the subindex $c$ refers to the colour as the conserved quantity. 
The QCD Lagrangian reads
\begin{equation}
    %  \mathcal{L}_{\text{QCD}} = -\frac{1}{4}G_{\mu\nu}^aG^{a\,\mu\nu} + \bar{q}^i\left( i\gamma^\mu D_\mu-m \right)^j_i q_j,
      \mathcal{L}_{\text{QCD}} = -\frac{1}{4}G_{\mu\nu}^aG^{a\,\mu\nu} + i \bar{q}^i \gamma^\mu D_\mu q_i,  \label{eq:lqcd}
    \end{equation}
    where $G_{\mu\nu}^a$ is the field tensor defined as
    \begin{equation}
      G_{\mu\nu}^a = \partial_\mu G_\nu^a - \partial_\nu G_\mu^a - g_s f^{abc}G_\mu^b G_\nu^c.
    \end{equation}

% Invariant under ... transformation, where ... are the group generators

    The associated gauge fields are the gluons, that are also massless similar to the photon in QED. Other than photons, however, gluons can interact with each other, a feature that arises from the non-commuting elements of the SU(3)$_c$. Interaction terms between gluons, 
\begin{equation}
    NOT SURE ABOUT THIS YET
\end{equation}
    , are therefore part of the QCD Lagrangian.
    %which is similar to the requirement of a massless photon in QED. 
    



\subsection{The electroweak model}
% - the chirality can be determined with... right-chiral fermions are singlets under SU(2)L transformations
% - This also means that no right-chiral neutrinos exist in the SM as they don't interact with any of the forces

% "Also called: The Glashow􏰁Weinb erg􏰁Salam Theory of Weak Interactions"
The weak and electromagnetic force are unified in the electroweak model developed by Glashow, Salam, Weinberg \TDnote{REF}{REF}.  
The unification is based on the symmetry group
\begin{equation}
  \text{SU(2)}_L \times \text{U(1)}_Y.
\end{equation}
The subindex $L$ reflects the empirical finding that only \emph{left-chiral}\footnote{
  The \emph{chirality} of a fermion can be determined with
  \begin{align}
    \psi &= \psi_L + \psi_R,  \\
    \psi_{L/R} &= \frac{1}{2} \left( 1 \mp \gamma^5 \right) \psi, 
  \end{align}
  where $\gamma^5 = i\gamma^0\gamma^1\gamma^2\gamma^3$. The chirality becomes identical to the helicity for massless particles. 
} (also denoted \emph{left-handed}) fermions interact via the weak interaction. 
The fermions are therefore grouped into left-handed doublets and right-handed singlets under SU(2)$_L$ transformations.
The associated conserved quantity is the \emph{weak isospin}, $T$, with values $T^3 = \pm \frac{1}{2}$ for SU(2)$_L$ doublets and $T^3 = 0$ for SU(2)$_L$ singlets.

The U(1)$_Y$ symmetry is associated to the \emph{hypercharge}, $Y$, and should not be mistaken with the QED gauge group.
The relation to the electromagnetic interaction and the physical electric charge $Q$ is
\begin{equation}
  Q = T^3 + \frac{Y}{2}.
\end{equation}

For the Lagrangian to be invariant under local SU(2)$_L$ $\times$ U(1)$_Y$ transformations, 
\begin{align}
  \psi_L &\rightarrow e^{iY\omega} e^{iT^a\omega^a} \psi_L  \\
  \psi_R &\rightarrow e^{iY\omega} \psi_R,
\end{align} 
where $T^a = \frac{\sigma^a}{2}$ ($a = 1, 2, 3$ and $\sigma^a$ denote the Pauli matrices) are the generators of the SU(2)$_L$ group and $\psi_L$ and $\psi_R$ are, respectively, representations of the left-handed and right-handed fermion fields, four new fields are introduced with the covariant derivative
\begin{equation}
  D_\mu = \partial_\mu + igW_\mu^aT^a + ig'B_\mu \frac{Y}{2}.
  \label{eq:covdevewk}
\end{equation}
The fields $W^a_\mu = W^1_\mu, W^1_\mu, W^3_\mu$ are the gauge fields of SU(2)$_L$ with associated coupling $g$ and $B_\mu$ is the gauge field of U(1)$_Y$ with coupling $g'$.
The gauge invariant electroweak Lagrangian reads
\begin{equation}
  \mathcal{L}_{\text{EWK}} = i\bar{\psi}^i_L\gamma^\mu D_\mu \psi^i_L + i\bar{\psi}^i_R\gamma^\mu D_\mu \psi^i_R - \frac{1}{4}W_{\mu\nu}^aW^{a\,\mu\nu} - \frac{1}{4} B_{\mu\nu}B^{\mu\nu},
  \label{eq:lagrangianewk}
\end{equation}
with the field tensors given by
\begin{align}
  W_{\mu\nu}^a &= \partial_\mu W_\nu^a - \partial_\nu W_\mu^a - g \epsilon^{abc} W^b_\mu W^c_\nu \label{eq:Wtensor} \\
  B_{\mu\nu} &= \partial_\mu B_\nu - \partial_\nu B_\mu.
\end{align}
The $\epsilon^{abc}$ are the structure constants of SU(2)$_L$. The third term on the right-hand side of \cref{eq:Wtensor} arises, as in the case of QCD, from the fact that SU(2) is non-abelian and gives rise to cubic and quartic self-interactions between the gauge fields $W_{\mu}^a$. The tensor $B_{\mu\nu}$ has the identical structure as the electromagnetic field strength tensor obtained in QED.

The Lagrangian in \cref{eq:lagrangianewk} describes 4 massless bosons. No terms quadratic in the vector fields are allowed due to the violation of local gauge invariance. Moreover, also the simple inclusion of fermion mass terms is forbidden, given that a term
\begin{equation}
  -m_f \bar{\psi} \psi = -m_f \left( \bar{\psi_R}\psi_L + \bar{\psi_L}\psi_R \right)
  \label{eq:fermionmassterm}
\end{equation}
violates SU(2)$_L$ gauge invariance, since $\psi_L$ transforms as a doublet and $\psi_R$ as a singlet.

The physical fields, \Wpm, \Zboson, $\gamma$ naturally appear in the Higgs mechanism explained below. 

% From experiments one expects two charged bosons $W^\pm$ and two neutral bosons, the $Z$ and the photon.


An overview of the fermion content of the electroweak model is shown in \cref{tab:ewfermioncontent}. Quarks undergoing the electroweak interaction are present as a mixture of the quark mass eigenstates. This phenomenon is described by the \emph{Cabibbo–Kobayashi–Maskawa (CKM) mechanism} \cite{doi:10.1143/PTP.49.652}.









\subsection{Spontaneous symmetry breaking and the Higgs mechanism}
- Local gauge invariance does not allow adding mass terms 

- Need mechanism to generate mass terms.
- Spontaneous symmetry breaking can do that (Reference to anderson)
- The SU(2)$_L$ $\times$ U(1)$_Y$ symmetry is spontaneously broken into U(1)$_\text{QED}$

- This mechanism is spontaneous symmetry breaking
- Simply put, the Lagrangian itself maintains the symmetry, but the state of lowest energy is not invariant and breaks the summetry. 

- 

% - Could add mass terms disregarding local gauge invariance but this would render theory unrenormalizable


% From Peskin and Schroeder
% - In the theory of sup erconductiv􏰁 ity􏰔 for example􏰔 the Ab elian gauge invariance of electromagnetism is broken by pairs of electrons that condense in the ground state of a metal􏰎


- including Higgs self couplings (see Master thesis)



\subsection{Fermion masses}
- The fermion masses are generated by interaction terms of the fermion fields and the Higgs field.
- This is possible because the left-chiral fermions appear as doublets under SU(2) transformations and can interact with the Higgs doublet.


\subsection{Renormalization}
% From modern particle physics book
% As shown by ‘t Hooft, only theories with local gauge invariance are renormalisable, such that the cancellation of all infinities takes place among only a finite number of interaction terms.

\subsection{Free parameters of the SM}
- SM makes predictions but some of the parameters are free

% From Brian but I think that is very useful to have here!

% Dickinson: In order to study physics at the highest energies and smallest distance scales, collider experiments focus on hard scatter interactions, in which there is a large momentum transfer (Q2) between two incident particles. 

% Dickinson Thesis:
% Perturbation theory: If the highest order term included in a calculation of  is linear in ↵S, the calculation is called “leading order” (LO). If the sum is completed up to the quadratic, cubic, or quartic term in ↵S, the calculation is called “next-to-leading order” (NLO), “next-to-next-to-leading order” (NNLO), or “next-to-next-to-next-to-leading order” (N3LO).


