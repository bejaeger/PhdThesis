

% From Master's thesis:
The SM of particle physics is a collection of relativistic quantum field theories (QFT) that describe the interactions between all known fundamental particles.
Mathematically, it is as a SU(3)$_C$ $\times$ SU(2)$_L$ $\times$ U(1)$_Y$ local gauge theory, the details of which are explained in this section.
% The mathematical formulation SU(3)$_C$ $\times$ SU(2)$_L$ $\times$ U(1)$_Y$ local gauge symmetry that gives rise to the interactions between all known fundamental particles.

%It evolved during the 60’s and 70’s due to a strong interplay between experimental observations and theoretical developments. 
The particle content of the SM is first described in \cref{subsec:particle-content}. \Cref{subsec:formalism} then briefly describes the theoretical principles that build the basis for the mathematical description of the SM, that is outlined in \cref{subsec:qed,subsec:qcd,subsec:ew-model,subsec:ewsymbreaking,subsec:fermion-masses,subsec:final-lagrangian,subsec:free-pars-sm}. This section concludes by describing the open questions of the SM that motivate precision measurements of the Higgs boson such as the one presented in this thesis.


%On a phenomenological level, fundamental physics can be described in terms of elementary particles and forces. 
% - Introduction to QFT and gauge group
% - Lagrangian formalism
% - local gauge invariance: without local gauge invariance:

% From Pich The Standard Model
% Thus, once a given phase convention has been adopted at one reference point x0, the same convention must be taken at all space-time points. This looks very unnatural.

% - highly successful in describing QED
% WHAT HAPPENS IN THIS SECTiION:

% - First overview of particles and forces
% - Formalism
% - QED, QCD, electroweak model
% - Problem: no masses for bosons -> Higgs mechanism
% - Final Lagrangian and parameters to be measured experimentally of the SM
% - Limitations of the SM

\subsection{Particles and Forces}
\label{subsec:particle-content}

\begin{figure}
  \newImageResizeCustom{0.99}{figures/theory/particles-infographic/particles-infographic.pdf}
  \caption[Overview of particles in the SM.]{Overview of the particles in the SM. Adapted from \ccite{CBurgardParticlesInfographic}. Values taken from \ccite{PDG2020}.}  
  \label{fig:particles-infographic}
\end{figure}


%In QFT all fundamental fields have an associated particle, that can be thought of as excitations (or vibrations) of the fields.
In QFT, nature is described in terms of fundamental fields and interactions between them. Each field is associated to an elementary particle that can be thought of as an excitation (or vibration) of the underlying field. This association allows describing fundamental physics in terms of elementary particles. 
%\footnote{To simplify terminology, the objects are mostly referred to as particles rather than fields in this chapter, but the relationship is important to be kept in mind.} 

% On a \TDnote{phenomenological}{other word?} level, fundamental physics can be described in terms of elementary particles and forces.
% All currently known particles and forces included in the SM are listed in \TDnote{REF}{REF}.
The particle content of the SM is summarized in \cref{fig:particles-infographic}. The elementary particles can be grouped into two main types: \emph{fermions} with half-integer spin, sometimes called \emph{matter particles}, and \emph{bosons} with integer spin, which are the \emph{mediators} (or \emph{force carriers}) of the fundamental forces.

The fermions can be grouped into three generations, each containing two leptons and two quarks. The different types of leptons and quarks are referred to as \emph{flavours}.
Each fermion has a corresponding antiparticle that has the same mass and spin but opposite internal quantum numbers and \emph{chirality}.
Quarks have an additional property called \emph{colour}, which can take three possible values typically labelled as red, green, and blue\footnote{In fact, also linear combinations of the three colours are realized in nature, as discussed later in this chapter.}.

The interactions between the fermions can be described by exchanges of so-called \emph{gauge bosons}.
The SM includes three of the four known fundamental interactions: The \emph{electromagnetic interaction}, the \emph{strong interaction}, and the \emph{weak interaction}. The electromagnetic and the weak interactions are unified in the \emph{electroweak theory}. The fourth known fundamental interaction, \emph{gravitation}, is not part of the SM but can be fully neglected in present particle physics experiments due to its weak strength.

The boson that mediates the electromagnetic interaction is the massless \emph{photon} that interacts with all charged fermions. Electromagnetic interactions are described by the theory of \emph{quantum electrodynamics} (QED). The strong interaction, described in \emph{quantum chromodynamics} (QCD), is transmitted via eight massless gluons. QCD acts on matter particles that carry colour, that is quarks and gluons themselves. 
Due to the property of \emph{confinement} in QCD, quarks cannot be found in isolation. They are confined within hadrons, that consist for example of a quark and an anti-quark (known as \emph{mesons}) or three quarks (known as \emph{baryons}).
The weak interaction is mediated by three massive gauge bosons, \Wplus, \Wminus, and \Zboson, and acts on all fermions. Similar to gluons the \Wpm and \Zboson bosons are able to interact with themselves.

The final particle of the SM is the electrically neutral \emph{Higgs boson}. It is the only spin-0 scalar particle and plays a special role in the SM in the mechanism of \emph{electroweak symmetry breaking}, which gives rise to the masses of the \Wpm and \Zboson bosons. More details in particular about physics involving the Higgs boson are provided in \cref{sec:higgs-phen}.


\subsection{Formalism and Principles}
\label{subsec:formalism}
% - Action: S = Int ( Lagrangian ) dt -> EOMs
% - Lagrangian = Int (Lagrange density ) d3x
% -> Lagrange density is commonly referred to as Lagrangian 
% - In QFT, we define EOMs for fields by specifying the Lagrange density (Lagrangian)

% Particles can be thought of as excitations (or vibrations) of corresponding fundamental fields. 
The equations of motion of a given physical system can be derived from the action,
\begin{equation}
  \label{eq:action}
  S = \int \Lagrangian d^3x dt,
\end{equation}
by following the action principle.\footnote{The derivation of the equations of motion can be found in any standard textbook about QFT, for example in \ccite{Peskin:1995ev}.}.
In QFT, the dynamics of a system are therefore expressed in terms of the Lagrange densities, \Lagrangian. 
The Lagrange densities, often simply denoted \emph{Lagrangian}, is a function of the fields, $\psi$, and their derivatives, $\partial \psi$, that is $\Lagrangian \rightarrow \Lagrangian\left(\psi, \partial\psi)\right)$. The fields are themselves dependent on the space-time coordinate, $\psi \rightarrow \psi(x)$\footnote{For cleaner notation, the dependence on $x$ is usually not explicitely mentioned in this thesis.}. 

For a given field, $\psi$, the principles of QFT demand to include all possible interaction terms in the Lagrangian that are related to the field\footnote{This follows what is sometimes referred to as the totalitarian principle of quantum mechanics, that satirically states: ``Everything not forbidden is compulsory''.}.
The number and type of terms that can be included, however, is strongly constraint by requiring the theory to be invariant under certain symmetry transformations of the fields. 
The SM is governed by the principle of \emph{local gauge invariance}. 
Local gauge invariance implies, that the physical content of the theory stays the same when performing certain re-definitions of the particle fields independently at every space-time point.
The structure of the Lagrangian is therefore determined by requiring invariance under field transformations of the sort,
\begin{equation}
  \psi \rightarrow e^{-iT} \psi,
\end{equation}
where $T$ is the generator of a certain symmetry group and generally dependent on the space-time coordinate, $T \rightarrow T(x)$. 
Gauge theories require the introduction of so-called gauge fields, that transform in a way so that the theory stays locally gauge invariant.
% Pich
% This is only possible if one adds an extra piece to the Lagrangian, transforming in such a way as to cancel the ∂μθ term in Eq. (6).
% In order to maintain the symmetry, the Lagrangian may need to be manipulated by introducing new fields.
These fields give rise to particle interactions that give rise to the fundamental forces of nature. 
Historically, the first local gauge theory that was established was QED. It follows a U(1) local gauge symmetry and entails the photon as the associated gauge field. 
The SM is a reflection of a SU(3)$_C$ $\times$ SU(2)$_L$ $\times$ U(1)$_Y$ local gauge symmetry that gives rise to the existence of all gauge bosons mentioned in the previous section. 
%The following sections provide an overview of the Lagrangian of the SM, by going through the different symmetry groups. 

Throughout this thesis, natural units are used, that is $c = \hbar = 1$. This leads to the mass, momentum, and energy of particles all being given in units of electronvolt (\eV).
In the equations in the following sections, the Einstein-summation convention is used, that is terms with equal indices are considered to be summed. If not mentioned otherwise, greek-letter indices take integer values from 1 to 4, while latin-character indices take integer values from 1 to 3.

% - renormalization:
% We consider that the physics is understood up to a given cut-off/scale Lambda.

%The SM is governed by the principle of local gauge invariance and the associated symmetry group is SU(3)$_C$ $\times$ SU(2)$_L$ $\times$ U(1)$_Y$. 

% the next important lesson is that the existence of these symmetries places suck an incredibly strong constraint on what the theory actually is.

% Sean
% QED: demanding all the terms in your Lagrangian being gauge invariant is enforcing the conservation of electric charge gauge
% This is a reflection of Noethers theory. Symmetry conservation -> associated quantity of the U(1) symmetry is charge

% Sean Carrol:
% W and Z bosons are the physical excitations from vibrations in the SU(2) to gauge field
%the existence of these fields giving rise to interactions giving rise to forces of nature comes from the gauge symmetry.
% the next important lesson is that the existence of these symmetries places suck an incredibly strong constraint on what the theory actually is.

% From Master thesis
% An essential part of the mathematical formulation of the SM is based on the postulation of local gauge invariance. It implies, that the physical content of the theory should stay the same when performing certain redefinitions of the particle fields independently at every space-time point. The SM follows a SU(3)×SU(2)L×U(1)Y symmetry group, where SU(3) is the gauge group of QCD and SU(2)L×U(1)Y the corresponding symmetry for the electroweak model (the meaning of the subindizes L and Y will be explained in the following sections). Historically, QED was the first well established gauge theory, following a U(1) symmetry. To demonstrate the principles of a gauge theory, which are the key concepts of the mathematical framework of the SM, the QED Lagrangian will be derived in the following. Subsequently the same principles are applied to describe the main aspects of the more complex theories of QCD and the electroweak model. A description of the mechanism to incorporate masses for the W± and Z bosons via breaking the electroweak symmetry is given thereafter. The following sections follow to a large extend the more detailed descriptions given in Refs. [19–21].


\subsection{Quantum electrodynamics}
\label{subsec:qed}
%The above derivations are shortly recapped: Starting from the Lagrangian of a freely moving relativistic fermion one can impose the requirement of local U(1) gauge invariance. This demands to add a new field $A_\mu$ toghether with an interaction term, which is incorporated in the Lagrangian by introducing a covariant derivative $D_\mu$. Additionally, the kinetic term in \cref{eq:kinetictermqed} needs to be added for the new field, which leads to the final Lagrangian for QED, shown in \cref{eq:Lagrangianqed}.
QED is the reflection of an underlying $U(1)$ local gauge symmetry of the complex-valued fermion fields, $\psi_f$, known as \emph{Dirac spinors}.
The Lagrangian that is invariant under $\psi_f \rightarrow e^{i \omega(x)} \psi_f$ transformations can be written as
\begin{equation}
  \mathcal{L}_{\text{QED}} = \sum_f \bar{\psi}_f(i\gamma^\mu D_\mu - m_f)\psi_f - \frac{1}{4}F_{\mu\nu}F^{\mu\nu},
  \label{eq:Lagrangianqed}
\end{equation}
where the sum goes over all electrically charged fermions with mass $m_f$.
Here, $D_\mu$ is the \emph{covariant derivative}, defined as
\begin{equation}
  D_\mu = \partial_\mu + ieA_\mu,
\end{equation}
which includes the gauge field $A_\mu$, associated with the photon. The requirement of local gauge symmetry prohibits terms quadratic in $A_\mu$, resulting in the prediction of the photon being massless.\footnote{Terms in the Lagrangian that are quadratic in a certain field give rise to particle masses.}
The so-called \emph{field tensors} are defined as
\begin{equation}
  F_{\mu\nu} = \partial_\mu A_\nu - \partial_\nu A_\mu,
\end{equation}
where $\gamma^\mu$ refers to the four $4x4$ gamma matrices, and $\partial_\mu$ are the partial derivatives.

The symmetry group associated to QED is denoted U(1)$_{\text{QED}}$ and the conserved quantity is the electric charge.\footnote{This is a reflection of Noether's theorem, which states that each symmetry is related to a conserved quantity.}
It should be noted that U(1)$_{\text{QED}}$ is not part of the original symmetry group of the SM. As explained below, U(1)$_{\text{QED}}$ arises from a SU(2)$_L$ $\times$ U(1)$_Y$ symmetry that is spontaneously broken.


\subsection{Quantum chromodynamics}
\label{subsec:qcd}
QCD is a non-abelian gauge theory associated to the SU(3)$_C$ symmetry. The subindex $C$ refers to the colour and is the conserved quantity under SU(3)$_C$ transformations.
The local gauge invariant QCD Lagrangian reads
\begin{equation}
  %  \mathcal{L}_{\text{QCD}} = -\frac{1}{4}G_{\mu\nu}^aG^{a\,\mu\nu} + \bar{q}^i\left( i\gamma^\mu D_\mu-m \right)^j_i q_j,
  \mathcal{L}_{\text{QCD}} = \sum_f \bar{\psi}_f(i\gamma^\mu D_\mu - m_f)\psi_f - \frac{1}{4}G_{\mu\nu}^aG^{\mu\nu}_{a},  \label{eq:lqcd}
\end{equation}
where the sum runs over all quark fields, $\psi_f$, where the fields are represented as spinor triplet. 
The covariant derivate is given by
\begin{equation}
    D_\mu = \partial_\mu + i g_s \frac{\lambda^a}{2} G_\mu^a,
\end{equation}
which includes eight gluon fields $G_\mu^a$ (with $a = 1, \ldots, 8$), as well as the strong coupling constant, $g_s$, and the Gell-Mann matrices, $\lambda^a$, which are the generators of the SU(3)$_C$ group.
The field tensors, $G_{\mu\nu}^a$, are defined as
\begin{equation}
  \label{eq:qcd-tensor}
  G_{\mu\nu}^a = \partial_\mu G_\nu^a - \partial_\nu G_\mu^a - g_s f^{abc}G_\mu^b G_\nu^c,
\end{equation}
where $f^{abc}$ are the structure constants of SU(3)$_C$. 
% Invariant under ... transformation, where ... are the group generators

The gluons are massless, as no term quadratic in the gluon fields are allowed when requiring local gauge invariance. Other than photons, however, gluons can interact with each other, a feature that arises from the non-commuting elements of the SU(3)$_C$ group. Triple and quartic self-interactions between gluons are therefore part of QCD because gluons themselves carry a combination of a colour and anti-colour. 
%which is similar to the requirement of a massless photon in QED. 



\subsection{The electroweak model}
\label{subsec:ew-model}
  \begin{table}
    \caption[Overview of the fermion content in the electroweak model.]{Overview of the fermion content in the electroweak model with the quantum numbers of the weak isospin $T^3$, hypercharge $Y$ and electric charge $Q$. They are grouped into left-handed doublets and right-handed singlets denoted with the subindex $L$ and $R$ respectively. The down-type quarks $d', s', b'$ are the eigenstates of the electroweak interaction and given by linear combinations of the mass eigenstates $d, s, b$. This mixing is described by the CKM matrix, see text. Since right-handed neutrinos are not undergoing any interaction they do not play a role in the Standard Model and are not listed here.}
    \label{tab:ewfermioncontent}
    \centering
      \begin{tabular}{c |@{}| c c c | c c c}
        \toprule
                                 & \multicolumn{3}{c}{Generation}          & \multicolumn{3}{|c}{Quantum number}                                                                                                            \\
                                 & 1$^{\text{st}}$                         & 2$^{\text{nd}}$                             & 3$^{\text{rd}}$                               & $T^3$          & $Y$            & $Q$            \\
        \midrule
        \multirow{3}{*}{Leptons} & \multirow{2}{*}{$\myvec{\nu_e                                                                                                                                                            \\ e}_L$} & \multirow{2}{*}{$\myvec{\nu_\mu \\ \mu}_L$} & \multirow{2}{*}{$\myvec{\nu_\tau \\ \tau}_L$} & $\frac{1}{2}$  & -1             & 0 \\
                                 &                                         &                                             &                                               & $-\frac{1}{2}$ & -1             & -1             \\
                                 & $e_R$                                   & $\mu_R$                                     & $\tau_R$                                      & 0              & -2             & -1             \\
        \midrule
        \multirow{3}{*}{Quarks}  & \multirow{2}{*}{$\myvec{u                                                                                                                                                                \\ d'}_L$}    & \multirow{2}{*}{$\myvec{c \\ s'}_L$}        & \multirow{2}{*}{$\myvec{t \\ b'}_L$}          & $\frac{1}{2}$  & $\frac{1}{3}$  & $\frac{2}{3}$ \\
                                 &                                         &                                             &                                               & $-\frac{1}{2}$ & $\frac{1}{3}$  & $-\frac{1}{3}$ \\
                                 & $u_R$                                   & $ c_R$                                      & $t_R$                                         & 0              & $\frac{4}{3}$  & $\frac{2}{3}$  \\
                                 & $d_R$                                   & $ s_R$                                      & $b_R$                                         & 0              & $-\frac{2}{3}$ & -$\frac{1}{3}$ \\
        \bottomrule
      \end{tabular}
  \end{table}



% - the chirality can be determined with... right-chiral fermions are singlets under SU(2)L transformations
% - This also means that no right-chiral neutrinos exist in the SM as they don't interact with any of the forces

% "Also called: The Glashow􏰁Weinb erg􏰁Salam Theory of Weak Interactions"
The weak and electromagnetic force are unified in the electroweak model~\cite{GLASHOW1961579,SALAM1964168,PhysRevLett.19.1264} by imposing local gauge invariance under transformations of the symmetry group
\begin{equation}
  \label{eq:ew-sym-group}
  \text{SU(2)}_L \times \text{U(1)}_Y.
\end{equation}
The subindex $L$ reflects the empirical finding that only \emph{left-chiral}\footnote{
  The \emph{chirality} of a fermion can be determined with the projection operators $P_L$ and $P_R$ like
  \begin{equation*}
    \psi       = P_L \psi + P_R \psi = \frac{1}{2} \left( 1 - \gamma^5 \right) \psi + \frac{1}{2} \left( 1 + \gamma^5 \right) \psi = \psi_R + \psi_L,                                \\
  \end{equation*}
  where $\gamma^5 = i\gamma^0\gamma^1\gamma^2\gamma^3$. The chirality becomes identical to the helicity for massless particles.
} (also denoted \emph{left-handed}) fermions interact via the weak interaction.
The fermions are therefore grouped into left-handed doublets, $\psi_L$, and right-handed singlets, $\psi_R$, under SU(2)$_L$ transformations. An overview of the fermion content is shown in \cref{tab:ewfermioncontent}.
The quarks participating in the electroweak interaction, labelled as $u', d', c'$, are a mixture of the quark mass eigenstates. Their relation is specified by the \emph{Cabibbo–Kobayashi–Maskawa (CKM) matrix} \cite{doi:10.1143/PTP.49.652}, $\pmb{V}$, as
% From Peskin
% The off diagonal terms in Vij allow weak􏲩interaction transitions b e􏲩 tween quark generations.
\begin{equation}
  \begin{pmatrix}
   d' \\
   s' \\
   b'
 \end{pmatrix}
 = 
 \pmb{V} 
 \begin{pmatrix}
   d \\
   u \\
   b
 \end{pmatrix}.
\end{equation}
The CKM matrix is unitary and fully specified with 4 parameters. It encodes the strength of the flavour-changing electroweak interactions.

\noindent The fermion fields transform as
\begin{align}
  \psi_L & \rightarrow e^{iY\omega} e^{iT^a\omega^a} \psi_L, \qquad (a = 1, 2, 3) \\
  \psi_R & \rightarrow e^{iY\omega} \psi_R,
\end{align}
under SU(2)$_L$ $\times$ U(1)$_Y$ transformations, where $T^a=\frac{\sigma^a}{2}$ are the Pauli matrices and generators of the SU(2)$_L$ group.
The associated conserved quantity is called \emph{weak isospin}, $T$, of which the third component is conserved in weak interactions and given by $T^{(3)} = \pm \frac{1}{2}$ for SU(2)$_L$ doublets and $T^{(3)} = 0$ for SU(2)$_L$ singlets.
The U(1)$_Y$ symmetry is associated to the \emph{hypercharge}, $Y$, and cannot be directly associated to the QED gauge group.
The relation to the electromagnetic interaction and the physical electric charge $Q$ is
\begin{equation}
  Q = T^{(3)} + \frac{Y}{2}.
\end{equation}

\noindent The local gauge invariant electroweak Lagrangian can be written as
\begin{equation}
  \mathcal{L}_{\text{EWK}} = \sum_f i\bar{\psi}_{f}\gamma^\mu D_\mu \psi_{f} - \frac{1}{4}W_{\mu\nu}^aW^{\mu\nu}_{a} - \frac{1}{4} B_{\mu\nu}B^{\mu\nu}, 
  \label{eq:lagrangianewk}
\end{equation}
where the sum runs over all fermions, $f$, including the left-handed and right-handed counterparts.
%, whose occurrence as left-handed and right-handed particles is explicitly mentioned.
The covariant derivative is defined as
\begin{equation}
  D_\mu = \partial_\mu + igT^aW_\mu^a + ig'\frac{Y}{2}B_\mu 
  \label{eq:covdevewk}
\end{equation}
and includes four gauge fields. The fields $W^a_\mu$ (with a = 1, 2, 3) are the gauge fields of SU(2)$_L$ with associated coupling $g$ and $B_\mu$ is the gauge field of U(1)$_Y$ with coupling $g'$.
The field tensors are given by
\begin{align}
  W_{\mu\nu}^a & = \partial_\mu W_\nu^a - \partial_\nu W_\mu^a - g \epsilon^{abc} W^b_\mu W^c_\nu, \label{eq:Wtensor} \\
  B_{\mu\nu}   & = \partial_\mu B_\nu - \partial_\nu B_\mu,
\end{align}
where $\epsilon^{abc}$ are the structure constants of SU(2)$_L$. The third term on the right-hand side of \cref{eq:Wtensor} arises because of the non-abelian nature of SU(2)$_L$ and gives rise to triple and quartic self-interactions between the gauge fields $W_{\mu}^a$.
The tensor $B_{\mu\nu}$ has the same structure as the electromagnetic field strength tensor obtained in QED.

\noindent The Lagrangian in \cref{eq:lagrangianewk} describes 4 massless bosons. No terms quadratic in the vector fields are allowed due to the requirement of local gauge invariance. Moreover, also the simple inclusion of fermion mass terms is forbidden, given that a term
\begin{equation}
  -m_f \bar{\psi} \psi = -m_f \left( \bar{\psi}_R\psi_L + \bar{\psi}_L\psi_R \right)
  \label{eq:fermionmassterm}
\end{equation}
violates SU(2)$_L$ gauge invariance, since $\psi_L$ transforms as a doublet and $\psi_R$ as a singlet. The physical fields, \Wpm, \Zboson, $\gamma$ naturally appear in the Higgs mechanism explained below.

% From experiments one expects two charged bosons $W^\pm$ and two neutral bosons, the $Z$ and the photon.

\subsection{Spontaneous symmetry breaking and the Higgs mechanism}
\label{subsec:ewsymbreaking}
%- Local gauge invariance does not allow adding mass terms
%- The SU(2)$_L$ $\times$ U(1)$_Y$ symmetry is spontaneously broken into U(1)$_\text{QED}$
%- This mechanism is spontaneous symmetry breaking
%- Simply put, the Lagrangian itself maintains the symmetry, but the state of lowest energy is not invariant and breaks the summetry.
% - Could add mass terms disregarding local gauge invariance but this would render theory unrenormalizable
\begin{figure}
  \newImageResizeCustom{0.75}{figures/theory/higgs-potential/higgs-potential.pdf}
  \caption[Two dimensional Higgs potential $V(\phi)$ with $\lambda > 0$ and $\mu^2 < 0$.]{Two dimensional Higgs potential $V(\phi)$ as in \cref{eq:higgspotential} with $\lambda > 0$ and $\mu^2 < 0$. The minimum occurs for different points on the sketched circle in the $(\phi_1, \phi_2)$ plane and has the value given in \cref{eq:higgsminima}.
  }
  \label{fig:higgspotential}
\end{figure}
An additional mechanism is needed in the SM to explain the finite masses of the weak gauge bosons. 
A principle known as \emph{spontaneous symmetry breaking}, that has been first explored in the field of condensed-matter physics, can be applied to elementary particle physics to generate the necessary mass terms without violating local gauge invariance.
This was first formulated by three independent research teams in 1964: Brout and Englert~\cite{PhysRevLett.13.321}, Higgs~\cite{PhysRevLett.13.508,HIGGS1964132}, and Guralnik, Hagan, and Kibble \cite{PhysRevLett.13.585}.\footnote{Their work was inspired by previous advancements in the theory of superconductivity~\cite{PhysRev.108.1175}, as well as Anderson, who proposed the mechanism of spontaneous symmetry breaking for generating mass terms in a non-relativistic scenario already in 1963~\cite{PhysRev.130.439}.}
Today the mechanism is usually called \emph{Higgs mechanism} or also \emph{electroweak symmetry breaking} (ESWB).
The basic principle is to allow the state of lowest energy to violate local gauge invariance while maintaining the gauge symmetry of the Lagrangian itself.

% From Peskin and Schroeder
% - In the theory of sup erconductiv􏰁 ity􏰔 for example􏰔 the Ab elian gauge invariance of electromagnetism is broken by pairs of electrons that condense in the ground state of a metal􏰎
%The mechanism is known as the \emph{Brout-Englert-Higgs mechanism} (or simply \emph{Higgs mechanism}) or also \emph{electroweak symmetry breaking} (ESWB).
%The underlying concept is that the Lagrangian itself maintains the symmetry, but the state of lowest energy is not invariant and breaks the gauge symmetry.
% was published almost simultaneously by three independent groups in 1964: by Robert Brout and François Englert;[3] by Peter Higgs;[4] and by Gerald Guralnik, C. R. Hagen, and Tom Kibble.[5][6][7]

The Higgs mechanism assumes a complex scalar field of the SU(2)$_L$ group,
\begin{equation}
  \phi = \frac{1}{\sqrt{2}} \myvec { \phi_1 + i \phi_3 \\ \phi_2 + i \phi_4},
\end{equation}
and introduces a potential of the form
\begin{equation}
  V(\phi) = \mu^2\phi^\dagger\phi + \lambda \left(\phi^\dagger\phi \right)^2.
  \label{eq:higgspotential}
\end{equation}
The Lagrangian 
\begin{equation}
  \mathcal{L}_{\text{Higgs}} = |D_\mu\phi|^2 - V(\phi), % - \frac{1}{4} W_{\mu\nu}^a W^{a\, \mu\nu} - \frac{1}{4} B_{\mu\nu} B^{\mu\nu},
  \label{eq:lagrangianhiggs}
\end{equation}
with $D_\mu$ defined as shown in \cref{eq:covdevewk}, is invariant under local SU(2)$_L$ $\times$ U(1)$_Y$ transformations.
The parameters of the potential, $\phi$, are specifically chosen to satisfy $\mu^2 < 0$ and $\lambda > 0$.
This choice provides a characteristic shape to the so-called \emph{Higgs potential}, which is depicted in \cref{fig:higgspotential}, and gives rise to a non-zero \emph{vacuum expectation value}, $v$, of the Higgs field.
%% We find that the Lagrangian for such a field
%% \begin{equation}
%% \end{equation}
%% is invariant under global SU(2) phase transformations
%It is invariant under gauge transformations but not the ground state, which can be found at
The minimum of the potential can be found on the circle of minima where
\begin{equation}
  |\phi| = \sqrt{ \frac{\mu^2}{2\lambda} } \equiv \frac{ v }{\sqrt{2}}.
  \label{eq:higgsminima}
\end{equation}
Once a ground state is chosen, the SU(2)$_L$ $\times$ U(1)$_Y$ symmetry becomes spontaneously broken.
% The symmetry is therefore spontaneously broken when a ground state is chosen. 
%The arbitrary choice of a ground state is said to spontaneously break the symmetry. 
Without loss of generality the ground state can be chosen to be
\begin{equation}
  \phi_0 = \frac{1}{\sqrt{2}} \myvec{0 \\ v},
  \label{eq:groundstate}
\end{equation}
that is $\phi_1 = \phi_2 = \phi_4 = 0$ and $\phi_3 = \frac{v}{\sqrt{2}}$. 
Expanding the field $\phi$ around the minimum to first order in the fields yields
\begin{equation}
  \phi(x) = e^{i\frac{\phi(x)}{v}}\frac{1}{\sqrt{2}} \myvec{ 0 \\ v + h(x) },
    \label{eq:higgsexp}
\end{equation}
where the extra term $e^{i\frac{\phi(x)}{v}}$ describes the fluctuations of the fields $\phi_1, \phi_2, \phi_4$ around the vacuum $\phi_0$.
These fields are known as the \emph{Goldstone bosons} and have no direct physical implications. They can be eliminated from the Lagrangian by choosing an appropriate gauge, the \emph{unitary gauge}, so that only one real, physical field, $h(x)$, remains:
\begin{equation}
  \phi(x) = \frac{1}{\sqrt{2}} \myvec{ 0 \\ v + h(x) }.
  \label{eq:expanded-groundstate}
\end{equation}
This field is known as the \emph{Higgs field} and associated to a neutral scalar boson, the \emph{Higgs boson}, $H$, with a mass of 
\begin{equation}
  m_H = \sqrt{2} \mu = \sqrt{2 \lambda} v.
\end{equation}
Inserting \cref{eq:higgsexp} into \cref{eq:lagrangianhiggs} and using the following linear combinations to obtain the physical gauge fields,
\begin{align}
  W_\mu^\pm &= \frac{1}{\sqrt{2}} \left( W_\mu^1 \mp iW_\mu^2 \right) \\
  Z_\mu &= \cos\theta_\text{w} W_\mu^3 - \sin\theta_\text{w} B_\mu \\
  A_\mu &= \cos\theta_\text{w} B_\mu - \sin\theta_\text{w} W_\mu^3,
\end{align}
where $\theta_\text{w}$ is the so-called \emph{weak mixing angle} defined as $\sin\theta_\text{w}^2 = \frac{g'^2}{g^2+g'^2}$, results in the following expression in the Lagrangian
\begin{equation}
  \mathcal{L}_m = \frac{1}{4} \left( v + H \right)^2  \left(g^2 W_\mu^+W^{-\,\mu} + \frac{g^2}{2\cos\theta_\text{w}^2} Z_\mu Z^\mu \right).
  \label{eq:lagrangianmasses}
\end{equation}
One can identify terms quadratic in $\Wpm$ and $Z$, generated by the non-vanishing expectation value, $v$, giving rise to the masses
\begin{align}
  m_W &= \frac{vg}{2}, \\
  m_Z &= \frac{m_W}{\cos \theta_\text{w}},
  \label{eq:boson-masses}
\end{align}
of the gauge bosons.
The mass of the $W$ boson can also be related to the Fermi constant, $G_F$, via $m_W = \frac{g}{4 * \sqrt{2}G_F}$. 
No field quadratic in $A_\mu$ appears, which reflects the fact that the photon is massless.

% DIFFERENT Formulation: It is worth while to decompose the Lagrangian into its different pieces:
The Lagrangian in \cref{eq:lagrangianmasses} can be re-written as follows after substituting the masses of the gauge bosons:
\begin{equation}
  \mathcal{L}_{\text{H,boson-coupling}} = m_W^2 W_\mu^+W^{-\,\mu} \left( \frac{2H}{v} + \frac{H^2}{v^2} \right) + \frac{1}{2} m_Z^2 Z_\mu Z^\mu \left( \frac{2H}{v} + \frac{H^2}{v^2} \right),
  \label{eq:higgsbosoncoupling}
\end{equation}
which shows that the interaction between the Higgs boson and the massive gauge bosons is proportional to the square of the mass of the coupled bosons and involves triplet ($V^\dagger VH$) and quartic ($V^\dagger VHH$) couplings.

%and choosing a specific gauge (the unitary gauge) the field $\phi$ can be written as

As shown below, introducing a complex SU(2)$_L$ doublet does not only give rise to the $\Wpm$ and $Z$ boson masses, but can also be used to construct mass terms for fermions. Hence, the Higgs field couples to all massive elementary particles.
A more detailed descriptions on experimental Higgs boson physics and an overview of the current state of knowledge is given in \cref{sec:higgsphysics}.

% The parameters of the Higgs potential $\mu^2 = \lambda v^2$ can be fixed for one combination by measuring parameters of the electroweak theory, but the physical Higgs mass cannot be predicted. 
% Moreover, the last two terms in \cref{eq:higgsselfcoupling} predict that the Higgs field couples to itself with cubic and quartic interactions.
% The couplings of the Higgs boson to the weak gauge bosons are already expressed in \cref{eq:lagrangianmasses}.

%While the Higgs boson was discovered experimentally in 2012 \cite{Aad:2012tfa,Chatrchyan:2012xdj}, and is by now seen in several different production and decay modes, the Higgs boson self-couplings are still searched for.


% In summary, adding a complex scalar Higgs field to the electroweak model, together with a potential that exhibits a non-zero vacuum expectation value, provides the ingredients to obtain mass terms for the weak gauge bosons $W^\pm$ and $Z$ when choosing a ground state of the Higgs field which spontaneously breaks the symmetry.
% The three Goldstone bosons that arise can be eliminated from the Lagrangian by using its underlying local gauge symmetry. 
% This results in three bosons acquiring a mass and the appearance of one remaining scalar field $H$. 
% The photon remains massless, because the U(1)$_{\text{QED}}$ symmetry is unbroken. 


\subsection{Fermion masses}
\label{subsec:fermion-masses}
The masses of the fermions can be explained by introducing interaction terms between the fermion fields and the Higgs field. 
This is possible because both the left-handed fermions and the Higgs field appear as doublets under SU(2)$_L$ transformations.

For leptons, only electrons, muons, taus, that appear in the lower part of the SU(2)$_L$ doublet (see \cref{tab:ewfermioncontent}), need to become massive, as neutrinos are assumed to be massless in the SM.
For up-type quarks to become massive, the fermion fields are coupled to the charge conjugate of the Higgs field that becomes
\begin{equation}
  \phi^C = \frac{1}{\sqrt{2}} \myvec{v + H \\ 0},
\end{equation}
when choosing a ground state similar to \cref{eq:expanded-groundstate}.

The interactions, that are known as \emph{Yukawa interactions}, can then be written in the form
\begin{equation}
  \mathcal{L}_\text{Yukawa} = - \sum_{i} Y_l^i \bar{\psi}^{i}_{L} \phi \psi^{i}_{R} - \sum_{ij} Y_{\text{u-type}}^{ij} \bar{\psi}^{i}_{L} \phi \psi^{i}_{\text{u-type},R} + Y_{\text{d-type}}^{ij} \bar{\psi}^{i}_{L} \phi^C \psi^{j}_{\text{d-type}, R} + \text{h.c.}, 
  \label{eq:lyukawa}
\end{equation}
where the first sum runs over all leptons, the second sum over all quarks, and h.c. stands for the Hermitian conjugate of the previous terms.
The left-handed fields are given as $\psi^{i}_{L}$ and include the three lepton- and quark doublets. 
The right-handed lepton fields are labelled as $\psi_R^i$ and the quark fields $\psi^{i}_{\text{u-type},L}$, $\psi^{i}_{\text{d-type},L}$.
The labels ``u-type'' and ``d-type'' refer to, respectively, up-type quarks, $u$, $s$, $t$, and down-type quarks, $d$, $c$, $b$. 
The so-called \emph{Yukawa couplings} are shown as $Y_l^i$ for leptons and $Y_{\text{u-type}}^{ij}$ and $Y_{\text{d-type}}^{ij}$ for up-type and down-type quarks, respectively. The $Y_{\text{u-type}}^{ij}$ and $Y_{\text{d-type}}^{ij}$ are $3 \times 3$ matrices accounting for the fact that the eigenstates of the weakly interacting quarks do not correspond to their mass eigenstates. 
%A rotation of the quark fields can be performed accounting for the mixing. 

% Mass terms for quarks can be added in a similar but slightly more involved way. The down-type quarks participating in the electroweak interaction (denoted $d'$, $s'$, $b'$) are a mixture of the quark mass eigenstates (denoted $d$, $s$, $b$). 
% For each quark doublet there are two mass terms generated. For lepton doublets only one mass term is added because right-handed neutrinos are not part of the SM. Assuming only one generation of fermions, with quarks $u$ and $d$ and leptons $e$ and $\nu_e$, the Yukawa term can therefore be written as
% \begin{equation}
%   \mathcal{L}_\text{Yukawa} = - Y_{ud} \bar{\psi}_{ud,L} \phi \psi_{u,R} - Y_{ud} \bar{\psi}_{ud,L} \phi \psi_{d,R} - Y_l \bar{\psi}_{e\nu_e,L} \phi \psi_{e,R} + \text{h.c.},
%   \label{eq:lyukawa}
% \end{equation}
% where $\psi_{l}$ stands for lepton fields and $\psi_{q}$ for the quark fields.

The mass terms appear from \cref{eq:lyukawa} once a ground state of $\phi$ is chosen and the Yukawa terms have been diagonalized resulting in nine independent parameters, $Y_f$. 
They take the form:
\begin{equation}
  m_f = \frac{Y_f v}{\sqrt{2}},
\end{equation}
which means that the coupling strength between the fermions and the Higgs field is proportional to the mass of the fermions.

%\subsection{Renormalization}
% From modern particle physics book
% As shown by ‘t Hooft, only theories with local gauge invariance are renormalisable, such that the cancellation of all infinities takes place among only a finite number of interaction terms.

\subsection{The Final Standard Model Lagrangian}
\label{subsec:final-lagrangian}
To summarize the previous sections, the final Standard Model Lagrangian is obtained from \cref{eq:lqcd,eq:lagrangianewk,eq:lagrangianhiggs,eq:lyukawa} and can be written as
\begin{align}
  \mathcal{L}_\text{SM} &= \mathcal{L}_\text{QCD} + \mathcal{L}_\text{EWK} + \mathcal{L}_\text{Yukawa} + \mathcal{L}_\text{Higgs} \\
   &= - \frac{1}{4}W_{\mu\nu}^aW^{\mu\nu}_{a} - \frac{1}{4} B_{\mu\nu}B^{\mu\nu} - \frac{1}{4}G_{\mu\nu}^aG^{\mu\nu}_{a} \\
   & \quad + \sum_f i \bar{\psi}_f\gamma^\mu D_\mu\psi_f \\
   & \quad - \sum_{i} Y_l^i \bar{\psi}^{i}_{L} \phi \psi^{i}_{R} - \sum_{ij} Y_{\text{u-type}}^{ij} \bar{\psi}^{i}_{L} \phi \psi^{i}_{\text{u-type},R} + Y_{\text{d-type}}^{ij} \bar{\psi}^{i}_{L} \phi^C \psi^{j}_{\text{d-type}, R} + \text{h.c.},  \\
   & \quad + |D_\mu\phi|^2 - \mu^2\phi^\dagger\phi + \lambda \left(\phi^\dagger\phi \right)^2
\end{align}
The covariant derivate includes all gauge fields,
\begin{equation}
  D_\mu = \partial_\mu + i g_s \frac{\lambda^a}{2} G_\mu^a + igT^aW_\mu^a + ig'\frac{Y}{2}B_\mu.
\end{equation}
The sum over $f$ in the third term includes all fermions, left-handed and right-handed.

% It satisfies a SU(3)$_C$ $\times$ SU(2)$_\text{L}$ $\times$ U(1)$_Y$ gauge symmetry and provides masses to the gauge bosons and fermions via the concept of electroweak symmetry breaking. This is manifested in the couplings of the Higgs field to fermions, which are proportional to the mass of the fermions, and the couplings to the gauge bosons, which are proportional to the mass squared of the gauge bosons.


\subsection{Free parameters of the SM}
\label{subsec:free-pars-sm}
The SM has been very successful in making predictions that were later confirmed by experimental data (see \cref{eq:sm-limitations}). However, the SM has many free parameters that need to be measured experimentally and cannot be derived from theoretical principles.  
In total there are 19 free parameters that can be represented with different parametrizations:
\begin{itemize}
  \item 9 fermion masses (or Yukawa couplings)
  \item 2 parameters describing the Higgs field: $\mu$ and $\lambda$ (or $v$ and $m_H$)
  \item 4 parameters to fully specify the CKM matrix, typically parametrized as 3 quark-mixing angles ($\theta_1$, $\theta_2$, $\theta_3$) and a CP-violating phase ($\theta_\delta$).
  \item 3 couplings constants: $\alpha$, $G_f$, $\alpha_s$ (or $g'$, $g_w$, $g_s$)
  \item 1 phase associated to CP violating terms in QCD, $\theta_{\text{QCD}}$
\end{itemize}
In total 14 parameters are associated with the Higgs field, four with the flavour sector, and only three with the gauge interactions. This underlines the special role the Higgs boson plays in the SM.


\subsection{Limitations of the SM}
\label{subsec:limitations}
Despite the extraordinary success of accurately predicting and explaining experimental measurements, the SM has several shortcomings.
There are several observable phenomona that cannot be explained by the SM as well as many open questions related to theory. 
There are models and theories beyond the SM that attempt to address these shortcomings, but experimental indication that either of them is realised in nature is still pending.
%none of them have made predictions that have been confirmed by experiments to date. 
Notable examples of such theories are supersymmetry, GUT, or more simple extensions of the SM such as the 2HDM \todo{REFs}. 
This section provides an overview of some of the biggest unsolved problems of the SM and fundamental physics in general.


% From Arnold:
% While several such theories have been developed that based on differing concepts attempt to improve on the shortcomings of the Standard Model, e.g. Supersymmetry or String Theory, experimental indications that either of them is actually realised in Nature are still pending.

%%%%%%%%%%%%%%%%%%%%%%%%%%%%%%%%%%%%%%%%%%%%%%%%%%%%%%%%%%%%%%
% Experimental ones (not sure yet if I should make this distinction)

\subsubsection{Dark matter}
The SM is able to explain the behavior of the currently observable matter in the universe, which makes up about 5\% of the energy of the universe. 
As known from astrophysics, such as from measurements of galaxy rotations \todo{REF}, another 27\% of the universe's energy is made up of so-called \emph{dark matter}. To date, the SM has no viable candidate to explain the nature of dark matter. 

% Another 26\% of the universe's energy is made up of so-called dark matter. 
% - The SM explains the observable mass which makes up about 5\% of the energy of the universe. 
% - The rest of the matter, making up about 26\% is known as dark matter. (known from measurements of orbiting galaxis)
% - The SM cannot explain the nature of dark matter, to date. 
% from Thomson
%velocity distributions of stars as they orbit the galactic centre)

\subsubsection{Dark energy}
About 68\% of the universe's energy constitutes so-called \emph{dark energy}. Dark energy is attributed to the non-zero cosmological constant that appears in Einstein's equation of general relativity and is also sometimes referred to as \emph{vacuum energy}. It is used to explain the acceleration of the expansion of the universe \todo{REF}. The SM provides no suitable explanation for the large value of the vacuum energy. 

%The SM has no explanation for the large value of the vacuum energy. 
% - Dark energy is the energy needed in order to explain the acceleration of the expansion of the universe. 
% - This is sometimes called vacuum energy and related to the cosmological constant problem. The SM has no explanation for the measured size of the vacuum energy.
% From Thomson
%- dark energy is attributed to a non-zero cosmological constant of Einstein's equations of general relativity

\subsubsection{Neutrinos}
Right-handed neutrinos are not part of the SM because they do not couple to any of the other particles.
Therefore, Dirac mass terms from Yukawa interactions (see \cref{subsec:Yukawa}) are typically not added to the SM. 
However, it is known that neutrinos oscillate\todo{REF}, which implies that they have finite masses.
While it is possible to add mass terms for neutrinos ad-hoc, without violating any of the theoretical principles, it is not clear exactly which terms to use, as the nature of the neutrinos is not settled\footnote{Neutrinos could for example either be Dirac fermions or Majorana fermions. Majorana fermions are particles that are their own anti-particles, which stands in contrast to Dirac fermions. Different mathematical descriptions are required to describe the different types of particles in the Lagrangian.}. 
Nevertheless, the SM is sometimes described with additional terms for the neutrinos, which adds an additional seven free parameters (three for the masses and four to describe the mixing between neutrinos).

%It is possible to add Dirac mass terms by implying the existence of right-handed neutrinos. 
%- One possibility is to add Dirac mass terms, which would add an additional 6 free parameters to the SM. 
%- However, it is not clear if Neutrinos are indeed sterile (Dirac) and unknown why Neutrino masses are so small and if they get their masses from the same process as the other SM particles
% Wikipedia on Majorana:
% A Majorana fermion (/maɪəˈrɑːnə ˈfɛərmiːɒn/[1]), also referred to as a Majorana particle, is a fermion that is its own antiparticle. They were hypothesised by Ettore Majorana in 1937. The term is sometimes used in opposition to a Dirac fermion, which describes fermions that are not their own antiparticles.
% The nature of the neutrinos is not settled—they may turn out to be either Dirac or Majorana fermions.
% From Pich 2012
%The experimental evidence of neutrino oscillations shows that νe, νμ and ντ are also mixtures of mass eigenstates. However, the neu

\subsubsection{Matter-antimatter symmetry}
It is widely believed that the universe was created with equal amounts of matter and antimatter.
However, the currently observable universe consists of significantly more matter than antimatter. 
The SM does not provide a theoretical explanation for this large asymmetry. 

% the SM cannot explain why there is more matter than antimatter in the observable universe. 
% - No mechanism in the SM can explain the asymmetry sufficiently. 
% From Thomson
% However, even if CP violation is observed in neutrino oscillations, it seems quite possible that the CP violation in the Standard Model is insufficient to explain the observed matter–antimatter asymmetry of the Universe.

%%%%%%%%%%%%%%%%%%%%%%%%%%%%%%%%%%%%%%%%%%%%%%%%%%%%%%%%%%%%%%
% Theoretical ones (not sure yet if I should make this distinction)

\subsubsection{Large number of free parameters}
As discussed in \cref{subsec:free-pars-sm}, the SM has several free parameters that need to be added to the theory ad hoc and cannot be derived from first principles. 
From an experimentalist point of view, this is not a problem per se, but it suggests that there might be a more fundamental theory with fewer parameters that can ultimately explain the relationships between the current SM parameters.

% Thomson: 
% [The SM] ... is a model constructed from a number of beauti- ful and profound theoretical ideas put together in a somewhat ad hoc fashion in order to reproduce the experimental data.

\subsubsection{Gravitation}
The current understanding of gravity is still shaped by Einstein's theory of general relativity. No quantum theory of gravity has been fully developed, which is why it is not part of the SM. 
The interaction strength of gravity is very weak compared to all other fundamental forces, which is why it can be safely neglected in current high energy physics experiment.
% Due to the small strength of gravity, compared to the other fundamental forces, gravity can be fully neglected in current high energy physics experiments. 
% - Quantum theory of gravity not fully developed. Versions that exist predict the graviton, the particle mediating gravity to be a spin 2 particle.
% - Classical theory of gravity, general relativity

\subsubsection{Strong CP problem}
No theoretical principle forbids CP violation in the strong interaction. However, no such process has been found and the strong CP phase, $\theta_{\text{QCD}}$, has been experimentally constraint to be smaller than $10^{-10}$ \todo{REF}. This is considered a \emph{fine-tuning} problem because it is highly unlikely that the reason for such a small interaction is due to chance alone. 
% Not sure if I should add that
The so-called Peccei-Quinn theory~\cite{PhysRevLett.38.1440} and modifications of it provide a solution by introducing a scalar field known as axion. However, no experimental evidence of the existence of such a particle has been found to date.
%there is no explanation of why these interactions are so small which makes
% - There is nothing that would forbid CP violation in the strong interaction, however, no CP-violating processes have been observed to date. 
% This is considered a ``fine-tuning'' problem.


% Not sure if I should include this it's kind of weird! I don't understand it, let's say that. Why is it a problem, can't you ALWAYS ask that question? Why is there this particle. Why are there electrons? Not sure if I get that quite right. 
% \subsubsection{The fermion generations}
% The SM has no explanation for why there are three generations of fermions. 


\subsubsection{Higgs boson}
While several physics problems were solved by introducing the Higgs boson to the SM, other mysteries remain. 
One of the biggest open questions is related to the mass of the Higgs boson itself. 
Assuming that there is no new physics up to the Planck scale, quantum loop corrections to the Higgs boson mass would be extremely large (at the order of the Planck scale). 
This means, that the bare mass of the Higgs boson\footnote{The observable mass of an elementary particle is given by the difference of two terms that are fully independent of each other, the particles bare mass and corrections to it from quantum fluctuations.}
needs to almost perfectly balance the quantum corrections to keep the effective Higgs mass in the \GeV\ range. The precision required for such a balance is extremely high and considered ``unnatural''. 
This is known as the hierarchy problem and is one of the main motivations for the search for new particles with masses in the \TeV\ range at the LHC. Their existence would affect the quantum contributions to the Higgs mass, eliminating the need of ``fine-tuning'' the bare Higgs mass to an incredible precision, as is assumed so far in the SM. 

The Higgs boson plays a special role in the SM because it is the only fundamental scalar. Its nature makes it a suitable candidate in many models beyond the SM to find new physics through couplings with the Higgs boson. 
This highly motivates performing precision measurements of the Higgs boson to be able to detect small deviations from the SM predictions. 
For an extensive review of different theoretical models, the interested reader is referred to \ccite{2019BHeinemann}.


% - Due to these inexplicabilities the Higgs boson (and the fact that it's the least well measured EM particle) is widely considered to be the answer to many other questions in fundamental physics

% Hierarchy Problem from Thomson:
% Just as quantum loop corrections contributed to the W-boson mass (see Section 16.4), quantum loops in the Higgs boson propaga- tor, such as those indicated in Figure 18.3, contribute to the Higgs boson mass. This in itself is not a problem. However, if the Standard Model is part of theory that is valid up to very high mass scales, such as that of a Grand Unified Theory ΛGUT ∼ 1016 GeV or the Planck scale ΛP ∼ 1019 GeV, these corrections become very large. Because of these quantum corrections, which are quadratic in Λ, it is difficult to keep the Higgs mass at the electroweak scale of 102 GeV. This is known as the Hierarchy problem. It can be solved by fine-tuning the new contributions to the Higgs mass such that they tend to cancel to a high degree of precision.

% Why Higgs measurements?
% The experimental study of the Higgs boson at the LHC is undoubtedly one of the most exciting areas in contemporary particle physics. Within the Standard Model, the Higgs boson is unique; it is the only fundamental scalar in the theory. Establish- ing the properties of the Higgs boson such as its spin, parity and branching ratios is essential to understand whether the observed particle is the Standard Model Higgs boson or something more exotic.