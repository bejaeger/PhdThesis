The theoretical framework described in \cref{sec:sm} allows making predictions about the interactions between the elementary particles. 
These predictions can be tested by studying the outcomes of particle collisions produced by accelerators such as the LHC. 
The LHC accelerates proton beams, which allows reaching very high center-of-mass energies for the collision events. 
The collision events have a complex phenomenology, since protons are not elementary particles.
% The phenomenology of proton-proton collisions ($pp$ collisions) is complex because protons are not elementary particles. 
Protons are compound states of so-called \emph{partons} made up of quarks and gluons that interact with each other according to the laws of QCD.
A correct understanding of QCD processes is therefore essential for the success of physics programs at $pp$ collider experiments.

In this section, the relevant aspects are described to move from QCD calculations to observables that can be measured in collider experiments such as the ATLAS detector.
\Cref{subsec:pp-collision-overview} provides a brief overview of the anatomy of a $pp$ collision event and introduces the key terminology. \Cref{subsec:collision-rates} summarizes basic observables, and \cref{subsec:factorisation} discusses the concepts necessary to compute theoretical particle interaction cross sections.
The theoretical predictions for the observables are quantified using simulated $pp$ collision events to compare them with the experimentally measured data. Details of the ATLAS simulation infrastructure are given in \cref{subsec:event-simulation}. 
% The $pp$ collisions are quantum mechanical by nature and therefore intrinsically probabilistic.
% Therefore, simulated $pp$ collision events generated with the \emph{Monte Carlo} (MC) method are used to compare the data against what is expected given the current knowledge of particle physics.


% The SM expectation of the observables is quantified using simluated $pp$ collision events 
% Later in this section, the procedure to simulate $pp$ collision events is described, 
% including the relevant details of QCD
% This section provides an overview of how to get from QCD calculations to physical observables that can be measured at collider experiments such as the ATLAS detector.

% The necessary computations are complex and affected by many aspects, and are continuously being improved by the theoretical particle physics community.
% This section provides an overview of the physical observables that are defined at hadron colliders, and gives details on the characteristics that need to be considered when simulating the collision events. 

% What is that sentence???
% As we will see we can separate the process of calculating cross sections between quarks and other soft QCD things.

\subsection{Overview of a proton-proton collision event}
\label{subsec:pp-collision-overview}
%A schematic overview of a \emph{hard} $pp$ scattering process is illustrated in \cref{fig:ppcol}.
As illustrated in \cref{fig:ppcol}, a hard $pp$ scattering process is characterized by a parton-parton interaction with large momentum transfer, known as a \emph{hard scatter}, and further low energy processes.
The partons participating in the hard scatter carry a momentum fraction of the proton described by \emph{parton distribution functions} (PDFs).
The hard scatter is accompanied by the so-called \emph{underlying event}, which consists of low energy interactions between the proton remnants that are not taking part in the hard scatter.
The partons in the initial and final states can radiate in the form of \emph{inital-state radiation} (ISR) and \emph{final-state radiation} (FSR), and a single parton can also split into two. Together this is referred to as \emph{parton showering}.
The final-state partons then undergo a process known as \emph{fragmentation} (also called \emph{hadronization}), due to the confining nature of QCD. They end up forming a spray of color-neutral hadrons, which are known as particle \emph{jets}. Many more details on the above-mentioned aspects can be found in standard textbooks on collider physics such as \ccite{Ellis:1991qj}. 

\begin{figure}[t]
  \newImageResizeCustom{0.8}{figures/anatomy/schematic_ppcollision.png}
  \caption[Schematic view of a proton-proton collision.]{Schematic view of a proton-proton collision. Details can be found in the text. Taken from Ref.~\cite{Bhatti2010}. }
  \label{fig:ppcol}
\end{figure}


\subsection{Collision rates}
\label{subsec:collision-rates}
%The whole physics program at collider experiments can more or less be broken down into
The main task of collider experiments is the measurement of event rates. 
%Comparing the event rates measured in data with the predictions from simulations allow for meaningful statistical inferences.
For a given process $p$, the event rate can be written as
\begin{equation}
  \frac{\mathrm{d}N}{\mathrm{d}t} = \sigma_p \mathcal{L},
\end{equation}
where $\sigma_p$ is the cross section for process $p$, and $\mathcal{L}$ the luminosity.
The cross section expresses the likelihood of any particular interaction to occur and can be written as
% PEskin/Schroeder
%The likelihood of any particular final state can be expressed in terms of the cross section, a quantity that is intrinsic to the colliding particles and therefore% allows comparison of two dierent experiments with dierent b eam sizes and intensities
\begin{equation}
  \label{eq:xsec}
  \sigma = \frac{|M|^2}{F} \int \text{d}Q,
\end{equation}
where $M$ is the so-called matrix element, d$Q$ the Lorentz-invariant phase-space factor, and $F$ the Lorentz-invariant flux factor \cite{Halzen:1984mc}. 
The matrix element contains all the dynamical information of the process under investigation, and the square of it represents the probability of going from a certain initial state to a final state. 
The luminosity, $\mathcal{L}$, is a measure of the number of particles per unit area and time and is a crucial performance parameter for particle colliders. It only depends on parameters of the accelerator and is given by
%is the proportionality constant that multiplies the cross section to determine the \emph{event rate} and 
\begin{equation}
  \mathcal{L} = f_rn_b\frac{N_A N_B}{A},
\end{equation}
where $N_A$ and $N_B$ are the numbers of particles in the colliding bunches $A$ and $B$ respectively, $f_r$ is the rotational frequency of the two bunches, $n_b$ the total number of bunches inside the accelerator, and $A$ the area of interaction. Assuming Gaussian shaped beam profiles the area of interaction can be written as $A = 4\pi \sigma_x \sigma_y$, where $\sigma_{x/y}$ are the horizontal and vertical beam widths.
The \emph{integrated luminosity}, 
\begin{equation}
  L_\text{int} = \int \mathcal{L} dt,
\end{equation}
is typically quoted to quantify the size of a dataset collected at a collider experiment.


\subsection{Partonic cross sections and parton distribution functions}
\label{subsec:factorisation}

% The incoming partons interact according to the laws of QCD.
An important feature of QCD is the energy dependence of the strong coupling constant, $\alpha_s$, as shown in \cref{fig:alphas}.\footnote{The $\alpha_s$ constant is also sometimes referred to as ``running'' coupling constant as it is, in fact, not a constant.} Especially the behavior at low energies where $\alpha_s$ diverges has important consequences, as it requires to separate the treatment of physics at low energies and high energies. 
At high energies, perturbative calculations of the matrix elements can be used because $\alpha_s$ becomes small. This behavior is also known as \emph{asymptotic freedom} and implies that strongly interacting particles behave like free particles at high energies. 
Conversely, the physics at low energies (known as \emph{soft QCD}) can only be described by non-perturbative models because the particles have a strong coupling. This leads to confinement of quarks inside hadrons.

\begin{figure}[t]
  \newImageResizeCustom{0.60}{figures/anatomy/alphas.png}
  \caption[Summary of measurements of the strong coupling constant $\alpha_s$.]{Summary of measurements of the strong coupling constant $\alpha_s$ as a function of the energy scale $Q$. Taken from \ccite{PDG2020}. }
  \label{fig:alphas}
\end{figure}
%\footnote{The fact that $\alpha_s$ becomes smaller at high energies is also known as \emph{asymptotic freedom} and implies that strongly interacting particles behave like free particles at high energies. Conversely, at small energies, the particles have a strong coupling. This characteristic of QCD leads to \emph{confinement} of quarks inside hadrons.}

Given these considerations, as well as the fact that protons are compound states of partons, the cross section of a hard-scattering process between two protons, $p_1$ and $p_2$, can be written as~\cite{Ellis:1991qj}
\begin{equation}
  \sigma(p_1p_2 \to Y) = \sum_{i,j} \int_0^1 \mathrm{d}x_1 \int_0^1 \mathrm{d}x_2 f_i(x_1,\mu_F^2) f_j(x_2,\mu_F^2) \hat{\sigma}_{ij \rightarrow Y}(x_1p_1,x_2p_2,\mu_F,\mu_R), 
  \label{eq:hhxsec}
\end{equation}
where $f_i(x,\mu_F^2)$ are the PDFs, and $\hat{\sigma}_{ij \rightarrow Y}$ is the partonic cross section for going from two initial partons $i$ and $j$ to the final state $Y$.
The parameter $\mu_F$, known as the \emph{factorization scale}, marks the boundary between the low energy processes and the perturbative regime. 
All processes with an energy below the value of $\mu_F$ are considered as part of the proton structure and are accounted for within the PDFs. 
This separation between the non-perturbative soft QCD processes and the hard partonic scattering cross section is known as \emph{factorisation theorem}.
% Hence, these non-perturbative soft processes are separated from the hard partonic scattering crosssection, a method that is also known as the \emph{factorization theorem}.
The sum in \cref{eq:hhxsec} runs over all possible combinations of initial state partons $i$ and $j$ inside the protons, and the integral goes over the momentum fractions $x_1$ and $x_2$ of the total proton momenta.

The PDFs are non-perturbative and quantify the probability of observing a parton $i$ in the proton with a momentum fraction $x_1$ of the total momentum of the proton.
They are typically evaluated at the factorization scale $\mu_F$. 
Currently, they cannot be predicted to the required precision from theoretical principles and must therefore be extracted from experimental data collected at dedicated scattering experiments.
Example PDFs are shown in \cref{fig:pdfs} for two different energy scales.
The so-called \emph{DGLAP (Dokshitzer-Gribov-Lipatov-Altarelli-Parisi) evolution equations} \cite{Dokshitzer:1977sg,GRIBOV197178,Altarelli:1977zs} allow transferring the PDFs between the different energy scales, which makes them universally applicable to any process. 
% Different physics groups provide computations of PDF sets; one of the most recent, the MMHT2014 PDF set, can be seen in \cref{fig:pdfs}. \todo{Update pdf set and plot}
%\Cref{fig:pdfs} shows the most recent PDFs from the MMHT2014 PDF set.
%There are different gorups, which provide PDF computations.
%\footnote{There are different groups, which provide PDF calculations. They are discussed in the most recent LHC Run 2 PDF recommendations \cite{Butterworth:2015oua}.}.
%which include a combination of the CT14 \cite{Dulat:2015mca}, MMHT2014 \cite{Harland-Lang:2014zoa} and NNPDF3.0 \cite{Ball:2014uwa} PDF sets.
%($\mathcal{O}(\alpha_s)$) Order of in mathematic form

\begin{figure}
  \newImageResizeCustom{0.95}{figures/anatomy/nnpdf.png}
  \caption[The NNPDF3.1 NNLO parton distribution functions at two different energy scales.]{The NNPDF3.1 NNLO parton distribution functions at two different energy scales, $\mu^2$, and associated 68\% confidence-level uncertainty bands. Taken from Ref.~\cite{2017NNPDF}.}
  \label{fig:pdfs}
\end{figure}

%The partonic cross-sections can be computed using Feynman rules.
%The determination of the matrix element (ME) is done with so called \emph{Feynman rules}, which 
The partonic cross section, $\hat{\sigma}_{ij \rightarrow Y}$, depends on the factorization scale, $\mu_F$, as well as another scale called \emph{renormalization scale}, $\mu_R$.
%Choosing its value sufficiently large, allows calculating the partonic cross section as a perturbation series in the running  coupling $\alpha_s$,
When $\mu_F$ is chosen sufficiently large, the partonic cross sections can be calculated as a perturbation series of the strong coupling $\alpha_s$~\cite{Ellis:1991qj},
\begin{equation}
  \hat{\sigma}_{ij \rightarrow Y} = \alpha^k_S \sum_{n=0}^{m} c^{(n)}\alpha_s^n,
  \label{eq:alphaexp}
\end{equation}
where the coefficients $c^{(n)}$ are functions of both the kinematic variables, $x_1$ and $x_2$, as well as the two scales, $\mu_F$ and $\mu_R$. The cross sections are calculated to fixed order by including all relevant combinations of incoming partons, $i$ and $j$, that may lead to the final state $Y$. Higher order terms are suppressed by higher orders of $\alpha_s$ and can often be neglected.
If the highest order term is linear, $m=0$, the calculation is called \emph{leading order} (LO). If $m=1$ ($m=2$), the calculation is called \emph{next-to-leading order} (NLO) (\emph{next-to-next-to-leading order}, NNLO), and so forth. 
The leading power $k$ in \cref{eq:alphaexp} is determined by the process under consideration. Most processes relevant to this thesis start contributing at $k=0$, while some sub-processes contribute with $k=2$.\footnote{For all processes of interest, cross sections are available at NLO, some of them at NNLO or even higher order. See chapter \cref{sec:data-mc-samples} for details.}
%For the Higgs gluon-fusion cross section there exist NNNLO calculations \cite{Anastasiou:2015ema}.}
The different processes that contribute to the cross sections can be depicted graphically with so-called \emph{Feynman diagrams} (also called \emph{Feynman graphs}). 
With the corresponding set of rules, known as \emph{Feynman rules}, the Feynman diagrams provide a graphical representation of the mathematical expressions found in the Lagrangian and can be used to compute the matrix elements (see \cref{eq:xsec}). 
The procedure is detailed in various textbooks, for example in \ccite{Griffiths:111880}.
% A system of rules was invented by Richard Feynman (known as \emph{Feynman rules}) to compute the matrix elements from the Lagrangians. 
% It makes use of graphical representations of the mathematical expressions, known as \emph{Feynman graphs}. More information on the procedure to compute matrix elements can be found in \todo{Maybe move the whole Feynman graph thingy to the partonic cross section below}.

% From previous section!
% A system of rules was invented by Richard Feynman (known as \emph{Feynman rules}) to compute the matrix elements from the Lagrangians. It makes use of graphical representations of the mathematical expressions, known as \emph{Feynman graphs}. 



When higher order terms are included in the matrix-element calculations, quantum loop corrections and real gluon emissions introduce divergences in the cross sections. 
% When including higher orders in $\alpha_s$ in the matrix-element calculation, loop corrections and real gluon emissions introduce divergences in the cross section. 
%\cref{fig:loops} illustrates these contributions with exemplary Feynman diagrams for different orders in $\alpha_s$.
%($\mathcal{O}(\alpha_s)$ and higher)
%singularities in the cross-section.
The infinities can be absorbed in the SM by a procedure known as \emph{renormalization}, by making the coupling parameters dependent on the renormalization scale, $\mu_R$.\footnote{Theories that allow for such a treatment are known as renormalisable, which is an essential property of the SM.}
The renormalization scale acts as a cut-off energy scale above which the processes that introduce infinities are not contributing anymore. 
The essential property of renormalization is that the physics descriptions below that cut off are independent of the dynamics of the theory above it. 
The drawback is that another arbitrary scale is introduced to the theoretical calculations in addition to the factorization scale.
In general, however, the more orders are included in the cross-section calculations, the less strong is the dependence on $\mu_F$ and $\mu_R$. 
The remaining dependence at fixed order is typically included as a systematic uncertainty in physics measurements.
%There are remaining dependencies at fixed order. Resulting systematic uncertainties are most often quantified by varying the scales over some reasonable range and are incorporated in the analysis.
In practice, the values of $\mu_F$ and $\mu_R$ are often chosen to be identical, with a typical value near the scale of momentum transfer for the hard scattering process under consideration.
Various textbooks, for example \ccite{Peskin:1995ev}, provide detailed discussions on the details of the renormalization procedure.
% From Arnold:
% Thus, in order to deal with the UV and collinear divergences, two non-physical scales are introduced. The dependence of observables on the choice of these scales naturally decreases with increasing accuracy of the calculations; it eventually vanishes when all-orders of the perturbative series are considered. A common choice is μ2 F = μ2 R = Q2, where Q2 represents the hard scale of the process under consideration; e.g. Q2 = M2 for the production of a resonance with mass M, or Q2 = p2 T in the case of the pair-production of massless particles with transverse momentum pT.

%\todo{say more about collinear and soft divergencies? -> Ja wegen infrared safe später beim jet algorithm}
% There exist different approaches for choosing the scale $\mu_R$, at which the coupling $\alpha_s$ is evaluated, known as \emph{renormalization schemes}, but there is no prove of any of them being correct.

\Cref{fig:xsec} shows the $pp$ collision cross sections for various processes, displaying a huge range of cross-section measurements over many orders of magnitudes. 
Given a center-of-mass energy of 13\,\TeV\ and a luminosity of $10^{33}\mathrm{cm^{-1}s^{-1}}$, the production of Higgs bosons, for example, is highly suppressed ($10^{-2}\,\text{events}/s$) compared to the total cross section ($10^8\,\text{events}/s$).
This necessitates a precise knowledge of the experimental setup and measurement resolutions. An example of an energy resolution measurement is discussed in \cref{chap:calibration} of this thesis. 

% An important feature is that the cross sections are not dependent on the renormalization scale, as well as the factorization scale, when infinite orders in $\alpha_s$ would be included. 
% There are remaining dependencies at fixed order. Resulting systematic uncertainties are most often quantified by varying the scales over some reasonable range and are incorporated in the analysis.
%However, it is known that the theoretical error on a quantity, which is calculated to $\mathcal{O(\alpha_s^n}$, is always $\mathcal{O(\alpha_s^{n+1})}$. The remaining theoretical error can be quantified
%MMHT2014 NNLO PDFs

% Figure 03a:
% Summary of several Standard Model total and fiducial production cross-section measurements (a) with associated references (b) and (c). The measurements are corrected for branching fractions, compared to the corresponding theoretical expectations. In some cases, the fiducial selection is different between measurements in the same final state for different center-of-mass energies √s, resulting in lower cross section values at higher √s.


%%%%%%%%%%%%%%%%%%%%%%%%%%%%%%%%%%%%%%%%%%%%%%
% NEED SECTION ON:
% Generation of MC events
% Comment from BERND to "Detector simulation" section in experimental chapter:
% I believe this statement "Simulations of the ATLAS detector are required in order to generate full Monte Carlo events” needs to be more precise. Why do we have to simulate collisions in the first place? Probably some discussion about quantum mechanics and its probabilistic nature which makes its way all the way to how we analyze data
\subsection{Event simulation and phenomenology}
\label{subsec:event-simulation}
% From ML paper: https://arxiv.org/pdf/1807.02876.pdf
% Particle discovery relies on the ability to accurately compare the observed detector response data with expectations based on the hypotheses of the Standard Model or models of new physics. While the processes of subatomic
% particle interactions with matter are known, it is intractable to compute the detector response analytically. As
% a result, Monte Carlo simulation tools, such as GEANT [6], have been developed to simulate the propagation
% of particles in detectors to compare with the data. The dedicated CWP on detector simulation [7] discusses the
% challenges of simulations in great detail. This section focuses on the machine learning related aspects

% They are an essential ingredient for extracting physics parameters from the collision data, as described later in this section.

The $pp$ interactions and the interactions of the particles with the detector follow the rules of QFT and are therefore probabilistic by nature.
Simulated $pp$ collision events are therefore generated with the \emph{Monte Carlo} (MC) method to compare the data against what is expected from the SM.
%given the current knowledge of particle physics.
%  are therefore used 
% They are an integral part at hadron colliders 
A simulated event must take into account all aspects of a $pp$ collision, including the matrix element and the non-perturbative regime, in particular the modelling of the hadronization and showering process. In addition, the detector response and geometry needs to be simulated, as well as other nuisances at hadron colliders, such as the underlying event or multiple $pp$ collisions that simultaneously produce detector signals.
The following provides details on the different aspects the simulation needs to cover.

% - MC tools are typically utilized for many of these steps.
% - Parameters of MC are tuned to fit the data, especially for the non-perturbative regime
% - An overview is given in the following
\begin{figure}[t]
  \newImageResizeCustom{0.95}{figures/anatomy/xsecs.pdf}
  \caption[Summary of production cross-sections measurements for $pp$ collisions at different center-of-mass energy.]{
    Summary of production cross-section measurements for $pp$ collisions at different center-of-mass energies, $\sqrt{s}$. Taken from \ccite{ATL-PHYS-PUB-2022-009}.
    }
  \label{fig:xsec}
\end{figure}

% Dandoy
% A good MC simulation will cover all aspects of particle evolution detailed in Section 2.1, including matrix element calculation, hadronization and showering of final states, and simulation of detector geometry and response.
\paragraph{Matrix element}
Matrix element calculations are performed with MC generators that randomly sample from the SM expectations to generate hard scatter events. 
There are different MC generators that come with varying precision in perturbation theory. The generators used for the \HWW analysis are summarized in \cref{sec:data-mc-samples}.

\paragraph{Parton showering}
Parton showering occurs because there is a finite quantum probability for a single parton to split into two. 
The \emph{QCD splitting functions}~\cite{Altarelli:1977zs} provide the theoretical foundation to describe this process and are used in dedicated parton-shower algorithms.
The probability for emitting soft and collinear radiation diverges, respectively, for low energies and small angles. 
Physical observables are therefore chosen such that they are insensitive to these processes and a cut off is applied in the parton-shower algorithms to avoid modelling this radiation explicitly.
% The theoretical foundation for the parton-shower models is provided by the \emph{QCD splitting functions}~\cite{Altarelli:1977zs}.
%There is also the possibility of hard gluon being emitted in the initial or final state, which spoils the measurement of the total transverse momentum of the event.

\paragraph{Matching} 
While the matrix element calculations take into account hard radiation that can be calculated perturbatively, the parton-shower algorithms rely on perturbative approximations of soft and collinear radiation. 
Combining both regimes appropriately without double counting is challenging.
To achieve this, different matching methods that rely on certain assumptions are used in different MC simulations. An overview can be found, for example, in \ccite{QCDIntroSchool}.

\paragraph{Hadronization}
The partons hadronize when their energy decreases and the strong coupling becomes large. There are several phenomenological approaches for modelling this non-perturbative process, one of which is the \emph{Lund-String-Model}~\cite{Andersson:1983ia}. 
It is based on the fact that the force between two strongly interacting particles is constant, implying that the potential increases linearly with increasing distance.
At a certain distance between the partons the potential energy is sufficient to create a quark-antiquark pair out of the vacuum. This procedure is repeated until neutral-colored hadrons are formed.
The Lund-String-Model is used, for example, in the \Pythia event generator. Other generators such as \HERWING\ or \Sherpa make use of an alternative model for hadron formation called \emph{cluster model}. Details can be found in \ccite{Buckley_2011}. 
\paragraph{Decays}
The particles produced in the scattering process or in the process of hadronization are often unstable and further decay. Many of these decays, for example decaying $b$-hadrons, provide useful information about the event. %Dedicated algorithms exist to find the particles which initiate the decay chains.
\paragraph{Underlying event}
The hard scatter is accompanied by many soft processes from the remaining partons in the colliding protons. These interactions produce signals in the detector that overlay the hard scattering process and need to be accounted for in the simulation.
\paragraph{Pile-up}
A nuisance in particular at $pp$ colliders is the overlap of the signals of more than one scattering event in the detector. This is known as \emph{pile-up}. A distinction is drawn between \emph{in-time pile-up}, referring to the amount of multiple interactions per bunch crossing, and \emph{out-of-time pile-up}, referring to overlapping events from previous $pp$ collisions or following bunches due to the long response time of parts of the detector.
Pile-up collisions predominantly consist of inelastic, low energy transfer $pp$ collision events. 
Their impact is typically taken into account by simulating them separately and overlaying them with the generated hard scatter.
\paragraph{Detector simulation}
%\todo{Not clear where to provide details of detector simulation. Here or in the experimental chapter. It might be more meaningful in the hadron collider physics chapter. We could have a small reference to this here from the experiment chapter that might go into more detail about ATLAS. We can keep it general here.}
The exact detector geometry and the detector response to different particles must be simulated.
% using the 
The simulation must consider all detector inefficiencies and the material distribution in the detector, as well as physics properties such as the interaction lengths, the lifetime, and decay rates of the particles.

\paragraph{Truth information}
Information on the event such as the total momentum or the type of the interacting particles are accessible in simulated events but not in data. This extra layer of information, referred to as \emph{truth information}, is crucial for many physics analyses and calibration measurements.

% \subsection{Soft QCD: Minimum bias, pile-up, ...?}
% % Mention in Event Generation subsection, see Master thesis!

% \todo{Not clear if a dedicated chapter is needed here. My preference is NO at the moment! Details can be provided int he dedicated JER chapter to not blow up the theory part.}

% \Rinote{}{Where else should I talk about pile-up!? what is pile-up? HERE! Pile-up reweighting? analysis section. Pile-up as nuisance for jet measurements? -> JER calibration chapter, pile-up in LHC section as something intrinsic to pp collision event?}

% ->  Ruthmann has a nice section about it!

% - I should explain concepts like luminosity blocks  / bunch spacing and stuff in Data Taking Section
% - Then I can explain different pile-up conditions here.
% - This will be valuable to understand the noise term measurement which exactly tries to measure the noise term!
% - Also look back at discussion on skype with Brian about pile-up (actual mu vs average mu and so on)

% Checkout this section for pile-up overlay

% https://indico.cern.ch/event/1003305/contributions/4236702/attachments/2202625/3728039/PileUpTaskForcePandPPlenaryMarch2021.pdf

