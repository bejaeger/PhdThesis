Introduce impact parameter cause it's mentioned here \cref{subsec:inner-detector}.


% \subsection{Particle colliders}
% \Minote{}{Maybe put this part in a theory, more general section (proton-proton collisions?).}
% A crucial performance indicator of a particle accelerator is its \emph{instantaneous luminosity} $\mathcal{L}$ as it is proportional to the \emph{event rate}, 
% \begin{equation}
%     \frac{\mathrm{d}N}{\mathrm{d}t} = \sigma_{P} \mathcal{L},
% \end{equation}
% for a given physics process $P$ with cross section $\sigma_{P}$.
% The instantaneous luminosity depends on parameters of the accelerator and is given by
% \begin{equation}
%   \mathcal{L} = f_rn_b\frac{N_p^2}{4\pi \sigma_x \sigma_y}, 
% \end{equation}
% when assuming Gaussian shaped beam profiles. Here, $f_r$ denotes the rotational frequency of the two bunches, $n_b$ the total number of bunches inside the accelerator, and $N_p$ the numbers of particles within each colliding bunch. 
% The total number of events after time $T$ is given after integrating with respect to time, 
% \begin{equation}
%   N_{\text{Events}} = \sigma_{P} \intLumi = \sigma_{P} \int_{0}^{T} \mathcal{L} \mathrm{d}t,
% \end{equation}
% and is a crucial quantity typically measured in high-energy-physics experiments. 
% It becomes clear that increasing the integrated luminosity \intLumi is one of the main goals of any collider experiment, as it is directly proportional to the number of expected events of a particular physics process. 

% Lumi defined by Mike:
% The luminosity measures the number of particles per unit area and time, and together with the probability of interaction (cross-section) determines the collision rate.


\subsection{Pile-up}
% Mention in Event Generation subsection, see Master thesis!

% Comment from Bernd:
%Before you can talk about "hard-scatter vertex or from pile-up”, you may have to introduce them. E.g. have a section on Hadron collider physics that goes through these terms?

\Rinote{}{Where else should I talk about pile-up!? what is pile-up? HERE! Pile-up reweighting? analysis section. Pile-up as nuisance for jet measurements? -> JER calibration chapter, pile-up in LHC section as something intrinsic to pp collision event?}

->  Ruthmann has a nice section about it!
From Sommer: "Additional in- elastic, minimum-bias like pp collisions (pile-up) are generated using Pythia8 and overlaid."
Scope:
- I should explain concepts like luminosity blocks  / bunch spacing and stuff in Data Taking Section
- Then I can explain different pile-up conditions here.
- This will be valuable to understand the noise term measurement which exactly tries to measure the noise term!
- Also look back at discussion on skype with Brian about pile-up (actual mu vs average mu and so on)

Checkout this section for pile-up overlay

https://indico.cern.ch/event/1003305/contributions/4236702/attachments/2202625/3728039/PileUpTaskForcePandPPlenaryMarch2021.pdf


%%%%%%%%%%%%%%%%%%%%%%%%%%%%%%%%%%%%%%%%%%%%%%
% NEED SECTION ON:
% Generation of MC events
% Comment from BERND to "Detector simulation" section in experimental chapter::
% I believe this statemtn "Simulations of the ATLAS detector are required in order to generate full Monte Carlo events” needs to be more precise. Why do we have to simulate collisions in the first place? Probably some discussion about quantum mechanics and its probabilistic nature which makes its way all the way to how we analyze data
\subsection{Event Generation}


