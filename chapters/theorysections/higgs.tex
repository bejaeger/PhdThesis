
% FROM MASTER THESIS
% The theoretical developments which lead to the prediction of the Higgs boson where discussed in the previous chapters.
% The experimental discovery of the Higgs boson was achieved in 2012 by the ATLAS and CMS Collaboration at the LHC \cite{Aad:2012tfa,Chatrchyan:2012xdj}.
% Since then many dedicated analyses have been measuring the properties of the Higgs boson \cite{2015arXiv150605669A,DIMARCO2016746}, as well as its couplings to other particles very well \cite{2016JHEP...08..045A}.
% %The LHC \RunOne\ results for Higgs boson measurements are reviewed in Ref.~\cite{2016JHEP...08..045A}.
% Up to now, all results are consistent within the given uncertainties with the predictions made by the SM.
% The current combination of the mass measurements from the ATLAS and CMS experiment of the Higgs boson is \cite{Aad:2015zhl}
% \begin{equation}
%   m_H = 125.09 \pm 0.21\text{(stat.)} \pm 0.11\text{(syst.)} \GeV.
% \end{equation}
% The most recent result analyzing data of \RunTwo\ from the ATLAS Collaboration is $m_H = 124.97\pm0.28\,\GeV$ \cite{ATLAS-CONF-2017-046} and from the CMS Collaborations $m_H = 125.26 \pm 0.21\,\GeV$ \cite{Sirunyan:2272260}.
% This section shortly discusses the most important processes to consider, when dealing with Higgs boson interactions.

The theoretical developments which lead to the prediction of the Higgs boson where discussed in the previous chapters.

- There is no indication that the found particle is not the SM. The contrary, the experimental evidence is overwhelming that the measured particle is the particle predicted by the SM.
- Properties of the Higgs boson in agreement with the spin-parity JP = 0+ predicted by the SM. 
- Cross-sections all in agreement with the SM predictions. Firmly establishing VBF production in this thesis.

==>> Higgs production and decay modes
- Higgs boson physics provides a rich field of different production and decay modes and experimental signatures
- Summarized different production and decay modes of the Higgs BRIEFLY with nice plots showing size
% (focusing on the details of WW decays)
- Different prod and decay modes allow accessing Higgs properties from various angles.

==>> experimental signatures
- Discussion of experimental signatures
  -> similar to Master thesis

==>> Measurements
- Describe what cross section measurements are 
- differential xsecs
- STXS -> different categories -> for later combination

==>> Experimental status
- Listing the most recent measurements?!
- Give outline of current status of Higgs measurments, focusing on cross section measurements.
- A more comprehensive and complete overview is given in \cref{PDG2020}.

- Mention where this thesis fits in!

% Wikipedia
%Since the Higgs field is scalar, the Higgs boson has no spin. The Higgs boson is also its own antiparticle, is CP-even, and has zero electric and colour charge.[163]

% The Standard Model spin-parity JP = 0+


% From PDG
%For a given mH, the sensitivity of a channel depends on the production cross section of the Higgs
% boson, its decay branching fraction, the reconstructed mass resolution, the selection efficiency and
% the level of background in the final state. For a low-mass Higgs boson (110 GeV < mH < 150 GeV)
% for which the SM width would be only a few MeV, five decay channels play an important role
% at the LHC. In the H → γγ and H → ZZ∗ → 4` channels, all final state particles can be
% very precisely measured and the reconstructed mH resolution is excellent (typically 1-2%). While
% the H → W+W− → ` +ν`` 0−ν¯`
% 0 channel has relatively large branching fraction, however, due
% to the presence of neutrinos which are not reconstructed in the final state, the mH resolution,
% obtained through observables sensitive to the Higgs boson mass such as the transverse mass, is poor
% (approximately 20%). The H → b ¯b and the H → τ +τ
% − channels suffer from large backgrounds and
% lead to an intermediate mass resolution of about 10% and 15% respectively.


% PDG
% In order to optimise search sensitivity and also to separate the various Higgs
% boson production modes, ATLAS and CMS split events into several mutually exclusive categories

% For phrasing from PDG
% None of the other production modes have been firmly established by the experiments
% individually. However, the table shows that, for the VBF production mode, the combination had a
% large sensitivity and produced a combined observation of 5.4σ, therefore establishing this process
% with a rate compatible with that expected in the SM.



\subsection{Higgs Boson Production Modes}
\label{sec:higgsprod}
As illustrated in \cref{fig:higgsprodfeyn}, the Higgs boson is produced at the LHC via four main production modes: \emph{gluon fusion} (ggF), \emph{vector-boson fusion} (VBF), \emph{Higgs-strahlung} from $W$ or $Z$ boson (VH) and $t\bar{t}$ fusion (ttH).
%- as illustrated in \cref{fig:higgsprodfeyn} with a choice of Feynman diagrams.
%associated production with a $t\bar{t}$ pair (also called \emph{$t\bar{t}$ fusion})
\Cref{fig:higgsprodxsec} shows the most recent results for the corresponding cross-section calculations with an indication of the achieved accuracy for each process.

The ggF process ($pp\rightarrow H$) is the leading mechanism at the LHC and characterized by a fermion loop coupling to the Higgs boson. Top-quark loops have the largest contribution due to the high mass of the top quark. The ggF cross-section is calculated with an effective theory to NNNLO in QCD and includes electroweak corrections at NLO precision \cite{Anastasiou:2016cez}.

The second largest contribution to Higgs boson production comes from vector-boson fusion ($pp\rightarrow qqH$), with the characteristic of two quarks accompanying the Higgs boson in the final state.
This feature can be exploited in the event selection to isolate and measure the subsequent decays of the Higgs boson. Additionally, the VBF production provides information about the couplings of the Higgs boson to the $W$ and $Z$ bosons. Observation of the VBF production mode of the Higgs boson was found by a combined measurement of the ATLAS and CMS Collaborations\cite{Khachatryan:2016vau}.

Similar features exhibits the Higgs-strahlung process ($pp \rightarrow WH$, $pp \rightarrow ZH$), which has a final state topology with one additional vector boson besides the Higgs boson. 

The Higgs boson production in association with a $t\bar{t}$ pair ($pp \rightarrow ttH$) is difficult to measure due to the relatively low production cross-section and the less prominent final state. It plays an important role in Higgs boson physics, since it allows direct access to the Yukawa coupling of the top quark. The most recent results of the ATLAS Collaboration yield evidence for the $t\bar{t}$-fusion production of the Higgs boson \cite{ATLAS-CONF-2017-077}.

Other Higgs boson production processes in association with $b\bar{b}$ ($pp\rightarrow bbH$) or a single-top quark ($pp \rightarrow tH$) have smaller contribution at the LHC at centre-of-mass energies of 13\,\TeV.

\captionsetup[subfloat]{captionskip=25pt} % space between subfloat caption and image
\unitlength=1mm
\begin{figure}
  \centering
\subfloat[Gluon fusion]{  
\begin{fmffile}{HiggsProduction_GluonFusion}
\begin{fmfgraph*}(60,22) 
\fmfleft{i1,i2}
\fmfright{o1,o2,o3}
\fmf{phantom}{i1,v1,o1} 
\fmffreeze 
\fmf{phantom}{i2,v2,o3}
\fmffreeze
\fmf{gluon, label=$g$, label.dist=+12.0}{i1,v1} 
\fmf{gluon, label=$g$, label.dist=+12.0}{i2,v2}
\fmf{fermion, label=$t$, label.dist=-12.0}{v1,v2,v3,v1}
\fmf{dashes, label=$H$,plain}{v3,o2}
\end{fmfgraph*}
\end{fmffile}
}
\quad
\subfloat[Vector-boson fusion]{
\begin{fmffile}{HiggsProduction_VBF}
\begin{fmfgraph*}(60,25) 
\fmfleft{i1,i2}
\fmfright{o1,o2,o3}
\fmf{phantom}{i1,v1,o1}
\fmffreeze
\fmf{phantom}{i2,v2,o3}
\fmffreeze
\fmfstraight
\fmf{phantom}{v1,v3,v2}
\fmffreeze
\fmf{fermion, label=$q$}{i1,v1}
\fmf{fermion, label=$q$}{i2,v2}
\fmf{fermion, label=$q^\prime$}{v1,o1}
\fmf{fermion, label=$q^\prime$}{v2,o3}
\fmf{boson, label=$W/Z$}{v1,v3,v2}
\fmffreeze 
\fmf{dashes, label=$H$,plain}{v3,o2}
\end{fmfgraph*}
\end{fmffile}
}
\\
\vskip1em
\subfloat[Higgs-strahlung]{
\begin{fmffile}{HiggsProduction_HiggsStrahlung} 
\begin{fmfgraph*}(60,25) 
\fmfleft{i1,i2}
\fmfright{o1,o2}
\fmf{fermion, label=$q$}{i1,v1}
\fmf{fermion, label=$q'$}{v1,i2}
\fmf{boson, label=$W/Z$}{v1,v2,o1}
\fmf{dashes, label=$H$}{v2,o2}
\end{fmfgraph*}
\end{fmffile}
}
\quad
\subfloat[$t\bar{t}$ fusion]{
\begin{fmffile}{HiggsProduction_ttbar}
\begin{fmfgraph*}(60,25) 
\fmfleft{i1,i2}
\fmfright{o1,o2,o3}
\fmf{phantom}{i1,v1,o1}
\fmffreeze
\fmf{phantom}{i2,v2,o3}
\fmffreeze
\fmfstraight
\fmf{phantom}{v1,v3,v2}
\fmffreeze
\fmf{gluon, label=$g$, label.dist=+12.0}{i1,v1}
\fmf{gluon, label=$g$, label.dist=+12.0}{i2,v2}
\fmf{fermion, label=$t$}{o3,v2,v3,v1,o1}
\fmf{dashes, label=$H$,plain}{v3,o2}
\end{fmfgraph*}
\end{fmffile}
}
\caption{Representative Feynman diagrams for the four main production modes of the Higgs boson at the LHC.}
\label{fig:higgsprodfeyn}
\end{figure}
\captionsetup[subfloat]{captionskip=5pt} % space between subfloat caption and image
%% \begin{figure}
%%   \newImageScale{figures/Higgs_Prod_XSec.pdf}{0.5}
%%   \caption{Higgs boson production cross-sections and their theory uncertainties for different values of the LHC centre of mass energy. \cite{YR4}}
%%   \label{fig:higgsprodxsec}
%% \end{figure}


\subsection{Higgs Boson Decay Channels}
The Higgs boson couples to all massive particles and thus has several decay modes. The \emph{branching ratios} of the most relevant decays are displayed in \cref{fig:higgsbr}. The decay to a pair of bottom quarks is the leading process followed by the $H\rightarrow WW^*$ decay mode.

An overwhelming fraction of events at a hadron collider, like the LHC, is dominated by QCD effects.
The Higgs boson decay to a pair of $b$-quarks with a purely hadronic final state is therefore very difficult to separate from other events.
Nevertheless it is possible to measure this decay by using for example the VBF or Higgs-strahlung production mode and exploiting the presence of the gauge bosons in the final state in the event selection. With this strategy evidence for the $H\rightarrow b\bar{b}$ decay was recently found in the VH production mode \cite{Aaboud:2017xsd,Sirunyan:2017elk}.

The most sensitive channels for the Higgs boson discovery in 2012 were the Higgs boson decay to two photons, the $H \rightarrow ZZ^* \rightarrow llll$ process, and the $H \to WW^*$ decay.
%Higgs boson initially was discovered in the decay to two photons and the $H \rightarrow ZZ^* \rightarrow llll$ channel \todo{REF}.
The first two decays have relatively small branching ratios but exhibit very clean final-state signatures due to the presence of leptons and photons.
This allows an excellent selection of signal events with a good control over the background.
Additionally, the mass of the Higgs boson can be fully reconstructed in these channels by computing the invariant mass of the reconstructed decay products.
%at a hadron collider, where an overwhelming fraction of events is dominated by QCD effects.
%This is equally important since it allows an excellent selection of signal events and a good control over the background. 
% the largest fraction of events are dominated by QCD effects.
%QCD dominated events
%events at a hadron collider, since the final states are dominated by QCD effects.

The $H \rightarrow WW^*$ decay also gives a good handle over the background when selecting leptonically decaying $W$ bosons.
The neutrinos stemming from the $W$ boson decays, however, prevent from the ability to fully reconstruct the mass of the Higgs boson, because they cannot be directly detected with the ATLAS experiment.
Results of the analysis of \HWW\ decays conducted by the ATLAS Collaboration and CMS Collaboration for data of \RunOne\ can be found in Ref.~\cite{PhysRevD.92.012006} and Ref.~\cite{2013arXiv1312.1129C}, respectively; a corresponding analysis of \RunTwo\ data is given in \cref{sec:ggfanalysis}, focusing on the gluon fusion production-mode.

A similar final state as the one in \HWW\ processes, can be exploited when looking at Higgs boson decays into $\tau$-leptons \cite{Aad:2015vsa,Sirunyan:2276465}. The Higgs boson decay to $Z\gamma$, where leptonically decaying $Z$ bosons provide the most sensible final state, or $\mu\mu$, are very rare and thus the Higgs boson is still searched for in these channels. % \cite{Aaboud:2017uhw,Aaboud:2017ojs}.

The final states with charm quarks and gluons, making up a considerable fraction of all Higgs decays, are currently too difficult to differentiate from other events at a hadron collider.

% The H → µµ
%decay is a sensitive channel in which the Higgs coupling to second-generation fermions can be measured
%with a clean final-state signature at the LHC.

%background events can be reduced very effectively
%, which have relatively small branching ratios. Nevertheless this was achieved due to the precise knowledge of the background.

%Altough the branching ratios for these decays are relatively small this could be achieved due to the excellent background rejection by selecting leptons, as well as the precise knowledge of the expected background in these decay modes.


%% (Left) The SM Higgs boson production cross sections as a function
%% of the center of mass energy, √
%% s, for pp collisions. The theoretical uncertainties [46]
%% are indicated as bands. (Right) The branching ratios for the main decays of the SM
%% Higgs boson near mH = 125GeV. The theoretical uncertainties [44,45] are indicated
%% as bands.

%% \begin{figure}
%%   \newImageScale{figures/Higgs_Prod_XSec.pdf}{0.5}
%%   \caption{Higgs boson production cross-sections and their theory uncertainties for different values of the LHC centre of mass energy. \cite{YR4}}
%%   \label{fig:higgsprodxsec}
%% \end{figure}

\captionsetup[subfloat]{captionskip=5pt} % space between subfloat caption and image
\begin{figure}
  \begin{center}
    \subfloat[]{
      \newImageScale{figures/theory/Higgs_Prod_XSec.pdf}{0.37}
      \label{fig:higgsprodxsec}
    }
    \subfloat[]{
      \newImageScale{figures/theory/SMHiggsBR.pdf}{0.37}
      \label{fig:higgsbr}
    }
  \end{center}
  \caption[(a) Higgs boson production cross-sections as a function of the LHC centre-of-mass energy. (b) Branching ratios of the Higgs boson for a mass range around 125\,\GeV.]{(a) Higgs boson production cross-sections as a function of the LHC centre-of-mass energy. (b) Branching ratios of the Higgs boson for a mass range around 125\,\GeV. The line widths represent the respective theory uncertainties. Taken from Ref.~\cite{YR4}.}
  %  \label{fig:higgsbr}
\end{figure}






\subsection{Properties of HWW Decays}
% THIS MIGHT actually fit in the analysis section
% as it directly related to what analysis selections we perform
% THINK ABOUT IT!
- VBF Higgs -> ww cross section must be exactly the SM value, otherwise VBS xsec converges!
-> Maybe add this to HWW analysis



\subsection{Higgs Boson Measurements}

\subsubsection{Inclusive cross-section measurements}
\subsubsection{Fiducial and differential cross-section measurements}
\subsubsection{Simplified Template Cross-section Measurements}

\subsection{Current Experimental Status}
