Despite the extraordinary success of accurately predicting and explaining an enormous number of experimental measurements, the SM has several shortcomings.
There are several observable phenomona that cannot be explained by the SM as well as several open questions related to theory. 
There are models and theories beyond the SM that attempt to address these shortcomings, but experimental indication that either one of them is realised in nature is still pending.
%none of them have made predictions that have been confirmed by experiments to date. 
Notable examples of such theories are supersymmetric theories (see e.g. \ccite{Golfand:1971iw,Volkov:1973ix,Wess:1974tw,Salam:1974ig,Wess:1974jb,Ferrara:1974pu,Fayet:1976et}), dark matter models (e.g. \ccite{Baumgart_2009,Kaplan_2009,Dienes_2012}), or more simple extensions of the SM such as the \emph{two-Higgs-doublet models} \cite{Branco_2012}.
This section provides an overview of some of the biggest unsolved problems of the SM and fundamental physics in general.

% HIGGS and Dark Matter
%https://home.cern/news/news/physics/atlas-probes-dark-matter-using-higgs-boson

% Maybe also read introduction of Deciphering the nature of the Higgs boson!


%%%%%%%%%%%%%%%%%%%%%%%%%%%%%%%%%%%%%%%%%%%%%%%%%%%%%%%%%%%%%%
% Experimental ones (not sure yet if I should make this distinction)
\subsubsection{Higgs boson}
While several physics problems were solved by introducing the Higgs boson to the SM, other mysteries remain. 
One of the biggest open questions is related to the mass of the Higgs boson itself. 
The mass of the Higgs boson receives virtual contributions from all energies up to the scale the SM is valid. 
Assuming that there is no new physics up to the Planck scale, these virtual corrections would be extremely large and the Higgs boson mass would be expected around the Planck scale. 
Therefore, the bare mass of the Higgs boson\footnote{The observable mass of an elementary particle is given by the difference of two terms that are fully independent of each other, the particles bare mass and corrections to it from quantum fluctuations.}
needs to almost perfectly balance the virtual corrections to keep the observable Higgs boson mass in the \GeV\ range. The precision required for such a balance is extremely high and often considered \emph{unnatural}.
%, meaning that it is highly unlikely that the Higgs boson mass is so low due to chance alone. 
This is also known as the \emph{hierarchy problem} and is one of the main motivations for the search for new particles with masses in the \TeV\ range at the LHC. 
Their existence would affect the quantum contributions to the Higgs boson mass, eliminating the need of ``fine-tuning'' the bare Higgs boson mass to an incredible precision, as is assumed in the SM. 

This is also related to the principle of ``separation of scales'', which implies that physics at one energy scale must not be dependent on the details of physical phenomena that are important only at much higher scales.
This notion has been at the core of physics for centuries and allows describing the world in terms of effective theories. 
If the Higgs boson couples to heavy particles beyond the SM they would affect the Higgs boson mass at lower energies and therefore break this principle.
%\todo{Maybe mention compositenes or 2HDMs?}

\subsubsection{Large number of free parameters}
As discussed in \cref{subsec:final-lagrangian}, the SM has several free parameters that need to be added to the theory ad hoc and cannot be derived from first principles. 
From an experimentalist point of view, this is not a problem per se, but it suggests that there might be a more fundamental theory with fewer parameters that can ultimately explain the relationships between some of the current SM parameters.

% Thomson: 
% [The SM] ... is a model constructed from a number of beauti- ful and profound theoretical ideas put together in a somewhat ad hoc fashion in order to reproduce the experimental data.

\subsubsection{Strong CP problem}
No theoretical principle forbids CP\footnote{CP refers to charge conjugation (C) and parity (P). CP violation refers to the violation of the combined CP-symmetry.} violation in the strong interaction. However, no such process has been found and the strong CP phase, $\theta_{\text{QCD}}$, has been experimentally constrained to be smaller than $10^{-10}$ (inferred from neutron electric dipole moment measurements \cite{Baker_2006}). This is considered another ``fine-tuning'' problem.
% Not sure if I should add that
The so-called Peccei-Quinn theory~\cite{PhysRevLett.38.1440} and modifications of it provide a solution by introducing a scalar field known as the axion. However, no experimental evidence of the existence of such a particle has been found to date.
%there is no explanation of why these interactions are so small which makes
% - There is nothing that would forbid CP violation in the strong interaction, however, no CP-violating processes have been observed to date. 
% This is considered a ``fine-tuning'' problem.

\subsubsection{Neutrinos}
Right-handed neutrinos do not couple to any of the other SM particles.
% Right-handed neutrinos are not part of the SM because they do not couple to any of the other SM particles.
%Neutrino Yukawa interactions (see \cref{subsec:fermion-masses}) are therefore typically not added to the SM. 
However, it is known that neutrinos oscillate \cite{Gonzalez_Garcia_2008}, implying that they have finite masses.
While it is possible to add mass terms from Yukawa interactions (see \cref{subsec:fermion-masses}) for neutrinos ad hoc, without violating any of the theoretical principles, it is not clear exactly which terms to add, as the nature of the neutrinos is not settled\footnote{It is still an open question whether neutrinos are Dirac fermions or Majorana fermions. Majorana fermions are particles that are their own anti-particles, which stands in contrast to Dirac fermions. Different mathematical descriptions are required to describe the different types of particles in the Lagrangian.}. 
Nonetheless, the SM is sometimes described including additional mass terms for the neutrinos, which adds an additional seven free parameters (three for the masses and four to describe the mixing between the neutrinos).

%It is possible to add Dirac mass terms by implying the existence of right-handed neutrinos. 
%- One possibility is to add Dirac mass terms, which would add an additional 6 free parameters to the SM. 
%- However, it is not clear if Neutrinos are indeed sterile (Dirac) and unknown why Neutrino masses are so small and if they get their masses from the same process as the other SM particles
% Wikipedia on Majorana:
% A Majorana fermion (/maɪəˈrɑːnə ˈfɛərmiːɒn/[1]), also referred to as a Majorana particle, is a fermion that is its own antiparticle. They were hypothesised by Ettore Majorana in 1937. The term is sometimes used in opposition to a Dirac fermion, which describes fermions that are not their own antiparticles.
% The nature of the neutrinos is not settled—they may turn out to be either Dirac or Majorana fermions.
% From Pich 2012
%The experimental evidence of neutrino oscillations shows that νe, νμ and ντ are also mixtures of mass eigenstates. However, the neu

%%%%%%%%%%%%%%%%%%%%%%%%%%%%%%%%%%%%%%%%%%%%%%%%%%%%%%%%%%%%%%
% Theoretical ones (not sure yet if I should make this distinction)

\subsubsection{Matter-antimatter symmetry}
It is widely believed that the universe was created with equal amounts of matter and antimatter.
However, the currently observable universe consists of significantly more matter than antimatter. 
The amount of CP violation observed in the SM in electroweak interactions is not sufficient to explain this large asymmetry.
%The SM does not provide a theoretical explanation for this large asymmetry. 
% the SM cannot explain why there is more matter than antimatter in the observable universe. 
% - No mechanism in the SM can explain the asymmetry sufficiently. 
% From Thomson
% However, even if CP violation is observed in neutrino oscillations, it seems quite possible that the CP violation in the Standard Model is insufficient to explain the observed matter–antimatter asymmetry of the Universe.

\subsubsection{Dark matter}
The SM is able to explain the behavior of the currently observable matter in the universe, which makes up about 5\% of the total energy of the universe. 
As known from astrophysics experiments such as measurements of galaxy rotations (see e.g. \ccite{Corbelli_2000}), another 27\% of the universe's energy is made up of another type of matter dubbed \emph{dark matter}. To date, the nature of dark matter is unknown.
% To date, the SM has no viable candidate to explain the nature of dark matter. 
% Another 26\% of the universe's energy is made up of so-called dark matter. 
% - The SM explains the observable mass which makes up about 5\% of the energy of the universe. 
% - The rest of the matter, making up about 26\% is known as dark matter. (known from measurements of orbiting galaxis)
% - The SM cannot explain the nature of dark matter, to date. 
% from Thomson
%velocity distributions of stars as they orbit the galactic centre)

\subsubsection{Dark energy}
About 68\% of the universe's energy constitutes so-called \emph{dark energy}, introduced to explain the acceleration of the expansion of the universe \cite{Riess_1998,Perlmutter_1999}. Dark energy is attributed to the non-zero cosmological constant that appears in Einstein's equations of general relativity~\cite{Einstein1916} and is also sometimes referred to as \emph{vacuum energy}. The SM provides no suitable explanation for the large value of the vacuum energy.

%The SM has no explanation for the large value of the vacuum energy. 
% - Dark energy is the energy needed in order to explain the acceleration of the expansion of the universe. 
% - This is sometimes called vacuum energy and related to the cosmological constant problem. The SM has no explanation for the measured size of the vacuum energy.
% From Thomson
%- dark energy is attributed to a non-zero cosmological constant of Einstein's equations of general relativity

\subsubsection{Gravitation}
The current understanding of gravity is still shaped by Einstein's theory of general relativity. No quantum theory of gravity has been fully developed, which is why it is not part of the SM. 
The interaction strength of gravity is very weak compared to all other fundamental forces, which is why it can be safely neglected in current high energy physics experiment.
% Due to the small strength of gravity, compared to the other fundamental forces, gravity can be fully neglected in current high energy physics experiments. 
% - Quantum theory of gravity not fully developed. Versions that exist predict the graviton, the particle mediating gravity to be a spin 2 particle.
% - Classical theory of gravity, general relativity


% Not sure if I should include this it's kind of weird! I don't understand it, let's say that. Why is it a problem, can't you ALWAYS ask that question? Why is there this particle. Why are there electrons? Not sure if I get that quite right. 
% \subsubsection{The fermion generations}
% The SM has no explanation for why there are three generations of fermions. 



% - Due to these inexplicabilities the Higgs boson (and the fact that it's the least well measured EM particle) is widely considered to be the answer to many other questions in fundamental physics

% Hierarchy Problem from Thomson:
% Just as quantum loop corrections contributed to the W-boson mass (see Section 16.4), quantum loops in the Higgs boson propaga- tor, such as those indicated in Figure 18.3, contribute to the Higgs boson mass. This in itself is not a problem. However, if the Standard Model is part of theory that is valid up to very high mass scales, such as that of a Grand Unified Theory ΛGUT ∼ 1016 GeV or the Planck scale ΛP ∼ 1019 GeV, these corrections become very large. Because of these quantum corrections, which are quadratic in Λ, it is difficult to keep the Higgs mass at the electroweak scale of 102 GeV. This is known as the Hierarchy problem. It can be solved by fine-tuning the new contributions to the Higgs mass such that they tend to cancel to a high degree of precision.

% Why Higgs measurements?
% The experimental study of the Higgs boson at the LHC is undoubtedly one of the most exciting areas in contemporary particle physics. Within the Standard Model, the Higgs boson is unique; it is the only fundamental scalar in the theory. Establish- ing the properties of the Higgs boson such as its spin, parity and branching fractions is essential to understand whether the observed particle is the Standard Model Higgs boson or something more exotic.