\chapter{The $H\rightarrow W^{\pm}W^{\mp^*}$ Analysis}
\label{chap:hww}

You can find some text snippets that can be used in papers in \texttt{latex/atlassnippets.sty}.
To use them, provide the \texttt{snippets} option to \texttt{atlasphysics}.

\section{Characteristics of the Signal and Background Processes}
\begin{itemize}
    \item Introduce all variables needed in the analysis
\end{itemize}

\section{Data and Monte Carlo Samples}


\section{Object Selection and Efficiency Correction}
\subsection{Lepton selection}
\subsection{Jet selection}
\subsection{Missing transverse energy}
\subsection{Overlap removal}
\label{subsec:overlap-removal}

\subsection{Pile-up reweighting}
In order to compare Monte Carlo simulated events with actual data, the amount of pile-up underlying the hard scatter needs to be account for. 
To this end, a method known as \emph{pile-up reweighting} assigns a dedicated \emph{pile-up weight} to each event so that the average pile-up in simulated collision events match the actual running conditions. The reweighting procedure is performed separately for the data recorded in the years 2015 and 2016, 2017, and 2018.



\section{Event Selection}
\section{VBF Analysis Regions}
\section{ggF Analysis Regions}

\section{Classification of VBF Signal and Background Using a Deep Neural Network}
\subsection{Training variables}
\subsection{Training hyperparameter optimization}
%\subsection{Prospects Studies for Common VBF and ggF HWW 2-jet Analysis with Multiclass Classification}

\section{Systematic Uncertainties}
\subsection{Systematic experimental uncertainties}
\subsection{Systematic theoretical uncertainties}
\subsection{Treatment of systematic uncertainties}

\section{\HWW VBF+ggF Couplings Analysis Fit Results}
\subsection{VBF signal results}
\subsection{Results of combined VBF+ggF analysis}

\section{The $H\rightarrow W^{\pm}W^{\mp^*}$ STXS Analysis}
\subsection{Results of combined STXS fit}
