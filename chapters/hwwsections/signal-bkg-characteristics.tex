Terminology introduction?

Signal is ggF, VBF Higgs production
Background is everything else

\subsection{The Signal Processes}

The Higgs couples to a pair of oppositely charged $W$ bosons via the process of EWSB that was discussed in \cref{sec:sm}. Due to the high mass of the \Wboson boson ($m_W = 80.354 \pm 0.007\,\GeV$ \cite{PDG2020}), only one of them is created on shell, the respective other boson is produced off shell indicated by a star in ``\HWW''.

The $W$ bosons are short-lived $\left( \tau \approx \mathcal{O}\left(10^{-25}\,\text{s}\right)\,\text{\cite{PDG2020}}  \right)$ so that only their decay products can be measured. The possible decay modes including their branching fractions are shown in \cref{fig:w-branching-ratios}. The analysis presented in this thesis targets \Wboson boson decays into \emph{light leptons} (simply referred to as leptons in the following), that is electrons and muons including their associated neutrinos.
As discussed in more detail in \cref{sec:object-selection}, the presented analysis targets \HWW decays that exhibit one electron and one muon in the final state.
% Belongs more to object selection!
%- Only muon+electron or electron+muon events considered, otherwise background from Drell-Yan overwhelming

%It can be assumed that the Higgs boson is produced at rest, so that the two \Wboson bosons are emitted in opposite directions in the transverse plane. 
%For this reason, and because of the property of the electroweak interaction to couple only to left-handed neutrinos as well as angular momentum conservation, the \HWWdet process exhibits a distinct kinematic feature that can be exploited in the event selection.
%It arises as a direct consequence of angular momentum conservation and the property of the electroweak interactions to only couple to left-handed neutrinos. 
Because of both the \Wboson bosons coupling only to left-handed neutrinos and the spin-0 nature of the Higgs boson, the \HWWdet process exhibits a distinct kinematic feature that can be exploited in the event selection.
The spin projections of the \Wboson bosons must sum to zero, because angular momentum must be conserved.
Similarly, the leptons from the \Wboson decays must have spin projections that are aligned, which ultimately leads to
an enhanced fraction of \HWWdet events in which the two leptons are emitted in the same direction at small angle \DPhill.
\Cref{fig:spin-correlations} illustrates this by showing the different spin configurations possible, including the resulting final state kinematics.\footnote{This provides a slightly simplified picture that disregards interference effects between different \Wboson boson polarisation states. When they are considered, however, similar conclusions can be drawn~\cite{Maina_2021}.}
As a consequence, several observables that are correlated to the angular separation between the leptons can be exploited to discriminate between \HWW decays and other background events. For example, the mass of the dilepton system,
\begin{equation}
    \Mll = \sqrt{\left| \Ptll \right|^2+ \left( E_{\ell\ell} \right)^2},
\end{equation}
is more likely to take smaller values, and the magnitude of the vectorial sum of the \pT of both leptons,
\begin{equation}
    \Ptll = |\pmb{p}_\text{T}^{\ell_1} + \pmb{p}_\text{T}^{\ell_2}|,
\end{equation}
is expected to be shifted towards larger values due to the angular correlations. 

The presence of the neutrinos in the final state has two important consequences:
First, they lead to events with significant missing transverse energy. This is amplified by the angular correlations described above, as the neutrinos are more likely to be emitted in the same direction. 
Second, not enough information is available about the longitudinal momentum of the Higgs boson in order to reconstruct its full mass. The mass of the Higgs boson can therefore only be reconstructed in the transverse plane, which is calculated as
\begin{equation}
  \Mt = \sqrt{ \left( E_{\ell\ell}+\MET \right)^2 - \left| \Ptll + \pmb{E}_{\text{T}}^{\text{miss}} \right|^2 },
\end{equation}
where $E_{\ell\ell} = \sqrt{ |\Ptll|^2 + \Mll^2}$ is the energy of the lepton system.

The signature of the final state depends on the production mode of the Higgs boson.
In the VBF production mode, two energetic jets are expected in the final state that are predominantly emitted in the forward direction and have a large invariant mass. Due to the mediating weak bosons not exchanging color, the hadronic activity between the jets is expected to be small. 
This prominent feature clearly distinguishes VBF produced \HWW events from events produced via the ggF production mode, where no additional jets are expected at leading order. Additional jets may, however, still arise (for both production modes) due to ISR and FSR. The mentioned characteristics are depicted in \cref{fig:ggF-VBF-Hww-feyn} and are what drive the event categorisation discussed in \cref{sec:event-categorisation}. 

In this analysis, the contributions from other Higgs production modes such as $VH$ or $ttH$ are marginal because of both their small cross sections as well as the fact that they have additional particles in the final state based on which their contributions can be suppressed.
% events can be efficiently rejected.

Events where the Higgs boson decays into a pair of $\tau$ leptons have similar final states as the \HWWdet\ decays when the $\tau$ leptons decay leptonically. 
Because both the branching fractions of $\tau$ leptons decaying into muons or electrons are small ($\approx$ 17-18 \% \cite{PDG2020}) and the process exhibits different kinematics, contributions from $H \to \tau\tau$ processes are small in this analysis. 
\todo{Might need to back up that statement}


\todo{update w decay ratios}
\begin{figure}
    \newImageResizeCustom{0.5}{figures/hww/introduction/wdecays/wdecays.pdf}
    \caption[Different decay modes of the $W$ boson with indication of their branching ratios.]{Different decay modes of the $W$ boson with indication of their branching ratios. Values are taken from Ref.~\cite{PDG2020}.}
    \label{fig:w-branching-ratios}
\end{figure}

\todo{update illustration}
\begin{figure}
    \newImageResizeCustom{0.7}{figures/hww/introduction/HWW_angular_correlation.png}
    \caption[Angular correlations in the \HWWdet\ decay.]{Illustration of the angular correlations in the \HWWdet\ decay. The narrow arrows indicate the flight direction of the particles, the double arrows visualize their spin. The spin projections of the $W$ bosons sum to zero and they decay into leptons with aligned spin projections. This results in an enhanced fraction of events with leptons that have a small opening angle as well as two neutrinos being emitted in opposite directions. Taken from Ref.~\cite{PhysRevD.92.012006}.}
    \label{fig:spin-correlations}
\end{figure}

\subsection{The Background Processes}
The following gives an overview over the most important background processes in this analysis.

\subsubsection{Continuum $WW$ background}
The non-resonant (or \emph{continuum}) $WW$ background is dominated by quark-initiated $WW$ processes (labelled $qqWW$) with only a small contribution coming from gluon-initiated processes ($ggWW$). 
Relevant Feynman diagrams are shown in \cref{FEYNMAN}
The production of two $W$ bosons gives rise to final states that contain the same physics objects as the signal process, which renders this background irreducible. 
A suppression of the $WW$ background can still be achieved by exploiting the distinct kinematics of the signal. For example, the $\mll$ observable tends to have a broader distribution for the non-resonant $WW$ processes compared to the signal. 
Another difference is that both $W$ bosons are produced on-shell in contrast to the signal process. 
This leads to the fact that the lepton \pT tends to be larger in continuum $WW$ processes than in signal processes.\todo{double-check the last statement}
\todo{Feynman}

\subsubsection{EW $WW$ background}
The contribution from $WW$ bosons produced via electroweak interactions such as VBS (\todo{EW6?? -> Check support note?}) is generally small, due to a small cross section. However, at large $m_{jj}$ the contribution can be significant relative to other background processes that fall much more steeply with increasing $m_{jj}$. 
\todo{Feynman}

\subsubsection{Top-quark backgrounds}
Collision events involving top quarks are abundant at $pp$ colliders because of their large production cross section, making them a major background in this analysis.
Processes with top quarks, collectively denoted as top-quark background, can be separated into $t\bar{t}$ processes, involving two top quarks, or processes with only a single top-quark. 
The single-top process relevant for this analysis is characterised by a $W$ boson radiated off a single top quark (labelled as $Wt$). 
The relevant Feynman diagrams are shown in \cref{FEYNMAN}.
The top quark decays into a $b$-quark and a $W$ boson in more than 99\% of all cases~\cite{PDG2020}. 
This gives rise to two $W$ bosons in the final state, which assimilates the signal process. The arising $b$-jets, however, can be used to reject the top-quark events effectively.
\todo{FEYNmAN}
% Since the $b$-jet identification is not perfect and the production cross section of top processes is large, top processes represents a major background. 


\subsubsection{\Ztautau\ background}
$Z$ bosons are produced with large cross sections at the LHC. The processes relevant for the \HWW\ analysis are Drell-Yan (DY) processes and \Zgamma processes in association with jets, as displayed in \cref{FIGURE}. 
The $Z$ boson decays with equal probabilities (3.4\%~\cite{PDG2020}\todo{double-check number}) into electrons, muons, or taus. 
The analysis specifically selects events with two leptons of different flavor, to minimize the otherwise large contamination from $\Zgamma \to ee$ and $\Zgamma \to \mu\mu$ processes. Only in rare cases where one of the two leptons of the same flavor cannot be identified and a jet is, for example, misidentified as a lepton does the event signature resemble the signal. 
The \Ztautau\ process, in contrast, constitutes a substantial background as the tau leptons can further decay into electrons or muons and their associated neutrinos (with a branching ratios of about 17-18\%~\cite{PDG2020}\todo{double-check number} each), giving rise to final states similar to the ones of the signal processes. 
The \Ztautau\ background can be suppressed effectively, if the $Z$ bosons are produced at rest, which is a good approximation for events with \ZeroJet. In these cases, the neutrinos from the tau decays are emitted in opposite directions, resulting in a small \MET, and for the same reason, the leptons are expected to have a large opening angle, which is contrary to the signal signature. 
If the $Z$ boson is produced with a significant transverse momentum, for example in cases where the $Z$ boson recoils against a jet, the \MET may be sizeable and the opening angle of the leptons is not necessarily large. 
Suppressing the \Ztautau\ background thus becomes more difficult in the \OneJet and \TwoJet categories. 


\subsubsection{$W$+jets and multi-jet background}
Processes with $W$ bosons in association with a jet (\Wjets) as well as multijet processes do not contain the same final state particles as the signal processes. However, they can contaminate the analysis regions if one or more of the reconstructed physics objects is misidentified as an isolated prompt lepton. 
Although such misidentifications are very rare, the $W$+jets and multijet production are sizeable backgrounds in this analysis due to their large production cross-section.
The $W$+jets process, shown in \cref{FIGURE}, is especially important because it requires only one physics object to be misidentified for it to mimic the final state of the signal process.
\todo{Write the above a bit more clearly. If the jet from the W+jets process is misidentified as a lepton that's it!}

Misidentifications arise from two main sources: jets that are misidentified as isolated leptons, or from non-prompt leptons from heavy hadron decays within jets that are also misidentified as isolated leptons.
The rate at which these misidentifications occur is difficult to estimate purely based on simulated event samples. For this reason, a dedicated data-driven method known as \emph{fake factor method} is used to estimate the background with misidentified leptons. Details are given in \cref{SECTION}.

% The detector signature of such events can be extremely similar to the signal process, making it difficult to reduce this background via selections on event observables.

\subsubsection{Other diboson and other backgrounds}
Aside from $WW$ production, other diboson processes can contaminate the analysis regions. 
The $WZ$, $ZZ$, and $W\gamma*$ processes can mimic the signal, if leptons or jets are lost in the reconstruction. 
$W\gamma$ and $Z\gamma$ processes can mimic the signature of the signal processes, if the photon is misidentified as an electron. This may occur, if the photon converts to an electron-positron pair and is reconstructed as a single isolated lepton. 
Since the reconstruction and identification efficiencies are reasonably high, the non-$WW$ diboson (labelled as Other $VV$) background is small in this analysis.
The same is true for processes involving three bosons (called \emph{triboson} processes), which contribute only marginally to the analysis regions because of their additional final state particles as well as their overall small production cross-section. 
