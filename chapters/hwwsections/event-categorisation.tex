
- Selections are made on event observables to define different regions of phase space that optimise the signal purity. 
- First a common selection is applied for all signal categories, then different reconstructed regions are defined that target the measurement of different production modes of the Higgs boson. 
- 
- A table of all cuts is provided

- Furthermore, the production modes are further subdivided into several kinematic bins in the STXS categorisation scheme that is discussed in ....

- Define ``vocabulary'' for regions, bins, categories, processes, ...



\subsection{Preselection}
A common event preselection is applied before the events are divided into specific analysis regions targeting different signal categories. 
As discussed in the previous sections, the analysis selects events with exactly two leptons with different flavour and opposite charge as well as satisfying trigger matching criteria. 
The transverse momenta of the leptons must satisfy \ptGT{22}\,GeV and \ptGT{15}\,GeV for the higher-\pT (\emph{leading}) and lower-\pT (\emph{subleading}) lepton, respectively.
To minimize the contamination from low mass meson resonances as well as di-tau backgrounds from low-mass Dell-Yan production, the invariant mass of the dilepton system is required to be $\mll > 10\,\GeV$.

The distribution of the number of jets in the event at the preselection level is shown in \cref{fig:njets}. 
The background composition varies significantly across jet bins, which motivates splitting the data sample into separate \Njet categories.
In total four mutually exclusive reconstructed regions are defined.
One region targets the VBF production mode and must satisfy \TwoJet. 
Three categories are defined targeting the ggF production mode: \ZeroJet, \OneJet, \TwoJet. 
An additional preselection is applied for the ggF categories on the missing transverse momentum, $\pTmiss > 20\,\GeV$, to exploit the presence of neutrinos in the signal.

The follow gives details on the specific cuts that are applied to each category and take into account the different event topologies as well as the different background composition.

\subsection{VBF 2-jet Analysis Regions}
The distinct VBF topology is exploited by rejecting events that contain either an additional jet with \ptGT{30}\,GeV between the two leading jets in pseudorapity (known as \emph{central jet veto} (CJV)) or a lepton at larger pseudorapidities than both of the jets (known as \emph{outside lepton veto} (OLV)). 
In addition, a selection is made on the invariant mass of the jets $\mjj > 120\,\GeV$, which has only minimal impact but serves to separate analyses targeting the $V(\to qq)H$ production mode. 

The VBF analysis is then performed by using a deep neural network model (DNN) that is trained to classify signal and background events. Details of the DNN analysis are described in \cref{sec:dnn}.


\subsection{ggF Analysis Regions}


\subsubsection{0-jet}
\subsubsection{1-jet}
\subsubsection{2-jet}

\subsection{STXS categorisation}
-> ggF is written as ggH
-> VBF is denoted as EW qqH that is more inclusive than VBF as it also includes V(->qq)H. 

- STXS categorisation plot
- Expected event yields in each stxs region (or maybe later!?)