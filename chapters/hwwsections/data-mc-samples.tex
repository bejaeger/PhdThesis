%
\begin{table}[h]
    \centering
    \caption{
      Overview of simulation tools used to generate signal and background processes, as well as to model the UEPS. The PDF sets are also summarised.
      Alternative event generators or quantities varied to estimate systematic uncertainties are shown in parentheses.}
    \label{tab:mcsamples}
  \scalebox{0.66}{
    \begin{tabular}{l l l l l}
    \hline\hline
    %%% Info mostly from https://gitlab.cern.ch/atlas-physics/pmg/documents/references/-/tree/master
    Process              & Matrix element                                              & PDF set                 & UEPS model                                           & Prediction order\\
                         & (alternative)                                               &                         & (alternative model)                                  &  for total cross section\\
    \hline
    ggF $H$              & \POWHEGBOXV{v2}~\cite{Hamilton:2013fea,Hamilton:2015nsa,Alioli:2010xd,Nason:2004rx,Frixione:2007vw}
    & \multirow{2}{*}{\texttt{PDF4LHC15 NNLO}~\cite{Butterworth:2015oua}} &\multirow{2}{*}{\PYTHIAV{8}~\cite{Sjostrand:2014zea}} & \multirow{2}{*}{N$^{3}$LO QCD + NLO EW~\cite{deFlorian:2016spz,Anastasiou:2016cez,Anastasiou:2015vya,Dulat:2018rbf,Harlander:2009mq,Harlander:2009bw,Harlander:2009my,Pak:2009dg,Actis:2008ug,Actis:2008ts,Bonetti:2018ukf,Bonetti:2018ukf}} \\
                         & NNLOPS~\cite{Nason:2009ai,Hamilton:2013fea,Campbell:2012am} &                         &    & \\
                         & (\MGFiveNLO)~\cite{Alwall:2014hca,Frederix:2012ps}          &                         & (\HerwigV{7})~\cite{Bellm:2015jjp} & \\
    VBF $H$              & \POWHEGBOXV{v2}~\cite{Nason:2009ai,Alioli:2010xd,Nason:2004rx,Frixione:2007vw}
                         & \texttt{PDF4LHC15 NLO}  & \PYTHIAV{8}        & NNLO QCD + NLO EW~\cite{Ciccolini:2007jr,Ciccolini:2007ec,Bolzoni:2010xr} \\
                         & (\MGFiveNLO)                                                &                         & (\HerwigV{7})                                        & \\
    $VH$ excl. $gg\to ZH$ & \POWHEGBOXV{v2}                                            & \texttt{PDF4LHC15 NLO}  & \PYTHIAV{8}  & NNLO QCD + NLO EW~\cite{Ciccolini:2003jy,Brein:2003wg,Brein:2011vx,Denner:2014cla,Brein:2012ne} \\
    \ttH                 & \POWHEGBOXV{v2}                                             & \texttt{NNPDF3.0NLO}    & \PYTHIAV{8}               & NLO~\cite{deFlorian:2016spz} \\
    $gg\to ZH$           & \POWHEGBOXV{v2}                                             & \texttt{PDF4LHC15 NLO}  & \PYTHIAV{8}               & NNLL~\cite{Altenkamp:2012sx,Harlander:2014wda} \\
  
    $qq \to WW$          & \SHERPAV{2.2.2}~\cite{Bothmann:2019yzt}                     & \texttt{NNPDF3.0NNLO}~\cite{Ball:2014uwa} & \SHERPAV{2.2.2}~\cite{Gleisberg:2008fv,Schumann:2007mg,Hoeche:2011fd,Hoeche:2012yf,Catani:2001cc,Hoeche:2009rj} & NLO~\cite{Buccioni:2019sur,Cascioli:2011va,Denner:2016kdg} \\
                         & ($Q_\text{cut}$)                                            &                         & (\SHERPAV{2.2.2}~\cite{Schumann:2007mg,Hoeche:2009xc}; $\mu_\text{q}$)  \\
  %% not used              & (\POWHEGBOXV{v2},                                           &                         & \multirow{2}{*}{(\Herwigpp~\cite{Bellm:2015jjp})}  \\
  %% not used              & \MGFiveNLO)                                                 &                         &  \\
  % used                 & (CKKW)                                                      &                         & (QSF/CSSKIN)  \\
    $qq \to WWqq$        & \MGFiveNLO~\cite{Alwall:2014hca}                             & \texttt{NNPDF3.0NLO}    & \PYTHIAV{8}                                      & LO \\
                         &                                                             &                         & (\HerwigV{7})                                        & \\
  $gg \to WW/ZZ$         & \SHERPAV{2.2.2}                                             & \texttt{NNPDF3.0NNLO}   & \SHERPAV{2.2.2}                                      & LO~\cite{Caola:2015rqy}  \\
  %$WZ/V\gamma^{\ast}/ZZ \to \ell\ell\ell\ell/\ell\ell\ell\nu$ & \SHERPAV{2.2.2}        & \texttt{NNPDF3.0NNLO}   & \SHERPAV{2.2.2}                                      & NLO~\cite{Cascioli:2013gfa} \\
  %Other $WZ/V\gamma^{\ast}/ZZ$ & \POWHEGBOXV{v2}                                       & CT10                    & \PYTHIAV{8}                                          & NLO~\cite{Cascioli:2013gfa} \\
  $WZ/V\gamma^{\ast}/ZZ$ & \SHERPAV{2.2.2}                                             & \texttt{NNPDF3.0NNLO}   & \SHERPAV{2.2.2}                                      & NLO~\cite{Cascioli:2013gfa} \\
  %% not used            & (CKKW)                                                      &                         & (CSS variation~\cite{Schumann:2007mg})\\
    $V\gamma$            & \SHERPAV{2.2.8}~\cite{Bothmann:2019yzt}                     & \texttt{NNPDF3.0NNLO}   &\SHERPAV{2.2.8}                                       & NLO~\cite{Cascioli:2013gfa} \\
  $VVV$                  & \SHERPAV{2.2.2}                                             & \texttt{NNPDF3.0NNLO}   &\SHERPAV{2.2.2}                                       & LO  \\
  %% not used            & (\MGFiveNLO)                                                &                         & (CSS variation~\cite{Schumann:2007mg,Hoeche:2009xc}) & \\
    $t\bar{t}$           & \POWHEGBOXV{v2}                                             & \texttt{NNPDF3.0NLO}    & \PYTHIAV{8}                                          & NNLO+NNLL~\cite{Beneke:2011mq,Cacciari:2011hy,Baernreuther:2012ws,Czakon:2012zr,Czakon:2012pz,Czakon:2013goa,Czakon:2011xx} \\  %~\cite{Frixione:2007nw}-~\cite{ATL-PHYS-PUB-2014-021}] \\
                         & (\MGFiveNLO)                                                &                         & (\HerwigV{7})                                        & \\
   $Wt$                  &\POWHEGBOXV{v2}                                              & \texttt{NNPDF3.0NLO}    & \PYTHIAV{8}                         & NNLO~\cite{Kidonakis:2010ux,Kidonakis:2013zqa} \\
                         & (\MGFiveNLO)                                                &                         & (\HerwigV{7})                                          & \\
    $Z/\gamma^{\ast}$    & \SHERPAV{2.2.1}                                             & \texttt{NNPDF3.0NNLO}   & \SHERPAV{2.2.1}                                      & NNLO~\cite{Anastasiou:2003ds} \\
                         & (\MGFiveNLO)                                                & \\
  \hline\hline
    \end{tabular}
  }
  \end{table}
  

This section first provides an overview of the data samples that are used in this analysis and then discusses the simulation infrastructure used for the simulated event samples.
Since the simulation infrastructure is described in great detail in the original publication \cite{HWWPaper}, a significant portion is cited directly. 

\subsection{Data samples}
\label{subsec:data-samples}
The proton-proton collision data analysed in this thesis were recorded by the ATLAS experiment during \RunTwo of the LHC between 2015 and 2018, and correspond to an integrated luminosity of 139\ifb\ at a center-of-mass energy $\sqrt{s} = 13\,\TeV$. 
The events were recorded when all relevant detector components were fully operational, and are considered as being of ``good'' data quality (see \cref{sec:run-2-data-taking}).

Given the targeted final state featuring one electron and one muon, a combination of unprescaled single-lepton triggers and one $e$--$\mu$ dilepton trigger~\cite{TRIG-2018-05,TRIG-2018-01} are used to select the data samples. 
The transverse momentum ($\pt$) threshold for single-electron(muon) triggers is 24(20)\,\GeV for the first year of data taking, while it is increased to 26~\GeV for both lepton flavors during the remainder of \RunTwo.\cite{HWWPaper}
The $e$--$\mu$ trigger requires a minimum $\pt$ threshold of 17\,\GeV for electrons and 14\,\GeV for muons.\cite{HWWPaper}


\subsection{Simulated event samples}
\label{subsec:simulated-event-samples}
%%%%%%%%%%%%%%%%%%%%%%%%%%%%%
% THIS IS COPIED SO FAR FROM 
% https://gitlab.cern.ch/atlas-physics-office/HIGG/ANA-HIGG-2018-47/ANA-HIGG-2018-47-PAPER/-/blob/master/sections/DataAndSimSamples.tex
%%%%%%%%%%%%%%%%%%%%%%%%%%

This analysis heavily relies on simulated $pp$ collision events to validate and optimize the analysis strategy and, more importantly, to extract the final signal strength and cross-section measurement after a fit to the data (as discussed in \cref{sec:stats-analysis}). 

Most signal and background processes are simulated using a combination of MC generators calculating the hard scatter, such as \Powheg, and programs that model the parton showering, hadronisation, and underlying event, such as \PythiaEight.
For all MC samples, the events were processed through the ATLAS detector simulation~\cite{SOFT-2010-01} based on
\textsc{Geant4}~\cite{Agostinelli:2002hh}.\cite{PLACEHOLDER:FOR:PAPER} 
The effect of pile-up is modelled by overlaying over the original event minimum-bias events generated with \PYTHIAV{8.186}~\cite{Sjostrand:2007gs} using the \nnpdftwo set of PDFs and the A3 tune~\cite{ATL-PHYS-PUB-2016-017}.


Each simulated event gets assigned a generator weight (\emph{MC weight}), which allows populating different phase space regions efficiently.
The statistical uncertainties on the MC predictions are given by the sum of the event weights squared,
\begin{equation}
    \sigma = \sqrt{\sum_i w_i^2},
\end{equation}
where $N$ is the total number of generated events (sometimes referred to as \emph{raw events}). Typically, the number of simulated raw events is significantly larger than the number of events expected in data, in order to ensure high statistical precision. 
The overall normalization of each process is determined by theoretical cross-section predictions.
The following provides details on the MC generators and the accuracies of the available cross-section calculations for each process. The content is quoted from \cite{PLACEHOLDER:FOR:PAPER} directly.


\subsubsection{Higgs boson samples}
Higgs boson production and decay into pairs of $W$ bosons or leptonically decaying $\tau$ leptons are simulated for each of the four main production modes: ggF, VBF, $WH$ and $ZH$.
%The Higgs boson cross sections, branching fractions, and uncertainties of the LHC Higgs Working group cross section are used~\cite{DjouadiKalinowski,Bredenstein,Handbook1,Handbook2}. 

The ggF production is simulated at next-to-next-to-leading order (NNLO) accuracy in QCD\footnote{When higher orders (NLO, NNLO) are specified, QCD is implied if not explicitly stated.} using the \POWHEG~NNLOPS program~\cite{Nason:2004rx, Frixione:2007vw, Alioli:2010xd, Hamilton:2013fea, Hamilton:2015nsa}, interfaced with \PYTHIAV{8.212}~\cite{Sjostrand:2014zea} for parton shower and non-perturbative effects.
The simulation achieves NNLO accuracy for arbitrary inclusive $gg\to H$ observables by reweighting the Higgs boson rapidity\footnote{The rapidity is defined in terms of a particle's energy $E$ and momentum in the direction of the beam pipe $p_z$ as $y = \frac{1}{2}\ln\frac{E+p_z}{E-p_z}$.} spectrum in Hj-MiNLO~\cite{Hamilton:2012np, Campbell:2012am, Hamilton:2012rf} to that of HNNLO~\cite{Catani:2007vq}.
%The PDF4LHC15 parton distribution function (PDF) set and the AZNLO tune of \PYTHIAV{8} are used. 
The gluon fusion prediction from the MC samples is normalized to the N$^3$LO cross section in QCD plus electroweak corrections at next-to-leading order (NLO)~\cite{deFlorian:2016spz, Anastasiou:2016cez, Anastasiou:2015vya, Dulat:2018rbf, Harlander:2009mq, Harlander:2009bw, Harlander:2009my, Pak:2009dg, Actis:2008ug, Actis:2008ts, Bonetti:2018ukf}.

VBF events are generated with \POWHEG~\cite{Nason:2004rx, Frixione:2007vw, Alioli:2010xd, Nason:2009ai}, interfaced with \PYTHIAV{8.230}~\cite{Sjostrand:2014zea}~with the dipole recoil option enabled to model the parton shower, hadronisation and underlying event.
The \POWHEG\ prediction is accurate to NLO in QCD corrections and tuned to match calculations with effects due to finite heavy-quark masses and soft-gluon resummations up to next-to-next-to-leading-logarithm (NNLL).
%The PDF4LHC15~\cite{PDF4LHC15} parton distribution function (PDF) set and the AZNLO tune~\cite{Aad:2014xaa} of \PYTHIAV{8} are used. 
The MC prediction is normalized to an approximate-NNLO QCD cross section with NLO electroweak corrections~\cite{PhysRevLett.99.161803, Ciccolini:2007ec, PhysRevLett.105.011801}.

The uncertainties due to the parton shower and hadronisation model for the ggF and VBF Higgs boson signal samples are
evaluated using the events in the nominal sample generated with \POWHEG\ but interfaced to an alternative showering program
\HERWIGV{7}~\cite{Bahr:2008pv,Bellm:2015jjp} instead of \PYTHIAV{8}. To estimate the uncertainty related to the matching between the matrix element and the parton shower for ggF and VBF production, MC events produced with the \MGFiveNLO~\cite{Alwall:2014hca} generator and interfaced to \HERWIGV{7} are used. They are accurate to NLO in QCD corrections and utilize the \texttt{NNPDF30\_nlo\_as\_0118}~\cite{Ball:2014uwa} parton distribution function (PDF) set.
In both cases, the H7UE set of tuned parameters~\cite{Bellm:2015jjp} and the \texttt{MMHT2014LO} PDF set~\cite{Harland-Lang:2014zoa} are used for the underlying event.

Higgs boson production in association with a vector boson ($WH$ and $ZH$, collectively referred to as $VH$) is simulated using \powhegbox~v2~\cite{Nason:2009ai,Alioli:2010xd,Nason:2004rx,Frixione:2007vw} and interfaced with \PYTHIAV{8.212}~\cite{Sjostrand:2014zea} for parton shower and non-perturbative effects.
The \POWHEG\ prediction is accurate to NLO for the production of $VH$ boson plus one jet. Samples for the loop-induced process $gg \to ZH$ are generated with \powhegbox~v2 interfaced to \PYTHIAV{8.235}.
%The PDF4LHC15 parton distribution function (PDF) set and the AZNLO tune of \PYTHIAV{8} are used.
The MC prediction is normalized to cross sections calculated at NNLO in QCD with NLO electroweak corrections~\cite{Ciccolini:2003jy, Brein:2003wg, Brein:2011vx, Denner:2014cla, Brein:2012ne} which includes the $gg \to ZH$ contribution at NNLO.

The ggF, VBF, and $VH$ Higgs boson samples use the \texttt{PDF4LHC15}~\cite{Butterworth:2015oua} PDF set and the AZNLO tune~\cite{STDM-2012-23} of \PYTHIAV{8}. The sample normalizations account for the decay branching fractions calculated with \textsc{HDECAY}~\cite{Djouadi:1997yw, Spira:1997dg, Djouadi:2006bz} and \textsc{PROPHECY4F}~\cite{Bredenstein:2006ha, Bredenstein:2006rh, Bredenstein:2006nk} assuming a Higgs boson mass of $\SI{125.09}{\GeV}$~\cite{HIGG-2014-14}. An uncertainty of 2.16\%~\cite{deFlorian:2016spz} is assigned to the \hww branching fraction, which includes the uncertainty on the Higgs boson mass. All Higgs boson samples are generated with a Higgs boson mass of 125 GeV, with the uncertainty on the Higgs boson mass being negligible for kinematic distributions.

The production of \ttH\ events is modelled using the \powhegbox~v2~\cite{Frixione:2007nw,Nason:2004rx,Frixione:2007vw,Alioli:2010xd,Hartanto:2015uka} generator at NLO with the \nnpdfnlo~\cite{Ball:2014uwa} PDF set.
The events are interfaced to \PYTHIAV{8.230}~\cite{Sjostrand:2014zea}~using the A14 tune~\cite{ATL-PHYS-PUB-2014-021}.
The prediction is normalized to the cross section computed at NLO QCD and NLO EW accuracy~\cite{deFlorian:2016spz}.
% The decays of bottom and charm hadrons are performed by \evtgen~v1.6.0~\cite{EvtGen}.
The sample is inclusive in Higgs decay modes and the cross section is computed assuming a Higgs boson mass of $\SI{125.09}{\GeV}$.

%% Now mentioned before ttH
%All Higgs boson sample normalizations account for the decay branching fractions calculated with \textsc{HDECAY}~\cite{Djouadi:1997yw, Spira:1997dg, Djouadi:2006bz} and \textsc{PROPHECY4F}~\cite{Bredenstein:2006ha, Bredenstein:2006rh, Bredenstein:2006nk}.

\subsubsection{Diboson and multi-boson samples}

To model the SM background, quark-initiated production of $WW$, $WZ$, $V\gamma^{\ast}$, and $ZZ$ involving the strong interaction are simulated with the \SHERPAV{2.2.2}~\cite{Bothmann:2019yzt} generator.
Fully leptonic final states are generated using matrix elements at NLO accuracy in QCD for up to one additional parton and at LO accuracy for up to three additional parton emissions.
For $V\gamma^{\ast}$, the $\gamma^{\ast}$ mass is generated with a lower bound of 4~\GeV.
Samples for the loop-induced processes $gg \to WW$ and $ZZ$ are generated using LO-accurate matrix elements for up to one additional parton emission.

For the quark-initiated $WW$ background, systematic uncerainties are evaluated
via samples simulated with alternative settings for the \SHERPAV{2.2.2}
generator. The uncertainty in the matching procedure is assessed by varying the
parameter $Q_\text{cut}$, which determines the transition between the matrix-element
and parton-shower domain~\cite{Hoeche:2009rj}. Specifically, alternative
samples with $Q_\text{cut}=15$ and $\SI{30}{\GeV}$ instead of the nominal value of
$Q_\text{cut}=\SI{20}{\GeV}$ are considered. The uncertainties in the shower
model are estimated via samples, in which either the resummation scale,
$\mu_\text{q}$, is increased or decreased by a factor of 2, or the alternative
recoil scheme described in Refs.~\cite{Schumann:2007mg,Hoeche:2009xc} is used.

The production of $V\gamma$ final states is simulated with the \SHERPAV{2.2.8}~\cite{Bothmann:2019yzt} generator.
Matrix elements are at NLO QCD accuracy for up to one additional parton and at LO accuracy for up to three additional parton emissions.

Triboson production ($VVV$) is simulated with the
\SHERPAV{2.2.2}~\cite{Bothmann:2019yzt} generator using factorised gauge-boson
decays. The matrix elements are accurate to NLO for the inclusive process and
to LO for up to two additional parton emissions.

For all nominal multi-boson samples generated with \SHERPA, the matrix element calculations are matched and merged with the \SHERPA parton shower based on Catani-Seymour dipole factorization~\cite{Gleisberg:2008fv,Schumann:2007mg} using the MEPS@NLO prescription~\cite{Hoeche:2011fd,Hoeche:2012yf,Catani:2001cc,Hoeche:2009rj}.
The virtual QCD corrections are provided by the \openloops\ library~\cite{Cascioli:2011va,Denner:2016kdg}.
The \nnpdfnnlo set of PDFs is used, along with the dedicated set of tuned parton-shower parameters developed by the \SHERPA authors.


\subsubsection{Electroweak samples}
Electroweak $WW$ production in association with two jets ($WWjj$) is
generated using \MGFiveNLO~\cite{Alwall:2014hca} with LO matrix elements using the \texttt{NNPDF2.3LO} PDF set.
For the nominal sample, \MGFiveNLO\ is interfaced with \PYTHIAV{8.244} using the
A14 tune to model the parton shower, hadronisation, and underlying event. An
alternative sample is utilized to evaluate the shower model uncertainty, for
which \MGFiveNLO\ is instead interfaced with \HERWIGV{7}.

Electroweak production of $\ell\ell jj$ final states are generated with \SHERPAV{2.2.1}, but using leading
order (LO) matrix elements with up to two additional parton emissions.

\subsubsection{\Zjets samples}

The quark- or gluon-initiated production of \Zjets is simulated with the \SHERPAV{2.2.1}~\cite{Bothmann:2019yzt} generator using NLO matrix elements for up to two partons, and LO matrix elements for up to four partons calculated with the Comix and \openloops libraries.
They are matched with the \SHERPA parton shower~\cite{Schumann:2007mg} using the MEPS@NLO prescription.
The \nnpdfnnlo set of PDFs is used and the samples are normalized to an NNLO prediction~\cite{Anastasiou:2003ds}.

An additional sample of \Zjets events is made with the
\MGFiveNLO~\cite{Alwall:2014hca} generator to evaluate the matching
uncertainty. The accuracy of the matrix elements is at LO for up to four
final-state partons. The \texttt{NNPDF2.3LO} PDF set~\cite{Ball:2012cx} is
used. Events are interfaced to \PYTHIAV{8.186}~\cite{Sjostrand:2007gs} with the
A14 tune~\cite{ATL-PHYS-PUB-2014-021} to model the parton shower,
hadronisation, and underlying event.
%The overlap between matrix element and parton shower emissions was removed using the CKKW-L merging procedure~\cite{Lonnblad:2001iq,Lonnblad:2011xx}. The A14 tune~\cite{ATL-PHYS-PUB-2014-021} of \PYTHIA[8] was used with the \NNPDF[2.3lo] PDF set~\cite{Ball:2012cx}. The decays of bottom and charm hadrons were performed by \EVTGEN[1.2.0]~\cite{Lange:2001uf}. The $V$+jets samples were normalized to a next-to-next-to-leading-order (NNLO) prediction~\cite{Anastasiou:2003ds}.

\subsubsection{Top-quark samples}

The production of \ttbar\ events is modelled using the \powhegbox~v2 generator at NLO with the \nnpdfnlo PDF set and the \hdamp\ parameter\footnote{The \hdamp\ parameter is a resummation damping factor and one of the parameters that controls the matching of \Powheg\ matrix elements to the parton shower and thus effectively regulates the high-\pt\ radiation against which the \ttbar\ system recoils.} set to $1.5\times$\mtop~\cite{ATL-PHYS-PUB-2016-020}.
In order to correct for a known mismodelling of the leading lepton \pt due to missing higher-order corrections, an NNLO reweighting is applied to the sample~\cite{NNLOReweighting}.
The events are interfaced to \PYTHIAV{8.230} to model the parton shower, hadronisation, and underlying event, with parameters set according to the A14 tune and using the \nnpdftwo set of PDFs~\cite{Ball:2012cx}.
% \input{top_isrfsr_powpy_short} describes the estimation of systematic uncertainties for ttbar and single top, to be used later? JONAS 

The associated production of top quarks with $W$ bosons (mainly $Wt$) is modelled using the \powhegbox~v2 generator at NLO in QCD using the five-flavor scheme and the \nnpdfnlo set of PDFs.
The diagram removal scheme~\cite{Frixione:2008yi} is used to remove interference and overlap with \ttbar\ production.
The events are interfaced to \PYTHIAV{8.230} using the A14 tune and the \nnpdftwo set of PDFs.
The decays of bottom and charm hadrons are performed by \evtgen~v1.6.0.
For all samples generated with \powhegbox~v2, the decays of bottom and charm hadrons are performed by \evtgen~v1.6.0.

\subsubsection{Single boson and multijet processes}

The \Wjets and multijet backgrounds are estimated from data. Generated samples of \Wjets and $Z$+jets events are used to validate the estimate and to determine the flavor composition uncertainties. These MC samples are generated using \POWHEG interfaced with \PYTHIAV{8.186}, with \SHERPAV{2.2.1}~\cite{Bothmann:2019yzt}, and with \MGFiveNLO~\cite{Alwall:2014hca,Frederix:2012ps} interfaced with \PYTHIAV{8.186}.

The MC generators, PDFs, and programs used for the underlying event and parton shower (UEPS) are summarized in Table~\ref{tab:mcsamples}.
The alternative generators or UEPS models used to estimate systematic uncertainties are also listed in parentheses.
Finally, the orders of the perturbative prediction for each sample are reported.


%A summary of the simulated signal and background samples is shown in Table~\ref{tab:mcsamples}.

\begin{table}[h]
    \centering
    \caption{
      Overview of simulation tools used to generate signal and background processes, as well as to model the UEPS. The PDF sets are also summarized.
      Alternative event generators or quantities varied to estimate systematic uncertainties are shown in parentheses.}
    \label{tab:mcsamples}
  \scalebox{0.66}{
    \begin{tabular}{l l l l l}
    \hline\hline
    %%% Info mostly from https://gitlab.cern.ch/atlas-physics/pmg/documents/references/-/tree/master
    Process              & Matrix element                                              & PDF set                 & UEPS model                                           & Prediction order\\
                         & (alternative)                                               &                         & (alternative model)                                  &  for total cross section\\
    \hline
    ggF $H$              & \POWHEGBOXV{v2}~\cite{Hamilton:2013fea,Hamilton:2015nsa,Alioli:2010xd,Nason:2004rx,Frixione:2007vw}
    & \multirow{2}{*}{\texttt{PDF4LHC15 NNLO}~\cite{Butterworth:2015oua}} &\multirow{2}{*}{\PYTHIAV{8}~\cite{Sjostrand:2014zea}} & \multirow{2}{*}{N$^{3}$LO QCD + NLO EW~\cite{deFlorian:2016spz,Anastasiou:2016cez,Anastasiou:2015vya,Dulat:2018rbf,Harlander:2009mq,Harlander:2009bw,Harlander:2009my,Pak:2009dg,Actis:2008ug,Actis:2008ts,Bonetti:2018ukf}} \\
                         & NNLOPS~\cite{Nason:2009ai,Hamilton:2013fea,Campbell:2012am} &                         &    & \\
                         & (\MGFiveNLO)~\cite{Alwall:2014hca,Frederix:2012ps}          &                         & (\HerwigV{7})~\cite{Bellm:2015jjp} & \\
    VBF $H$              & \POWHEGBOXV{v2}~\cite{Nason:2009ai,Alioli:2010xd,Nason:2004rx,Frixione:2007vw}
                         & \texttt{PDF4LHC15 NLO}  & \PYTHIAV{8}        & NNLO QCD + NLO EW~\cite{Ciccolini:2007jr,Ciccolini:2007ec,Bolzoni:2010xr} \\
                         & (\MGFiveNLO)                                                &                         & (\HerwigV{7})                                        & \\
    $VH$ excl. $gg\to ZH$ & \POWHEGBOXV{v2}                                            & \texttt{PDF4LHC15 NLO}  & \PYTHIAV{8}  & NNLO QCD + NLO EW~\cite{Ciccolini:2003jy,Brein:2003wg,Brein:2011vx,Denner:2014cla,Brein:2012ne} \\
    \ttH                 & \POWHEGBOXV{v2}                                             & \texttt{NNPDF3.0NLO}    & \PYTHIAV{8}               & NLO~\cite{deFlorian:2016spz} \\
    $gg\to ZH$           & \POWHEGBOXV{v2}                                             & \texttt{PDF4LHC15 NLO}  & \PYTHIAV{8}               & NNLO QCD + NLO EW~\cite{Altenkamp:2012sx,Harlander:2014wda} \\
  
    $qq \to WW$          & \SHERPAV{2.2.2}~\cite{Bothmann:2019yzt}                     & \texttt{NNPDF3.0NNLO}~\cite{Ball:2014uwa} & \SHERPAV{2.2.2}~\cite{Gleisberg:2008fv,Schumann:2007mg,Hoeche:2011fd,Hoeche:2012yf,Catani:2001cc,Hoeche:2009rj} & NLO~\cite{Buccioni:2019sur,Cascioli:2011va,Denner:2016kdg} \\
                         & ($Q_\text{cut}$)                                            &                         & (\SHERPAV{2.2.2}~\cite{Schumann:2007mg,Hoeche:2009xc}; $\mu_\text{q}$)  \\
  %% not used              & (\POWHEGBOXV{v2},                                           &                         & \multirow{2}{*}{(\Herwigpp~\cite{Bellm:2015jjp})}  \\
  %% not used              & \MGFiveNLO)                                                 &                         &  \\
  % used                 & (CKKW)                                                      &                         & (QSF/CSSKIN)  \\
    $qq \to WWqq$        & \MGFiveNLO~\cite{Alwall:2014hca}                             & \texttt{NNPDF3.0NLO}    & \PYTHIAV{8}                                      & LO \\
                         &                                                             &                         & (\HerwigV{7})                                        & \\
  $gg \to WW/ZZ$         & \SHERPAV{2.2.2}                                             & \texttt{NNPDF3.0NNLO}   & \SHERPAV{2.2.2}                                      & LO~\cite{Caola:2015rqy}  \\
  %$WZ/V\gamma^{\ast}/ZZ \to \ell\ell\ell\ell/\ell\ell\ell\nu$ & \SHERPAV{2.2.2}        & \texttt{NNPDF3.0NNLO}   & \SHERPAV{2.2.2}                                      & NLO~\cite{Cascioli:2013gfa} \\
  %Other $WZ/V\gamma^{\ast}/ZZ$ & \POWHEGBOXV{v2}                                       & CT10                    & \PYTHIAV{8}                                          & NLO~\cite{Cascioli:2013gfa} \\
  $WZ/V\gamma^{\ast}/ZZ$ & \SHERPAV{2.2.2}                                             & \texttt{NNPDF3.0NNLO}   & \SHERPAV{2.2.2}                                      & NLO~\cite{Cascioli:2013gfa} \\
  %% not used            & (CKKW)                                                      &                         & (CSS variation~\cite{Schumann:2007mg})\\
    $V\gamma$            & \SHERPAV{2.2.8}~\cite{Bothmann:2019yzt}                     & \texttt{NNPDF3.0NNLO}   &\SHERPAV{2.2.8}                                       & NLO~\cite{Cascioli:2013gfa} \\
  $VVV$                  & \SHERPAV{2.2.2}                                             & \texttt{NNPDF3.0NNLO}   &\SHERPAV{2.2.2}                                       & NLO  \\
  %% not used            & (\MGFiveNLO)                                                &                         & (CSS variation~\cite{Schumann:2007mg,Hoeche:2009xc}) & \\
    $t\bar{t}$           & \POWHEGBOXV{v2}                                             & \texttt{NNPDF3.0NLO}    & \PYTHIAV{8}                                          & NNLO+NNLL~\cite{Beneke:2011mq,Cacciari:2011hy,Baernreuther:2012ws,Czakon:2012zr,Czakon:2012pz,Czakon:2013goa,Czakon:2011xx} \\  %~\cite{Frixione:2007nw}-~\cite{ATL-PHYS-PUB-2014-021}] \\
                         & (\MGFiveNLO)                                                &                         & (\HerwigV{7})                                        & \\
   $Wt$                  &\POWHEGBOXV{v2}                                              & \texttt{NNPDF3.0NLO}    & \PYTHIAV{8}                         & NNLO~\cite{Kidonakis:2010ux,Kidonakis:2013zqa} \\
                         & (\MGFiveNLO)                                                &                         & (\HerwigV{7})                                          & \\
    $Z/\gamma^{\ast}$    & \SHERPAV{2.2.1}                                             & \texttt{NNPDF3.0NNLO}   & \SHERPAV{2.2.1}                                      & NNLO~\cite{Anastasiou:2003ds} \\
                         & (\MGFiveNLO)                                                & \\
  \hline\hline
    \end{tabular}
  }
  \end{table}
  