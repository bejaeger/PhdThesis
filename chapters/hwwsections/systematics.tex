This analysis is affected by several sources of systematic uncertainties.
They can be grouped into \emph{experimental uncertainties} and \emph{theoretical uncertainties}.
The experimental uncertainties are related to the detector performance and associated to the identification, reconstruction, and trigger efficiencies, as well as the limited knowledge of the scale and resolution of energy and momentum measurements.
The theoretical uncertainties for both the signal and background processes arise from the assumptions made in the MC simulations as well as theoretical cross-section calculations. 
% In a multi-category analysis, the sign and magnitude of these experimental uncertainties are calculated separately in each category.
All uncertainties are taken into account by constructing respective varied MC templates for the statistical model and using them to define the corresponding nuisance parameters, as described in \cref{chap:statistics}.

\subsection{Experimental uncertainties}
Experimental uncertainties are typically derived in separate calibration measurements, each of which provides a set of sometimes many sources of uncertainty to be considered in physics analyses.
For example, the uncertainties related to the jet energy resolution arise from different sources such as pile-up conditions, method non-closures, etc., which is explained in detail in \cref{chap:calibration}.
% More details on the uncertainties related to jets, in particular the jet energy resolution, are provided in \cref{chap:calibration}. 
In the statistical analysis, the experimental uncertainties are constraint in the likelihood by Gaussian functions with widths corresponding to the nominal uncertainties derived.
The experimental uncertainties can be further divided into uncertainties affecting the entire event, referred to as \emph{scale-factor} (SF) uncertainties, and uncertainties specific to individual physics objects, referred to as \emph{four-vector} (P4) uncertainties.
While the former is applied to the events as a flat scale factor and thus cannot change the shape of the final discriminant, the latter can cause events to migrate between different histogram bins and thus induce a shape difference between the nominal and the varied template.
A complete list of experimental uncertainties considered is provided in \cref{tab:exp-uncertainties}, including references to the relevant calibration measurements where applicable.
The uncertainty on the luminosity is obtained using the LUCID-2 detector~\cite{ATLAS-CONF-2019-021} and only applied to processes not estimated with CRs. 
The uncertainty on the pile-up reweighting procedure is estimated by varying the scale factor used for the reweighting. 
Uncertainties on the physics objects included in the \MET calculation, that is, electrons, muons, and jets, are propagated to the \MET object. In addition, specific \MET uncertainties are included to account for uncertainties of the \MET soft term.
The uncertainties on the jet energy resolution correspond to the ones discussed in \cref{chap:calibration}. 

% From MASTER thesis:
% Experimental uncertainties are usually derived externally by dedicated calibration measurements, based on actual data measurements. The constraint terms for the introduced nuisance parameters are represented by Gaussians with a width of the provided nominal uncertainty. Since the subsidiary measurements are generally based on many events, this assumption is supported by the central limit theorem.
The uncertainties associated to the background with misidentified leptons are also grouped into the experimental uncertainties. They enter via the uncertainties derived for the fake factors, as explained in \cref{subsec:misid-bkg}, and thus only impact the normalisation of the statistical templates. They are discussed in \cref{subsec:misid-bkg}.
% NOT SURE IF I SHOULD DO THE FOLLOWING! START WITHOUT, maybe reconsider
% - Discuss some of the largest SHAPE uncertainties in VBF? 
%     -> I studied it quite a bit so could definitely add some content. 
%     -> In results section?
% A review of all the values of the uncertainties would go beyond the scope of this thesis and luckily many of them are negligible for the final results. The following highlights the most important experimental uncertainties for the VBF analysis, to which the author of this thesis contributed. 
% QUESTIONS:
% - Mention NP names!?
% Ralf's thesis: Nope
% Carsten's thesis: Yep
% Hannah Arnold: Yep
% Dickinson: Nope and very short. Includes 'constraint type' when mentioning systematics
% From paper:
% The largest sources of experimental uncertainties affecting the ggF measurement are from the 푏-jet 517 identification, the pile-up modelling, the jet energy resolution, and the Mis-Id background estimate. For the 518 VBF measurement, the largest source of experimental uncertainty comes from the 퐸miss T reconstruction.
\begin{table}[ht]
    \begin{center}
        \begin{tabular}{l c c}
    \toprule
    Source & Number of NPs & Type \\
    \midrule
    \textbf{Event} & & \\
    \midrule
    Luminosity \cite{ATLAS-CONF-2019-021} & 1 & SF \\
    Pileup reweighting & 1 & SF \\
    \midrule
    \textbf{Electrons} \cite{EGAM-2018-01} & & \\
    \midrule
    Trigger \cite{TRIG-2018-05} & 1 & SF \\
    Reconstruction & 1 & SF \\
    Identification & 35 & SF \\
    Isolation & 1 & SF \\
    Energy scale & 1 & P4 \\
    Energy resolution & 1 & P4 \\
    \midrule
    \textbf{Muons} \cite{MUON-2018-03} & &  \\
    \midrule
    Trigger \cite{TRIG-2018-01} & 2 & SF \\
    Reconstruction & 2 & SF \\
    %Identification & 2 \\
    Track-to-vertex association & 2 & SF \\
    Isolation & 2 & SF \\
    Energy scale & 3 & P4 \\
    Energy resolution & 2 & P4 \\
    \midrule
    \textbf{Jets} \cite{JETM-2018-05} & & \\
    \midrule
    Energy scale (JES) & 28 & P4 \\
    Energy resolution (JER) & 13 & P4\\
    JVT efficiency \cite{ATLAS-CONF-2014-018} & 1 & SF \\
    $b$-tagging \cite{FTAG-2018-01} & 12 & SF \\
    \midrule
    \pmb{$E_{\rm T}^{\mathrm{miss}}$} \cite{PERF-2016-07} & & \\
    \midrule
    Energy scale of soft term & 2 & P4 \\
    Energy resolution of soft term & 2 & P4 \\
    \bottomrule
\end{tabular}

    \end{center}
    \caption{Summary of experimental uncertainties considered, including their total number of nuisance parameters (NPs) and a specification whether they represent scale-factor (SF) uncertainties or four-vector (P4) uncertainties.
    }
    \label{tab:exp-uncertainties}
\end{table}



\subsection{Theoretical uncertainties}
The nominal MC simulations for both the signal and background processes rely on a set of assumptions that must be made because of the imprecise knowledge of various effects. Assumptions include the choice of renormalisation and factorisation scale, the choice of PDFs, the value of experimentally measured parameters like $\alpha_s$, or assumptions about non-perturbative effects such as the parton showering, hadronization, and underlying event model. 
The uncertainties arising from such assumptions are determined either by directly varying parameters within the nominal MC simulation and comparing the predictions, or by comparing the nominal predictions with alternative simulations. 
The uncertainties can affect both the overall normalisation of the signal and background processes and the shape of the final discriminant.  
% Imprecise knowledge of the parton showering process and underlying event, missing higher order calculations, and uncertainties in the strong coupling constant $\alpha_s$ as well as the PDFs. In some cases, further process-specific uncertainties are taken into account.
% - Derived by 
% - Affect normalisation and shape of final discriminant

\subsubsection{Signal uncertainties}
Uncertainties affecting the signal processes can be separated into two types: uncertainties that only affect the predicted SM cross section and thus only impact the overall normalisation of the process, and uncertainties arising from the acceptance effects due to the limited detector coverage or analysis selections. The latter can also impact the shape of the final discriminants. 
In the case of cross-section measurements, only acceptance uncertainties must be included in the statistical model. 
For the measurement of signal strengths, both uncertainty components are included. 

For both the ggF and VBF signal process, the uncertainties are estimated that arise from the choice of PDF, including the uncertainties on $\alpha_s$, the choice of parton shower model, and matrix element matching.
The PDF uncertainty is estimated by taken the standard deviation of in total 30 independent Hessian-reduced error sets corresponding to the PDFLHC15 set \cite{Butterworth:2015oua}. 
The parton shower and matrix element matching uncertainty is derived from the sample comparisons indicated in \cref{tab:mcsamples}.
Cross-section uncertainties arising from both missing higher-order contributions in the cross-section calculations and migration effects on the ggF cross sections in different \Njets categories are estimated according to the descriptions provided in \ccite{deFlorian:2016spz,Stewart:2011cf,Stewart:2013faa,Liu:2013hba,Boughezal:2013oha,Gangal:2013nxa}.
This includes uncertainties from the choice of factorization and renormalization scales, the choice of resummation scales, and the ggF migrations between the \ZeroJet and \OneJet regions or between the $N_{\text{jet}} \ge 1$  and \TwoJet categories.
For the ggF migration uncertainties in the \TwoJet categories the so-called Steward Tackmann procedure is used, where the inclusive cross sections in categories $N_{\text{jet}} \ge N$ and $N_{\text{jet}} \ge N+1$ are assumed to be independent and therefore the uncertainty on the exclusive cross section in the $N_{\text{jet}} = N$ can be estimated by the sum in quadrature of the uncertainties of the inclusive cross sections.
Furthermore, an uncertainty of 2.2\% is applied to the signal cross sections assumed, to account for the uncertainty on the \HWW branching fraction. 
%OR THIS: where uncertainties in exclusive jet bins are estimated based on scale variations in inclusive bins.

\paragraph{STXS signal uncertainties}
% Same uncertainties but for each truth bin:
% ONLY cross-section uncertainties, from INT NOTE: As such, the samples or weight variations used to evaluate the signal uncertainties 2509 must be normalized to the same cross-section at truth-level in each STXS bin, prior to examining their 2510 effects on the yield in signal region. 
The sources of uncertainties on the signals split into different STXS bins are the same as for the inclusive ggF and VBF signal processes. 
To ensure that only acceptance uncertainties are considered, the samples used to evaluate the uncertainties are normalized in each STXS bin to the same theoretical cross-sections, prior to comparing their effects on the yields and shapes in the STXS SRs. 
Dedicated QCD scale uncertainties are used for the STXS signal samples that specifically cover uncertainties due migrations between different STXS bins. 

% From paper
%For signal processes, the approach described in Refs. [11, 124] is used for estimating the variations due 528 to the impact of higher-order contributions not included in the calculations and migration effects on the 529 푁jet ggF cross sections. In particular, the uncertainty from the choice of factorization and renormalization 530 scales, the choice of resummation scales, and the ggF migrations between the 0-jet and 1-jet phase-space 531 bins or between the 1-jet and ≥ 2-jet bins are considered [11, 125–128].
%from renormalization and factorization scale choices (known as \emph{QCD scale} uncertainties),

% Dickinson
% Uncertainties on the predicted Standard Model cross section of process p are not included on the measurement of cross section p. However, these uncertainties are included on measurements of the signal strength μp, since this result parameterizes the cross section in terms of the SM prediction. In addition, if a process q 6= p is not measured simultaneously, but is fixed to the SM expectation, then theory uncertainties on the inclusive SM q are included on the measurement of both p and μp.
% Migrationuncertaintiesareincludedonmeasurements of both p and μp.
\subsubsection{Background uncertainties}
%The following provides an overview of the estimation of the theoretical uncertainties underlying the background predictions. 
% \paragraph{Common uncertainty estimates between background processes}
%The method used to estimate some background uncertainties are common for different processes. 
The uncertainties related to the various background estimates are derived in a similar way as the uncertainties for the signals.  
\paragraph{Common methods} The same methods are used for different processes to derive some of the background uncertainties. 
%Some background uncertainties are estimated with the same methods for different processes. 
%They are referenced in the paragraphs below, that provide specifics on the uncertainties of different processes, and include:
\begin{itemize}
    \item Uncertainties arising from the choice of the nominal renormalisation ($\mu_R^\text{nom}$) and factorisation scales ($\mu_F^\text{nom}$), together denoted \emph{QCD scales}, are estimated using both coherent and incoherent variations $\{\mu^{\text{var}}_R , \mu^{\text{var}}_F \} = \{v_R, v_F \} \times \{\mu^{\text{nom}}_R , \mu^{\text{nom}}_F \}$ with $v_R, v_F = 0.5, 1, 2$ excluding the combinations $\{v_R, v_F \} = \{0.5, 2\}, \{2, 0.5\}$. 
    The maximum of these 7 different variations are taken as the total scale uncertainty.
    %his uncertainty is referred to as \emph{7-point QCD scale} uncertainty.
    \item The uncertainty associated to the choice of using the \nnpdfnnlo (or \nnpdfnlo) ~\cite{Ball:2014uwa} set of PDFs is estimated by the standard deviation of the predictions of 100 \nnpdfnnlo (or \nnpdfnlo) replicas.
    % \begin{equation}
    %     \sigma_{\text{PDF}} = \sqrt{ \frac{1}{99} \sum_{i=1}^{100} \left(  n^\text{var}_i - \mu \right) }
    % % \text{Error}^{R}_{J}(PDF)=\sqrt{\frac{1}{99}\sum_{i=1}^{100}\left(\left.\frac{NSR^R_J}{NCR_J}\right|_i-\left\langle\frac{NSR^R_J}{NCR_J}\right\rangle\right)^2}.
    % \end{equation}
    \item The uncertainty associated to the strong coupling constant, $\alpha_s$, is estimated by taking the symmetrized average of the expected event count when varying $\alpha_s$ up and down, 
     \begin{equation}
        \sigma_{\alpha_s}^\text{up/down} = 1 \pm 0.5 \frac{ \text{max}(N^\text{up},N^\text{down}) - \text{min}(N^\text{up},N^\text{down})}{N^\text{nom}}
        % \text{Error}^{R}_{J}(\alpha_s)=\frac{1}{2}\left[\frac{NSR^R_J(\alpha_s^{\text{high}})}{NCR_J(\alpha_s^{\text{high}})}-\frac{NSR^R_J(\alpha_s^{\text{low}})}{NCR_J(\alpha_s^{\text{low}})}\right].
    \end{equation}
    where $N^\text{nom}$ are the nominally expected yields and $N^\text{up}$ and $N^\text{down}$ are the expected yields when $\alpha_s$ is varied up and down, respectively.
    \item The uncertainty due the choice of the parton shower model is estimated by comparing the nominal MC simulations with an alternative simulation setup as indicated in \cref{tab:mcsamples}. 
    % \item The uncertainty due to the choice of the matching procedure between the matrix element and parton shower is also estimated by comparing the nominal MC simulations with an alternative simulation setup as indicated in \cref{tab:mcsamples}.  
    % \item Shower uncertainty based on comparison between alternative shower models (usually \textsc{Pythia8} vs \textsc{Herwig7}) indicated in \cref{tab:mcsamples}.
    % \item Matching uncertainty based on comparison between different matrix element generators as indicated in \cref{tab:mcsamples}.
\end{itemize}

\noindent The following provides details on the specific uncertainties considered for each background process. 
\paragraph{$qq \to WW$} 
The uncertainties due to the QCD scale choices, the choice of PDF, and $\alpha_s$ are estimated as discussed above.
% Uncertainties due to scale choices are estimated with the 7-point QCD scale uncertainty and the PDF uncertainty with the set of 100 NNPDF3.0NLO replicas as explained above. 
% PDF uncertainties:
% The PDF and $alpha_s$ uncertainties are evaluated from the standard deviation of 100 NNPDF3.0NLO replicas. 
% The PDF replicas and alphahigh,low s are included as alternative weights in the nominal \Sherpa samples.
The uncertainty due to the choice of parton shower model is covered by two separate comparisons based on generator-level samples.
The first accounts for the uncertainty due to the resummation scale choice (denoted \emph{QSF} uncertainty) in the nominal sample that determines the matching of the NLO matrix elements with the parton shower. 
To assess the uncertainty, the resummation scale is varied by a factor of two up and down.
In the second comparison, the momentum recoil is evaluated differently in the parton shower provided by the \Sherpa generator (\emph{CSSKIN} uncertainty). The strategy used in the nominal MC simulation is described in \cite{Hoeche:2009rj}, the alternative approach is detailed in \cite{Schumann:2007mg}. 
The nominal \Sherpa sample uses a cut at $Q=20\,$GeV to separate the phase spaces covered by the matrix element calculations and the parton shower. This scale is arbitrary and the associated uncertainty due to that choice is estimated by varying it up and down such that $Q = 15\,\GeV$ and $Q = 30\,\GeV$ (\emph{CKKW} uncertainty). The uncertainty is estimated based on generator-level sample comparisons.
% \todo{Not sure if this is maybe too much detail?}
% \todo{What's the difference between resummation and matching/merging uncertainty?}
% \todo{Introduce abbreviations, QSF, CSSKIN, CKKW?}

% QSF
% - Resummation scale (QSF)
% -> Take larger variation
% % CSSKIN
% - Parton shower recoil scheme

% CKKW
% - Merging scale (CKKW)
% -> Take larger variation
%The QSF, CSSKIN, and CKKW uncertainty are all estimated using generator-level samples.

\paragraph{$gg \to WW$} 
The $gg \to WW$ process contributes around 10\% to the total $WW$ background in most signal regions. The \Sherpa sample used is generated only at leading order precision. For the ggF and VBF \TwoJet categories, a conservative uncertainty of $+100/-50$\% is therefore used. For the ggF \OneJet and \TwoJet categories, the NLO scale uncertainties using the latest theoretical predictions as shown in \ccite{Grazzini_2020} are used.

\paragraph{EW $qq \to WWqq$}
%Uses MadGraph+Pythia8 at reco level with the following uncertainties derived.
The uncertainties from the choice of QCD scales, choice of PDF, and $\alpha_s$ are estimated as described above.
% QCD scale:
%For this LO sample, varying the μR (renormalization) scale has no effect because the ME does not include 2342 any powers of αs, but varying the factorization scale μF is expected to give a conservative uncertainty 2343 band that covers differences between the LO and NLO predictions (see, for example, Figure 3 of [121])
The shower uncertainty is estimated by comparing the nominal \MADGRAPH+\PythiaEight sample with a sample that interfaces \MADGRAPH and \HerwigV{7}.
% Shower
% MadGraph+Pythia8 vs MadGraph+Herwig7 
An additional uncertainty due to the choice of electroweak scale and the missing higher orders are estimated based on the leading-log approximation \cite{Denner_2019}, yielding an additional normalisation uncertainty of $\pm 15$\%. 

% EW scale (alpha)
% Int note: The main effect of missing EW scales is expected to come from the NLO EW correction.
% Paper: he EW 푊푊 process, which contributes most significantly in the highest VBF DNN bin, 538 is assigned an additional normalization uncertainty of 15% due to NLO EW corrections, as calculated 539 using the leading-log approximation [129].

% PDF+alpha_s 
% SAME AS qqWW!
% IntNote: 100 NNPDF replicas around 2335 the nominal. Similarly, the effect of αs was evaluated using three replicas with different settings of αs.


\paragraph{Background with top quarks}
The uncertainties arising from the choice of parton shower model, the choice of QCD scales, and choice of PDF, is estimated as described above. 
% Shower uncertainty
% Powheg+Pythia8 and Powheg+Herwig7
% QCD scale with 7-point envelope
% LIKE qqWW!
% PDF uncertainty with 100 NNPDF3.0 replicas (MISSES ALPHA S!?)
The uncertainty due to the matching procedure between matrix element calculations and the parton shower are estimated by comparing the predictions of MC simulations based on \Powheg+\PythiaEight and \aMCATNLO+\PythiaEight. 
The DNN output distributions for both of these samples are shown in \cref{fig:dnn:smoothing}. 
% Matching uncertainty
% Int Note: The difference between the prediction of Powheg+Pythia8 and the aMCAtNlo+Pythia8 sample is taken as a matching uncertainty between 
In addition, uncertainties are considered due to additional initial state (ISR) or final state radiation (FSR). The parameters determining the amount of ISR and FSR are varied up and down.
% Initial State Radiation (ISR) and Final State Radiation (FSR)
% -> missing hpdamp factor!?
For the $Wt$ process only, an additional uncertainty is considered due to interference effects between processes with single top quarks and $t\bar{t}$ processes. This uncertainty uses an alternative diagram removal scheme \cite{Frixione:2008yi}. %(DR instead of DS scheme) . 
% Single Top Interference (Wt only)


\paragraph{\Ztautau background}
% QCD scale: 7-point envelope
% LIKE qqWW
The uncertainty due to the choice of QCD scales, the choice of PDF, and  $\alpha_s$ is estimated as described above. 
% PDF: NNPDF3.0 100 NNPDF replicas
% alpha_s: two NNPDF3.0nnlo αs variations.
The nominal \SHERPAV{2.2.1} generator handles the simulation of the matrix element and parton shower with a custom tune. To estimate the uncertainty arising from that choice the nominal sample is compared to a sample generated with \textsc{MadGraph5\_aMC@NLO+Pythia8} and the A14 tune. 

% GENERATOR:
% INT NOTE: \Sherpa 2.2.1 handles simulation of the matrix element and parton shower with a custom tune. Uncertainty 2447 associated with this choice of generator is evaluated by comparing with MadGraph5_aMC@NLO, which 2448 is interfaced to Pythia8 with the A14 tune for modelling of the parton shower and underlying event.

\paragraph{Other backgrounds}
The overall normalisation of the $V\gamma$ sample is assigned an uncertainty of $+100/-50\%$ to account for potential mismodelling of this background. For Other $VV$ and triboson processes, a 12\% uncertainty is applied for similar reasons. 
% Int note: A flat -50%/+100% uncertainty on the overall normalization of V γ processes is applied to account for 2494 potential mismodelling of this background. For Non-WW diboson processes, a 12% normalization 2495 uncertainty is applied. This uncertainty was evaluated for the Z+jets fake factor estimate (section 7.2) 2496 and includes contributions from QCD scale, PDF and αS. This uncertainty is also applied for triboson 2497 processes.



% \subsection{Discussion of most impactful uncertainties}

% - THINK ABOUT ADDING THIS SECTION!
% - MIGHT INDEED BE RELEVANT 
% - COULD ALSO ADD THIS TO THE END OF RESULTS SECTION or an IMPROVEMENTS SECTION
