% From Hannah Arnold Thesis
% - using the HistFactory tool [326] and the RooStats/ RooFit framework [327, 328].

The VBF and ggF signals are measured using a binned likelihood fit of the MC template to data in the SRs and CRs. The uncertainties are incorporated as NPs in the likelihood. The procedure is explained in detail in \cref{chap:statistics}. 
This section first provides details on the statistical model, including the treatment of uncertainties, and discusses how the likelihood fit is validated.

\subsection{Background templates}
All background processes discussed in \cref{sec:bkg-estimation} are included in the likelihood. 
The normalization of processes for which CRs are defined is controlled by freely floating NFs in the fit. All other processes are normalized to their respective theoretical cross sections.

\subsection{Cross-section fit inputs}
The signal cross sections are parametrized as signal strengths in the likelihood and extracted from a simultaneous fit to the data in multiple regions.
%depending on the target of the fit, for example,
% \begin{equation}
%     \label{eq:incl-ggF-VBF-signal-strengths}
%     \mu_{\text{ggF/VBF}} = \frac{ [ \sigma_{\text{ggF/VBF}}  \times \BR(H \to WW^*) ]_\text{meas} } { [ \sigma_{\text{ggF/VBF}} \times \BR(H \to WW^*) ]_\text{SM}},
% \end{equation}
For the measurement of the inclusive ggF and VBF production cross sections a fit is performed to all nominal SRs and CRs in the \ZeroJet, \OneJet, and \TwoJet categories, as defined in \cref{sec:event-categorization,sec:bkg-estimation}. 
The ggF and VBF cross sections are the two unconstrained PoIs, parametrized as shown in \cref{eq:incl-ggF-VBF-signal-strengths}.
The same regions are used for the measurement of the combined ggF and VBF cross section, using a single PoI to scale the ggF and VBF yields. 
For the measurement of 11 STXS cross sections, all STXS SRs and CRs are used as defined in \cref{subsec:STXS-categorization,subsubsec:stxs-crs}.

The final fit discriminant in the ggF regions is the \mT distribution. The same binning is used in all regions: [0-90, 90-100, 100-110, 110-120, 120-130, 130-$\infty$]. 
The VBF region uses the DNN distribution as final discriminant with a binning as described in \cref{subsec:fina-model-validation}.

\TDinote{}{Make graphic of all regions included in fits}
% From Paper
%The cross sections for the ggF and VBF production modes are determined in a simultaneous fit to all 574 nominal SRs and CRs in the 푁jet = 0, 푁jet = 1, and 푁jet ≥ 2 categories. The ggF and VBF cross sections are 575 the two unconstrained POIs in this fit. 
%A second fit is performed using these same regions, but measuring a 576 single POI for the combined ggF and VBF yield.
%In both fits, the other Higgs production modes are fixed 577 to their expected yields. 
%A third fit is made to all the STXS regions, where the 11 cross sections measured 578 are POIs. No nuisance parameters are significantly pulled or constrained in any of the fits.

\subsection{Treatment of uncertainties}
The sources of systematic uncertainties are discussed in \cref{sec:stats-analysis}. 
Different uncertainties are treated differently in the fit, and before the fit is performed, several uncertainties are preprocessed depending on their type and size. 

\paragraph{Uncertainty correlation}
Most systematic uncertainties are fully correlated between the SRs and CRs in all analysis categories. Only the theoretical uncertainties affecting the backgrounds are uncorrelated between the four different categories. The same is true for the NFs. This is a conservative approach and ensures that potential mismodellings observed in one of the categories do not impact another category. 

\paragraph{Transfer of uncertainties}
% From INT note
In some regions the available statistical power is not sufficient to correctly estimate some of the uncertainties. 
In these cases, the uncertainty can be evaluated in a more inclusive region with more events and then applied to the individual regions. This is useful especially for the regions used in the STXS measurement. The specific inclusive region is chosen and only used if the uncertainty derived in the inclusive region is statistically compatible with the uncertainties in the individual regions. 

\paragraph{Pruning and symmetrizing of uncertainties}
Only a few of the systematic uncertainties derived have a significant variation with respect to the nominal MC template. To increase the numerical stability of the fit and reduce the time it takes for the fit to converge, uncertainties are pruned from the likelihood if they do not satisfy certain threshold criteria.
For a given NP, the normalization and shape components are pruned independently of each other.
The following cases result in an uncertainty for a given sample being pruned from a region:
\begin{itemize}
    \item A normalization uncertainty is smaller than 0.5\%.
    \item A four-vector normalization uncertainty is smaller than 20\% of the statistical uncertainty. 
    \item A shape uncertainty has no significant slope ($p$-value $< 0.05$).
\end{itemize}
The nominal or alternative samples that have only small yields often lead to large, unphysical variations, due to statistical fluctuations. 
Therefore, a normalization uncertainty is also pruned if it is larger than 80\%. 
This criterion is carefully monitored to ensure that not meaningful uncertainty is pruned.
In addition, some experimental four-momentum uncertainties and uncertainties on STXS signal samples are pruned manually if they are not affected by the above criteria but are clearly dominated by statistical fluctuations and thus deemed unphysical.
Furthermore, the shape component of uncertainties of minor backgrounds ($V\gamma$, \Zgamma) are pruned, mainly because meaningful shape uncertainties cannot be estimated with the available MC samples. 

Statistical fluctuations can also cause the up and down variations to be asymmetric around the nominal value. 
For normalization uncertainties whose up and down variations differ by a factor of two or have the same sign, the larger variation is fixed and also used as ``opposite'' variation by mirroring it with respect to nominal.
Similarly, for shape uncertainties that have at least one bin where the up and down variations have the same sign, the full shape uncertainty is symmetrized by fixing the larger variation and mirroring it with respect to nominal for use as the ``opposite'' variation in each bin.

\paragraph{Smoothing of shape uncertainties}
For the DNN discriminant in the VBF category, a smoothing procedure is applied to background shape uncertainties whose estimates are subject to large fluctuations. 
The smoothing procedure is based on a bin merging strategy together with the 353QH running median algorithm~\cite{Friedman353QH}. 
An example of an uncertainty processed in such a way is shown in \cref{fig:dnn:smoothing}.
\begin{figure}[th]
    \centering
    % \subfloat[]
    %     {
        \newImageResizeCustom{0.6}{figures/plots/shape-uncertainty/ttbar-matching-smoothed.pdf}
        % }
        % \subfloat[]
        % {
        %     \newImageResizeCustom{0.48}{figures/plots/CKKW-smooth.pdf}
        % }
    {\caption{DNN output distribution for the nominal \ttbar sample (\Powheg+\PYTHIAV), an alternative sample (\aMCATNLO+\PYTHIAV) including the symmetrized distribution with respect to the nominal, and smoothed distributions based on the difference between the nominal and alternative sample. The alternative sample is used to estimate the uncertainty due to the choice of matching procedure between matrix element calculations and the parton shower. The smoothed distribution is used in the statistical analysis as varied template. More information on the smoothing procedure can be found in the text.
    \label{fig:dnn:smoothing} }}
\end{figure}
\TDinote{}{Update plot!}

\subsection{Validation of likelihood fit}
The statistical model is validated by monitoring the values of the NPs obtained from the fit (post-fit values). 
This is done for each signal category first, before running the combined analysis. A representative list of NPs obtained a fit to only the VBF-enriched \TwoJet SR and its associated VBF CRs is shown in \cref{fig:fit:breakdown}. 


\Cref{fig:fit:breakdown} lists the post-fit values of the individual NPs, including their impact on the signal strength $\mu_{\rm VBF}$.
\TDinote{}{To be incorporated!}
\TDinote{}{Include pull and breakdown plot for VBF}
\paragraph{Pulls}
\paragraph{Constraints}
\paragraph{Breakdown of uncertainties}

% A listing of the results for the individual parameters obtained in the fit is shown in Fig. 4.25. The total uncertainty has been broken down to the impacts of each individual nuisance parameter (and for certain groups of nuisance parameters). Shown are the breakdowns for the fit to the observed data, as well as for a control fit to an Asimov data set generated with the assumption of μ = 1. The postfit values and uncertainties of the individual nuisance parameters are shown as well, where the constrained nuisance parameters have been normalized to their respective constraints.


\begin{figure}[th]
    \centering
        \newImageResizeCustom{0.6}{figures/plots/breakdown/ranking_asimov_observed_thesis_muVBF.pdf}
    {\caption{Impact of individual NPs on the VBF signal strength $\mu$ for the nominal VBF analysis, shown for the fit to observed data (green) and the Asimov dataset (orange). The leading 20 NPs are shown in decreasing order of their impact on $\mu$ (top x-axis).
    The impact of a given NP is derived by comparing the uncertainty on $\mu$ resulting from a fully unconditional fit and a fit with the respective NP fixed to its nominal value. The comparison is based on the square root of the difference in quadrature of the two post-fit uncertainties on $\mu$. 
    The dots indicate the pull (bottom x-axis), that is, the post-fit values of the NPs, $\hat{\theta}$, relative to their nominal values, $\theta_0$. 
    The associated error bars show the post-fit uncertainties of the NPs relative to their nominal uncertainties. 
    The pull is zero by definition for the fit to the Asimov dataset.
% (b): Breakdown
% of the contribution of groups of NPs to the uncertainty on the V H
% PoI. The sum in quadrature of the individual contributions differs
% from the total uncertainty due to correlations between the NPs. Both
% Figures are obtained from a (1+1)-PoI fit. Published in Ref. [5].
    \label{fig:fit:breakdown} }}
\end{figure}



% Combination:
% From nature:
% The statistical analysis of the data relies on a likelihood formalism, where the likelihood functions describing each of the input measurements are multiplied to obtain a combined likelihood. The observed yield in each category of reconstructed events follows a Poisson distribution whose parameter is the sum of the predicted signal and background contributions. The predicted signal yield is split into the different production and decay processes, so it can be parameterized as a function of dedicated parameters of interest depending on the tested model.


