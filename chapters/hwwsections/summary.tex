% From nature
% The statistical analysis of the data relies on a likelihood formalism, where the likelihood functions describing each of the input measurements are multiplied to obtain a combined likelihood. The observed yield in each category of reconstructed events follows a Poisson distribution whose parameter is the sum of the predicted signal and background contributions. The predicted signal yield is split into the different production and decay processes, so it can be parameterized as a function of dedicated parameters of interest depending on the tested model.

% This chapter presents cross-section measurements of \HWWdet decays, with the Higgs boson produced via the ggF and VBF production mode, using the full dataset collected at \RunTwo of the LHC. 
% Prior measurements of these processes were performed by the ATLAS collaboration. 
% Using data from \RunOne of the LHC the \HWW process was established by reporting its observation with a discovery significance of 6.1 standard deviations~\cite{HIGG-2013-13}. 
% An analysis using a partial \RunTwo dataset reported ggF and VBF production cross sections times \HWW branching fractions of $11.4^{2.2}_{-2.1}$ pb and $0.50^{0.29}_{-0.28}$~\cite{HIGG-2016-07}.
% The analysis presented here uses the full \RunTwo dataset and in addition to the increase in dataset size implements several improvements compared to the previous \RunTwo results. Most noteworthy, the discrimination of the VBF signal is performed using a deep neural network (DNN) instead of a boosted decision tree, ggF events with two or more jets in the final state are included in the measurement, and measurements of cross sections in the kinematic regions defined in the STXS framework are reported for the first time using \HWW events~\cite{HWWPaper}. 
% The analysis is published in \ccite{HWWPaper} and yields some of the most precise Higgs cross-section meausurements to date.\footnote{Comparable results based on the full \RunTwo dataset have also been reported by the CMS collaboration~\cite{Sirunyan_2021}.}
% The results are also used as input to a combined Higgs boson measurement published in \ccite{NaturePaper} where they make an essential contribution.

% This is obseved:
% $11.4^{2.2}_{-2.1}$ pb and $0.50^{0.29}_{-0.28}$
% 12.0 ± 1.4 pb and 0.75 +0.19 −0.16 pb,

All measurements presented are consistent with the SM predictions. 
% Previous \RunTwo results of the ATLAS collaboration reported corresponding cross sections of the ggF and VBF production mode of $11.4^{2.2}_{-2.1}$~pb and $0.50^{0.29}_{-0.28}$~pb~\cite{HIGG-2016-07}.
Previous \RunTwo results of the ATLAS collaboration reported corresponding cross sections of the ggF and VBF production mode of $11.4^{+2.2}_{-2.1}$~pb\footnote{Broken down into statistical and systematic uncertainties the results are $\sigmaGGF = 11.4^{+1.2}_{-1.1}\,\text{(stat.)}^{+1.8}_{-1.7}\,\text{(syst.)}$~pb} and $0.50^{+0.29}_{-0.28}$~pb\footnote{Broken down into statistical and systematic uncertainties the results are  $\sigmaVBF= 0.50^{+0.24}_{-0.22}\,\text{(stat.)}\pm 0.17\,\text{(syst.)}$~pb}, respecitvely~\cite{HIGG-2016-07}.
The measurements presented here thus result in a 40\% and 60\% smaller total relative uncertainty on the ggF and VBF measurement, respectively. In part, this improvement is due to the increased size of the dataset by almost a factor of four.
In addition, the ggF measurement benefits from the inclusion of the ggF \TwoJet category and the VBF measurement is drastically improved by the implementation of a DNN as final fit discriminant. 
The previous analysis, which used a boosted decision tree (BDT) for the VBF signal discrimination, reached an expected discovery significance of $2.6\,\sigma$ for the VBF signal, which is to be compared to $6.2\,\sigma$ achieved by the analysis presented. 
Direct comparisons between the performance of the BDT and the DNN with an otherwise identical analysis strategy showed an improvement of about 20-30\% in the discovery significance depending on the binning choice. % solely due to the change of the discriminant. 
These developments allowed observing the VBF production mode of the Higgs boson for the first time using only \HWW decays.
Both the ggF and VBF measurements also benefit from improvements in the estimation of the background from misidentified leptons, substantially reducing their impact on the measurement. 
Furthermore, improvements in ATLAS reconstruction software, identification algorithms, and improved object calibration measurements are of benefit. 
An example of the latter is the JER measurement presented in \cref{chap:calibration}. 
Both the ggF and VBF measurements are dominated by theoretical uncertainties. 
The VBF measurement, in particular, is largely dominated by uncertainties due to the parton shower model. 
A better understanding of parton shower effects and more accurate theoretical calculations are needed to improve the measurements in future iterations.~\cite{J_ger_2020} 

This chapter also presented STXS measurements, performed by the ATLAS collaboration for the first time using \HWW decays. 
The full \RunTwo dataset allowed measuring in total 11 Higgs boson production cross sections split in different kinematic regions. 
While the results of the inclusive cross-section measurements are largely dominated by systematic uncertainties, the measurement of 6 of the 11 STXS cross sections are still dominated by statistical uncertainties. 

The analysis presented is also crucial input to combined Higgs boson measurements~\cite{NaturePaper} that includes various other analyses than the one presented, studying different combinations of Higgs boson production and decay processes.
% The ones with the largest contributions are analyses of $H \to ZZ$, $H \to \gamma\gamma$, and $H \to \tau\tau$ decays that consider Higgs boson production via all major production modes. 
These results were presented in \cref{chap:higgs}.
The combined measurements of both inclusive VBF and ggF cross sections as well as STXS cross sections benefit greatly from the \HWW analysis.
Furthermore, measurements of \HWW decays provide the most precise measurements of the Higgs boson coupling to vector bosons, which is particularly driven by the VBF measurement, because the $HVV$ vertex appears twice in the VBF, \HWW diagram (see \cref{sec:signal-bkg-characteristics}). 
Using the $\kappa$-framework, the $HV$ coupling-strength modifier is measured to be $\kappa_{V} = 1.035 \pm 0.031$, assuming the same modifier for the $HW$ and $HZ$ coupling.
Treating the $HW$ and $HZ$ coupling independently, the $HW$ coupling is measured with relative uncertainties of about 5-10\%, depending on the assumed model. All results are found to be consistent with the SM expectations. 

% The predicted signal yield is split into the different production and decay processes, so it can be parameterized as a function of dedicated parameters of interest depending on the tested model.




