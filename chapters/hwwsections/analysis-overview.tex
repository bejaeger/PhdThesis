The \HWW events that can be best separated from backgrounds at the LHC are the ones where the two $W$ bosons decay leptonically.
This analysis therefore analyzes events with two leptons in the final state. 
The most relevant background contributions come from $WW$ production, events with top quarks, $W$ production in association with jets ($W$+jets) where a jet is misidentified as a lepton, other diboson processes, and Drell-Yan processes. In particular Drell-Yan events are produced at large cross sections at the LHC. To reduce their contamination only final states with different-flavor leptons are selected, that is, with exactly one reconstructed muon and electron.
Given the features of the signal and background processes, that are discussed in detail below, the analysis is divided into four signal categories separated by jet multiplicity and production mode.
Three categories target the ggF production mode by separating events into a \ZeroJet, \OneJet, and \TwoJet category, and one category targets the VBF production mode, also requiring \TwoJet. 
This categorization is illustrated in \cref{fig:signal-categorization}. 
\begin{figure}
    \newImageResizeCustom{0.7}{figures/hww/introduction/analysis-categorization.png}
    \caption{Schematic view of the signal categories defined in the \HWW analysis.}
    \label{fig:signal-categorization}
\end{figure}
Each of the categories is first optimized and validated separately before being combined in the statistical analysis.
The categories targeting the ggF signal use event selections to increase the purity of signal events. 
The VBF signal is discriminated using a DNN that distinguishes VBF signal from non-VBF events. 
The development of the DNN is described in \cref{sec:dnn}. 
The descriptions and results presented in this chapter supersede similar measurements performed with the same dataset but based on a previous version of the analysis~\cite{ATLAS-CONF-2021-014}.
The main differences between the analyses are changes in the simulated samples used, improved treatment of systematic uncertainties, and an improved DNN discriminant in the VBF category. 

\TDinote{}{Maybe put the following in the stats-analysis section}
To extract the signals and measure their cross sections times \HWW branching fraction, a likelihood fit is performed simultaneously in all categories. 
The signal yields are parametrized as signal strengths in the likelihood, 
\begin{equation}
    \label{eq:incl-ggF-VBF-signal-strengths}
    \mu_{\text{signal}} = \frac{ [ \sigma_{\text{signal}}  \times \BR(H \to WW^*) ]_\text{meas} } { [ \sigma_{\text{signal}} \times \BR(H \to WW^*) ]_\text{SM}}, 
\end{equation}
with the signal definition depending on the type of measurement that is performed.
The analysis presented performs three types of measurements:
\begin{itemize}
    \item Measurement of the inclusive VBF and ggF production cross sections by extracting two signal strengths $\mu_{\text{VBF}}$ and $\mu_{\text{ggF}}$ in a \emph{2-PoI fit}.
    \item Measurement of the combined VBF and ggF production cross section, treating the VBF and ggF signal with a common signal strength $\mu_{\text{VBF+ggF}}$ in a \emph{1-PoI fit}.
    \item Measurement of differential cross sections separated in kinematic regions following the STXS framework (termed \emph{STXS measurement}). In total 11 signal strengths are measured targeting the ggF and VBF production mode in an \emph{11-PoI fit}. 
\end{itemize}
In addition, the excess of the VBF signal over the background expectation is reported in terms of its discovery significance, yielding, for the first time based on a single Higgs decay mode, the observation of the VBF production mode.

% \section{Previous Results and Objectives of Measurement}
% \label{sec:prev-results}
