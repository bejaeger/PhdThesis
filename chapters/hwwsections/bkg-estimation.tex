\todo{Provide table with backgrounds and characteristics...see S Gadatsch thesis Table 5.1 page 126???}

One of the crucial tasks of a physics analysis is to estimate the different background contributions in the SR as accurately as possible.
In this thesis, three different methods are applied for this purpose: 
The top quark, \Zjets, and continuum $WW$ background (not for the VBF category) are estimated using subsidiary measurements in so-called control regions (CRs), processes with misidentified leptons are estimated using a data-driven method known as the \emph{fake factor method}, and all other background contributions are estimated solely based on simulated samples normalised to the theoretical cross sections for the specific processes.
The following provides an overview of the different estimates.

\subsection{Backgrounds estimated with control regions}

- Explain NF method (I guess a formula would be great?)
- Reference table with all selection criteria

\subsubsection{WW}

- Mention mostly selections

\subsubsection{Top}
- Mention mostly selections
\subsubsection{Zjets}
- Mention mostly selections

\subsection{Backgrounds with misidentified leptons}
- Explain fake factor method!

\subsection{Backgrounds estimated with simulated samples}
- Mention validation region
- List xsecs?