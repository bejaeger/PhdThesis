%\todo{Provide table with backgrounds and characteristics...see S Gadatsch thesis Table 5.1 page 126???}
%One of the crucial tasks of a physics analysis is to estimate the different background contributions in the SR as accurately as possible.
To estimate the contribution of the various backgrounds in the SRs as accurately as possible, three different methods are used in the \HWW\ analysis:
The top-quark, \Zgamma, and non-resonant $qq \to WW$ backgrounds are estimated using subsidiary measurements in so-called \emph{control regions} (CRs), processes with misidentified leptons are estimated using a data-driven method known as the \emph{fake factor method}, and all other background contributions are estimated based on simulated samples normalized to the theoretical cross sections for the specific processes.
The following provides an overview of the different estimation strategies.
% An overview of the background contamination in each of the nominal SRs is shown in \cref{sec:analysis-overview}, \cref{fig:bkg-composition}. 
% \begin{figure}[t]
%     \newImageResizeCustom{1}{figures/plots/bkg-fractions}
%     \caption{Background fractions in nominal SRs after a fit to the data. 
%     \label{fig:bkg-fractions}
%     }
% \end{figure}
% \TDinote{}{Refer to cutflow?}

\subsection{Backgrounds estimated with control regions}
Regions that are pure enough in a specific process and orthogonal to the signal categories can be used to measure the normalization of that process via a \emph{normalization factor} (NF) by simultaneously fitting these CRs with the SRs (see \cref{chap:statistics}).
The CRs are also used to validate the agreement between MC simulation and data in various variables prior to the fit.
% The latter is done after deriving pre-fit normalization factors (NFs) for each process that has a corresponding CR.\footnote{Given $N$ CRs and $M > N$ processes, $N$ NFs are derived by solving the set of linear equations that equate the sum of expected number of background events (times the NF, where applicable) and the number of data events. The statistical uncertainties on the NFs are estimated using of toys, that is, repeating the outlined procedure with several randomly varied inputs within the uncertainties. The standard deviation of the resulting distribution is used as an uncertainty on the NFs.}
% This guarantees a fair comparison between data and MC simulation since potential discrepancies are corrected based on the data. 

All CR selections are summarized in \cref{tab:CRsSelection}.
The CRs are defined separately for each analysis category and have at least one criterion that ensures orthogonality between the CRs and SRs.
In addition, selections similar to the SR selections are applied, to ensure that the events in the SRs and CRs have similar kinematics and topologies.
The precision with which the NFs can be determined is dependent on the purity of the process targeted by a CR in that CR. \Cref{tab:cr-purities} summarizes the purities in all CRs used in the analysis.
Distributions of \mT for the ggF CRs and the DNN output for the VBF CRs after a fit to data are provided in Appendix~\ref{app:post-fit-cr-dists}. 
%The pre-fit NFs for the CRs related to the \TwoJet VBF category are shown in \cref{tab:pre-fit}. 

\begin{table}[!ht]
    \caption[Event selection criteria used to define the control regions.]{
        Event selection criteria used to define the control regions in the \HWW\ analysis.
        Every control region selection starts from the selection labeled ``Preselection'' in Table~\ref{tab:HWWselection}. Details on the variables used are given in the text.}
    \label{tab:CRsSelection}
    \centering
    \resizebox{\textwidth}{!}{
        \renewcommand{\arraystretch}{1.4}
\begin{tabular}{c|| c | c | c | c}
    \toprule
    %\hline\hline
    %  CR                         & $\ZeroJet$                      & $\OneJet$      & TwoJet ggF             & $\TwoJet$, VBF\\
    CR                                  & \ZeroJet ggF                                         & \OneJet ggF                                                   & \TwoJet ggF                    & \TwoJet VBF                    \\
    %\hline
    \midrule
    \multirow{8}{*}{$qq\rightarrow WW$} & \multicolumn{3}{c|}{$\Nbjetsub=0$}                   &                                                                                                                                 \\ \cline{2-4}
                                        & $\dphillMET > \pi/2$                                 & \multicolumn{2}{c|}{$\mll\,{>}\,80~\GeV$}                     &                                                                 \\ \cline{3-4}
                                        & $\pTll\,{>}\,30~\GeV$                                & $|\mtt-m_Z|\,{>}\,25~\GeV$                                    & $\mtt<m_Z-25~\GeV$             &                                \\
                                        & $55\,{<}\,\mll\,{<}\,110~\GeV$                       & $\maxmtlep\,{>}\,50~\GeV$                                     & $\mTtwo\,{>}\,165~\GeV$        &                                \\ \cline{4-4}
                                        & $\dphill\,{<}\,2.6$                                  &                                                               & fail central-jet veto          &                                \\
                                        &                                                      &                                                               & or fail outside-lepton veto    &                                \\ \cline{4-4}
                                        &                                                      &                                                               & $|\mjj-85| > 15~\GeV$          &                                \\
                                        &                                                      &                                                               & or\ \ \ \ $\dyjj > 1.2$        &                                \\
    \midrule
    %\hline
    %  top quark     & \multicolumn{2}{c}{$\trkmet\,{>}20\GeV$} \\
    \multirow{10}{*}{Top quark}   & \multirow{2}{*}{$\Nbjetbetween>0$}                   & $\Nbjetnom = 1$                                               & \multirow{2}{*}{$\Nbjetsub=0$} & \multirow{2}{*}{$\Nbjetsub=1$} \\
                                        &                                                      & $\Nbjetbetween  = 0$                                          &                                &                                \\ \cline{3-5}
                                        & $\dphillMET > \pi/2$                                 & \multicolumn{3}{c}{\phantom{ccccc}$\mtt\,{<}\,m_Z - 25~\GeV$}                                                                   \\ \cline{3-5}
                                        & $\pTll\, > \, 30~\GeV$                               & $\maxmtlep\,{>}\,50~\GeV$                                     & $\mll\,{>}\,80~\GeV$           &                                \\
                                        & $\dphill\,{<}\,2.8$                                  &                                                               & $\dphill\,{<}\,1.8$            &                                \\
                                        &                                                      &                                                               & $\mTtwo\,{<}\,165~\GeV$        &                                \\ \cline{4-4}
                                        &                                                      &                                                               & fail central-jet veto          & central-jet veto               \\
                                        &                                                      &                                                               & or fail outside-lepton veto    & outside-lepton veto            \\ \cline{4-4}
                                        &                                                      &                                                               & $|\mjj-85| > 15~\GeV$          &                                \\
                                        &                                                      &                                                               & or\ \ \ \ $\dyjj > 1.2$        &                                \\
    %\hline
    \midrule
    \multirow{8}{*}{$Z/\gamma^{\ast}$}  & \multicolumn{4}{c}{\phantom{ccccc}$\Nbjetsub=0$}                                                                                                                                       \\ \cline{2-5}
                                        & \multicolumn{2}{c|}{\phantom{ccccc}$\mll < 80~\GeV$} & $\mll\,{<}\,55~\GeV$                                          & $\mll\,{<}\,70~\GeV$                                            \\\
                                        & \multicolumn{2}{c|}{no $\pTmiss$ requirement}        &                                                               &                                                                 \\\cline{2-4}
                                        & $\dphill > 2.8$                                      & \multicolumn{2}{c|}{$\mtt > m_Z - 25~\GeV$}                   & $|\mtt - m_Z| \leq 25~\GeV$                                     \\ \cline{3-4}
                                        &                                                      & $\maxmtlep\,{>}\,50~\GeV$                                     & fail central-jet veto          & central-jet veto               \\
                                        &                                                      &                                                               & or fail outside-lepton veto    & outside-lepton veto            \\ \cline{4-4}
                                        &                                                      &                                                               & $|\mjj-85| > 15~\GeV$          &                                \\
                                        &                                                      &                                                               & or\ \ \ \ $\dyjj > 1.2$        &                                \\
    %\hline
    %$VV$    & same-sign leptons     & same-sign leptons     \\
    %        & all SR cuts           & all SR cuts           \\
    \bottomrule
    %\hline\hline
\end{tabular}

    }
\end{table}
\begin{table}[h!!]
    \caption[Purity of the targeted process in each control region.]{
        Purity of the targeted process in each control region. The statistical uncertainties are small and therefore not indicated.}
    \label{tab:cr-purities}
    \centering
    \begin{tabular}{c|| c | c | c | c}
        \toprule
        CR          & \ZeroJet ggF & \OneJet ggF & \TwoJet ggF & \TwoJet VBF \\
        \midrule
        Top quark   & 89\%         & 98\%        & 71\%        & 97\%        \\
        \Zgamma     & 94\%         & 76\%        & 76\%        & 77\%        \\
        $qq \to WW$ & 67\%         & 34\%        & 39\%        &             \\
        \bottomrule
    \end{tabular}
\end{table}

\subsubsection{Top backgrounds}
The $t\bar{t}$ and $Wt$ processes are normalized jointly using the same NF, extracted from dedicated top-quark CRs.
The relative contribution of $t\bar{t}$ and $Wt$ processes is comparable between the CRs and SRs and its uncertainty is accounted for by treating the relevant sources of uncertainties separately.

The top-quark CRs in the \ZeroJet, \OneJet, and VBF-enriched \TwoJet\ category are defined by requiring the presence of $b$-jets.
For the \ZeroJet\ category, at least one $b$-jet with $20 < \pT < 30\,$GeV is required, i.e.  $\Nbjetbetween > 0$.
For the \OneJet\ category, exactly one $b$-jet with $\pT > 30\,$GeV and no $b$-jet with $20 < \pT < 30\,$GeV is required, and for the VBF \TwoJet\ category exactly one $b$-jet with $\pT > 30\,$GeV is required.
These specific selections ensure that the ratio between $t\bar{t}$ and $Wt$ processes is similar between the CR and the SR without compromising the purity of top-quark processes in the CR.
For the ggF-enriched \TwoJet\ region, an alternative strategy is used that does not rely on $b$-jet requirements, but still retains a high purity of top-quark events. The CR is defined by requiring $\mll > 80\,$GeV and $\mTtwo < 165\,$GeV\footnote{A definition of this variable is provided later in this section in \cref{eq:mt2-definition}.}, where the latter selection ensures orthogonality to the $WW$ CR. This definition reduces the uncertainties from flavor tagging selections and is therefore preferable to a CR defined using $b$-jet criteria.
Distributions of several variables in the VBF, top-quark CR are shown in \cref{fig:topcr:dnn-inputs}.
For these comparisons, pre-fit NFs are derived based on the CRs.\footnote{The pre-fit NFs are derived as follows: Given $N$ CRs and $M > N$ processes, $N$ NFs can be derived by solving the set of linear equations that equate the sum of expected number of background events (times the NF for the applicable $N$ processes) and the number of data events. The statistical uncertainties on the NFs are estimated using of toys, i.e., repeating the outlined procedure with several randomly varied inputs within the uncertainties. The standard deviation of the resulting distribution is used as an uncertainty on the NFs.}
They show an excellent agreement between the data and MC simulation.
%Similar comparisons are performed for the other top-quark CRs.

\newcommand{\crmodellingcaption}[1]{Distributions of (a) \mjj, (b) \dyjj, (c) \dphill, (d), \mll, (e) \lepetacent, and (f) \METSig in the VBF, #1. The \Ztautau and top-quark processes are normalized using pre-fit normalization factors extracted from their corresponding CRs (NF$_{\Ztautau} = 1.00 \pm 0.04$, NF$_{\text{top}} = 1.00 \pm 0.01$). The yellow band corresponds to the square root of the sum in quadrature of the MC statistical uncertainties and the normalization component of the dominant experimental systematic uncertainties (JER, JES, Flavor-tagging, and \MET uncertainties). The various uncertainty components are discussed in more detail in \cref{sec:systematics}.}
\newcommand{\crmodellingcaptionshort}[1]{Distributions of (a) \mjj, (b) \dyjj, (c) \dphill, (d), \mll, (e) \lepetacent, and (f) \METSig in the VBF, #1.}
\newcommand{\vrmodellingcaption}[1]{Distributions of (a) \mjj, (b) \dyjj, (c) \mlonejone, (d), \mll, (e) \lepetacent, and (f) \METSig in the VBF, #1. The \Ztautau, top-quark, and $qq \to WW$ processes are normalized using pre-fit normalization factors extracted from their corresponding CRs (NF$_{\Ztautau} = 1.01 \pm 0.04$, NF$_{\text{top}} = 1.00 \pm 0.01$, NF$_{WW} = 0.84 \pm 0.02$). The yellow band corresponds to the square root of the sum in quadrature of the MC statistical uncertainties and the normalization component of the dominant experimental systematic uncertainties (JER, JES, Flavor-tagging, and \MET uncertainties). The various uncertainty components are discussed in more detail in \cref{sec:systematics}.}
\newcommand{\vrmodellingcaptionshort}[1]{Distributions of (a) \mjj, (b) \dyjj, (c) \mlonejone, (d), \mll, (e) \lepetacent, and (f) \METSig in the VBF, #1.}
\newcommand{\topcrplotdir}{figures/220605-Thesis/topcr/plots}

\FloatBarrier
\sixPlots{\crmodellingcaption{top-quark CR}}{fig:topcr:dnn-inputs}{\topcrplotdir/../mjj/plots/run2-emme-CutVBF_TopControl_2jet-Mjj-log.pdf}{\topcrplotdir/run2-emme-CutVBF_TopControl_2jet-DYjj-lin.pdf}{\topcrplotdir/run2-emme-CutVBF_TopControl_2jet-DPhill-lin.pdf}{\topcrplotdir/run2-emme-CutVBF_TopControl_2jet-Mll-lin.pdf}{\topcrplotdir/../lepcent/plots/run2-emme-CutVBF_TopControl_2jet-contOLV-lin.pdf}{\topcrplotdir/run2-emme-CutVBF_TopControl_2jet-METSig_broad-lin.pdf}{\crmodellingcaptionshort{top-quark CR}}
%{figures/hww/presel/topcr/run2-emme-CutVBF_TopControl_2jet-PtTot-lin.pdf}{figures/hww/presel/topcr/run2-emme-CutVBF_TopControl_2jet-contOLV-lin.pdf}{run2-emme-CutVBF_TopControl_2jet-DNNoutputG_Fit_finelowDNN2-log.pdf}

\subsubsection{\Ztautau background}
Similar as for top-quark processes, the \Zgamma\ background is normalized using dedicated CRs.
The corresponding CR in the \ZeroJet category requires a large opening angle of the leptons of $\DPhill > 2.8$.
For the \OneJet and two \TwoJet categories, the \mtautau observable is exploited to separate the \Ztautau CRs\footnote{The CRs used to normalize the \Zgamma contribution are denoted \Ztautau CRs as \Ztautau events constitute by far the dominant contribution.} from the SRs by selecting the region around the nominal $Z$ boson mass.
The specific selections are $\mtautau - m_Z < 25\,$GeV for the \OneJet and ggF-enriched \TwoJet category and $|\mtautau - m_Z| \le 25\,$GeV for the VBF-enriched \TwoJet category.
%The latter includes the upper tail of the \mtautau distribution to increase the statistical power of the CR. 

In order to increase the statistical power of the CRs, some selections made in the SR are relaxed or dropped.
In the \ZeroJet and \OneJet category, no \pTmiss requirement is applied and the \mll\ selection is relaxed to $\mll < 80\,$GeV.
In the VBF-enriched \TwoJet category, the $\mll$ requirement is relaxed to $\mll < 70\,$GeV and the region above $\mtautau = m_Z + 25\,$GeV is kept.

Distributions of several variables in the VBF, \Ztautau CR are shown in \cref{fig:zttcr:dnn-inputs}. Similar to the case for the VBF, top-quark CR, an excellent agreement between the data and MC simulation is observed. %Similar comparisons are performed for the other \Ztautau CRs. 

\newcommand{\zttcrplotdir}{figures/220605-Thesis/zttcr/plots}
\sixPlots{\crmodellingcaption{\Ztautau CR}}{fig:zttcr:dnn-inputs}{\zttcrplotdir/../mjj/plots/run2-emme-CutVBF_ZtautauControl_2jet-Mjj-log.pdf}{\zttcrplotdir/run2-emme-CutVBF_ZtautauControl_2jet-DYjj-lin.pdf}{\zttcrplotdir/run2-emme-CutVBF_ZtautauControl_2jet-Ml0j0-lin.pdf}{\zttcrplotdir/run2-emme-CutVBF_ZtautauControl_2jet-Mll-lin.pdf}{\zttcrplotdir/../lepcent/plots/run2-emme-CutVBF_ZtautauControl_2jet-contOLV-lin.pdf}{\zttcrplotdir/run2-emme-CutVBF_ZtautauControl_2jet-METSig_broad-lin.pdf}{\crmodellingcaptionshort{\Ztautau CR}}

\subsubsection{Non-resonant $WW$ backgrounds}
In the ggF categories, the $qq \to WW$ background is controlled with CRs by modifying the selections on the \mll\ observable and applying additional selections to purify the regions.

In the \ZeroJet $WW$ CR, a requirement of $55 < \mll < 110$\,GeV is applied, where the lower boundary ensures orthogonality between the $WW$ CR and the \ZeroJet\ SR, and the upper boundary reduces the contamination from top-quark processes.
In addition, the \DPhill\ selection is relaxed to $\DPhill < 2.6$, rejecting the majority of \Ztautau processes while retaining many of the $qq \to WW$ events.
The $WW$ CR in the \OneJet\ category is defined by modifying the selections $\mll > 80\,$GeV and $|m_{\tau\tau} - m_Z > 25|\,$GeV, which ensure orthogonality as well as a high purity of $qq \to WW$ events.
The \TwoJet categories suffer from a large contribution of top-quark processes, which makes it difficult to define regions pure in $WW$ processes.
For the ggF-enriched \TwoJet category, a CR is nonetheless found by requiring $\mll > 80\,$GeV and $\mTtwo > 165\,$GeV. The \mTtwo\ variable is defined as
\begin{equation}
    \label{eq:mt2-definition}
    \mTtwo^2 = \underset{{p\mkern-9.5mu/}_1+{p\mkern-9.5mu/}_2={p\mkern-9.5mu/}_\text{T}}{\textrm{min}} \left[\textrm{max}\{\mt^2(p_\text{T}^a,{p\mkern-9.5mu/}_1),\mt^2(p_\text{T}^b,{p\mkern-9.5mu/}_2)\}\right],
\end{equation}
where the minimization goes over all possible two-momenta, ${p\mkern-9.5mu/}_{1,2}$, such that their sum gives the observed missing transverse momentum ${p\mkern-9.5mu/}_\text{T}$, and where each of $p_\text{T}^a$ and $p_\text{T}^b$ is the combined transverse momentum of a charged lepton and a jet~\cite{HWWPaper}.
% \begin{figure}[ht]
%     \newImageResizeCustom{0.6}{\paperfiguredir/CRs/MT2_for_ggF2jet_dataMCRatio.pdf}
%     \caption[Distribution of the \mTtwo\ variable in the ggF-enriched \TwoJet category.]{
%         Distribution of the \mTtwo\ variable used in the definition of the $WW$ CR for the ggF-enriched \TwoJet category. The dashed line indicates where the selection on the observable is made. The distributions are normalized to their nominal yields, before the final fit to all SRs and CRs (pre-fit normalizations). The hatched band shows the normalization component of the total pre-fit uncertainty. Figure and caption taken from \ccite{HWWPaper}.}
%     \label{fig:njet-dist}
%   \end{figure}
For \ttbar events, the \mTtwo variable has an upper bound around the mass of the top quark, making it possible to reduce the \ttbar contribution by requiring \mTtwo to be sufficiently large.
For the VBF-enriched \TwoJet category, no $WW$ CR pure enough in $WW$ processes can be defined.
The $qq \to WW$ background is therefore estimated from simulated samples normalized to their theoretical cross section in the statistical analysis.
To validate the consistency between MC simulation and data for $qq \to WW$ processes, a $WW$ validation region (VR) is defined by requiring $\mt > 130$ and $\mTtwo > 165$ in addition to all preselection requirements.
The distributions of various observables in the VBF, $WW$ VR is shown in \cref{fig:wwvr:dnn-inputs}.

\newcommand{\wwvrplotdir}{figures/220605-Thesis/wwvr/plots}
\FloatBarrier
\sixPlots{\vrmodellingcaption{$WW$ VR}}{fig:wwvr:dnn-inputs}{\wwvrplotdir/../mjj/plots/run2-emme-CutVBF_WWControl_2jet-Mjj-log.pdf}{\wwvrplotdir/run2-emme-CutVBF_WWControl_2jet-DYjj-lin.pdf}{\wwvrplotdir/run2-emme-CutVBF_WWControl_2jet-DPhill-lin.pdf}{\wwvrplotdir/run2-emme-CutVBF_WWControl_2jet-Mll-lin.pdf}{\wwvrplotdir/../lepcent/plots/run2-emme-CutVBF_WWControl_2jet-contOLV-lin.pdf}{\wwvrplotdir/run2-emme-CutVBF_WWControl_2jet-METSig_broad-lin.pdf}{\vrmodellingcaptionshort{$WW$ VR}}

\subsubsection{Control regions for the STXS measurement}
\label{subsubsec:stxs-crs}
For the STXS measurement, the CRs defined above are further split into separate CRs that are aligned with the STXS SRs. Due to the limited statistical precision, the same split as in the SRs cannot be performed for the CRs. Thus, a compromise is made between statistical power and granularity, leading to a total number of 21 CRs in the ggF categories and 6 CRs in the VBF categories.
The full list of CRs is summarized in \cref{tab:STXSCRsSelection}.
Each selection specified is applied to each type of CR listed in the first column, i.e., to three CRs for the ggF categories (the $WW$, top-quark, and \Ztautau CR) and to two CRs for the VBF categories (top-quark and \Ztautau CR).
\begin{table}[ht]
    \caption[Event selection criteria used to define the STXS control regions.]{
        Event selection criteria used to define the control regions used in the STXS measurement.
        The selections start from the nominal CR selections summarized in \cref{tab:CRsSelection}.
        %In each category the selections are applied to all nominal CRs, that is . 
    }
    \label{tab:STXSCRsSelection}
    \centering
    \resizebox{\textwidth}{!}{
        \renewcommand{\arraystretch}{1.4}
\begin{tabular}{c || l | l | l | l}
    \dbline
    %\hline\hline
    %  CR                         & $\ZeroJet$                      & $\OneJet$      & TwoJet ggF             & $\TwoJet$, VBF\\
    CRs                            & \ZeroJet ggF                     & \OneJet ggF                & \TwoJet ggF          & \TwoJet VBF                                 \\
    %\hline
    \hline\hline
    $qq\rightarrow WW$ (for ggF cat.), & \multirow{4}{*}{Same as nominal} & $\pTH < 60~\GeV$           & $\pTH < 200~\GeV$    & $350 \leq \mjj < 700~\GeV, \pTH < 200~\GeV$ \\
    $t\bar{t}$/$Wt$,               &                                  & $60 \leq \pTH < 120~\GeV$  & $\pTH \geq 200~\GeV$ & $\mjj \geq 700~\GeV, \pTH < 200~\GeV$       \\
    and $Z/\gamma^{\ast}$          &                                  & $120 \leq \pTH < 200~\GeV$ &                      & $\mjj \geq 350~\GeV, \pTH \geq 200~\GeV$    \\
                                   &                                  & $\pTH \geq 60~\GeV$        &                      &                                             \\
    \hline
    Total number of CRs            & 3                                & 12                         & 6                    & 6                                           \\
    \dbline
    %\hline\hline
\end{tabular}

    }
\end{table}

% % NFs table
% \begin{table}[!h]
%     \centering
%     \renewcommand{\arraystretch}{1.6}
%       \caption{
%         Pre-fit normalization factors which scale the corresponding estimated yields in all regions; the dash indicates where a MC-based normalization is used. 
%         The uncertainties are statistical only.
%         %The quoted uncertainties include only the statistical contributions.
%     }
%     \label{tab:vbf-prefit-NFs}
%     \scalebox{1.0}{
%     \begin{tabular}{c c c c}
%     \toprule
%     Category      &   $t\bar{t}/Wt$     &    $Z/\gamma^{\ast}$  & $WW$       \\
%     \midrule
%     \TwoJet VBF 2x2  &  $1.00 \pm 0.01$ &  $1.00\pm 0.04$ & -- \\
%     \TwoJet VBF 3x3 &    $1.00 \pm 0.01$  &  $1.01\pm 0.04$ & $0.84 \pm 0.02$ \\
%     \bottomrule
%     \end{tabular}
% }
% \end{table}

\subsection{Backgrounds estimated with simulated samples}
Several processes are estimated using simulated samples normalized to their theoretical cross sections.
These include the background contributions from $gg \to WW$, EW~$WW$, Other $VV$, and triboson processes, as well as the Higgs boson processes not considered as signal in this analysis, which are both Higgs bosons produced via $ttH$ and $VH$ and processes with $H\to\tau\tau$ decays. In addition, in the VBF-enriched \TwoJet category, the $qq \to WW$ background is estimated using simulated samples and validated in a validation region as described above.
%{figures/hww/presel/topcr/run2-emme-CutVBF_WWControl_2jet-PtTot-lin.pdf}{figures/hww/presel/topcr/run2-emme-CutVBF_WWControl_2jet-contOLV-lin.pdf}{run2-emme-CutVBF_WWControl_2jet-DNNoutputG_Fit_finelowDNN2-log.pdf}
% not sure if I want to add the following
% The underlying assumption is that the misidentification efficiency only depends on properties of the jet misidentified as a lepton, not on the remainder of the event, which is reasonable given the that the jet and lepton reconstruction only takes information from specific detector regions.

\subsection{Backgrounds with misidentified leptons}
\label{subsec:misid-bkg}
The $W$+jets and multijet processes enter the SRs if one ($W$+jets) or two (multijet) leptons are misidentified.
The rate at which these misidentifications occur is difficult to estimate purely based on simulated event samples.
Their contributions are therefore estimated using a data-driven technique known as \emph{fake factor method}. In this method, a control data sample is defined and used to extrapolate the expected number of events to the analysis regions via an extrapolation factor known as \emph{fake factor}.
The control sample is defined with alternative object selection criteria that are orthogonal to the nominal selections and designed to be enriched in candidate events with misidentified leptons.
The fake factor is derived in a separate analysis and can be regarded as the probability of a lepton being misidentified, given the control sample selections.
For each signal category, the same event selections are applied to the control sample as in the nominal analysis. Details on the fake factor method including a description of the systematic uncertainties that must be considered are given in Appendix~\ref{app:fake-factor-method}.

%%%%%%%%%%%%%%%%%%%%%%%
% Pre-fit Normalization Factors
% From here: https://gitlab.cern.ch/atlas-physics-office/HIGG/ANA-HIGG-2018-47/ANA-HIGG-2018-47-INT1/-/blob/master/sections/STXSAnalysesStrategy-vbf.tex
% \begin{tabular}{c c c}
%     \hline
%     \hline
%      Phase space             &   NF$^\mathrm{\Ztt}$       & NF$^\mathrm{top}$ 	\\
%      \hline
%      \mjjm{350}{700} and \pthle{200}	 & $1.01 \pm 0.06$ &   $1.02 \pm 0.01$				\\
%      \mjjge{700} and \pthle{200}    &   $0.90 \pm 0.10$     & $1.01 \pm 0.03$			\\
%      \mjjge{350} and \pthge{200}   &   $1.04 \pm 0.12$     & $0.94 \pm 0.04$			\\
%     \hline
%     \hline
%     \end{tabular}
%     \caption{Normalisation factors (NF) obtained from the VBF STXS CRs and their uncertainties for the \Ztt and Top backgrounds pre-fit.}
%     \label{tab:vbfstxs:PreNormalisationFactors}
%     \end{table}
