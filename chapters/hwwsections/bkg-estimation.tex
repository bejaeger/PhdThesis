\todo{Provide table with backgrounds and characteristics...see S Gadatsch thesis Table 5.1 page 126???}

One of the crucial tasks of a physics analysis is to estimate the different background contributions in the SR as accurately as possible.
In this thesis, three different methods are applied for this purpose: 
The top quark, \Zjets, and continuum $qqWW$ background (not for the VBF category) are estimated using subsidiary measurements in so-called control regions (CRs), processes with misidentified leptons are estimated using a data-driven method known as the \emph{fake factor method}, and all other background contributions are estimated solely based on simulated samples normalised to the theoretical cross sections for the specific processes.
The following provides an overview of the different estimation strategies, treated separately for each analysis category.

\subsection{Backgrounds estimated with control regions}

- Explain NF method (I guess a formula would be great?)
- CRs defined separately for each analysis category.
- Reference table with all selection criteria.
- Target is to construct regions as pure as possible in a specific background.
- Reference table with pre-fit NFs as well as purities.

- Generally, CRs apply the same SR selections unless mentioned otherwise.

\subsubsection{Continuum $WW$ backgrounds}
The $qqWW$ background is controlled with CRs separated from the SRs using modified selections on the \mll\ observable and purified with additional selections specific to each analysis category.

In the \ZeroJet $WW$ CR, a requirement of $55 < \mll < 110$\,GeV is applied, where the lower boundary ensures orthogonality between the $WW$ CR and the \ZeroJet\ SR, and the upper boundary reduces the contamination from top-quark processes.
In addition, compared to the SR selection, the \DPhill selection is relaxed to $\DPhill < 2.6$, rejecting the majority of \Ztautau processes while retaining many of the $qqWW$ events.
The $WW$ CR in the \OneJet\ category is defined with modified selections: $\mll > 80\,$GeV and $|m_{\tau\tau} - m_Z > 25|\,$GeV, which ensure orthogonality as well as a region pure in $qqWW$ events.
The \TwoJet categories suffer from a large contribution of top-quark processes, which makes it difficult to define regions pure in $WW$ processes.
For the ggF-enriched \TwoJet category, a CR is nonetheless found by requiring $\mll > 80\,$GeV and $\mTtwo > 165\,$GeV. The \mTtwo\ variable is defined as
\begin{equation}
    \mTtwo^2 = \underset{{p\mkern-9.5mu/}_1+{p\mkern-9.5mu/}_2={p\mkern-9.5mu/}_\text{T}}{\textrm{min}} \left[\textrm{max}\{\mt^2(p_\text{T}^a,{p\mkern-9.5mu/}_1),\mt^2(p_\text{T}^b,{p\mkern-9.5mu/}_2)\}\right],
\end{equation}
where the minimization goes over all possible two-momenta, ${p\mkern-9.5mu/}_{1,2}$, such that their sum gives the observed missing transverse momentum ${p\mkern-9.5mu/}_\text{T}$, and where each of $p_\text{T}^a$ and $p_\text{T}^b$ is the combined transverse momentum of a charged lepton and a jet~\cite{PLACEHOLDER:PAPER:CITATION}.
For the VBF-enriched \TwoJet category, no $WW$ CR pure enough in $WW$ processes can be defined. 
The $qqWW$ background is therefore estimated from simulated samples normalised to their theoretical cross section. 
To validate the consistency between the simulated samples and data, a $WW$ validation region (VR) is used, defined by requiring \todo{WW VR selections}. 

\todo{WW VR figure}

\todo{FIGURE}
The purity of the $WW$ process in the CRs as well as the normalisation factors are summarized in \cref{table-NFs-purities}.


\subsubsection{Top}
The $t\bar{t}$ and $Wt$ processes are normalised with a combined template using the same NF, extracted from dedicated top-quark CRs.
The relative contribution of $t\bar{t}$ and $Wt$ processes are comparable between the CRs and SRs and its uncertainties are accounted for by treating the relevant sources of uncertainties separately.

The top-quark CRs in the \ZeroJet, \OneJet, and VBF-enriched \TwoJet\ category are defined by requiring the presence of $b$-jets. 
For the \ZeroJet\ category, a $b$-jet with $20 < \pT < 30\,$GeV is required.
For the \OneJet\ category, exactly one $b$-jet with $\pT > 30\,$GeV and no $b$-jet with $20 < \pT < 30\,$GeV is required. These specific selections ensure that the ratio between $t\bar{t}$ and $Wt$ processes is similar between the CR and the SR without compromising the purity of top-quark processes in the CR.
The top-quark CR in the VBF-enriched \TwoJet category is defined by requiring the presence of exactly one $b$-jet with $\pT > 30\,$GeV.
In the ggF-enriched \TwoJet\ region, an alternative strategy is used that does not rely on $b$-jet requirements, while still retaining a high purity of top-quark events. The CR is defined by requiring $\mll > 80\,$GeV and $\mTtwo < 165\,$GeV, where the latter selection ensures orthogonality to the $WW$ CR. This definition reduces the uncertainties from flavour tagging selections and is therefore preferable to a CR defined using $b$-jet criteria.

The purity of the $WW$ process in the CRs as well as the normalisation factors are summarized in \cref{table-NFs-purities}.

\subsubsection{\Ztautau background}
The \Ztautau\ background is normalised using dedicated CRs, defined separately for each analysis category. 
The CR in the \ZeroJet category requires a large opening angle of the leptons of $\DPhill > 2.8$.
For the \OneJet and two \TwoJet categories, the \mtautau observable is exploited to separate the \Ztautau CRs from the SRs by selecting the region around the nominal $Z$ boson mass.
The specific selections are $\mtautau - m_Z < 25\,$GeV for the \OneJet and ggF-enriched \TwoJet category and $|\mtautau - m_Z| \le 25\,$GeV for the VBF-enriched \TwoJet category. 

In order to increase the statistical precision of the CRs, some selections made in the SR are relaxed or dropped.
In the \ZeroJet and \OneJet category, no \pTmiss requirement is applied and the \mll\ selection is relaxed to $\mll < 80\,$GeV. 
In the VBF-enriched \TwoJet category, the $\mll$ requirement is relaxed to $\mll < 70\,$GeV and the region above $\mtautau = m_Z + 25\,$GeV is additionally considered.

The purity of the \Ztautau\ process in the CRs as well as the normalisation factors are summarized in \cref{table-NFs-purities}.

\todo{Table with bkg selection HIGHLIGHTING the important cuts for orthogonality}

\subsection{Backgrounds with misidentified leptons}
- Explain fake factor method!

\subsection{Backgrounds estimated with simulated samples}
- Mention validation region
- List xsecs?