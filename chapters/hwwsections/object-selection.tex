
\subsection{Lepton selection}
\subsection{Jet selection}

\TDinote{}{Checkout JVT extension to 120 GeV}

\subsection{Missing transverse energy}
\subsection{Overlap removal}
\label{subsec:overlap-removal}

The inputs to the \emph{anti-$k_T$} algorithm are typically used also in other object reconstruction algorithms such as in the reconstruction of electrons and photons (see \cref{sec:electron-photon-reconstruction}).
To avoid double consideration of detector signals in the event reconstruction a dedicated procedure known as \emph{overlap removal} is performed in physics analyses that resolves these ambiguities. The procedure used for the work presented in this thesis is described in the relevant analysis chapter in \cref{subsec:overlap-removal}.
\Rinote{}{Not sure where exactly this belongs. Also need to make sure that this reflects the correct understanding of overlap removal}


\subsection{Pile-up reweighting}
In order to compare Monte Carlo simulated events with actual data, the amount of pile-up underlying the hard scatter needs to be account for.
To this end, a method known as \emph{pile-up reweighting} assigns a dedicated \emph{pile-up weight} to each event so that the average pile-up in simulated collision events match the actual running conditions. The reweighting procedure is performed separately for the data recorded in the years 2015 and 2016, 2017, and 2018.
