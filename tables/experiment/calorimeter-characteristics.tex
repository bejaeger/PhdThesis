\begin{table}[t]
    \centering
    \begin{tabular}{l | l l l}
        \toprule
                                                 & Active material               & Passive material         & Coverage                \\
        \midrule
        %LAr electromagnetic calorimeter barrel &  liquid argon     & lead             & $\approx 24 X_0$    & \absetaST{1.475}         \\
        %LAr electromagnetic calorimeter end-cap &  liquid argon     & lead             & $\approx 22 X_0$    & \absetaBT{1.375}{3.2}    \\
        LAr electromagnetic                      & \multirow{2}{*}{Liquid argon} & \multirow{2}{*}{Lead}    & \multirow{2}{*}{\absetaST{3.2}} \\
        calorimeter                              &                               &                          &                                 \\
        \midrule
        Tile calorimeter                         & Plastic scintillators                 & Steel                    & \absetaST{1.7}                  \\
        \midrule
        LAr hadronic end-caps                    & Liquid argon                  & Copper                   & \absetaBT{1.5}{3.2}             \\
        \midrule
        \multirow{2}{*}{LAr forward calorimeter} & \multirow{2}{*}{Liquid argon} & Copper (1st layer)       & \multirow{2}{*}{\absetaBT{3.1}{4.9}}             \\
                                                 &                               & Tungsten (2nd/3rd layer) &             \\
        \bottomrule
    \end{tabular}
    \caption{
        Characteristics of the different calorimeter systems, including the active and passive material used, and their \abseta coverage. Taken from \ccite{PERF-2007-01}.}
    \label{tab:calorimeter-characteristics}
\end{table}
%%%%%%%%%%%%%%%%%%%%%%%%%%%%%%%%%%%%%%%%%%%%%%%%%%%%%%%%%%%%%%%%
% Version with stopping power


% \begin{table}
%     \centering
%     \begin{tabular}{l | l l l l}
%         \toprule
%                                                  & Active material               & Passive material      & Stopping power                          & \abseta coverage                     \\
%         \midrule
%         %LAr electromagnetic calorimeter barrel &  liquid argon     & lead             & $\approx 24 X_0$    & \absetaST{1.475}         \\
%         %LAr electromagnetic calorimeter end-cap &  liquid argon     & lead             & $\approx 22 X_0$    & \absetaBT{1.375}{3.2}    \\
%         LAr electromagnetic                      & \multirow{2}{*}{Liquid argon} & \multirow{2}{*}{lead} & \multirow{2}{*}{$\approx 22-38\,X_0$}   & \multirow{2}{*}{\absetaST{3.2}}      \\
%         calorimeter                              &                               &                       &                                         &                                      \\
%         \midrule
%         Tile calorimeter                         & Scintillators                 & Steel                 & $\approx 7-10\,\lambda$               & \absetaST{1.7}                       \\
%         \midrule
%         LAr hadronic end-caps                    & Liquid argon                  & Copper                & $\approx TBD\,\lambda$                 & \absetaBT{1.5}{3.2}                  \\
%         \midrule
%         \multirow{4}{*}{LAr forward calorimeter} & \multirow{4}{*}{Liquid argon} & Copper                & $\approx 28\,X_0$ /                     & \multirow{2}{*}{\absetaBT{3.1}{4.9}} \\
%                                                  &                               & (1st layer)           & $\approx 3\,\lambda$                  &                                      \\
%                                                  &                               & Tungsten              & \multirow{2}{*}{$\approx 7\,\lambda$} & \multirow{2}{*}{\absetaBT{3.1}{4.9}} \\
%                                                  &                               & (2nd/3rd layer)       &                                         &                                      \\
%         \bottomrule
%     \end{tabular}
%     \caption{
%         Characteristics of the different calorimeter systems, including the active and passive material that is used, the approximate stopping power in terms of either the radiation length ($X_0$) or interaction length ($\lambda_0$) depending on whether the system targets electromagnetic or hadronic showers, and the \abseta coverage.
%         The stopping power is given as a range as it strongly depends on the $\eta$ region. \cite{PERF-2007-01}}
%     \label{tab:calorimeter-characteristics}
% \end{table}