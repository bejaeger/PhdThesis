
%!TEX TS-program = lualatex
\documentclass{standalone}

\usepackage[compat=1.1.0]{tikz-feynman}
\usetikzlibrary{positioning}

\begin{document}
\providecommand{\defaultDist}{1.5cm}%
\providecommand{\qqAngle}{45}%
\providecommand{\ttAngle}{45}%
\providecommand{\vWAngle}{25}%
\providecommand{\lepDist}{1.0cm}%
\providecommand{\lepAngle}{35}%

%
\begin{tikzpicture}
  \begin{feynman}

    \begin{scope}[node distance=\defaultDist]

      \vertex (v12) at (0, 0);
      \vertex (v11) at ([xshift=-\defaultDist]v12);

      \vertex (q1) at ([shift=(180-\qqAngle:\defaultDist)]v11) {\(q\)};
      \vertex (q2) at ([shift=(180+\qqAngle:\defaultDist)]v11) {\(\bar{q}\)};

      \vertex (t1) at ([shift=( \ttAngle:\defaultDist)]v12);
      \vertex (t2) at ([shift=(-\ttAngle:\defaultDist)]v12);

      \vertex (w1) at ([shift=( \vWAngle:\defaultDist)]t1);
      \vertex (w2) at ([shift=( -\vWAngle:\defaultDist)]t2);

    \end{scope}
    \begin{scope}
      \vertex[anchor=west] (v3) at ([shift=({-\vWAngle:\defaultDist})]t1) {\(\bar{\nu}_{\tau}\)};
      \vertex[anchor=west] (v4) at ([shift=({\vWAngle:\defaultDist})]t2) {\(\nu_{\tau}\)};
    \end{scope}

    \begin{scope}
      \vertex[anchor=west] (l1) at ([shift=({\lepAngle:\lepDist})]w1) {\(\ell^{+}\)};
      \vertex[anchor=west] (v1) at ([shift=({-\lepAngle:\lepDist})]w1) {\(\nu\)};
      \vertex[anchor=west] (l2) at ([shift=({\lepAngle:\lepDist})]w2) {\(\ell^{-}\)};
      \vertex[anchor=west] (v2) at ([shift=({-\lepAngle:\lepDist})]w2) {\(\bar{\nu}\)};
    \end{scope}

    \path (v12) -- node[shift=({ 110:0.4cm})] {\(\tau^{+}\)} (t1);
    \path (v12) -- node[shift=({-110:0.4cm})] {\(\tau^{-}\)} (t2);
    \path (v11) -- node[shift=({-100:0.4cm})] {\(Z/\gamma^{\ast}\)} (v12);
    \path (t1) -- node[shift=({ 110:0.4cm})] {$W^{+}$} (w1);
    \path (t2) -- node[shift=({-110:0.4cm})] {$W^{-}$} (w2);
    \diagram*{
      (q1) -- [fermion] (v11) -- [fermion] (q2);
      (v11) -- [boson] (v12),
      (t2) -- [fermion] (v12) -- [fermion] (t1);
      (w1) -- [boson] (t1) -- [fermion] (v3),
      (v4) -- [fermion] (t2) -- [boson] (w2),
      (l1) -- [fermion] (w1) -- [fermion] (v1),
      (v2) -- [fermion] (w2) -- [fermion] (l2),
    };
  \end{feynman}
\end{tikzpicture}
%
\end{document}
