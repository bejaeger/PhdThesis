%!TEX TS-program = lualatex

\documentclass{standalone}

\usepackage[compat=1.1.0]{tikz-feynman}
% \usetikzlibrary{positioning}

\begin{document}
\providecommand{\defaultDist}{1.5cm}%
\providecommand{\tChannelDist}{1.9cm}%
\providecommand{\lepDist}{1.0cm}%
\providecommand{\lepAngle}{35}%
%
%
\begin{tikzpicture}
  \begin{feynman}
    \begin{scope}
      \vertex (q1) at (0, 0){\(q\)};
      \vertex (q2) at ([xshift=\defaultDist]q1);
      \vertex (q3) at ([yshift=-\tChannelDist]q2);
      \vertex (q4) at ([xshift=\defaultDist]q3){\(q'\)};
      \vertex (g1) at ([xshift=-\defaultDist]q3){\(g\)};
      \vertex (W1) at ([xshift=\defaultDist]q2);
    \end{scope}

    \begin{scope}
      \vertex[anchor=west] (l1) at ([shift=({\lepAngle:\lepDist})]W1) {\(\ell^{+}\)};
      \vertex[anchor=west] (v1) at ([shift=({-\lepAngle:\lepDist})]W1) {\(\nu\)};
    \end{scope}
    \path (q2) -- node[shift=({ 110:0.4cm})] {$W^{+}$} (W1);
    \diagram*{
      (q1) -- [fermion] (q2) -- [fermion] (q3) -- [fermion] (q4);
      (g1) -- [gluon] (q3);
      (q2) -- [boson] (W1);
      (l1) -- [fermion] (W1) -- [fermion] (v1),
    };
  \end{feynman}
\end{tikzpicture}
\end{document}
