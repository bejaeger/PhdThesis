%!TEX TS-program = lualatex

\documentclass{standalone}

\usepackage[compat=1.1.0]{tikz-feynman}
\usetikzlibrary{positioning}

\begin{document}
\providecommand{\defaultDist}{1.5cm}%
\providecommand{\topAngle}{30}%
%
\begin{tikzpicture}
  \begin{feynman}

    \begin{scope}[node distance=\defaultDist]

      \vertex (t3) at (0, 0);

      \vertex (t2) at ([shift=(180+\topAngle:\defaultDist)]t3);
      \vertex (t1) at ([shift=(   -\topAngle:\defaultDist)]t2) {\(\bar{t}\)};
      \vertex (t4) at ([shift=(180-\topAngle:\defaultDist)]t3);
      \vertex (t5) at ([shift=(   +\topAngle:\defaultDist)]t4) {\(t\)};

      \vertex (g1) at ([xshift=-\defaultDist]t4) {\(g\)};
      \vertex (g2) at ([xshift=-\defaultDist]t2) {\(g\)};

      \vertex (h1) at ([xshift=\defaultDist]t3) {\(H\)};

    \end{scope}

    \diagram*{
      (t1) -- [fermion] (t2) -- [fermion] (t3) -- [fermion] (t4) -- [fermion] (t5),
      (g1) -- [gluon] (t4),
      (g2) -- [gluon] (t2),
      (t3) -- [scalar] (h1),
    };
  \end{feynman}
\end{tikzpicture}
%
\end{document}
