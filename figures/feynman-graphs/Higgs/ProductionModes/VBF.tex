%!TEX TS-program = lualatex

\documentclass{standalone}

\usepackage[compat=1.1.0]{tikz-feynman}
\usetikzlibrary{positioning}

\begin{document}
\providecommand{\defaultDist}{1.5cm}%
\providecommand{\vvAngle}{45}%
\providecommand{\qqAngle}{20}%
%
\begin{tikzpicture}
  \begin{feynman}

    \begin{scope}[node distance=\defaultDist]

      \vertex (vX2)  at (0, 0);
      \vertex (v11) at ([shift=(180-\vvAngle:\defaultDist)]vX2);
      \vertex (v21) at ([shift=(180+\vvAngle:\defaultDist)]vX2);

      \vertex (q11) at ([xshift=-\defaultDist]v11)           {\(q\)};
      \vertex (q12) at ([shift=( \qqAngle:\defaultDist)]v11) {\(q'\)};
      \vertex (q21) at ([xshift=-\defaultDist]v21)           {\(q\)};
      \vertex (q22) at ([shift=(-\qqAngle:\defaultDist)]v21) {\(q'\)};

      \vertex (h1) at ([shift=(  0:\defaultDist)]vX2) {\(H\)};

    \end{scope}

    \diagram*{
      (q11) -- [fermion] (v11) -- [fermion] (q12);
      (q21) -- [fermion] (v21) -- [fermion] (q22);
      (v11) -- [boson, edge label=\(V\)]   (vX2) -- [boson, edge label=\(V\)]   (v21);
      (vX2) -- [scalar] (h1),
    };
  \end{feynman}
\end{tikzpicture}
%
\end{document}
