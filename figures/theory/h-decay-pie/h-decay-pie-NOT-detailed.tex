% Based on example at https://texample.net/media/tikz/examples/TEX/pie-chart-color.tex

\documentclass{standalone}

\usepackage{tikz}
\usetikzlibrary{positioning}
\usetikzlibrary{calc}

\begin{document}
%
\newcommand{\outerSlice}[6]{
  % Command to plot the outer slice of the circle.
  % Need to have \currentAngle defined
  % (it is automatically updated after drawing)
  % arguments:
  %  1: stop angle (relative to start angle)
  %  2: explode angle
  %  3: inner radius
  %  4: outer radius
  %  5: explode radius
  %  6: color
  \draw[pie style, fill={#6}]
    [shift=(#2:#5)]
    (\currentAngle:#3) -- (\currentAngle:#4)
    arc (\currentAngle:\currentAngle+#1:#4)
    -- (\currentAngle+#1:#3)
    arc (\currentAngle+#1:\currentAngle:#3);

  \pgfmathparse{\currentAngle+#1}
  \xdef\currentAngle{\pgfmathresult}
}%
%
\newcommand{\outerSliceWLabel}[8]{
  % Same as \outerSlice, but also adds a label with pin.
  % arguments:
  %  1-6: see \outerSlice command above
  %  7: pin start coordinate
  %  8: label

  % Draw label
  \node[anchor=west] (to) at (#7) {#8};

  % Draw line from start coordinate to label
  % [xshift=-1.0cm] empirically found to look good.
  \path (#2:#5) -- ++(\currentAngle+0.5*#1:#4) node[shape=coordinate] (from) {};
  \draw [pin style] (from) [xshift=-1.0cm]-- (to);

  \outerSlice{#1}{#2}{#3}{#4}{#5}{#6}
}%
%
%
% BRs in percent
\providecommand{\BRGammaGamma}{ 0.227  }%
\providecommand{\BRZZ}        { 2.619  }%
\providecommand{\BRWW}        {21.37   }%
\providecommand{\BRTauTau}    { 6.272  }%
\providecommand{\BRbb}        {58.24   }%
\providecommand{\BRcc}        { 2.891  }%
\providecommand{\BRZGamma}    { 0.1533 }%
\providecommand{\BRMuMu}      { 0.02176}%
\providecommand{\BRgg}        { 8.187  }%
%
\providecommand{\BRMuMuZGamma}{ 0.17506}%
%
\providecommand{\BRevev}{0.2518}%
\providecommand{\BRmvmv}{0.2994}%
\providecommand{\BRevmv}{0.5038}%
\providecommand{\BRtvlv}{1.283}%
\providecommand{\BRlvqq}{9.396}%
\providecommand{\BRqqqq}{9.636}%
\providecommand{\BRlvlvSF}{0.5512}%
%
\providecommand{\bbbar}{\ensuremath{b\bar{b}}}%
\providecommand{\ccbar}{\ensuremath{c\bar{c}}}%
%
% Colors taken from https://github.com/Gnuplotting/gnuplot-palettes
\definecolor{gnuplot-set1-0}{HTML}{E41A1C}% red
\definecolor{gnuplot-set1-1}{HTML}{377EB8}% blue
\definecolor{gnuplot-set1-2}{HTML}{4DAF4A}% green
\definecolor{gnuplot-set1-3}{HTML}{984EA3}% purple
\definecolor{gnuplot-set1-4}{HTML}{FF7F00}% orange
\definecolor{gnuplot-set1-5}{HTML}{FFFF33}% yellow
\definecolor{gnuplot-set1-6}{HTML}{A65628}% brown
\definecolor{gnuplot-set1-7}{HTML}{F781BF}% pink
\definecolor{gnuplot-set2-0}{HTML}{66C2A5}% teal
\definecolor{gnuplot-set2-1}{HTML}{FC8D62}% orange
\definecolor{gnuplot-set2-2}{HTML}{8DA0CB}% lilac
\definecolor{gnuplot-set2-3}{HTML}{E78AC3}% magenta
\definecolor{gnuplot-set2-4}{HTML}{A6D854}% lime green
\definecolor{gnuplot-set2-5}{HTML}{FFD92F}% banana
\definecolor{gnuplot-set2-6}{HTML}{E5C494}% tan
\definecolor{gnuplot-set2-7}{HTML}{B3B3B3}% grey
\definecolor{gnuplot-set3-0}{HTML}{8DD3C7}% teal
\definecolor{gnuplot-set3-1}{HTML}{FFFFB3}% banana
\definecolor{gnuplot-set3-2}{HTML}{BEBADA}% lilac
\definecolor{gnuplot-set3-3}{HTML}{FB8072}% red
\definecolor{gnuplot-set3-4}{HTML}{80B1D3}% steel blue
\definecolor{gnuplot-set3-5}{HTML}{FDB462}% adobe orange
\definecolor{gnuplot-set3-6}{HTML}{B3DE69}% lime green
\definecolor{gnuplot-set3-7}{HTML}{FCCDE5}% mauve
%
\definecolor{gnuplot-set3-3-5}{HTML}{FFD7D2}% shades of gnuplot-set3-3
\definecolor{gnuplot-set3-3-4}{HTML}{FFAFA6}% shades of gnuplot-set3-3
\definecolor{gnuplot-set3-3-3}{HTML}{FB7F72}% shades of gnuplot-set3-3
\definecolor{gnuplot-set3-3-2}{HTML}{E73826}% shades of gnuplot-set3-3
\definecolor{gnuplot-set3-3-1}{HTML}{B61100}% shades of gnuplot-set3-3
%
% Choose one of the previously defined color sets
\providecommand{\colorset}{gnuplot-set3}%
%
% Variables defining layout
\def\angle{0}%           starting angle
\def\radius{3}%          radius
\def\radiusInner{2}%     inner radius for WW
\def\radiusExplode{0.3}% offset for WW center
%
\begin{tikzpicture}[
  pie style/.style={draw=black, very thin, line join=round},
  pin style/.style={black!60, very thin, shorten <=0.1 cm},
  scale=1,]

  % Coordinates for aligned labels outside of circle
  \coordinate [shift=(10:\radius*3.7/3)] (label_evmv)       at (0, +0.00*\radius/3);
  \coordinate [shift=(10:\radius*3.7/3)] (label_lvlvSF)     at (0, +0.45*\radius/3);
  \coordinate [shift=(10:\radius*3.7/3)] (label_tvlv)       at (0, +0.90*\radius/3);
  \coordinate [shift=(10:\radius*3.7/3)] (label_GammaGamma) at (0, -0.60*\radius/3);
  \coordinate [shift=(10:\radius*3.7/3)] (label_MuMuZGamma) at (0, -1.00*\radius/3);

  % Draw individual pie slices
  % gamma gamma
  \draw[pie style, fill={\colorset-1}]
    (0,0) -- (\angle:\radius)
    arc (\angle:\angle+\BRGammaGamma*3.6:\radius) -- cycle;
  % Set gamma gamma label
  \node[anchor=west] at (label_GammaGamma) {$\gamma\gamma$};
  \path (\angle+0.5*\BRGammaGamma*3.6:\radius) node[shape=coordinate] (from) {};
  \draw [pin style] (from) [xshift=-1.0cm] -- (label_GammaGamma);

  \pgfmathparse{\angle+\BRGammaGamma*3.6}
  \xdef\angle{\pgfmathresult}

  % ZZ
  \draw[pie style, fill={\colorset-0}]
    (0,0) -- (\angle:\radius)
    arc (\angle:\angle+\BRZZ*3.6:\radius) -- cycle;
  \node at (\angle+0.5*\BRZZ*3.6:0.8*\radius) {$ZZ$};
  \pgfmathparse{\angle+\BRZZ*3.6}
  \xdef\angle{\pgfmathresult}

  \pgfmathparse{\angle+\BRWW*3.6*0.5}
  \xdef\angleExplode{\pgfmathresult}

  % WW
  \draw[pie style, fill={\colorset-3}]
    (0,0) -- (\angle:\radius)
    arc (\angle:\angle+\BRWW*3.6:\radius) -- cycle;
  \node  at (\angle+0.5*\BRWW*3.6:0.45*\radius) {$W^+W^-$};
  \xdef\currentAngle{\angle}
  \pgfmathparse{\angle+\BRWW*3.6}
  \xdef\angle{\pgfmathresult}

  % outer slice around WW
  % \outerSliceWLabel{\BRevmv*3.6}  {\angleExplode}{\radiusInner}{\radius}{\radiusExplode}{\colorset-3-1}{label_evmv}  {{\color{red}$e\nu\mu\nu$}}
  % \outerSliceWLabel{\BRlvlvSF*3.6}{\angleExplode}{\radiusInner}{\radius}{\radiusExplode}{\colorset-3-2}{label_lvlvSF}{$e\nu e\nu/\mu\nu\mu\nu$}
  % \outerSliceWLabel{\BRtvlv*3.6}  {\angleExplode}{\radiusInner}{\radius}{\radiusExplode}{\colorset-3-3}{label_tvlv}  {$\tau\nu\ell\nu$}
  % \outerSlice{\BRlvqq*3.6}  {\angleExplode}{\radiusInner}{\radius}{\radiusExplode}{\colorset-3-4}
  % \node [shift=(\angleExplode:\radiusExplode)] at (\currentAngle-0.5*\BRlvqq*3.6:0.83*\radius) {$\ell\nu qq$};
  % \outerSlice{\BRqqqq*3.6}  {\angleExplode}{\radiusInner}{\radius}{\radiusExplode}{\colorset-3-5}
  % \node [shift=(\angleExplode:\radiusExplode)] at (\currentAngle-0.5*\BRqqqq*3.6:0.82*\radius) {$qqqq$};

  % gg
  \draw[pie style, fill={\colorset-2}]
    (0,0) -- (\angle:\radius)
    arc (\angle:\angle+\BRgg*3.6:\radius) -- cycle;
  \node at (\angle+0.5*\BRgg*3.6:0.7*\radius) {$gg$};
  \pgfmathparse{\angle+\BRgg*3.6}
  \xdef\angle{\pgfmathresult}

  % bb
  \draw[pie style, fill={\colorset-6}]
    (0,0) -- (\angle:\radius)
    arc (\angle:180:\radius)
    arc (180:270:\radius)
    arc (270:\angle+\BRbb*3.6:\radius)
    -- cycle;
    % Draw the arc in steps. Otherwise too much whitespace is added.
  \node at (\angle+0.5*\BRbb*3.6:0.5*\radius) {$\bbbar$};
  \pgfmathparse{\angle+\BRbb*3.6}
  \xdef\angle{\pgfmathresult}

  % cc
  \draw[pie style, fill={\colorset-5}]
    (0,0) -- (\angle:\radius)
    arc (\angle:\angle+\BRcc*3.6:\radius) -- cycle;
  \node at (\angle+0.5*\BRcc*3.6:0.8*\radius) {$\ccbar$};
  \pgfmathparse{\angle+\BRcc*3.6}
  \xdef\angle{\pgfmathresult}

  % tau tau
  \draw[pie style, fill={\colorset-4}]
    (0,0) -- (\angle:\radius)
    arc (\angle:\angle+\BRTauTau*3.6:\radius) -- cycle;
  \node [yshift=0.08 cm] at (\angle+0.5*\BRTauTau*3.6:0.82*\radius) {$\tau^+\tau^-$};
  \pgfmathparse{\angle+\BRTauTau*3.6}
  \xdef\angle{\pgfmathresult}

  % other
  \draw[pie style, fill={\colorset-7}]
    (0,0) -- (\angle:\radius)
    arc (\angle:\angle+\BRMuMuZGamma*3.6:\radius) -- cycle;
  \node[anchor=west] at (label_MuMuZGamma) {others};
  \path (\angle+0.5*\BRMuMuZGamma*3.6:\radius) node[shape=coordinate] (from) {};
  \draw [pin style] (from) [xshift=-1.0cm] -- (label_MuMuZGamma);
  \pgfmathparse{\angle+\BRMuMuZGamma*3.6}
  \xdef\angle{\pgfmathresult}

  % make bounding box symmetric so that circle is in center
  \useasboundingbox let \p1=(current bounding box.north east),
                        \p2=(current bounding box.south west),
                        \n{x}={max(abs(\x1),abs(\x2))}, \n{y}={max(abs(\y1),abs(\y2))} in
    (-\n{x}, 0) -- (\n{x}, 0);
    % (-\n{x}, -\n{y}) rectangle (\n{x}, \n{y}); % this would also add space to top and bottom

\end{tikzpicture}
%
\end{document}
