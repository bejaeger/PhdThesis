% Standard model of physics
% Author: Carsten Burgard
\documentclass[border=10pt]{standalone}
\usepackage{tikz}
\usetikzlibrary{calc,positioning,shadows.blur,decorations.pathreplacing}
\usepackage{etoolbox}
\usepackage{pgfplots, pgfplotstable}

\begin{document}

\def\angle{0}
\def\radius{2.5}
\def\cyclelist{{"blue","red","green","orange"}}
\newcount\cyclecount \cyclecount=-1
\newcount\ind \ind=-1
  \begin{tikzpicture}
    \foreach \percent/\name in {
      10.7/e$\nu$,
      10.6/$\mu\nu$,
      11.4/$\tau\nu$,
      67.4/hadrons,
    } {
      \ifx\percent\empty\else                 % If \percent is empty, do nothing
      \global\advance\cyclecount by 1       % Advance cyclecount
      \global\advance\ind by 1              % Advance list index
      \ifnum3<\cyclecount                   % If cyclecount is larger than list
      \global\cyclecount=0                %   reset cyclecount and
      \global\ind=0                       %   reset list index
      \fi
      \pgfmathparse{\cyclelist[\the\ind]}   % Get color from cycle list
      \edef\color{\pgfmathresult}           %   and store as \color
      % Draw angle and set labels
      \draw[fill={\color!50},draw={\color!50}] (0,0) -- (\angle:\radius) arc (\angle:\angle+\percent*3.6:\radius) -- cycle;
      \node at (\angle+0.5*\percent*3.6:0.7*\radius) {\percent\%};
      \node[pin=\angle+0.5*\percent*3.6:\name] at (\angle+0.5*\percent*3.6:\radius) {};
      \pgfmathparse{\angle+\percent*3.6}    % Advance angle
      \xdef\angle{\pgfmathresult}           %   and store in \angle
      \fi
    };
  \end{tikzpicture}


\end{document}